\documentclass{article}

\usepackage[utf8]{inputenc}
\usepackage{enumerate}
\usepackage{amsfonts}
\usepackage{amsmath}
\usepackage{amsthm}
\usepackage{blindtext}
\usepackage{graphicx}
\usepackage[numbers]{natbib}
\usepackage{amssymb}
\usepackage{mathtools}
\usepackage{stmaryrd}
\usepackage{tikz-cd}
\usepackage{relsize}
\usepackage{mathrsfs}
\usepackage{tikzit}
\usepackage{ upgreek }
\usepackage[normalem]{ulem}
\usepackage{quiver} 




\newtheorem{theorem}{Theorem}
\newtheorem{lemma}{Lemma}
\newtheorem{corollary}{Corollary}
\newtheorem{conjecture}{Conjecture}
\newtheorem{proposition}{Proposition}
\theoremstyle{definition}
\newtheorem*{definition}{Definition}
\newtheorem{remark}{Remark}
\newtheorem{experiment}{Experiment}
\newtheorem{proposition-definition}{Proposition-Definition}


\graphicspath{ {./images/} }
\numberwithin{figure}{section}
\setcounter{section}{-1}


\title{The Verlinde Formula for MTCs}
\author{Milo Moses}


\begin{document}


\maketitle

\newcommand{\RR}{\mathbb{R}}
\newcommand{\HH}{\mathbb{H}}
\newcommand{\NN}{\mathbb{N}}
\newcommand{\QQ}{\mathbb{Q}}
\newcommand{\CC}{\mathbb{C}}
\newcommand{\FF}{\mathbb{F}}
\newcommand{\ZZ}{\mathbb{Z}}
\newcommand{\Zcal}{\mathcal{Z}}
\newcommand{\Ncal}{\mathcal{N}}
\newcommand{\LL}{\mathscr{L}}
\newcommand{\TT}{\mathcal{T}}
\newcommand{\Ccat}{\mathscr{C}}
\newcommand{\Dcat}{\mathscr{D}}
\newcommand{\Ecat}{\mathscr{E}}
\newcommand{\st}{\,\,\mathrm{s.t.}\,\,}
\newcommand{\mm}{\mathfrak{m}}
\newcommand{\pp}{\mathfrak{p}}
\newcommand{\Hom}{\mathrm{Hom}}
\newcommand{\Aut}{\mathrm{Aut}}
\newcommand{\Frac}{\mathrm{Frac}}
\newcommand{\tr}{\mathrm{tr}}
\newcommand{\op}{\mathrm{op}}
\newcommand{\res}{\mathrm{res}}
\newcommand{\im}{\mathrm{im}}
\newcommand{\ev}{\mathrm{ev}}
\newcommand{\coev}{\mathrm{coev}}
\newcommand{\id}{\mathrm{id}}
\newcommand{\coker}{\mathrm{coker}}
\newcommand{\SL}{\mathrm{SL}}
\newcommand{\End}{\mathrm{End}}
\newcommand{\Rep}{\bold{Rep}}
\newcommand{\Set}{\bold{Set}}
\newcommand{\Vecc}{\bold{Vec}}
\newcommand{\Top}{\bold{Top}}
\newcommand{\Grp}{\bold{Grp}}
\newcommand{\Hilb}{\bold{Hilb}}
\newcommand{\Bord}{\bold{Bord}}
\newcommand{\Cat}{\bold{Cat}}
\newcommand{\0}{\left|0\right>}
\newcommand{\1}{\left|1\right>}
\newcommand{\nullclass}{\left|\bold{0}\right>}
\newcommand{\alphaclass}{\left|\alpha\right>}
\newcommand{\betaclass}{\left|\beta\right>}
\newcommand{\alphabetaclass}{\left|\alpha\beta\right>}
\newcommand{\ppsi}{\left|\psi\right>}
\newcommand{\pphi}{\left|\phi\right>}
\newcommand{\func}{\mathrm{func}}
\newcommand{\bigleadsto}{\mathlarger{\mathlarger{\mathlarger{\leadsto}}}}
\newcommand{\vin}{\rotatebox[origin=c]{-90}{$\in$}}

\begin{equation*}
\tikzfig{temp}
\end{equation*}

One of the most powerful equations in the theory of modular tensor categories (MTCs) is the so-called \textit{Verlinde formula}. This formula was first conjectured by Verlinde \cite{verlinde1988fusion}, and proven the following year by Moore-Seiberg \cite{moore1989classical}. There are now many Verlinde-type formulas. Most imporantly, there is one for vertex operator algebras \cite{huang2008vertex} and one in algebraic geometry \cite{faltings1994proof}.

To begin our exposition, we set notation.  Fix a modular tensor category $\Ccat$, with set $\LL$ of isomorphism classes of simple objects. The $S$-matrix of $\Ccat$ is the matrix whose rows and collumns are labeled by $\LL$, and whose $(a,b)$ entry is

\begin{equation*}
\tikzfig{S-matrix}
\end{equation*}

where $\beta$ is the braiding on $\Ccat$, $\tr$ is the categorical trace on $\Ccat$, and $A,B$ are representatives of $a,b$. The fusion coefficients of $\Ccat$ are the unique non-negative integers $N^{\--,\--}_{\--}$ such that for all $a,b\in \LL$

$$A\otimes B \cong \bigoplus_{c\in \LL}N^{a,b}_{c} C$$

where $A,B,C$ are representatives of $a,b,c$. The quantum dimension of an object $a\in \LL$ is $d_a=\tr(\id_A)$. The Verlinde formula allows one to express the fusion coefficients in terms of the $S$-matrix entries:

\begin{equation}
N^{a,b}_{c}=\sum_{e\in \LL} \frac{S_{a,e} S_{b,e}(S^{-1})_{c,e}}{d_e}.\tag{\textasteriskcentered}
\end{equation}

This can be restated as saying that the $S$-matrix diagonalizes the fusion coefficents in the sense that if we let

$$N^{a}=(N^{a,b}_c)_{(b,c)\in \LL^2}$$

be the ``fusion matrix" corresponding to $a\in \LL$, then $SN^aS^{-1}$ is diagonal and its entries are normalized $S$-matrix values. That is,

$$D^a=SN^{a}S^{-1}$$

is diagonal and its $(b,b)$ entry is $s_{a,b}/d_b$. From this, (\textasteriskcentered) follows by expanding the equality $N^{a}=S^{-1} D^{a}S$.

We now give a proof. Let $\CC[\LL]$ be the vector space freely generated by $\LL$. Let $K_{\CC}(\Ccat)$ be algebra structure on $\CC[\LL]$ given by

$$\left| a\right>\cdot \left| b\right>=\left| a\otimes b \right>=\sum_{c\in\LL}N^{a,b}_c\left| c\right>.$$

This is the (complexified) Grothendieck ring of $\Ccat$. Let $\CC[\LL]^{\func}$ be the $\CC$-algebra obtained by endowing $\CC[\LL]$ with pointwise multiplication. We claim that the map $\mu: K_{\CC}(\Ccat)\to\CC[\LL]^{\func}$ sending $\left| a\right>\in K_{\CC}(\Ccat)$ to

\begin{equation*}
\tikzfig{grothendieck-ring-iso}
\end{equation*}

is an isomorphism of $\CC$-algebras. Here, we use Schur's lemma to identify morphisms $f:B\to B$ with the unique number $\lambda\in \CC$ such that $f=\lambda\cdot \id_B$. Taking trace, we have

\begin{equation*}
\tikzfig{mu-iso-formula}
\end{equation*}

It is a general category theoretic fact that natural transformations commute with direct sums, so it is clear from expanding that $\mu$ adds over directs sums, making it a group homomorphism. To verify that $\mu$ is a morphism of algebras, we compute as follows:

\begin{equation*}
\tikzfig{algebra-morphism}
\end{equation*}

We now verify $\mu$ is an isomorphism. Choosing a simple object $\left| a\right>\in K_{\CC}(\Ccat)$ it is clear that

$$\mu(\left| a\right>)=\sum_{b\in \LL}\frac{s_{a,b}}{d_b}\left| b\right>.$$

Thus, $\mu$ is given on the level of vector spaces by a scaled $S$-matrix. Since $S$ is invertible we get that $\mu$ is bijective as desired. Note the key use of the fact that quantum dimensions are non-zero in MTCs. The definition of $K_{\CC}(\Ccat)$ exactly says that left multiplication is represted by the fusion matrix $N^a$. The computation

\begin{align*}
[a]\cdot \mu^{-1}(\left| b\right>)&=\mu^{-1}\left(\left(\sum_{c\in \LL}\frac{s_{a,c}}{d_c}\left| c\right>\right)\cdot  \left| b\right>\right)\\
&=\mu^{-1}\left(\frac{s_{a,b}}{d_b}\left|b\right>\right)=\frac{s_{a,b}}{d_b}\cdot \mu^{-1}\left(\left|b\right>\right).
\end{align*}

shows $\mu^{-1}\left(\left|b\right>\right)$ is an eigenvector for $N^a$ with eigenvalue $s_{a,b}/d_b$. Since $\{\mu^{-1}(\left| b\right>)\}_{b\in\LL}$ is a basis for $K_{\CC}(\Ccat)$ this gives a diagonalization of $N^a$. Moreover, the formula for $\mu$ on simple objects tells us that after re-scaling collumns the change of basis matrix  is exactly $S$. Hence,

$$S N^a S^{-1}=D^a$$

with $D^a$ the diagonal matrix whose $(b,b)$ entry is $s_{a,b}/d_b$, as desired.


\bibliographystyle{alpha}
\bibliography{ref}


\end{document}






