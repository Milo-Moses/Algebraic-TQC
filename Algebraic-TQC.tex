\documentclass{article}

\usepackage{enumerate}
\usepackage{vwcol} 
\usepackage[margin=1.5in]{geometry}



\usepackage[utf8]{inputenc}
\usepackage{amsfonts}
\usepackage{amsmath}
\usepackage{amsthm}
\usepackage{blindtext}
\usepackage{graphicx}
\usepackage[numbers]{natbib}
\usepackage{amssymb}
\usepackage{mathtools}
\usepackage{stmaryrd}
\usepackage{tikz-cd}
\usepackage{relsize}
\usepackage{mathrsfs}
\usepackage{tikzit}
\usepackage{upgreek}
\usepackage[normalem]{ulem}
\usepackage{quiver} 


\newtheorem*{definition}{Definition}


\newtheorem{theorem}{Theorem}[section]
\newtheorem{lemma}{Lemma}[section]
\newtheorem{corollary}{Corollary}[section]
\newtheorem{conjecture}{Conjecture}[section]
\newtheorem{proposition}{Proposition}[section]
\theoremstyle{definition}
\newtheorem{remark}{Remark}[section]
\newtheorem{experiment}{Experiment}[section]
\newtheorem{proposition-definition}{Proposition-Definition}[section]


\graphicspath{ {./images/} }
\numberwithin{figure}{section}
\setcounter{section}{-1}


\newcounter{quantumdim}
\newcounter{fusionsystem}




\title{The Algebraic Theory\\ of \\ Topological Quantum Information}
\author{by Milo Moses}

\date{\textit{California Institute of Technology} \\ [2ex] \today}


\begin{document}


\maketitle


\newcommand{\RR}{\mathbb{R}}
\newcommand{\SO}{\mathrm{SO}}
\newcommand{\marginfootnote}[1]{\marginpar{\footnotemark}\footnotetext{#1}}



\newcommand{\HH}{\mathbb{H}}
\newcommand{\NN}{\mathbb{N}}
\newcommand{\QQ}{\mathbb{Q}}
\newcommand{\CC}{\mathbb{C}}
\newcommand{\FF}{\mathbb{F}}
\newcommand{\ZZ}{\mathbb{Z}}
\newcommand{\Zcal}{\mathcal{Z}}
\newcommand{\Ncal}{\mathcal{N}}
\newcommand{\LL}{\mathscr{L}}
\newcommand{\TT}{\mathcal{T}}
\newcommand{\Ccat}{\mathscr{C}}
\newcommand{\Dcat}{\mathscr{D}}
\newcommand{\Ecat}{\mathscr{E}}
\newcommand{\st}{\,\,\mathrm{s.t.}\,\,}
\newcommand{\mm}{\mathfrak{m}}
\newcommand{\pp}{\mathfrak{p}}
\newcommand{\dd}{\bold{d}}
\newcommand{\Hom}{\mathrm{Hom}}
\newcommand{\Aut}{\mathrm{Aut}}
\newcommand{\Frac}{\mathrm{Frac}}
\newcommand{\tr}{\mathrm{tr}}
\newcommand{\op}{\mathrm{op}}
\newcommand{\res}{\mathrm{res}}
\newcommand{\im}{\mathrm{im}}
\newcommand{\ev}{\mathrm{ev}}
\newcommand{\coev}{\mathrm{coev}}
\newcommand{\id}{\mathrm{id}}
\newcommand{\coker}{\mathrm{coker}}
\newcommand{\SL}{\mathrm{SL}}
\newcommand{\End}{\mathrm{End}}
\newcommand{\Rep}{\bold{Rep}}
\newcommand{\Set}{\bold{Set}}
\newcommand{\Vecc}{\bold{Vec}}
\newcommand{\Top}{\bold{Top}}
\newcommand{\Grp}{\bold{Grp}}
\newcommand{\Hilb}{\bold{Hilb}}
\newcommand{\Bord}{\bold{Bord}}
\newcommand{\FPdim}{\mathrm{FPdim}}
\newcommand{\Cat}{\bold{Cat}}
\newcommand{\func}{\mathrm{func}}
\newcommand{\0}{\left|0\right>}
\newcommand{\1}{\left|1\right>}
\newcommand{\nullclass}{\left|\bold{0}\right>}
\newcommand{\alphaclass}{\left|\alpha\right>}
\newcommand{\betaclass}{\left|\beta\right>}
\newcommand{\alphabetaclass}{\left|\alpha\beta\right>}
\newcommand{\ppsi}{\left|\psi\right>}
\newcommand{\pphi}{\left|\phi\right>}
\newcommand{\bigleadsto}{\mathlarger{\mathlarger{\mathlarger{\leadsto}}}}
\newcommand{\vin}{\rotatebox[origin=c]{-90}{$\in$}}


\begin{abstract}
This book aims to give a comprehensive account of the algebraic theory of topological quantum information. It is intended to be accessible both to mathematicians unfamiliar with quantum mechanics and theoretical physicists unfamiliar with category theory. Additionally, this text should make a good reference for working researchers in the field. A primary focus of this text is the balancing of powerful algebraic generalities with concrete examples, principles, and applications.
\end{abstract}


\newpage

\tableofcontents

\newpage


\section{Preface}
\label{Preface}


This book is a mathematical treatment of topological quantum information, with a focus on formal algebraic aspects and a special eye towards topological quantum computation. This manuscript began as an extended set of notes from a course on topological quantum field theory given by Zhenghan Wang in the winter of 2022 at UC Santa Barbara. Through his courses, his private tutoring, and his reccomendations, Zhenghan took me from a state of almost complete ignorance of mathematical physics to being a young researcher in the field. I am greatly emdebted to him for this, and it is certain that this book would not have existed without his guidance.

Great pains have been taken to make this book as pedagogical and accessable as possible. The hope is that it should be readable by both mathematicians unfamilar with quantum mechanics as well as theoretical physicists unfamiliar with category theory. A primary focus of this text is the balancing of powerful algebraic generalities with concrete examples, principles, and applications.

[WORK: add a list of other references the reader could use - past textbooks and expository articles]

[WORK: add paragraph that gets the reader excited about the material!]

[WORK: add a section which outlines the structure of the book]

\newpage

\section{Overview}
\label{overview}

\subsection{Conception introduction}

\subsubsection{Motivation and applications}

I will take as a definition \textit{topological quantum information} to be the study of information in topological quantum systems. A topological quantum system is some mathematical or physical system which is in a fundamental sense described by the mathematics of quantum mechanics and topology. The term \textit{quantum system} here is used in contrast to \textit{classical systems}. The flow of current through a conducting copper wire is described perfectly well by classical electro-magnetism, whereas the flow of current through a superconducting niobium-titanium wire necessarily requires quantum mechanics for its description.

The term \textit{topological system} is used in contrast to what might be called \textit{geometric systems}, though the term “geometric system” is a nonstandard one. In a geometric system, measurable quantities and phenomena depend on quantitative local aspects of your system - the distance between wires, the exact shape of a sample, the curvature of a sheet. In a topological system, measurable quantities and phenomena depend on more qualitative global aspects of your system - whether two wires cross or not, whether a sample is connected or not, whether a sheet curves into a ball or has a boundary.

I say that this book is about “topological quantum information” and not “topological quantum systems” for two reasons. The first is to highlight the fact that topological quantum systems \textit{do} have local geometric descriptions, but we will mostly be ignoring them in favor of focusing on global topological properties. The beauty of topological systems lies exactly in the fact that this global perspective retains all the essential information in the system. The second reason I use the term \textit{information} is to highlight this book’s eye towards topological quantum computing. 

Since Peter Shor’s 1994 discovery of an efficient factoring algorithm on quantum computers \cite{shor1994algorithms}, the primary goal of quantum information theorists has been to harness quantum information sufficiently well so that it can be used to make an efficient scalable quantum computer. One of the major hurdles in achieving this goal is that quantum information is \textit{fragile}. Information in quantum computers is oftentimes stored in the quantum degrees of freedom of individual atoms. Small amounts of noise coming from nearby electromagnetic fields or imperfections in experimental devices are often enough to affect the information being stored, resulting in \textit{errors} in the computation. In the early days of quantum computing it was not at all clear whether or not there was any way around this problem. Perhaps the inherent fragility of quantum information would make quantum computers impossible. This turned out not to be the case.

The beautiful observation is that errors are not nearly as catastrophic in \textit{topological} quantum systems. Errors are typically local, and by definition the information topological systems does not depend on local properties. Hence, under suitable conditions, topological systems are naturally error resistant! In the same way that invariants of topological spaces are supposed to be invariant under deformations in pure mathematics, information in topological systems is invariant under errors in mathematical physics. Hence, to solve the problem of noise all one has to do is make a \textit{topological} quantum computer! This observation was made in 1997 and is due independently to Alexei Kitaev and Michael Freedman \cite{kitaev2003fault, freedman1998p}. Since then topological quantum computing has grown and evolved, finding its way into almost every modern proposal for fault-tolerant quantum computing.

The first approach to topological quantum computing is to use a physical material, some literal condensed collection of atoms, which naturally behaves as a topological quantum system. These exist and have been studied for a long time. For instance, a two dimensional sheet of graphene becomes topological when it is subjected to low temperatures ($\approx$5 degrees Kelvin) and large magnetic fields ($\approx$15 Teslas) \cite{bolotin2009observation}. Topological quantum materials which can be used to make scalable quantum computers require intricate experiments to operate, which has been the most prominent roadblock in this approach.

The second approach to topological quantum computing is to artificially construct a topological system on top of a geometric one. The function of a quantum computer, almost by definition, is to simulate quantum systems. In particular, it can simulate \textit{topological} quantum systems. Since topological systems are resistant to local errors, this means that the original computer which is simulating the topological system will itself become resistant to local noise! This works exactly as described as long as the simulation itself is local, that is, local effects in the original system correspond to local effects in the simulated system. This technique of simulating a topological system to inherit its error-resistant properties is known as \textit{topological quantum error correction}. The advantage of this approach is that it works on any physical hardware available. The disadvantage is that to perform useful computations one must pass through the topological quantum error correction. In the end you will have a physical quantum system, simulating a topological system, which itself is simulating some target quantum system. The additional layer adds a hefty amount of overhead, which can eat up the majority of runtime and resources in your computer. It is for this reason that \textit{efficient} topological quantum error correction is an important and active area of research.

[WORK: add schematic: TQI => TQM, and TQI => TQEC]

Of course, the above discussion presents only one motivation for topological quantum information and only one example of an application. Topological quantum materials open a whole world of potential applications, and it seems likely they will play an important role in the techonologies of the future. Some proposed applications include [WORK: list examples of applications]. This breadth of potential applications is due in part to the number of different types of topological materials which have been discovered or theorized. This includes [WORK: list examples of topological materials]. The contents of this book certainly do not provide the entire picture for many of these materials. However, the hope is that it gives a picture of the algebraic structures which lie within them, hence helping readers think both concretely and conceptually about these materials and applications.

[WORK: add list of mathematical ideas as well, like fractons and non-semisimple topological order.]

\subsubsection{Mathematical picture}

The term \textit{topological quantum system} is broad. To get a rigorous mathematical subject, we will focus on a specific type of topological quantum system known as a \textit{topologically ordered} quantum system. Topological order is much more precise, though there are still conflicting definitions in the literature. Specifically, I will be focusing on \textit{(2+1)-dimensional} topological order. Here, I am using the physicist convention of using (2+1)D to refer to two space dimensions and one time dimension. That is, I will be discussing a completely flat topologically ordered system. For example, a single sheet of graphene at low temperatures and large magnetic fields exhibits (2+1)D topological order, and a two dimensional sheet of individually trapped atoms with topological quantum error correction being applied also exhibits (2+1)D topological order.

The naive mathematical description of topological order is \textit{topological quantum field theory}. In a sense, we can \textit{define} a topologically ordered system to be a system which admits a description in terms of topological quantum field theory. Geometric quantum systems require non-topological quantum field theory to describe.

While they are useful in many contexts, topological quantum field theories do have several difficulties associated with them. The most prominent of which is that their definition requires a large amount of essentially redundant data. This makes them hard to construct, inaccessible to computational methods, and opaque in their realizations of important phenomena. One solution to this problem is to dig down to the core algebraic data lying within topological quantum field theories. This algebraic data contains far less redundancy than the original topological quantum field theory - it only deals with essential information. This makes the algebraic data thus easier to construct, more accessible to computational methods, and more clear in its realization of important phenomena. The algebraic structure which houses the algebraic data of a (2+1)D topological quantum field theory is known as a \textit{modular tensor category}. These algebraic structures are the main mathematical object of this text, and it is because of this focus that this book is called the \textit{algebraic theory} of topological quantum information

Once one has a modular tensor category it is easy to manipulate the stored information to perform computations. This gives us the overall schema of our mathematical discussion. In Chapter [ref] we will give mathematical examples of topological order. In Chapter [ref] we will axiomatize topological order in terms of topological quantum field theory. In Chapter [ref] we will construct the theory of modular tensor categories. In Chapter [ref] we will use the tools we have established to detail several algorithms and procedures for topological quantum computaiton. Two introductory chapters are also included: Chapter [ref] which establishes the basic theory of finite dimensional quantum systems and Chapter [ref] which establishes general category theory. In Chapter [ref] we describe further structures in topologically ordered system which lie beyond plain modular tensor categories. This includes domain walls (which are modeled as bimodule categories over modular tensor categories), symmetry enriched topological orders (which are modeled as G-crossed modular tensor categories), and Fermionic topological orders (which are modeled as super-modular tensor categories).

[add schema]

\subsubsection{History of the subject}

Like with any sufficiently rich subject, the history of topological quantum information can be traced back as far as one wants. So let me do exactly that. The first use of topology in information science was roughly 2600 BCE, with South American \textit{Quipu} \cite{ascher1981code}. Quipu are intricate knotted strings typically made out of cotton fibers. The knots in the string are used to store various types of information, typically numbers. The Quipu store their information in knot invariants, and hence hold \textit{topological} information.

Quipu were so successful that they remained the primary method of information processing in much of South America for thousands of years. They reached their peak of usage in the 15th century with the Inca empire. The Inca empire was the largest pre-Columbian empire ever in the western hemisphere, with over ten million subjects and spanning over two million square kilometers. Despite their intricate and powerful government, the Incas had \textit{no written language}. This distinguished them from its contemporary empires, such as the Mali, Mongolian, or Chinese empires, which all relied on the written word. The success of the Inca empire can be seen as a testament to the versatility and power of knot invariants. The difference between the Inca and modern proposals for topological quantum computers is that instead of the strings being made out of cotton fibers they are made out of the spacetime trajectories of quasiparticles in topological systems.

Similarly, the history of topological methods in quantum mechanics can be traced back to the origins of quantum mechanics. There is a 1931 paper of Paul Dirac \cite{dirac1931quantised} which introduces many of the ideas which would become foundational to topological quantum mechanics. In the 1950s, explicitly topological ideas such as the Aharanov-Bohm effect \cite{aharonov1959significance} and the theory of point defects by Tony Skyrme \cite{skyrme1962unified} were beginning to emerge. In the 1970s nontrivial abstract topological considerations were leading to novel contributions to contemporary physics, such as the theoretical description of the A-phase of superfluid Helium-3 \cite{anderson1977phase} and the theory of phase transitions in the xy model proposed by Kosterlitz-Thouless \cite{kosterlitz1973ordering}. These results which were associated with the 1996 and 2016 Nobel prize respectively.

It was in the 1980s, however, that topology established itself as one of the leading themes in condensed matter physics. The discovery of the quantum Hall effect in 1980 \cite{klitzing1980new} and the subsequent discovery of the fractional quantum Hall effect in 1982 \cite{tsui1982two} gave the first examples of topologically ordered systems in our modern sense of the word, and resulted in the 1985 and 1998 Nobel prizes respectively. These systems, along with their nontrivial algebraic descriptions, gave theorists the license to dream big about what possibilities could be in front of them. This led to major work by theorists such as Frank Wilczek \cite{wilczek1982quantum, arovas1985statistical}, Duncan Haldane \cite{haldane1983nonlinear, haldane1988model}, and others.

The most notable of these theorists for our present manuscript is Edward Witten, with his introduction of \textit{topological quantum field theory} in 1988 \cite{witten1988topological}. This work not only put the modern developments in experiment within a larger context, but it also connected these developments to a parallel story which had been developing within pure mathematics. Namely, knot theory. In 1984 Vaughn Jones discovered his landmark knot invariant, which was powerful in its ability to distinguish between non-equivalent knots \cite{jones1997polynomial}. This marked the first major progress in the field since Alexander's invariant in 1928 \cite{alexander1928topological}. However, Jones’ construction was steeped in opaque subfactor theory, so much so that the fact that it resulted in knot invariant felt almost like a happy accident. Hence, a widespread topic on the mind of contemporary mathematicians was how to properly interpret the Jones invariant, and how to construct other invariants like it. Witten seemed to answer both. After defining topological quantum field theory, he showed how the Jones invariant could be obtained as observable quantities in a certain theory \cite{witten1989quantum}! This shocking result gave a new interpretation of the Jones invariant in terms of mathematical physics which was much more appealing to experts. Seeing as the Jones invariant was constructed from a topological quantum field theory, it was natural to expect that other theories might give new invariants which could distinguish between even more knots. This vision of invariants in low-dimensional topology constructed using topological quantum field theory became known as \textit{quantum topology}, and evolved into its own discipline in the following years.

This brings us to 1997. Quantum topology is a well developed area in pure mathematics, and topological themes in condensed matter physics are at the forefront of the field. The open problem is how to construct a fault tolerant quantum computer. Peter Shor had recently discovered his factoring algorithm \cite{shor1994algorithms}, and there was debate about whether scalable quantum error correction was possible \cite{landauer1995quantum}. This led to two independent proposals for topological quantum computation in the same year. One was by the mathematician Michael Freedman \cite{freedman1998p}. His vision was very simple. A recent paper had shown that computing the Jones invariant of knots was in general an NP-hard problem \cite{jaeger1990computational}. However, by the work of Witten, the Jones invariants of knots were observables in certain topological quantum field theories. Hence, if one could construct physically a topologically ordered system which was described by Witten’s topological quantum field theory then the Jones polynomial of knots could be computed efficiently by making measurements on the system. Hence, one would obtain a very powerful computer. This was Freedman’s proposal.

The other proposal was made by theoretical physicist Alexei Kitaev \cite{kitaev2003fault}. His proposal was much more precise. He gave a toy model for a certain family of topologically ordered systems. He then outlined a technique for storing and manipulating information within these systems. The powerful observation was that every quantum algorithm could be performed using the technique he had outlined \cite{mochon2003anyons}.

In the subsequent years Freedman and Kitaev teamed up with collaborators Zhenghan Wang, Michael Larsen, and others to study the new field of topological quantum information and the possibility of constructing a topological quantum computer. One of the first major results was that no topological quantum computer could be more powerful than a standard quantum computer \cite{freedman2002simulation}. This went against Freedman’s original hope to solve NP-hard problems using topological quantum computers. Freedman’s mistake was in asserting that topological quantum computers could compute the Jones polynomial. The nature of the measurements which give the Jones invariant in topological quantum field theory imply they will always be approximate. Approximating the Jones invariant in this way is strictly computationally easier than evaluating the Jones invariant exactly. In fact, this way of approximating the Jones invariant is not NP-complete - it can only be used to solve problems which could efficiently be solved using standard quantum computers.

The second major result of Freedman, Kitaev, Wang, and Larsen was the converse of their first result \cite{freedman2002modular}. Namely, they showed that every quantum algorithm can be efficiently run on a topological quantum computer. They do this by showing that every quantum algorithm can be efficiently reinterpreted in terms of computing the Jones invariant of some knot. In this way, computing the Jones invariant is a \textit{universal problem} for quantum computation. They then formalize Freedman’s ideas about topological quantum field theory, and show directly that realistic operations on a topologically ordered quantum system described by Witten’s quantum field theory can be used to compute the Jones invariants of knots. Quantum systems which can be used to run any quantum algorithm are known as \textit{universal} quantum systems.

Together, these two results show that in a real sense topological quantum computing is equally powerful as standard quantum computing with quantum circuits. This laid the groundwork for fruitful studies of fault-tolerant universal topological quantum computing, both using error correcting codes and physical materials. This has resulted in a great number of important results, which we will discuss at length throughout the rest of this manuscript.

\subsection{Technical introduction}

\subsubsection{Principles of topological quantum information}

In this section I will lay out the general principles of topological quantum information. As an organization tool, I will introduce these principles one by one as I construct a sample topological system. This example is meant to be representative of the systems we will encounter throughout this text, and within the broader field of topological quantum information. As a further organization tool, I will construct this example with the stated goal of obtaining a topological quantum computer.

Our system will be flat, containing only \textit{two spatial dimensions}. Our system will be mostly homogenous, essentially identical everywhere, at the exception of finitely many localized regions. These regions will differ substantially from the top-dimensional homogeneous bulk. These localized regions are called \textit{quasiparticles}. The beauty of systems like these is that they behave as though homogeneous bulk were empty, and the quasiparticles were fundamental particles within the bulk. In fact, in its algebraic description, these topological systems are \textit{identical} to ones in which the homogenous bulk is empty and the quasiparticles are fundamental particles. This is where the term quasiparticle arises. It is important however to remember that in most relevant applications the bulk is \textit{not} empty and the quasiparticles are \textit{not} fundamental particles. The bulk will typically be some highly entangled wavefunction, and the quasiparticles will be emergent phenomena made up of smaller microscopic degrees of freedom.

Our aim is to build a computer. In general this requires three components:

\begin{enumerate}
\item A method of stored information;
\item A method of manipulating information;
\item A method of reading out information.
\end{enumerate}

Information is stored in the state of the system - the bulk is described by some wavefunction, and the details of that wavefunction encodes information. Our method for manipulating information is \textit{braiding}. Braiding is the process whereby quasiparticles are moved along continuous paths around one another. There are two important points about braiding to keep in mind. The first is that braiding changes the state of the system. Even though the quasiparticle content of the system may be identical before and after the braid, the overall wavefunction of the system will change - there is more to the state of the system than just the positions of the quasiparticles. The second point is that the way that the state of the system changes \textit{only depends on the topology of the braid}, and not the geometry. Small deformations in the path taken by the quasiparticles does not affect the result - only global changes, like whether a path is taken clockwise or counterclockwise, makes a difference. This invariance is due to the fact that our system is topological. In generic geometric systems we expect the exact path taken by quasiparticles matters a great deal. The independence of the details of the paths is extremely specific to topological systems, and in the present setting is the \textit{defining topological feature} of our system.

It is at this point that we can already see we have succeeded in our goal of making our computation fault-tolerant. Noise in the system will correspond to uncontrolled perturbations in the trajectories of the quasiparticles. This uncontrolled movement won’t change global properties of paths taken, and hence will not change the action of the braids on the system. That is, small errors will not affect computation! Of course, large enough errors could unintentionally make one quasiparticle wind around another. This would change the topology of the braid and hence ruin the computation. These errors are controllable, however, by moving the quasiparticles far apart and limiting the magnitude of the noise.

The final step in making our computer is to introduce a method for reading out information. This is done using \textit{fusion}. Fusion is the process whereby two quasiparticles are brought together, resulting in a single quasiparticle. In sufficiently complicated topological systems the result of fusion depends on the state of the overall system. That is, the result of fusion can be used as a way of reading out information about the state. In its most basic form, when two quasiparticles fuse they can either result in a localized region which is identical to the homogenous bulk or is different from the homogenous bulk. If they result in a localized region identical to the bulk we say that the two quasiparticles have \textit{annihilated} each other. In a real sense this can be seen as the difference between constructive and destructive interference. Two waves can either have destructive interference and annihilate each other, or they can constructively interfere and result a new wave. Whether or not two quasiparticles annihilate can depend on the overall state of the system, and hence gives a method for reading out information.

In some situations, the result of fusion can even be nondeterministic. In this case the fusion can be repeated multiple times, which allows one to measure the \textit{probability} that two quasiparticles will annihilate each other. These probabilities are a rich source of data, and will serve as our way of reading out information in the current setting. The fact that our system is topological implies that the result of fusion does not depend on the specifics of the path taken between the two quasiparticles, and hence this method of readout preserves the invariance of our computation to noise. This gives us a full picture of topological quantum computation:

[WORK: initialize state => braid quasiparticles => fuse]

To make the above discussion more concrete, we will give a more specific worked example. In this example we use a specific topological order known as the \textit{Fibonacci particle theory}, to run Shor’s efficient quantum factorization algorithm \cite{shor1994algorithms}. The input of Shor’s algorithm is a positive integer. The output of Shor’s algorithm is the factorization of that integer. Shor’s algorithm is \textit{efficient} in the sense that it uses polynomially many quantum logic gates to arrive at its answer relative to the size of the input. Throughout this passage we will use \textit{efficient} and \textit{polynomially sized} interchangeably. The Fibonacci particle theory is a specific topological order, which describes in an algebraic fashion how the overall state changes when quasiparticles are braided and fused.

The first step in running Shor’s algorithm on a Fibonacci quantum computer is to translate the positive integer input into a certain braid. This is done using an efficient classical algorithm. The second step is to run this braid on a Fibonacci quantum computer. This is done by initializing some prescribed state with a collection of quasiparticles, and then braiding those quasiparticles by the braid which was obtained in the previous step. This initialization and braiding is performed repeatedly, and after every time two of the quasiparticles are fused. This gives us a real number, which is the probability that the two quasiparticles annihilate after the braiding. An efficient classical algorithm is then used to take this real number and obtain from it the factorization of the original input. Since all of these steps are efficient, it gives a topological quantum algorithm for factoring integers. The schematic for this process is shown below:

[WORK: pos int => braid => real number => factorization]

The magic in the above procedure is the existence of these two classical algorithms: a first one for encoding integers into braids and a second one for decoding real numbers into factorizations. These algorithms are nontrivial. They are due to Freedman-Larsen-Kitaev-Wang \cite{freedman2002modular}. In fact, Freedman-Larsen-Kitaev-Wang showed that any problem which can be efficiently solved using a quantum circuit can also be solved using the Fibonacci particle theory, via a similar method of efficient classical preprocessing and postprocessing. It is in this sense that the Fibonacci theory is \textit{universal} for quantum computation.

The final step of this process would be to create a physical topological system which is described by the Fibonacci theory, which would serve as our quantum computer. In the realm of materials, the most promising approach seems to be to use specially tuned versions of the fractional quantum Hall system \cite{zhu2015fractional}. While these materials are theorized to host quasiparticles described by the Fibonacci theory, the difficulty of the experiment makes it inaccessible to current technology. There has been progress made on topological quantum error correcting codes which work by simulating the Fibonacci theory \cite{schotte2022quantum, schotte2022fault}. However these codes at the current moment have structural issues and require an unbearable amount of overhead to run, making them unfeasible to use on modern computers.

Progress on topological quantum computing has thus been focused on realizing topological particle theories other than the Fibonacci theory. These other theories can be constructed in more workable materials, and can be simulated as topological quantum error correcting codes with less overhead. The drawback of these other theories is that they are typically less computationally powerful, meaning that they require more tricks and techniques to achieve universal quantum computing. There are a great number of different proposals for how to achieve universal topological quantum computing, based on different particle theories, different methods of encoding information, different methods of manipulating information, and different methods of reading out information. It is an exciting time to be a theorist in the field of topological quantum information.

[WORK: Add boxes with four principless? 2D, quasiparticles, braiding, fusion.]

\subsubsection{Defects in ordered media}

We will now work through a complete mathematical example of a family of topological systems. Seeing as no assumption of familiarity of quantum mechanics is made of the reader, we will study a \textit{classical} topological system. Many of the important subtleties of topological quantum information are already present in the classical case. Topological classical information is a smaller subject than topological quantum information - the reader should have a relatively complete grasp of the subject by the end of the chapter. Much of the discussion in this chapter is taken from an excellent review article by Mermin [ref]. The goal of this chapter is both to give a family of relevant examples to the reader, as well as to give a crash-course on the relevant topology.

The family of systems we will describe goes by many names. In communities of experimentally focused physicists it goes by the name \textit{ordered media}. In mathematical physics communities it goes by the name \textit{gauge theory}. In pure mathematics it would be described as \textit{homotopy theory}. We will construct a system based on every topological space $M$. We will call $M$ the order space of our theory.

To describe a system in physics, the first step is to define the space of possible states of the system. In this case, states will correspond to \textit{continuous maps $\phi: \RR^2\to M$}. We now give physical intuition for this choice of state space. The choice of $\RR^2$ as a source represents the underlying physical space. We are working on an infinite flat plane. A function $\phi: \RR^2\to M$ amounts to choosing a value $\phi(x)$ for every point $x\in \RR^2$. In this way we imagine our system as being made up of infinitely many objects, one placed at each point in $\RR^2$, each of which has an internal state space $M$. The fact that $\phi$ must be continuous is a compatibility condition between the states of the objects at nearby points. It says that nearby objects must have similar states. We now list some examples:

\begin{itemize}
\item Classical XY model of a 2D electron gas. In this example, $M$ is the circle. An electron can be modeled as point particles with a magnetic dipole pointing in some direction. This magnetic dipole is known as the \textit{spin} of the election, and can point in any direction in the plane. The topological space of possible direction in the plane is a circle. The fact that nearby electrons must have similar spins is known as Hund’s rule, and is the most fundamental incarnation of ferromagnetism. It is physically derived as a consequence of the Pauli exclusion principle.

\item A-phase of superfluid Helium-3.  [WORK: fill in example]

\item Biaxial nematics. [refs] In this example, $M$ is a certain quotient of the special orthogonal group $\SO(3)$ of rotations in three dimensional space. The objects at every point in the biaxial nematic should be thought of as small rectangles with unequal side lengths. These rectangles can be oriented in any direction in three dimensional space. In practice these objects will often be molecular compounds. [WORK: what’s with the continuity condition?] They will not be exactly rectangular, but have the same symmetry group as a rectangle which is enough for the model to be accurate. To compute the space of possible orientations of a small rectangle, we work by the method of symmetries. Choosing some reference orientation to start with, every rotation in three dimensional space brings the rectangles to new orientations. The space of orientations of the rectangle is hence equal to $\SO(3)$ modulo the rotations which fix the rectangle. That is, $M$ is equal to $\SO(3)$ modulo the symmetry group of a rectangle.
\end{itemize}

We will now analyze these systems. In doing this analysis we will want to use the ideas of \textit{deformation} and \textit{topological equivalence}. Of course, these ideas are vague and require rigorous notations to make precise. We make such a digression now.

\begin{definition}
Let $X,Y$ be topological spaces, and let $f,g: X\to Y$ be continuous maps. A \textbf{homotopy} between $f$ and $g$ is a map $H: X\times [0,1] \to Y$ such that $H(x, 0)=f(x)$ and $H(x,1)=g(x)$ for all $x\in X$.
\end{definition}

The idea is that the maps $H(\cdot,t):X\to $Y for $t\in [0,1]$ are a continuously deforming family. The first map $H(\cdot, 0)$ is equal to $f$, and as $t$ moves forward the maps $H(\cdot,t)$ slowly change until at $t=1$ they become $H(\cdot,1)=g$. If there exists a homotopy between two maps, then we say those maps are \textit{homotopic}. It is straightforward to verify that homotopy is an equivalence relation. This is our relevant notation of topological equivalence.

We can use homotopies to get a picture of dynamics in our system. We consider a continuous map $H: \RR^2 \times \RR_{\geq 0}\to M$. The map $\phi_t=H(\cdot,t): \RR^2\to M$ represents the state of the system at time $t$. In this sense, the system is \textit{evolving by homotopies}. The following proposition is an important part of understanding the nature of the system:

\begin{proposition}
Let $M$ be a path connected topological space. Every pair of continuous maps $\phi_0,\phi_1:\RR^2\to M$ is homotopic.
\end{proposition}
\begin{proof}
Consider the map $H_0: \RR^2\times [0,1]\to M$ given by $H_0(x,t)=\phi_0(tx)$. Since multiplication in $\RR^2$ is continuous, $H_0$ is continuous. Hence, $H_0$ gives a homotopy between the constant map $H_0(\cdot,0)=\phi_0(0)$ and $H_0(\cdot,1)=\phi_0(x)$. Similarly, the map $H_1:\RR^2\times [0,1]\to M$  given by $H_1(x,t)=\phi_1(tx)$ gives a homotopy between the constant map $H_1(\cdot,0)=\phi_1(0)$ and $H_1(\cdot,1)=\phi_1(x)$.

Now, since $M$ is path connected, there exists a continuous map $\ell: [0,1]\to M$ such that $\ell(0)=\phi_0(0)$ and $\ell(1)=\phi_1(0)$. Thus, the map $H_\ell:\RR^2\times [0,1]\to M$ given by $H_\ell(x,t)=\ell(t)$ establishes a homotopy between the constant maps $H_\ell(\cdot,0)=\phi_0(0)$ and $H_\ell(\cdot,1)=\phi_1(0)$. Composing the homotopies from $\phi_0(x)$ to $\phi_0(0)$, $\phi_0(0)$ to $\phi_1(0)$, and $\phi_1(0)$ to $\phi_1(x)$, we arrive at a homotopy between $\phi_0(x)$ and $\phi_1(x)$ as desired.
\end{proof}

The above proposition can be interpreted as saying that as our system evolves in time, we can expect it to go from any state to any other state. In other words, all of the possible states of our system are topologically equivalent and hence it contains \textit{no topologically invariant information}. This means that we will need to add more complexity to the system before we can study topological information.

This complexity will come from introducing quasiparticles. These quasiparticles go by many names. In the theory of ordered media they are known as defects. In gauge theory they are known as particles. In homotopy theory they are known as point singularities. For the sake of brevity, we will use the term defect. A defect is a point at which we will drop our condition that the state $\phi:\RR^2\to M$ be continuous. This is done by making $\phi$ \textit{undefined} at certain points. Our new system is called \textit{ordered media with finitely many defects}. The state space consists of pairs $(S,\phi)$, where $S\subset \RR^2$ is a finite set and $\phi: \RR^2\\ S\to M$ is a continuous map.

Dynamics in our new system can involve the movement of defects. Let $\gamma_1… \gamma_n: \RR_{\geq 0} \to \RR^2$ be continuous paths, with $ \gamma_i(t)\neq \gamma_j(t)$ for all $i\neq j$, $t\in [0,1]$. These paths represent the trajectories of the defects, subject to the condition that no two defects collide. Continuous maps

$$H:\RR^2\times \RR_{\geq 0} \backslash \bigcup_{t\geq 0}\bigcup_{i=1}^{n}(\gamma_i(t),t)\to M$$

represent dynamics in our system. At every time $t\geq 0$, the system is in the $n$-defect state

$$\phi_t=H(\cdot,t): \RR^2\backslash \bigcup_{i=1}^{n}\gamma_i(t)\to M.$$

We will call such maps \textit{defect-mobile homotopies}. If the trajectories $\gamma_i$ are constant with respect to $t$ for all $i$, we will call $H$ a \textit{defect-fixed homotopy}. This now leads us to two central questions:

\begin{enumerate}
\item What information is invariant under defect-mobie homotopy?
\item What information is invariant under defect-fixed homotopy?
\end{enumerate}


[WORK: define path, path homotopy, loop]

We first examine the information which is invariant under defect-mobile homotopy. Let $(\{*\},\phi)$ be a state with one defect. Let $\alpha:[0,1]\to \RR^2$ be a continuous map with $\alpha(0)=\alpha(1)$, which winds around $\{*\}$ counterclockwise exactly once. Postcomposing, we get a map $(\phi\circ \alpha): [0,1]\to M$ with $(\phi\circ\alpha)(0)=(\phi\circ\alpha)(1)$. Now, it is a standard fact from homotopy theory that all counterclockwise loops around $\{*\}$ are homotopic. Hence, difference choices of $\alpha$ will induce homotopic maps $(\phi\circ\alpha):[0,1]\to M$. Moreover, if $\phi$ varies under defect-mobile homotopy, the homotopy class of map $[0,1]\to M$ will remain invariant. Hence, we have arrived at an invariant under defect-mobie homotopy: the homotopy class of map $U(1) \to M$ given by choosing a loop around the defect and postcomposing with $\phi$.

When $(S,\phi)$ is a state with multiple defects, this same approach still works. Now, let $\alpha_s:[0,1]\to\RR^2$ be continuous maps for every $s\in S$, such that $\alpha(0)=\alpha(1)$, $\alpha_s$ winds around s counterclockwise exactly once, and $\alpha_s$ does not wind around any other point $s’\neq s$, $s’\in S$. The homotopy classes of the maps $(\phi\circ \alpha_s)$ are invariants of the state under defect mobile homotopy.


\subsubsection{The fundamental group}



\subsubsection{Topological classical computation}









\bibliographystyle{alpha}
\bibliography{ref}


\end{document}






