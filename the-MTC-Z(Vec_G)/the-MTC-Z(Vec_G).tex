\documentclass{article}

\usepackage[utf8]{inputenc}
\usepackage{enumerate}
\usepackage{amsfonts}
\usepackage{amsmath}
\usepackage{amsthm}
\usepackage{blindtext}
\usepackage{graphicx}
\usepackage[numbers]{natbib}
\usepackage{amssymb}
\usepackage{mathtools}
\usepackage{stmaryrd}
\usepackage{tikz-cd}
\usepackage{relsize}
\usepackage{mathrsfs}
\usepackage{tikzit}
\usepackage{ upgreek }
\usepackage[normalem]{ulem}
\usepackage{quiver} 




\newtheorem{theorem}{Theorem}
\newtheorem{lemma}{Lemma}
\newtheorem{corollary}{Corollary}
\newtheorem{conjecture}{Conjecture}
\newtheorem{proposition}{Proposition}
\theoremstyle{definition}
\newtheorem*{definition}{Definition}
\newtheorem{remark}{Remark}
\newtheorem{experiment}{Experiment}
\newtheorem{proposition-definition}{Proposition-Definition}


\graphicspath{ {./images/} }
\numberwithin{figure}{section}
\setcounter{section}{-1}


\title{The Modular Tensor Category $\mathcal{Z}(\text{Vec}_G)$}
\author{Milo Moses}


\begin{document}


\maketitle

\newcommand{\RR}{\mathbb{R}}
\newcommand{\HH}{\mathbb{H}}
\newcommand{\NN}{\mathbb{N}}
\newcommand{\QQ}{\mathbb{Q}}
\newcommand{\CC}{\mathbb{C}}
\newcommand{\FF}{\mathbb{F}}
\newcommand{\ZZ}{\mathbb{Z}}
\newcommand{\Zcal}{\mathcal{Z}}
\newcommand{\Ncal}{\mathcal{N}}
\newcommand{\LL}{\mathscr{L}}
\newcommand{\TT}{\mathcal{T}}
\newcommand{\Ccat}{\mathscr{C}}
\newcommand{\Dcat}{\mathscr{D}}
\newcommand{\Ecat}{\mathscr{E}}
\newcommand{\st}{\,\,\mathrm{s.t.}\,\,}
\newcommand{\mm}{\mathfrak{m}}
\newcommand{\pp}{\mathfrak{p}}
\newcommand{\Hom}{\mathrm{Hom}}
\newcommand{\Aut}{\mathrm{Aut}}
\newcommand{\Frac}{\mathrm{Frac}}
\newcommand{\tr}{\mathrm{tr}}
\newcommand{\op}{\mathrm{op}}
\newcommand{\res}{\mathrm{res}}
\newcommand{\im}{\mathrm{im}}
\newcommand{\ev}{\mathrm{ev}}
\newcommand{\coev}{\mathrm{coev}}
\newcommand{\id}{\mathrm{id}}
\newcommand{\coker}{\mathrm{coker}}
\newcommand{\SL}{\mathrm{SL}}
\newcommand{\End}{\mathrm{End}}
\newcommand{\Rep}{\bold{Rep}}
\newcommand{\Set}{\bold{Set}}
\newcommand{\Vecc}{\bold{Vec}}
\newcommand{\Top}{\bold{Top}}
\newcommand{\Grp}{\bold{Grp}}
\newcommand{\Hilb}{\bold{Hilb}}
\newcommand{\Bord}{\bold{Bord}}
\newcommand{\Cat}{\bold{Cat}}
\newcommand{\0}{\left|0\right>}
\newcommand{\1}{\left|1\right>}
\newcommand{\nullclass}{\left|\bold{0}\right>}
\newcommand{\alphaclass}{\left|\alpha\right>}
\newcommand{\betaclass}{\left|\beta\right>}
\newcommand{\alphabetaclass}{\left|\alpha\beta\right>}
\newcommand{\ppsi}{\left|\psi\right>}
\newcommand{\pphi}{\left|\phi\right>}
\newcommand{\func}{\mathrm{func}}
\newcommand{\bigleadsto}{\mathlarger{\mathlarger{\mathlarger{\leadsto}}}}
\newcommand{\vin}{\rotatebox[origin=c]{-90}{$\in$}}


.[WORK: Give introduction]

[WORK: Define $\Vecc_G$]

[WORK: Define $\Zcal(\Vecc_G)$]

[WORK: Show that $\Zcal(\Vecc_G)$ is the category of $G$-graded representations of $G$. State that there is a ``duality" between the $G$ we are taking reps of and the $G$ that is doing the grading - a few words about Hopf algebra interpretation?]

[WORK: Show that simple objects correspond to chosing a conjugacy class and an irrep. State this is part of a more general duality between conjugacy classes and irreps (e.g. Fourier transform when $G$ is abelian). In particular, both sets have the same size.]

[WORK: Compute fusion coefficents]

[WORK: Compute quantum dimensions]

[WORK: Compute braiding coefficients]

[WORK: Find twisting coeffients]

[WORK: State Verlinde formula, in the form given by Burnside. Give an elementary proof, and then give a proof using the more general Verlinde formula.]


\bibliographystyle{alpha}
\bibliography{ref}


\end{document}






