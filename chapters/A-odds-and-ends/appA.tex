\section{Odds and ends}

[WORK: I'm not sure where to do this, but I'd like to make a little comment about non-semisimple modular categories. There are inherent limits to the applications of MTCs to quantum topology, as shown in these papers \cite{reutter2023semisimple, davis2011axiomatic}. This motives going beyond semisimplicity.

A nice paper about this from Zhenghan's perspective is \cite{chang2024modular}.

The canonical reference is \cite{creutzig2021qft}. The summary is that ``the only physical thing is the derived category". Makes me look at derived categories differently. Another important paper in this area is \cite{shimizu2019non}.

 ]


\subsection{Topological quantum field theories}

\subsubsection{Overview}

Topological quantum field theory (TQFT) is an important player in the field of topological quantum information. For the purposes of this manuscript, TQFT is treated meerly as a perspective on topological order. TQFT can be connected more directly and deeply to physics using the machinery of effective field theory, which we will not discuss. The TQFT perspective is summarized as follows:

$\newline$

\fbox{\parbox{\dimexpr\linewidth-2\fboxsep-2\fboxrule\relax}{

\textbf{The TQFT perspective:} topological order should be studied in terms of the way topological systems react to being put on different manifolds, and by the way they react to topological manipulations on those manifolds.
}}

$\newline$

We now elaborate on what this means. Start with some topological order, perhaps the toric code or some other Kitaev quantum double model. When we refer to \textit{putting the topological order on a manifold}, we mean the following. First, we draw some lattice the manifold, and add Hilbert spaces on the edges of the lattice. Then, we consider the Hamiltonian on the lattice associated with the relevant topological order. This Hamiltonian has a ground space (the zero-energy subspace), which we refer to as the state-space of the topological order on the manifold.

Thus, associated to every two dimensional topological order, we have an assignment from closed surfaces to Hilbert spaces, which sends a surface to its state-space on that surface. We can call this assignment $V$, depicted below as follows:

[WORK: add picture. Surface with holes, under $V$, gets assigned a Hilbert space.]

These Hilbert spaces are the basic objects of study in the TQFT perspective. For instance, their dimensions give information about the anyon types and fusion rules of the topological order:

[WORK: add table. 
V(sphere)=1;

V(torus)= number of anyons types...;

V(two-holdes torus) = $\sum_{a,b,c\in \cL}(N^{a,b}_c)^2$.
]

[WORK: I think that I should add a heuristic derivation of these state-space dimensions. I want to use the example of the two-holed torus, and not having a basis is causing me trouble.]

Of course, without any additional structure there is no information in a vector space beyond its dimension. For this reason, to make the TQFT perspective useful one must consider not only the dimensions of these vector spaces but also the way they react to topological manipulations of manifolds. In the remainder of this appendix, we will dicuss exactly what these extra manipulations are, use them to define an object called a TQFT, and explore the ramifications of this perspective.

\subsubsection{Dehn twists in the toric code}

In the TQFT perspective, we are interested in studying the way that state-spaces of topological order on different manifolds behave under topological manipulations. We illustrate the sort of topological manipulations we are interested in though a paradigmatic example, a \textit{Dehn twist} on the torus, illustrated below:

[WORK: dehn twist on torus, decomposed into cut, twist, and glue.]

The crucial point is that this topological manipulation on the torus induces a linear transformation on the state-space of any topological order on the torus. The TQFT perspective says that studying the linear transformation on the state-space of the torus induced by the Dehn twist is a good thing to do.

We now describe this induced linear transformation, using the explicit example of the toric code. Consider a square lattice on the torus, with qubits placed at edges, equipped with the toric code Hamiltonian. The Dehn twist on the torus is a continuous map from the torus to itself. Acting on the level of lattices, the Dehn twist sends the old lattice to a new lattice as follows:

[WORK: use four-by-four square lattice, write out Dehn twist action explicitely.]

We make a few observations. Firstly, we observe that the permutation on the level of lattice sites induces a linear map on the level of Hilbert spaces. Just like how there is a linear ``swap" map $\bC^2\otimes \bC^2\to \bC^2\otimes \bC^2$ which sends $\ket{b_0}\otimes \ket{b_1}$ to $\ket{b_1}\otimes \ket{b_0}$, there are linear maps on any tensor-product Hilbert space induced by permuting the components.

Secondly, we observe that the Hamiltonian is \textit{not} invariant under this linear map. This is an immediate corrolary of the fact that the lattice edges are not invariant under the map, and thus plaquette terms which used to act on square faces now act on slanted parallelograms. However, the new Hamiltonian is still manifestly a toric code Hamiltonian, just associated to a different lattice. Thus, the ground space of the old Hamiltonian and the new Hamiltonian are both canonically isomorphic to $\bC[H^1(T^2;\bZ_2)]$. Thus, identifying the ground space of both Hamiltonians with $\bC[H^1(T^2;\bZ_2)]$, we find that the Dehn twist induces a linear map

$$\bC[H^1(T^2;\bZ_2)]\to \bC[H^1(T^2;\bZ_2)].$$

This linear map can be described entirely explicitely. [WORK: describe the map.]

\subsubsection{Defining TQFT}

We saw in the last section that some topological transformations on surfaces induce linear maps on the state-space associated to those surfaces under a topological order. In general, we can associated linear maps on state-space to any self-homeomorphism $f:M\to M$ of a surface $M$.

The induced linear map on state-space is the the same as before. To define state-spaces, we choose a lattice on $M$. The function $f$ induces a map from this lattice on $M$ to a new lattice. This new lattice has a new Hamiltonian associated with it, in the same topological order. Since the ground states of a topologically ordered Hamiltonian do not depend on the details of the lattice and only on the topology of the manifold, the state-space of the original lattice can be canonically identified with the state-space of the new lattice. Permuting the tensor factors in the Hilbert space via $f$, we thus get a linear map from the vector space $V(M)$ to itself.

We observe that any two maps $f:M\to M$ which can be continuously deformed from one to another will indcue the same linear map on $V(M)$. This is for the following reason. Suppose that $f_0,f_1:M\to M$ can be deformed from one to the other. As $f_0$ deforms to $f_1$, the image of the lattice will deform continuously as well. This means that the image of the lattice under $f_0$ and $f_1$ will be isomorphic as lattices, and thus the maps induces by $f_0$ and $f_1$ will be the same. [WORK: add some detail to this argument, it feels too loose.]

Thus, every element of the \textit{mapping class group} of $M$ induces a map $V(M)\to V(M)$, where the mapping class group is defined as the space of self-homeomorphisms $f:M\to M$ modulo continuous deformations:

$$\MCG(M)=\text{(self-homeomorphisms $f:M\to M$ )}/\text{(continuous deformations)}.$$

The summary of the above dicussion is that assocaited to every topological order we have representations

$$\rho_{M}:\MCG(M)\to \Aut(V(M))$$

for every surface $M$. These representations admit a much more fine-grained study of topological order than just the dimensions of $V(M)$.

Of course, not every collection of mapping class group representations will be induced by some topological order. These is a compatibility condition between the mapping class group representations of different surfaces. This compatibility condition comes from the following observation. Suppose we are given two surfaces $\Sigma_{g}$, $\Sigma_{g'}$ of genus $g$ and $g'$ respectively. There is a topological transformation one can do to go from the disjoint union $\Sigma_{g}\sqcup\Sigma_{g'}$ to the surface $\Sigma_{g+g'}$ of genus $g+g'$. This goes as follows. First, we take our surfaces $\Sigma_{g}$, $\Sigma_{g'}$. Then, we cut small holes into each of them. Then we connect these holes by gluing in a cylinder. This process is shown below:

[WORK: add process showing fusion of $\Sigma_{g}$, $\Sigma_{g'}$ to $\Sigma_{g+g'}$.]

Like with self-homeomorphisms, this process induces a linear map on vector spaces. [WORK: The details of this linear map are more subtle than before. My way of doing it is to use local purfiability to trace out a little bit extra, and then glue in a cylinder of the reference state in the vaccuum sector. Not sure what the most elementary way of saying/doing this is...]

In quantum mechanics, the Hilbert space associated with two combined systems is the tensor product of their Hilbert spaces. Thus, the state-space Hilbert space associated with $\Sigma_{g}\sqcup\Sigma_{g'}$ is $V(\Sigma_{g})\otimes V(\Sigma_{g'})$. Thus, the cutting-and-gluing process gives a linear map we call $Z_{g,g'}$:

$$Z_{g,g'}:V(\Sigma_{g})\otimes V(\Sigma_{g'})\xrightarrow{} V(\Sigma_{g+g'}).$$

[WORK:

I want to say a little bit about $Z_{g,g'}^\dagger$. It is described geometrically by projecting onto the space of states with trivial charge around the annulus connecting $\Sigma_{g}$ and $\Sigma_{g'}$ in the connect sum, tracing out the cylinder, then filling in the holes. The key point to observe is that $Z_{g,g'}$ is an isometric embedding. That is,

$$Z_{g,g'}^\dagger \circ Z_{g,g'}=\id_{V_{g}\otimes V_{g'}}.$$
]


[WORK: This map should be

\begin{align*}
\bC[\cL]\otimes \bC[\cL]&\xrightarrow{}\sum_{a,b,c\in \cL}B(V^{a,b}_c)\\
\ket{a}\otimes \ket{b}&\mapsto \sum_{c\in \cL}\sqrt{\frac{d_c}{d_ad_b}}\ket{\id_{V^{a,b}_c}}
\end{align*}

but I don't have the setup for this to be a substantive statement yet.
]

Every element of $\MCG(\Sigma_g)\times \MCG(\Sigma_{g'})$ induces an element of $\MCG(\Sigma_{g+g'})$ as follows. Think of $\Sigma_{g+g'}$ as $\Sigma_{g}$ connecting with a cylinder to $\Sigma_{g'}$. Then, a pair $(f,f')\in \MCG(\Sigma_g)\times \MCG(\Sigma_{g'})$ acts on $\Sigma_{g+g'}$ by first removing the cyliner, then acting by $f$ on $\Sigma_{g}$ and by $f'$ on $\Sigma_{g'}$, and then by reattacing the cylinder at the new locations of the holes. Finally, so that the image of this map is the same as the original manifold, the cylinder is slid across the manifolds back to its original location. The final step of this process is ambiguous because the cylinder could be slid multiple ways, but all of these ways are equivalent up to deformations and thus we get a well-defined element of $\MCG(\Sigma_{g+g'})$.

We can now put all of the maps we have defined together into a commutative diagram which gives the compatibility between the different mapping class group representations. Our maps between $V_{g}\otimes V_{g'}$ and $V_{g+g'}$ come together to give a map

\begin{align*}
\Aut(V_{g}\otimes V_{g'})&\xrightarrow{}\Aut(V_{g+g'}),\\
h & \mapsto Z_{g,g'}\circ h\circ Z_{g,g'}^\dagger
\end{align*}

which is a group homomorphism because $Z^\dagger_{g,g'}\circ Z_{g,g'}=\id_{V_{g}\otimes V_{g'}}$. All these maps fit into the below diagram, which is immediately seen to be commutative after expanding the definitions:

% https://q.uiver.app/#q=WzAsNCxbMCwwLCJcXE1DRyhcXFNpZ21hX3tnK2cnfSkiXSxbMCwxLCJcXE1DRyhcXFNpZ21hX2cpXFx0aW1lcyBcXE1DRyhcXFNpZ21hX3tnJ30pIl0sWzEsMSwiXFxBdXQoVl9nXFxvdGltZXMgVl97Zyd9KSJdLFsxLDAsIlxcQXV0KFZfe2crZyd9KSJdLFsyLDMsIlpfe2csZyd9IiwyLHsic3R5bGUiOnsidGFpbCI6eyJuYW1lIjoiaG9vayIsInNpZGUiOiJ0b3AifX19XSxbMSwwLCIiLDAseyJzdHlsZSI6eyJ0YWlsIjp7Im5hbWUiOiJob29rIiwic2lkZSI6InRvcCJ9fX1dLFsxLDIsIlxccmhvX3tnfVxcb3RpbWVzIFxccmhvX3tnJ30iXSxbMCwzLCJcXHJob197ZytnJ30iXV0=
\[\begin{tikzcd}
	{\MCG(\Sigma_{g+g'})} & {\Aut(V_{g+g'})} \\
	{\MCG(\Sigma_g)\times \MCG(\Sigma_{g'})} & {\Aut(V_g\otimes V_{g'})}
	\arrow["{\rho_{g+g'}}", from=1-1, to=1-2]
	\arrow[hook, from=2-1, to=1-1]
	\arrow["{\rho_{g}\otimes \rho_{g'}}", from=2-1, to=2-2]
	\arrow["{Z_{g,g'}^\dagger \circ (\--)\circ Z_{g,g'}}"', hook, from=2-2, to=1-2]
\end{tikzcd}\]

Of course, the above analysis has no been rigorous. It can't be, since we do not have a rigorous definition of topological order! However, what we can do now is \textit{define} a TQFT in terms of this data we have constructed. Namely, we have the following:

\begin{defn}[TQFT] A \textit{topological quatum field theory} (TQFT) is the following data:

\begin{enumerate}
\item A collection of Hilbert spaces $V_{g}$ for every integer $g\geq 0$;
\item A unitary representation

$$\MCG(\Sigma_{g})\xrightarrow{}\Aut(V_{g})$$

for every $g\geq 0$;
\item Linear maps

$$Z_{g,g'}: V_g\otimes V_{g}\to V_{g+g'}$$

for all $g,g'\geq 0$
\end{enumerate}

Such that:

\begin{enumerate}

\item $V_{0}=\bC$;

\item For all $g,g'\geq 0$,

$$Z_{g,g'}^\dagger \circ Z_{g,g'}=\id_{V_{g}\otimes V_{g'}}.$$

\item For all $g,g'\geq 0$, the diagram

\[\begin{tikzcd}
	{\MCG(\Sigma_{g+g'})} & {\Aut(V_{g+g'})} \\
	{\MCG(\Sigma_g)\times \MCG(\Sigma_{g'})} & {\Aut(V_g\otimes V_{g'})}
	\arrow["{\rho_{g+g'}}", from=1-1, to=1-2]
	\arrow[hook, from=2-1, to=1-1]
	\arrow["{\rho_{g}\otimes \rho_{g'}}", from=2-1, to=2-2]
	\arrow["{Z_{g,g'}^\dagger \circ (\--)\circ Z_{g,g'}}"', hook, from=2-2, to=1-2]
\end{tikzcd}\]

commutes.

\item .[WORK: I bet I need more axioms. What are they?]
\end{enumerate}
\end{defn}

\subsection{Quasitriangular weak Hopf algebras}

[WORK:

Weak Hopf algebras were introduced in \ref{bohm1996coassociative}. A good early source about them is \cite{nikshych2004semisimple}.

Weak Hopf algebras are relevant to the algebraic theory of topological quantum information because the representation category of a weak Hopf algebra is a fusion category. Adding more structure to the weak Hopf algebra gets you all the way up to modular categories. This is Tannaka duality in action. The reference for tannaka duality for modular categories is \cite{pfeiffer2009tannaka}.

They are also intimately linked to the theory of module categories. This was first established in \cite{ostrik2003module}, and then was shown much more explicitely in \cite{kitaev2012models}.
]

\subsection{Quantum groups}

\subsection{Subfactors}

\subsection{Vertex operator algebras}

[WORK:

The connection between vertex operator algebras and topological order comes through conformal field theory. VOAs are at their heart tools for conformal field theory. Of course, since algebraically conformal field theory and topological field theories are so similar, this means that well beahved VOAs describe topological order.

This was first proved in the landmark paper of Huang \cite{huang2005vertex}. Of course, there are versions for $G$-crossed and fermionic theories - \cite{huang2021representation, carpi2023vertex}.

One very nice thing to be aware of is the work of Nikita Sopenko. He is able to prepare topologically ordered states using vertex operator algebras, thus realizing the implicit program in the topological order interpretation of Huang's work \cite{sopenko2023topological}.

A big thing in all of this is the Kazhdan-Lusztig correspondence, which I do not understand very well. A great reference seems to be \cite{tan2020vertex}.

]