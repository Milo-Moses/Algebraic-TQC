\section{Modular categories}
\label{modular-categories}

\Note{I am using nonstandard language. I am using the term ``modular category" to refer to what most authors call ``modular tensor categories". I do this because modular category is shorter than modular tensor category and I want to be succinct. For many people (notably Zhenghan) modular category means potentially non-semsimple modular category. For instance, Turaev's original definition did not include the simisimplicity axiom and he called them modular categories. The term tensor is used to highlight semisimplicity by many authors, so I can highlight semisimplicity with a footnote.}

\Note{There's an issue in this treatment. There is one piece of data beyond the scope of an modular categories - the chiral central charge. This is a remnant of the fact that the bulk-to-boundary correspondane is not exact because the boundary can have stacked $E_8$ phases, see \cite{bonderson2021measuring}. Of course this is not something to dive into now. However, I want to be maximally honest - point out that there is a unique piece of topologically invariant information beyond the scope of modular categories. Maybe include somewhere (as an exercise?) the treatment of chiral central charge mod 8? I think this would make for a good footnote. I'm realizing now that the fact that the chiral central charge is a root of unity is part of Vafa's theorem. So, it makes the most sense for this to be in the number theory section.}

\subsection{Overview}

\subsubsection{Introduction}

In this chapter we will be giving a detailed analysis of modular categories, the abstract algebraic structures used to describe anyons in topological order. We recall below how this fits into the general framework of this book:

\begin{equation*}
\tikzfig{mathematical-outline-MTC}
\end{equation*}

Describing exactly what an anyon is and how it can transform in terms of states and unitary operators on a Hilbert space can be difficult. However, describing abstractly how these transformations compose with one another can be done realtively simply. Hence we take a composition-first category-theoretic approach to anyons. We will make heavy use of the diagramatic language of braided monoidal categories established in chapter \ref{category-theory}. Concretely, we will think of a modular category as being the category with the following data:

\begin{equation*}
\left(\substack{
\mathbf{objects:}\text{ finite collections of anyons}\\
\mathbf{morphisms:}\text{ motions/behaviors of anyons}
}\right)
\end{equation*}

\Note{In a sense the above picture is not correct. Suppose I give you the object $2A\oplus B$. What ``finite collection of anyons" does this correspond to? None! It is more like a composite object which has two fusion channels to $A$ and one fusion channel to $B$. You could also maybe think of it as a probability distribution over $A$ and $B$.}

We have the following general pictuture for our algebraic theory:

\begin{dict}\label{topo-phase-dictionary} Topological phases of matter are algebraically described by unitary modular categories $\cC$ called the {\em anyon theory} of the phase, and a choice of integer $c_-\in \bZ$ called the {\em chiral central charge} of the phase.
\end{dict}

\begin{rem} Throughout these notes, there are some aspects of the general dictionary \ref{topo-phase-dictionary} that we have not emphasized. For instance, we have not emphasized that the modular categories describing topological phases are supposed to be {\em unitary} modular categories - we will discuss this later, but by analagy we can say that a unitary modular category is related to a modular category just like a Hilbert space is related to a vector space.

Another aspect we have not emphasized is the {\em chiral central charge}. This is a real invariant of topological phases which is beyond the description of modular categories. In particular, there are topological phases with no non-trivial anyons (and hence correspond to the trivial modular category) yet are non-trivial as topological phases. These are called {\em invertible} topological phases. Invertible topological phases are paramaterized by an integer invariant, their chiral central charge. Not every pair $(\cC,c_-)$ describes a valid topological phase. In particular, $\cC$ determines $c_-$ modulo $8$ but there is no other condition. The way that $\cC$ determins $c_-$ modulo $8$ is known as the Gauss-Milgram formula and is discussed in subsection \ref{vafa-theorem-unitarity-chiral-central-charge}. 
\end{rem}

\begin{rem}
The major structures of a modular tensor category can be motivated by considering abstractly the possible motions and behaviors of anyons. The most basic thing anyons can do is move anyons around each other - this is known as {\em braiding}. If the anyons touch each other then they can congeal into a single  anyon - this is known as {\em fusion}. Even if there are no anyons in a system, however, there is always something possible. Anyons can be spontaneously created, so long as every anyon which is created comes along with its corresponding antiparticle. This is known as {\em pair-creation}. These three operations are the fundamental structures which we will building into modular categories:

\begin{enumerate}
\item braiding;
\item fusion;
\item pair-creation.
\end{enumerate}
\end{rem}

\begin{rem}
Another potentially useful way of thinking about modular categories comes from analogy with classical physics. We saw in chapter \ref{overview} that topological classical systems have an algebraic description in terms of finite groups. Namely, quasiparticles in the system of ordered media with order space $M$ is algebraically characterized by the fundamental group $\pi_1(M,m)$ of $M$ relative to some basepoint $m\in M$. Seeing as topological order is a vast quantum generalization of classical ordered media, we can think of modular categories as being a vast quantum generalization of finite groups. Every finite group $G$ induces a modular category $\fD(G)$, by first constructing the Kitaev quantum double model based on that finite group and then describing its anyons. Many modular categories do not come from the group-theoretical construction.
\end{rem}

\subsubsection{Using the final product}

Before developping the theory of modular categories, it is good to get a feel for what using the final product is like. A modular category itself will be a big infinite structure, with infinitely many objects and infinitely many morphisms between those objects. However, all modular categories are in a real sense \textit{finitely generated}. What we mean by this is that plugging in a finite number of objects and morphisms, the rest of the obejcts and morphisms can be recovered by the abstract rules encoded in the formalism. In this way, the axioms of modular categories are not only neccecary by the fact that they restrict which categories can be modular categories, but they are also vital in the fact that they allow us to generate a full description of anyons from a minimal collection of data. For practically-minded readers, this can be viewed as the main motivation for putting so much work into defining modular categories..

\begin{ex}
To highlight how the formalism used to define an object impacts its finitely-generated description, we take an example from group theory. Consider the 3-strand braid group $B_3$. This group has ininitely many elements and the group operation $B_3\times B_3\to B_3$ naively takes an infinite amount of data to describe. However, the presentation

$$B_3=\Braket{\sigma_1, \sigma_2 | \sigma_1 \sigma_2 \sigma_1 = \sigma_2 \sigma_1 \sigma_2}$$

gives  completely finite description of $B_3$. It is important to note, however, that this presentation would \textit{not} have been enough to recover $B_3$ if we had just been told that $B_3$ is a monoid. The fact that $B_3$ is a group implied the existence of elements $\sigma_1^{-1}$, $\sigma_{2}^{-1}$, and defined how they interacted with $\sigma_1,\sigma_2$. We see in this way that the axioms of a group not only serve as a restriction on what mathematical objects are allowed to be groups, but they also serve as a compression technique. They give the rules by which a minimal collection of data can be used to generate the rest.
\end{ex}


An important step in going from modular categories to their description in terms of a finite set of data is in comming up with an efficient standard way of descrbing morphisms in a modular category. This is done using skeletonization, as discussed in section \ref{skeletonization}. A large table of these descriptions are found in appendix \ref{anyon-data}. We now give a worked example of how this data is used to compute observable quantities.

\begin{ex}
\Note{add toric code modular category data}
\end{ex}

\begin{ex}
Or, for a more complicated example, we can consider the data for $G=S_3$:
\Note{add $G=S_3$ modular category data}
\end{ex}

\begin{ex}
\Note{ add good example, computing some probability of annhilation of two anyons in $G=S_3$.}
\end{ex}

\subsection{First properties}

\subsubsection{Definition}

In this section we define modular categories, which are the main mathematical subject of this book. Seeing as lots of data is involved, we spread out the definition over a series of steps as to not overload the senses. These intermediate definitions are also important in their own right, because they will be used in other places in the algebraic theory of topological phases.


\begin{defn}[Fusion category] A fusion category is the following data:

\begin{enumerate}
\item A category $\cC$;
\item The structure of a right-rigid monoidal category on $\cC$;
\item The structure of a $\bC$-linear category on $\cC$.
\end{enumerate}

Such that:

\begin{enumerate}
\item The tensor product functor $\otimes: \cC\times \cC\to \cC$ induces bilinear maps hom-spaces;
\item There is an equivalence $\cC \cong \Vec_\bC^n$ as $\bC$-linear categories;
\item $\End_\cC(\bone)\cong \bC$ as $\bC$-vector spaces
\end{enumerate}


\end{defn}

A fusion category is part of the way towards having all of the requisite structures of a modular category: it has a method for fusion inherited from the tensor product, and it has half of a method for pair-creation coming from right-rigidity. The $\bC$-linearity allows us to think of hom-spaces as vector spaces, which allows us to treat hom-spaces as quantum systems. The condition (1) is a comptability between the $\bC$-linear structure and the monoidal structure. The conditions (2)-(3) are strong niceness and finiteness conditions - we will explain them in detail later.

\begin{defn}[Spherical fusion category] A spherical fusion category is the following data:

\begin{enumerate}
\item A fusion category $\cC$;
\item A left-rigid structure on $\cC$.
\end{enumerate}

Such that:

\begin{enumerate}
\item The left-rigid and right-rigid structures on $\cC$ satisfy the axioms of a pivotal structure on $\cC$;
\item For every object $A\in \cC$ and for every morphism $f: A \to A$, we have

\begin{equation*}
\tikzfig{spherical-category}
\end{equation*}
\end{enumerate}
\end{defn}

A spherical fusion category now has a structure for fusion, and a full structure for pair-creation. The 2nd condition is known as the \textit{spherical axiom}. We will explain this axiom in more detail later.

\begin{defn}[Pre-modular category] A pre-modular category is the following data:

\begin{enumerate}
\item A spherical fusion category $\cC$;
\item A braided structure on $\cC$.
\end{enumerate}

No extra compatibility conditions are required.
\end{defn}

A pre-modular category has all of the structure we require of a modular category: fusion, pair-creation, and braiding. The final axiom it is missing is a non-degeneracy condition. The non-degeneracy condition is subtle in its interpretation, and we will explain it several different ways throughout this chapter.

\begin{defn}[Modular category] A modular category is a pre-modular category satisfying the following condition. Let $A\in \cC$ be an object. If

\begin{equation*}
\tikzfig{non-degeneracy}
\end{equation*}

for all $B\in \cC$, then $A\cong \bone$.
\end{defn}


\subsubsection{Anyons in modular categories}

Modular categories are supposed to be theories of anyons in topological order. So, now that we have the definition of modular category, it is natural to ask: what do anyons mathematically correspond to, in modular categories? The answer lies within the condition in a fusion category $\cC$ that there is an equivalence $\cC\cong \Vec_\bC^n$ as $\bC$-linear categories. We explore the importance of this condition.

\begin{rem}
Suppose we are given an object $V=(V_1,V_2...V_n)\in \Vec_\bC^n$. For all $1\leq i\leq n$, let  $\bC_i\in \Vec_\bC^n$ denote the object which has dimension zero in every index $j\neq i$ and is equal to $\bC$ in index $i$. We observe the isomorphism

\begin{align*}
V& \cong \bigoplus_{i=1}^n (0... V_i ... 0)\\
& \cong \bigoplus_{i=1}^n (0... \bC^{\dim (V_i)}.. 0)\\
& \cong \bigoplus_{i=1}^n \dim(V_i)\cdot \bC_i
\end{align*}

where $\dim(V_i)\cdot (\bC_i)=\bC_i\oplus \bC_i...\oplus \bC_i$, $\dim(V_i)$ many times. This computation shows that any object in $\Vec_\bC^n$ can be decomposed into irriducible components $\bC_i$. These objects $\bC_i$ are in a real sense the building blocks of $\Vec_\bC^n$. In the language of definition \ref{simple-object-definition}, the objects $\cC_i$ are the simple objects of $\Vec_{\bC}^n$.
\end{rem}

\begin{defn}\label{simple-object-definition} A \textit{simple object} $A$ in a fusion category $\cC$ is an object which has no direct sum decomposition into smaller objects. That is, $A\ncong B\oplus C$ for any non-zero objects $B,C\in \cC$ where $\oplus$ denotes the biproduct in $\cC$.
\end{defn}

\begin{dict}\label{anyon-types-definition}
Isomorphism classes of simple objects in $\cC$ correspond to anyon types in a topological phase described by the MTC $\cC$.
\end{dict}

\begin{defn} For all fusion categories $\cC$, we define $\cL(\cC)$ (or simple $\cL$ when $\cC$ is clear from context) to the the set of isomorphism classes of simple objects. By entry \ref{anyon-types-definition} of the physics-math direction, elements of $\cL(\cC)$ correspond bijectively to anyon types in any topological phase described by $\cC$.
\end{defn}

\begin{prop}\label{fusion-category-simples} Let $\cC$ be a fusion category. The biproduct of any two elements in $\cC$ exists, and $\cC$ has a zero object. The set $\cL=\cL(\cC)$ is finite. Choose an object $X\in \cC$. There exist unique nonnegative integers $c_{[A]}$, $[A]\in \cL$ such that 

$$X\cong \bigoplus_{[A]\in \cL}N_{[A]}\cdot A.$$
\end{prop}
\begin{proof} We first show that $\cC$ has biproducts. Let $F:\cC\to \Vec_{\bC}^n$, $G:\Vec_{\bC}^n\to \cC$ be a pair of functors which induces an equivalence of categories, for some $n\geq 1$ Let $A,B\in \cC$ be objects. Since $G$ and $F$ are fully faithfull, the universal property of the direct sum $F(A)\oplus F(B)$ guarantees that $G(F(A)\oplus F(B))$ will be a direct sum of $G(F(A))$ and $G(F(B))$. Since $G(F(A))\cong A$ and $G(F(B))\cong B$, we conclude that $\cC$ has biproducts. The object $G(0)$ is a zero object for $\cC$.

We now prove that $\cC$ has finitely many isomorphism classes of simple objects. It is clear that an object $A\in \cC$ is simple if and only fi $F(A)\in \Vec_{\bC}^n$ is simple. Thus, since $G$ serves as an inverse, $F$ establishes a bijection between isomorphism classes of simple obejcts in $\cC$ and isomorphism classes of simple objects in $\Vec_{\bC}^n$. Every simple obejct in $\Vec_{\bC}^n$ will be isomorphic to $\bC_i$ for some $1\leq i \leq n$. Hence, there are $n$ simple objects in $\Vec_{\bC}^n$. Hence, there are $n$ simple objects in $\cC$, which is finite. The unique direct sum decomposition is clearly true in $\Vec_{\bC}^n$. It is immediate that it passes to a unique direct sum decomposition in $\cC$.
\end{proof}

\begin{prop}[Schur's Lemma] Let $\cC$ be a fusion category. An object $A\in \cC$ is simple if and only if its endomorphism ring $\End(A)$ is one-dimensional. Additionally, if $A,B\in \cC$ are nonisomorphic simple objects then $\Hom(A,B)=0$.
\end{prop}
\begin{proof} Let $F:\cC\to \Vec_{\bC}^n$, $G:\Vec_{\bC}^n\to \cC$ be a pair of $\bC$-linear functors which establishes an equivalence between $\cC$ and $\Vec_{\bC}^n$ as $\bC$-linear categories. The simple objects in $\Vec_{\bC}^n$ are all isomorphic to $\bC_i$ for some $1\leq i \leq n$. We compute that

$$\dim\left(\Hom_{\Vec_{\bC}^n}\left(\bigoplus_{i=1}^n n_i \bC_i , \bigoplus_{i=1}^n m_i \bC_i\right)\right)=\sum_{i=1}^n n_im_i.$$

As a corrolary of this formula, we find that if $A=\bigoplus_{i=1}^n n_i\bC_i$ then $\dim(\End_{\Vec_{\bC}^n}(A))=\sum_{i=1}^nn_i^2$. Clearly, this dimension is equal to one if $A=\bC_i$ for some $1\leq i\leq n$, and is greater than one otherwise. As a second corrolary, we compute that $\Hom(\bC_i,\bC_j)=0$ whenever $i\neq j$. The functor $G$ induces a bijection between isomorphism classes of simple objects in $\Vec_{\bC}^n$ and isomorphism classes of simple objects in $\Vec_{\bC}^n$, and it induces vector space isomorphisms on hom spaces. This means that the results for $\Vec_{\bC}^n$ translate to the desired result on $\cC$.
\end{proof}

\begin{rem}
Schur's lemma is a first verification that simple objects are a good choice of mathematical characterization of anyons. If $A,B$ are distinct anyon types, then there should not be any physical process which goes from one to another. There is no physical mechanism for locally turning one anyon type into another. This is captured by the formula $\Hom(A,B)=0$. Similarly, given an anyon $A$, there is no nontrivial action that can be locally performed on $A$. This comes from the fact that information is topologically protected, and thus cannot be changed by acting on a single particle - topological information processing requires global braiding between multiple particles. This is encoded in the fact that $\Hom(A,A)\cong \bC$ is one dimensional and hence consists only of trivial phase gates.
\end{rem}

\begin{rem}
As an application of Schur's lemma, we can observe that the monoidal unit $\bone$ is a simple object in every fusion category. By our physics-math dictionary, this means that $\bone$ corresponds to an anyon type. This anyon $\bone$ is the trivial ``no-anyon" type, corresponding to a quasiparticle which happens to be the same as the homogenous bulk.
\end{rem}

\begin{dict}\label{vacuum-anyon-definition} The monodial unit $\bone $ corresponds to the trivial anyon type, which is the same as the homogenous bulk (sometimes called the  \textit{vaccuum} anyon type).
\end{dict}

\begin{rem} Continuing in our expansion of the physics-math dictionary, we can assert that dual objects (defined using the right-rigid structure on a fusion category) correspond to antiparticles (entry \ref{antiparticle-definition}). This is consistent with the fact that anyon types correspond to simple objects, by proposition \ref{simple-dual-simple}.
\end{rem}

\begin{dict}\label{antiparticle-definition}
For every simple object $A$ in a modular category $\cC$, the dual object $A^*$ corresponds to the {\em antiparticle} of $A$.
Expanding our physics-math dictionary, we say that for every anyon $A$ its \textit{antiparticle} is the dual $A^*$ which comes from right-rigidity. This gives a valid anyon type by proposition \ref{simple-dual-simple}.
\end{dict}

\begin{prop}\label{simple-dual-simple} Let $\cC$ be a fusion category. If $A\in \cC$ is a simple object, then so is $A^*$.
\end{prop}
\begin{proof} By proposition \ref{rigidity-functorial} duality induces a bijection on hom-spaces. Since composition is bilinear, duality is thus an isomorphism of vector spaces on all hom-spaces. Hence, for all $A\in \cC$ there is an isomorphism $\Hom(A,A)\cong \Hom(A^*,A^*)$.  So, $\dim \Hom(A,A)= 1$ if and only if $\dim \Hom(A^*,A^*)=1$, and thus the result follows from Schur's lemma.
\end{proof}

\begin{dict}\label{composite-system-def}
The tensor product $\otimes$ physically corresponds to joining anyons, forming a composite anyon configuration. That is, the object $A\otimes B$ corresponds to the configuration with one $A$-type anyon and one $B$-type anyon.
\end{dict}

\begin{rem} By entry \ref{composite-system-def} of the physics-math dictionary and proposition \ref{fusion-category-simples}, it behooves us to take a moment to interpret the direct sum decomposition

\begin{equation}
A\otimes B \cong \bigoplus_{[C]\in \cL}N^{A,B}_{C}\cdot C,
\end{equation}

where $A,B\in \cC$ are objects in a fusion category. We call the non-negative integers $N^{A,B}_C$ the {\em fusion coefficients} of $\cC$. We observe that

\begin{equation}\label{fusion-rule-dimensions}
\dim \Hom(A\otimes B, C)=N^{A,B}_C.
\end{equation}

Morphisms in $\cC$ correspond to physical processes. So, equation \ref{fusion-rule-dimensions} can be interpreted as saying that whenever $N^{A,B}_{C}>0$, there is a physical process $A\otimes B\to C$. That is, there is a physical process which goes from the composite system of one $A$-type anyon and one $B$-type anyon and ouputs one $C$-type anyon. Neccecarily, such a process can be interpreted as {\em fusion} of $A$ and $B$ to $C$. Moreover, equation \ref{fusion-rule-dimensions} tells us that the space of possible physical processes which fuse $A$ and $B$ to $C$ (known as {\em fusion channels} $A\otimes B\to C$) is $N^{A,B}_C$-dimensional.
\end{rem}

\subsubsection{States in modular categories and unitarity}

\Note{This whole section only discusses \textit{pure} states. For a good discussion of mixed states and entropy, see \cite{bonderson2017anyonic}}

It is now worth reflecting on what exactly states correspond to in modular categories. Certainly, objects in modular categories are \textit{not} quantum systems. They don't have vector space structure. Objects correspond to anyon configurations, which are classical observables. The hom-spaces are what have vector space structure, by $\bC$-linearity. However, they are not yet Hilbert spaces. It is exactly for this reason we need to define {\em unitary modular categories}. These are modular categories in which hom-spaces have Hilbert space structure.

\begin{defn}[Unitary fusion category] A unitary fusion category is the following data:

\begin{enumerate}
\item An fusion category $\cC$.
\item (Conjugation) A linear map $\dagger: \Hom(A,B)\to \Hom(B,A)$ for all $A,B\in \cC$.
\end{enumerate}

Such that:

\begin{enumerate}
\item (Unitarity) Given $f:A\to A$ an endomorphism of $A\in \cC$, define


$$\tr(f)=\ev_A \circ (\id_{A^*}\otimes f)\circ \left(\ev_A\right)^{\dagger}.$$

The map $\left<\cdot|\cdot\right>:\Hom(A,B)\times \Hom(A,B)\to \bC$ defined by $\left<f|g\right>=\tr(f^{\dagger}\circ g)$ is an inner product, endowing $\Hom(A,B)$ with the structure of a Hilbert space.
\item $\left(f^{\dagger}\right)^{\dagger}=f$ for all $f\in \Hom(A,B)$, $A,B\in \cC$.

\item $(f\circ g)^{\dagger}=g^{\dagger}\circ f^{\dagger}$ for all $f\in \Hom(B,C)$,$g\in \Hom(A,B)$, $A,B,C\in \cC$.

\item $(f\otimes g)^{\dagger}=f^{\dagger}\otimes g^{\dagger}$ for all $f\in \Hom(A,B)$,$g\in \Hom(C,D)$, $A,B,C,D\in \cC$.
\item $\left(\coev_A\right)^{\dagger}\circ (f \otimes \id_{A^*})\circ \coev_A=\tr(f)$ for all $A\in \cC$
\end{enumerate}
\end{defn}

\begin{rem}
Unitary fusion categories make for a pleasant object of study because by proposition \ref{UFC-pivotal} the distinguished maps $(\ev_A)^{\dagger}:1\to A^{*}\otimes A$ and $(\coev_A)^{\dagger}:A\otimes A^{*}\to 1$ induce a pivotal structure.
\end{rem}

\begin{prop}\label{UFC-pivotal} Let $\cC$ be a unitary fusion category. The maps $\ev_{A}^L = (\coev_{A})^{\dagger}$ and $\coev^L_{A}=(\ev_A)^{\dagger}$ give a left-rigid structure on $\cC$. This left-rigidity endows $\cC$ with the structure of a spherical fusion category.
\end{prop}
\begin{proof}\Note{This is actually subtler than I expected. Either give a proof here, or postpone it to the unitarity section}
\end{proof}

\begin{defn}[Unitary pre-modular category] A unitary modular category is the following data:

\begin{enumerate}
\item A modular category $\cC$;
\item (Conjugation) A linear map $\dagger: \Hom(A,B)\to \Hom(B,A)$ for all $A,B\in \cC$.
\end{enumerate}

Such that:

\begin{enumerate}
\item Forgetting the left-rigid structure and braiding, $\left(\cC,\dagger\right)$ forms a unitary fusion category.
\item $\left(\ev_{A}^R\right)^{\dagger}=\coev_{A}^L$;
\item $\left(\coev_{A}^R\right)^{\dagger}=\ev_{A}^L$;
\item $\left(\beta_{A,B}\right)^{\dagger}=\beta_{B,A}^{-1}$.
\end{enumerate}
\end{defn}

\begin{defn}[Unitary modular category] A unitary modular category is a unitary pre-modular category which satisfies the non-degeneracy axiom of a modular category.
\end{defn}

\begin{rem} The compatibility conditions for the twist are chosen so that the definition of trace as a modular category and the definition of trace as a unitary fusion category coincide.
\end{rem}

\begin{dict} For any phase whose anyons as described by a unitary modular category $\cC$.

\begin{equation}\label{dict-state-formula}
\left(\substack{\text{states of topological order $\cC$} \\ \text{on the sphere $S^2$} \\ \text{with anyon configuraiton $A_1,A_2...A _n$}}\right)
=
\left(
\substack{
\text{normalized vectors in the Hilbert space}\\
\Hom_\cC(\bone, A_1\otimes A_2... \otimes A_n)
}
\right)
\end{equation}

where by ``anyon configuration $A_1,A_2...A_n$" we mean that the state has anyons present in $n$ sites, arranged left to right on a line segment in the sphere, with corresponding anyon type $A_1,A_2...A_n$.
\end{dict}

\begin{rem} It is not immediately clear where in equation \ref{dict-state-formula} we chose the sphere as the physical space. As one source of motivation, we can observe the following corollary of equation \ref{dict-state-formula}:

\begin{equation*}
\dim\left(\substack{\text{Hilbert space of topological order $\cC$} \\ \text{on the sphere $S^2$} \\ \text{with exactly one anyon of type $A$}}\right)
=
\dim \Hom(\bone, A)
=
\begin{cases}
1 & A=\bone \\ 
0 & \text{otherwise}.
\end{cases}
\end{equation*}

Hence, equation \ref{dict-state-formula} tells us that if the sphere has exactly one anyon on it then that anyon type must be trivial. Moreover, there is unique state on the sphere with no anyons. This is consistent with our general principles about topological order on the sphere.

\Note{ sketch nice argument for why there is a unique ground state on the sphere. What I'm struggling with here is why an anyon type in a region must neccecarily be detectable by its surrounding region.}
\end{rem}

\begin{dict} For any modular category $\cC$,:

\begin{equation}\label{dict-state-formula-2}
\left(\substack{\text{states of topological order $\cC$} \\ \text{on the infinite flat plane $\bR^2$} \\ \text{with anyon configuraiton $A_1,A_2...A _n$}}\right)
=
\left(
\substack{
\text{normalized vectors in the Hilbert space}\\
\Hom_\cC\left(\bigoplus_{[B]\in \cL}B, A_1\otimes A_2... \otimes A_n\right)
}
\right)
\end{equation}
\end{dict}

\begin{rem}
Replacing $\bone$ with $\bigoplus_{[B]\in \cL} B$ reflects the differences between the sphere and the infinite flat plane.

\Note{sketch nice argument for why states on infinite flat plane are determined by their overall charge. The subtelty here is exactly the same as the one for the sphere. Think about it then put it down.}

\Note{What happens for higher genus surfaces? I should add a few words about them. Zhenghan says all of this is contained in Turaev's book about the Reshetikhin-Turaev construction.}
\end{rem}

\begin{rem}
The anyon configurations in equations \ref{dict-state-formula} and \ref{dict-state-formula-2} are always assumed to be linear. The main reason to do this is because it makes the mathematics much simpler. If we kept track of the positions of each of the anyons in two dimensional space it would add more pieces of data and structures to keep track of. Seeing as every anyon configuration can be pushed onto a one-dimensional space, only working with a one-dimensional configuration does not affect the generality of the answers and hence it is prefered.
\end{rem}

\begin{rem} The formula $\Hom_{\cC}(\bone, A_1\otimes A_2...\otimes A_n)$ encodes the fact that states can be specified by their history. A good first question to ask when seeing the Hilbert space $\Hom_\cC(\bone, A_1\otimes A_2... A_n)$ is {\em why} this should describe a state with anyon configuration $A_1$... $A_n$. The answer is that states can be described their history. \Note{give good example of making a state by specifying its history; argue why it has to be this way in general.}
\end{rem}

\begin{rem}
In the definition of an modular category hom-spaces are vector spaces and not Hilbert spaces, so this choice of physics-math correspondance is incorrect as literally written. To make this definition work, all of the hom-spaces of the modular category $\cC$ should be equipped with Hilbert space structures. Furthermore, the natural operators we wish to perform like braiding should all be unitary with respect to these inner products. This amounts to adding a large number of compatibility conditions on the Hilbert space structures. A modular category with this choice of structure is known as a \textit{unitary} modular category.
\end{rem}

For this reason, the correct algebraic structure to underlie the theory of topological order is not a modular category, but a unitary modular category. We have chosen to not emphasize this before because the difference between unitary modular categories and non-unitary modular categories is very small. \Note{talk about uniqueness + positive q.d. criterion this will make more sense once we write the actual section about unitarity. A good thing to emphaize is that unitary modular categories don't let you use less data in your definition, and you can still do essentially everything you want to do. It's just way more cumbersome. They're all equivalent but you still have to choose, c.f. the fact that the category of vector spaces and Hilbert spaces with linear maps as morphisms are equivallent}.

\subsubsection{Topological charge measurement}

When two anyons are fused together, they will form a superposition of other anyon types. Measuring the result of the fusion will collapse the answer into a specific anyon type. The outcome of this measurment is an observable quantity, which allows for the measurement of topological quantum information. In many cases this is the \textit{only} local observable quanitity. We give the formalism behind computing these probabilities now.

\Note{ do this right - I don't know it well but it shouln't be hard to learn. Don't introduce anything too general, like trace or whatnot. Just quantum dimension, which should already have been introduced in previous chapter.}

\Note{ The correct reference for this subsection is \cite{bonderson2021measuring}. The paper \cite{cong2017universal} claims to introduce the term topological charge measurement and gives a nice formal treatmenet. Clarifying the situation seems important.}
 


\subsection{The modular category toolkit}

In this section, we will introduce and prove the basic facts about the most important structures in the theory of modular categories. These facts and structures are the tools used for solving problems about the algebraic theory of anyons. 

\Note{ I don't have a section on fusion coefficients yet. I guess this isn't a problem, because there isn't that much to say. I would like to have the associativity of fusion cooficients and the fact that braiding $\implies$ commutative said somewhere explicitely, though. Find a place?}

\subsubsection{Trace}

The first structure to define in the theory of modular categories is the \textit{trace}. Let $\cC$ be a spherical fusion category. Given any object $A\in \cC$ and any endomorphism $f:A\to A$, we define the \textit{trace of $f$} by the following formula:

\begin{equation*}
\tikzfig{trace}
\end{equation*}

Initially, the trace is a morphism,  $\tr(f):\bone\to\bone$. However, we will choose to think of the trace of a morphism as a \textit{complex number}, $\tr(f)\in \bC$. This can be done because the definition of a fusion category $\End(\bone)\cong \bC$. This isomorphism can be made canonical by identifying an endomorphism $g\in \End(\bone)$ with the unique $\lambda\in \bC$ such that $g = \lambda \cdot \id_{\bone}$.

The trace is used mainly as a tool for linearization. Morphisms and objects are hard to describe, but the trace is a complex number.

\begin{prop} Let $\cC$ be a spherical fusion category. For all $A,B\in \cC$, $f\in \End(A)$ the following claims are all true:

\begin{enumerate}
\item $\tr: \End(A)\to \bC$ is a linear map of vector spaces,
\item $\tr(f^{*})=\tr(f)$,
\item $\tr(f\oplus g)=\tr(f)+\tr(g)$ for all $g\in \End(B)$,
\item $\tr(f\otimes g)=\tr(f)\cdot \tr(g)$ for all $g\in \End(B)$,
\item $\tr(h\circ g)=\tr(g\circ h)$ for all $g:A\to B$, $h:B\to A$.
\item Trace is preserved by functors. That is, let $\cC,\cD$ be spherical categories with traces $\tr_{\cC},\tr_{\cD}$ respectively. Let $F:\cC\to \cD$ be a pivotal functor. We have that $\tr_{\cC}(f)=\tr_{\cD}(F(f))$;
\end{enumerate}

\end{prop}
\begin{proof} We prove the claims one by one.

\begin{enumerate}
\item This follows immediately from the bilinearity of composition.

\item This is a straightforward computation.

\item \Note{do proof. This uses facts about the direct sum we haven't established yet.}

\item Using Proposition [ref] we compute

\begin{equation*}
\tikzfig{trace-tensor}
\end{equation*}

\item Using Proposition [ref] we find that

\begin{equation*}
\tikzfig{composition-commutes}
\end{equation*}

\item \Note{do the proof. It's not very hard, but involves a diagram and uses pivotality of the functor.}

\end{enumerate}

This completes the proof.
\end{proof}

With these properties in hand, we can explicitely compute the trace using a straightforward procedure:

\begin{cor} Let $f:A\to A$ be an endomorphism in a fusion category $\cC$. Fix a decomposition $A\cong \bigoplus_{i\in I}A_i$ of $A$ into simple objects $A_i$. Moreover, we take the decomposition such that if $A_i\cong A_j$ then $A_i=A_j$. We can decompose

$$\Hom(A,A)\cong \Hom(\bigoplus_{i\in I} A_i,\bigoplus_{i\in I}A_i)=\bigoplus_{i\in I, j\in I}\Hom(A_i,A_j).$$

Let $M$ be the matrix whose collums and rows are labeled by $I$, and whose $(i,j)$ entry is $0$ if $A_i\not\cong A_j$ and $\lambda \cdot d_{A_i}$ if $A_i=A_j$, where $\lambda\in \bC$ is the unique value such that the $\Hom(A_i,A_j)$ component of $f$ is $\lambda\cdot \id_{A_i}$. We have that

$$\tr_{\cC}(f)=\tr_{\Vec}(M).$$

\end{cor}
\begin{proof} Suppose that $A=A_0\oplus A_1$ is the direct sum of two objects, not neccecarily simple. By proposition [ref] we have a canonical decomposition

$$(A_0\oplus A_1)\otimes (A_0\oplus A_1)^*\cong (A_0\otimes A_0^*) \oplus (A_0\otimes A_1^*) \oplus (A_1\otimes A_0^{*})\oplus (A_1\otimes A_1^*).$$

Suppose that $h:A\to A$ is an endomorphism. We can decompose $h=h_{A_0,A_0}+h_{A_0,A_1}+h_{A_1,A_0}+h_{A_0,A_1}$ as a sum of morphisms which restrict to maps $A_i\to A_j$. We find that $\coev_{A\oplus B}$ restricts to a map whose codomain is  $(A_0\otimes A_0^*) \oplus (A_1\otimes A_0)^*$ and similarly $\ev$ restricts to a map whose domain is $(A_0\otimes A_0^*) \oplus (A_1\otimes A_0)^*$, since $\coev_{A\oplus B}=\coev_{A}\oplus \coev_{B}$ and $\ev_{A\oplus B}=\ev_{A}\oplus \ev_{B}$.

Hence, in the definition of trace, we find that the cross terms $h_{A_0,A_1}+h_{A_1,A_0}$ act by zero since they send the codomain of $\coev_{A\oplus B}$ to elements with no effect on the map $\ev_{A\oplus B}$. Moreover, we compute in this way that $\tr(h)=\tr(h_{A_0,A_0})+\tr(h_{A_1,A_1})$.

In this way, the trace splits over direct sums and only picks out diagonal elements. Applying this result inductively reduces the proof to the case that $A$ is a simple object. This follows directly from the definition of quantum dimension.
\end{proof}

\subsubsection{Duality}

Duality is baken into our definition of modular categories as a fundamental part of the structure. It acts in a very controlled way on fusion cooefficients:

\begin{prop}Let $\cC$ be a fusion category and let $A,B,C\in \cC$ be simple objects. We have the following:

\begin{enumerate}[(i)]
\item (Anti-involution) $N^{A,B}_C=N^{B^*,A^*}_{C^*}$;
\item (Frobenius reciprocity) $N^{A,B}_C = N^{A^*,C}_B = N^{C, B^*}_{A}$.
\end{enumerate}

\end{prop}
\begin{proof} Part (i) follows from the fact that the duality functor is fully faithful and monoidal from propositon [ref], so

$$N^{A,B}_{C}=\dim \Hom(C,A\otimes B)=\dim \Hom(A^*\otimes B^*,C^*)=N^{B^*,A^*}_{C^*}.$$

Part (ii) follows from the following computation. Consider the map

\begin{equation*}
\tikzfig{fusion-coof-is-one}
\end{equation*}

Since composition is bilinear, $i$ is a linear map. The map

\begin{equation*}
\tikzfig{fusion-coof-proof}
\end{equation*}

serves as an inverse for $i$ by rigidity. Hence, we conclude that

$$N^{A,B}_{C}=\dim \Hom(C,A\otimes B) = \dim \Hom( A^*\otimes C, B) = N^{A^*,C}_{B}.$$

The third equality in Frobenius reciprocity follows from an identical argument, and hence we conclude the proof.
\end{proof}

In particular, we can describe the fusion rules of a simple object with its dual:

\begin{cor} Let $\cC$ be a fusion category. Let $A,B\in \cC$ be simple objects. We find that

$$
N^{A,B}_{\bone}=N^{B,A}_{\bone}=
\begin{cases}
1 & B\cong A^*\\
0 & \text{otherwise}.
\end{cases}$$
\end{cor}
\begin{proof} This follows from Frobenius reciprocity and Schur's lemma:

$$N^{A,B}_{\bone}=N^{A^*,\bone}_{A}=\dim(\Hom(A,A^*))=
\begin{cases}
1 & B\cong A^*\\
0 & \text{otherwise}.
\end{cases}$$
\end{proof}

Specializing even more, we get the following corollary:

\begin{cor} If $\cC$ is a fusion category, then $A\cong A^{**}$ for all $A\in \cC$.
\end{cor}
\begin{proof} Since $N^{A^*,A^{**}}_{\bone}>0$, we conclude that $A^{**}\cong A$ by the $(iii)\implies (i)$ implication in proposition [ref].
\end{proof}

However, despite this corollary, we \textit{cannot} conclude that every fusion category admits a pivotal structure. The isomorphism $A\cong A^{**}$ may fail to form a monoidal natural transformation. It is an open problem whether or not every fusion category admits a pivotal structure, and it is furthermore an open problem whether every pivotal fusion category admits a spherical structure \cite{etingof2005fusion}.


\subsubsection{Quantum dimension and Frobenius-Perron dimension}

\Note{ include something about global quantum dimension $\cD$.}

Our next tool to discuss is the \textit{quantum dimension}. Given any spherical fusion category $\cC$ and any object $A\in \cC$, we define its quantum dimension using the following formula:

\begin{equation*}
\tikzfig{quantum-dimension}
\end{equation*}

As usual, we identitfy $d_A$ with a complex number via the canonical isomorphism $\End(\bone)\cong \bC$. The quantum dimension is clearly equal to the trace of the identity map on $A$, $d_A=\tr(\id_A)$. The first properties of quantum dimension follow from our general analysis of trace:

\begin{prop} For every spherical fusion category $\cC$ and any objects $A,B\in \cC$, we have the following formulas:

\begin{enumerate}[(i)]
\item If $A\cong B$, then $d_A=d_B$;
\item $d_{A\oplus B}=d_{A}+d_B$;
\item $d_{A\otimes B}=d_{A}\cdot d_{B}$;
\item $d_{A^*}=d_A$.
\item $d_A\neq 0$.
\end{enumerate}
\end{prop}
\begin{proof}$\,$
\begin{enumerate}[(i)]
\item Let $f:A\cong B$ be an isomorphism. Using Proposition [ref] we find

$$d_A=\tr(\id_{A})=\tr(f^{-1}\circ f)=\tr(f\circ f^{-1})=\tr(\id_{B})=d_B.$$

\item This follows from proposition [ref].
\item This follows from proposition [ref].
\item This follows from proposition [ref].
\item From proposition [ref], we know that $A\otimes A^* \cong \bone \oplus X$ for some $X\in \cC$ which does not have any factors of $\bone$ in its direct sum decomposition. The map $\coev^R_A: \bone \to A\otimes A^*$ is thus a non-zero scalar times the inclusion $\bone \hookrightarrow{} \bone\oplus X$, and the map $\ev^L_A: A\otimes A^*\to \bone$ is a non-zer scalar times the projection $\bone \oplus X \xrightarrow{}\bone$. Since inclusion composed with projection is the identity, we find that $\ev^{L}_{A}\circ \coev^R_{A}$ is a non-zero scalar times the identity, as desired.
\end{enumerate}
\end{proof}

The above propositions tell us that the values $d_A$ as $A$ ranges over isomorphism classes of simple objects determs all the other values of $d_A$. Moreover, proposition [ref] tells us that the quantum dimensions of simple objects determines the trace of \textit{every} endomorphism! Hence, computing $d_{A}$ for each isomorphism class $[A]\in \cL$ is an important step in analysing an modular category. The following formula and its linear-algebraic reformulaion are the primary insight in performing the computation:

\begin{prop} Let $\cC$ be a spherical fusion category.

\begin{enumerate}[(i)]
\item Let $A,B\in \cC$ be simple objects. We have that

$$d_Ad_B=\sum_{[C]\in \cL}N^{A,B}_C d_C.$$

\item Let $A\in \cC$ be a simple object. Define an operator

 \begin{align*}
N^{A}:\bC[\cL]&\xrightarrow{} \bC[\cL].\\
\ket{[B]}&\mapsto \sum_{[C]\in \cL} N^{A,B}_C \ket{[C]}
\end{align*}

Define $\bold{d} = \sum_{[B]\in \cL} d_{B}\ket{[B]}\in \bC[\cL]$. We have that

$$N^{A}\bold{d}=d_{A}\bold{d}.$$

\end{enumerate}

\end{prop}
\begin{proof} From proposition [ref], we have an isomorphism

$$A\otimes B \cong \bigoplus_{[C]\in \cL}N^{A,B}_{C}\cdot C$$

and thus

$$\tr(\id_{A\otimes B})=\tr(\id_{\bigoplus_{[C]\in \cL}N^{A,B}_{C}\cdot C}).$$

Expanding using the rules in proposition [ref] gives part (i). Part (ii) follows from expanding the definition of the linear operator and applying part (i).
\end{proof}

We now make commentary about the above proposition. It tells us that $d_A$ is an eigenvalue of $N^A$. Since $N^A$ is an operator with integer coefficients, this immediately tells us that $d_A$ is the root of polynomial with integer coeffients. Namely, the characteristic polynomial of $N^A$. We can even be more precise about the nature of $d_A$:

\Note{On the MathOverflow question ``Modular Tensor Categories: Reasoning behind the axioms", a commentator said ``You also have to assume that the categorical dimensions arising from the pivotal structure are all real. This is called the spherical axiom". Why is this the same thing as the spherical axiom? What is the motivation? I should include this as a remark on the below proposition.}

\begin{cor} Let $\cC$ be a spherical fusion category. The quantum dimensions of all simple objects in $\cC$ are real numbers.
\end{cor}
\begin{proof}\Note{ do proof.}
\end{proof}

The question is whether or not the quantum dimensions are \textit{positive} real numbers. We recall that we defined a unitarizable spherical fusion category to be one in which the quantum dimensions are all positive. It is at this point that this becomes relevant. In particular, if $\cC$ is unitarizable then its quantum dimensions are eigenvalues of $N^A$, and their corresponding eigenvector $\bold{d}$ has positive real entries. There is a theorem about eigenalues of non-negative matrices with positive eigenvectors:

\begin{thrm}[Frobenius-Perron theorem, \cite{etingof2016tensor}] Let $B$ be a square matrix with nonnegative real entries.

\begin{enumerate}[(i)]
\item $B$ has a non-negative real eigenvalue. The largest non-negative real eigenvalue $\lambda(B)$ of $B$ dominates the absolute values of all other eigenvalues $\mu$ of $B$: $|\mu|\leq \lambda(B)$. Moreover, there is an eigenvector of B with non-negative entries
and eigenvalue $\lambda(B)$.
\item If $B$ has strictly positive entries then $\lambda(B)$ is a simple positive eigenvalue, and the corresponding eigenvector can be normalized to have strictly positive entries. Moreover, $|\mu| < \lambda(B)$ for any other eigenvalue $\mu$ of $B$.
\item If a matrix $B$ with non-negative entries has an eigenvector $v$ with strictly
positive entries, then the corresponding eigenvalue is $\lambda(B)$.
\end{enumerate}
\end{thrm}

We call the largest positive real eigenvalue of a matrix its \textit{Frobenius-Perron eigenvalue}. The Frobenius-Perron theorem tells us the following:

\begin{cor} Let $\cC$ be a unitarizable spherical fusion category. Let $A\in \cC$ be a simple object. The quantum dimension $d_A$ is equal to the Frobenius-Perron eigenvalue of $N^A$.
\end{cor}
\begin{proof} Since $\cC$  is unitarizable, the vector $\bold{d} = \sum_{[B]\in \cL} d_{B}\ket{[B]}\in \bC[\cL]$ has positive entries and has eigenvalue $d_A$. Hence, $d_A$ is the Frobenius-Perron eigenvalue of $N^A$ as desired.
\end{proof}

In this chapter we will mostly work with generic spherical fusion categories with no conditions on unitarizability. Hence, it is useful to make the following definition. Let $A\in\cC$ be a simple object in a spherical fusion category. We define

$$\FPdim(A)=(\text{Frobenius-Perron eigenvalue of $N^A$}).$$

When $\cC$ is unitarizable, $\FPdim(A)=d_A$. Many formulas about quantum dimension in the unitary world apply to the Frobenius-Perron dimension in the non-unitary world. An interesting observation is that the definition of quantum dimension strongly uses the spherical structure on $\cC$. However, the Frobenius-Perron dimension only uses the fusion coeffients, and those are well-defined in any fusion category. Hence, the Frobenius-Perron dimension also derives utility from being applicable in a broader set of situations than the quantum dimension.

We now give an alternate interpretation of the Frobenius-Perron dimension in terms of growth in tensor powers. This sort of alternate perspective of dimension applies to several types of objects outside the scope of tensor category theory \cite{coulembier2024growth}.

\begin{prop} Let $\cC$ be a fusion category, and let $A\in \cC$ be a simple object.

\begin{enumerate}[(i)]
\item $\FPdim(A)=\lim_{n\to\infty}\dim(\Hom(A^{\otimes n},A^{\otimes n}))^{1/(2n)}$
\item $\FPdim(A)=\lim_{n\to\infty}\dim(\Hom(\bone,A^{\otimes n}))^{1/n}$
\item $\,$

$$\FPdim(A)=\lim_{n\to\infty}(\text{\# of simple objects in the direct sum decomposition of $A^{\otimes n}$})^{1/n}.$$
\end{enumerate}
\end{prop}
\begin{proof}\Note{I can do a good part of this when $\cC$ is unitarizable, so that its largest eigenvalue is strictly larger than all of the others. When there are multiple large eigenvalues all of the same size then the proofs go wrong. Is there something about the structure of $N^A$ I can exploit? Are these theorems true for fusion categories, or do I need to pass to unitarizable fusion categories?}
\end{proof}

This proposition can be interpreted as saying that the simple object $A$ has $\FPdim(A)$ internal degrees of freedom ``on average". Elements of the vector space $\Hom(\bone,A^{\otimes n})$ correspond to states in the system with $n$ anyons of type $A$ arranged in a line. If the internal configuration space of each anyon was $\FPdim(A)$-dimensional, then the overall dimension would be $\FPdim(A)^n$. By Proposition [ref], $\FPdim(A)^n$ is approximately $\Hom(\bone,A^{\otimes n})$ for large $n$. Hence, each anyon has approximately $\FPdim(A)$ internal degrees of freedom. Of course, $\FPdim(A)$ has no reason to be an integer! In the Fibonacci theory $\FPdim(\tau)=\phi=1.61...$. Frobenius-Perron dimension just gives an average amount for large values.

\Note{re-do this explanation way better + add diagram for it.}

\subsubsection{Twist}

In this section we will discuss \textit{twists}. The twist is a subtle concept, which we have not explicitely mentioned up to now. The idea is that anyons can \textit{rotate in place}. Since the space of endomorphisms of an anyon is one dimensional, this rotation must act by a phase. This phase is physically relevant, and can be measured in experiment.

For example, consider the $Y$-type on the toric code. It consists of the fusion of an $X$-type anyon and a $Z$-type anyon, as shown below:

\Note{ add figure of Y as a thick X and Z together; could be hard to draw these nice}

Twisting $Y$ in place will correspond to twisting $X$ and $Z$ around each other. This twsiting thus results in a phase of $-1$. In general, we can imagine anyons as having some thickness to them. Anyons are not localized at points - they are localized at small regions. Twisting this region all the way around can be viewed visually as

\Note{ twisted anyon.}

This is the twist. One way of working with the twist is to work with thickened diagrams, where strings are replaced with ribbons. While popular in some parts of the literature, we will continue to work with string diagrams for simplicity. The key observation is that the twist can be constructed using string diagramatic structures we already have as follows:

\Note{twist as a swirl diagram, compared with ribbon.}

Hence, letting $\cC$ be a pre-modular fusion category, we \textit{define} the twist $\theta_{A}$ of an object $A\in \cC$ to be

\begin{equation*}
\tikzfig{graphical-twist}
\end{equation*}

For every simple object $A\in \cC$, the map $\theta_A\in \End(A)$ can be identified with the unique complex number $\lambda$ such that $\theta_A=\lambda\cdot \id_A$. Equivilantly, we can identify $\theta_A$ with the complex number $\lambda=\tr(\theta_A)/d_A$ which gives the graphical formula

\begin{equation*}
\tikzfig{twist}
\end{equation*}

We can reinterpret all other twist-like maps in terms of $\theta$:

\begin{lem} Let $\cC$ be a pre-modular fusion category. We have that

\begin{equation*}
\tikzfig{twist-alternatives}
\end{equation*}

\end{lem}
\begin{proof} To begin we show that

\begin{equation*}
\tikzfig{twist-equality}
\end{equation*}

When $A$ is simple, this follows from the spherical axiom. Taking the trace of both sides gives the same formula for $\theta_A$ as a figure-eight. Additionally, pushing through duals it is clear that both sides in the above proposed equality are natural isomorphisms. Natural isomorphisms are determined by their action on simple objects because they commute with direct sums. Hence, we conclude that the sides are equal for all objects.

To get that the two reversed formulas are equal to $\theta_{A}^{-1}$, it suffices to compose with $\theta_A$ and use string-diagram manipulations to show that it results in the identity. This is a simple exercise and is left as an exercise to the reader.
\end{proof}

We now characterize the key properties of the twist:

\begin{prop} Let $\cC$ be a pre-modular fusion category. The twists $\theta$ induce a monoidal natural isomorphism $\id_{\cC}\xrightarrow{\sim}\id_{\cC}$. Additionally, $\theta$ satisfies the identity

$$\theta_{A\otimes B}=\beta_{B,A}\circ \beta_{A,B}\circ (\theta_{A}\otimes \theta_{B})$$

for all $A,B\in \cC$, and $\theta_{A^*}=(\theta_A)^*$.
\end{prop}
\begin{proof} Naturality of $\theta$ follows from pushing through duals. The formula $\theta_{A\otimes B}=\beta_{B,A}\circ \beta_{A,B}\circ (\theta_{A}\otimes \theta_{B})$ comes from manipulating string diagrams to get the equation

\begin{equation*}
\tikzfig{twist-naturality}
\end{equation*}


Finally, $\theta_{A^*}=(\theta_A)^*$ comes from the string-diagram manipulation and proposition [ref]:

\begin{equation*}
\tikzfig{twist-duality}
\end{equation*}

as desired.
\end{proof}

The naive reason to care about twists is that they descrbe a physically relevant quantity and hence should be studied. The more subtle reason to care about twists is that they are the most efficient way of encoding the spherical structure on $\cC$. A spherical structure is first and foremost a pivotal structure, meaning that it has a right and left rigid structure which are compatible. Given a spherical structure one can always obtain twists. Conversely, given a right-rigid structure and twists one can recover the left-rigid structure via the formulas

\begin{equation*}
\tikzfig{co-dual}
\end{equation*}

In this way, giving a spherical structure on a right-rigid monoidal category is the \textit{same} as giving a twist structure. This is codified in the following lemma:

\begin{prop}[Deligne's twisting lemma, \cite{yetter1992framed}] Let $\cC$ be a right-rigid braided monoidal category. Every pivotal structure on $\cC$ naturally gives a twist natural transformation $\theta:\id_\cC\to\id_\cC$. This assignment induces a canonical bijection between the set of pivotal structures on $\cC$ and the set of natural isomorphism $\theta:\id_\cC\to\id_\cC$ satisfying $\theta_{A\otimes B}=\beta_{B,A}\circ \beta_{A,B}\circ (\theta_A\otimes \theta_B)$ for all $A,B\in \cC$.

Moreover, restricting the assignment to the space of spherical structures on $\cC$ induces a canonical bijection between the set of spherical structures on $\cC$ and the set of isomorphisms $\theta:\id_\cC\to\id_\cC$ satisfying $\theta_{A\otimes B}=\beta_{B,A}\circ \beta_{A,B}\circ (\theta_A\otimes \theta_B)$ for all $A,B\in \cC$ and $\theta_{A^*}=(\theta_A)^*$.
\end{prop}
\begin{proof} We already showed in proposition [ref] that every spherical category gives a twist natural transformation satisfying the desired axioms. Restriciting the proof to only a possibly non-spherical pivotal category still gives a twist natural transformation satisfying $\theta_{A\otimes B}=\beta_{B,A}\circ \beta_{A,B}\circ (\theta_A\otimes \theta_B)$ for all $A,B\in \cC$. The heart of the proof is showing that the formulas [ref] induce pivotal and spherical structures with the twist satisfies the right axioms. The process of inducing a pivotal structure and inducing a twist are inverses to one another because

\begin{equation*}
\tikzfig{graphical-twist-reverse}
\end{equation*}

To begin, we assume that $\theta_{A\otimes B}=\beta_{B,A}\circ \beta_{A,B}\circ (\theta_A\otimes \theta_B)$ and we seek to prove that the corresponding $\ev^{L}$, $\coev^L$ maps induce a pivotal structure. We first axiom of pivotality follows from use of the axiom $\theta_{A\otimes B}=\beta_{B,A}\circ \beta_{A,B}\circ (\theta_A\otimes \theta_B)$:

\begin{equation*}
\tikzfig{something-property-proof}
\end{equation*}

The second axiom of pivotality follows from the use of the naturality of $\theta$:

\begin{equation*}
\tikzfig{morphism-duals-agree-proof}
\end{equation*}

Finally, we assume that $(\theta_A)^*=\theta_{A^*}$ and we seek to prove the spherical axiom. Taking the dual of theta we can get all of the equalities in Lemma [ref]. Applying them we get that

\begin{equation*}
\tikzfig{spherical-proof}
\end{equation*}

as desired.
\end{proof}

\subsubsection{Functors, natural transformations, and equivalence}

In this section, we will talk about functors, natural transformations, and equivalences between fusion, spherical, pre-modular, and modular categories. Given a topological order, there is \textit{not} a unique modular category describing it. There is a unique modular category \textit{up to equivalence}. Hence, the notion of equivalence of categories is baked into our physics-math correspondance so it is important that we state it explicitely.

Functors which do not induce equivalences of categories are also physically relevant. In certain contexts, a functor $F:\cC\to \cD$ is used to model a \textit{phase transition} from $\cC$ to $\cD$. We will see a lot more functors and natural transformations between modular categories throughout the book, especially in chapter [ref].

Even though structures in categories require a lot of compatibility conditions, the conditions on the functors do not. This means that we have the following:

\begin{itemize}
\item The correct notion of functor between fusion categories is $\bC$-linear monoidal functor. There is no compatibility condition required between the $\bC$-linear structure and the monoidal structure. The correct notion of natural transformation between $\bC$-linear monoidal functors is a monoidal natural transformation.

\item The correct notion of functor between spherical fusion categories is $\bC$-linear pivotal monoidal functor. There is no compatibility condition required between the $\bC$-linear structure and the pivotal monoidal structure. The correct notion of natural trasnformation is monoidal natural transformation.

\item The correct notion of functor between pre-modular categories is $\bC$-linear pivotal braided monoidal functor. There is no compatibility condition required between the $\bC$-linear structure, pivotal monoidal structure, or braided monoidal structure. The correct notion of natural transformation is monoidal natural transformation.

\item The correct notions of functors/natural transformations for modular categories are the same as for pre-modular categories.

\end{itemize}


\Note{: this section is very short. I don't have much to say, actually. Should this be moved? Maybe I keep a very short section? I don't know.}

\subsubsection{Deligne tensor product}

In the theory of any class of mathematical object, an important consideration is the ways in which examples can be put together to give new examples. In the case of fusion categories, this basic operation is known as the \textit{Deligne tensor product}. Given any fusion categories $\cC$, $\cD$, their Deligne tensor product $\cC\boxtimes \cD$ is a new fusion category. The Deligne tensor product of spherical fusion categories will be equipped with the structure of a spherical fusion category, and the Deligne tensor product of (pre-)modular categories will be equipped with the structure of a (pre-)modular category.

Physically, the Deligne tensor product corresponds to \textit{stacking}. Consider two sheets of material. We choose two modular categories $\cC$, $\cD$. We enodow the top sheet with the structure of a topologically ordered quantum system described by $\cC$ and we endow the bottome with the structure of a topologically ordered quantum system described by $\cD$. The algebraic description of this bilayer system is $\cC\boxtimes \cD$. This can be viewed as the physical definition of $\cC\boxtimes \cD$.

\Note{ add bilayer system diagram}

We now mathematically define the Deligne tensor product.

\begin{defn} Let $\cC,\cD$ be a $\bC$-linear categories, isomorphic as a $\bC$-linear categories to $\Vec_{\bC}^{n}$, $\Vec_{\bC}^m$ respectively. We define a Deligne tensor product of $\cC$ and $\cD$ to be be the following data:

\begin{enumerate}
\item A $\bC$-linear category $\cC\boxtimes \cD$;
\item A $\bC$-linear functor $\cC\times\cD \xrightarrow{} \cC\boxtimes \cD$.
\end{enumerate}

Such that:

\begin{enumerate}
\item Every object $X\in \cC\boxtimes \cD$ has a direct sum decomposition

$$X\cong \bigoplus_{i=1}^n A_i\boxtimes B_i$$

for some $n\geq 1$, $A_i\in \cC$, $B_i\in \cD$.

\item There is an equality of vector spaces

$$\Hom_{\cC\boxtimes\cD}(A\boxtimes B,A'\boxtimes B')=\Hom_{\cC}(A,A')\otimes \Hom_{\cD}(B,B').$$

\item Given any $A,A',A''\in \cC$, $B,B',B''\in \cD$, $f:A\to A'$, $f':A'\to A''$, $g:B\to B'$, $g':B'\to B''$, the diagram

% https://q.uiver.app/#q=WzAsMyxbMCwwLCJBXFxib3h0aW1lcyBCIl0sWzEsMCwiQSdcXGJveHRpbWVzIEInIl0sWzIsMCwiQScnXFxib3h0aW1lcyBCJyciXSxbMCwxLCJmXFxib3h0aW1lcyBnIl0sWzEsMiwiZidcXGJveHRpbWVzIGcnIl0sWzAsMiwiKGYnXFxjaXJjIGYpXFxib3h0aW1lcyAoZydcXGNpcmMgZikiLDIseyJjdXJ2ZSI6M31dXQ==
\[\begin{tikzcd}
	{A\boxtimes B} & {A'\boxtimes B'} & {A''\boxtimes B''}
	\arrow["{f\boxtimes g}", from=1-1, to=1-2]
	\arrow["{(f'\circ f)\boxtimes (g'\circ f)}"', curve={height=18pt}, from=1-1, to=1-3]
	\arrow["{f'\boxtimes g'}", from=1-2, to=1-3]
\end{tikzcd}\]

commutes.
\end{enumerate}

\end{defn}

We know state the main existence/uniqueness result about the Deligne tensor product:

\begin{prop} Let $\cC,\cD$ be $\bC$-linear categories isomorphic as $\bC$-linear categories to $\Vec_{\bC}^n$ and $\Vec_{\bC}^m$ respectively. There exists a Deligne product $\cC\boxtimes \cD$ for $\cC$ and $\cD$. Moreover, given any other deligne tensor product $\cC\boxtimes' \cD$ of $\cC$ and $\cD$ there exists a unique functor $F: \cC\boxtimes \cD\xrightarrow{} \cC\boxtimes' \cD$ making the diagram

% https://q.uiver.app/#q=WzAsMyxbMCwwLCJcXENcXHRpbWVzXFxEY2F0Il0sWzEsMCwiXFxDXFxib3h0aW1lcyBcXERjYXQiXSxbMSwxLCJcXENcXGJveHRpbWVzJyBcXERjYXQiXSxbMCwxXSxbMCwyXSxbMSwyLCJGIl1d
\[\begin{tikzcd}
	{\cC\times\cD} & {\cC\boxtimes \cD} \\
	& {\cC\boxtimes' \cD}
	\arrow[from=1-1, to=1-2]
	\arrow[from=1-1, to=2-2]
	\arrow["F", from=1-2, to=2-2]
\end{tikzcd}\]

commute. This functor is an equivalence of categories.
\end{prop}
\begin{proof} It is clear that $\Vec_{\bC}^n\boxtimes \Vec_{\bC}^m = \Vec_{\bC}^{nm}$. Every equivalence of categories $\cC \to \cC'$ induces an equivalence of categories $\cC\boxtimes \cD\to\cC'\boxtimes \cD$. Hence, since $\cD$ and $\cD$ are equivalent to $\Vec_{\bC}^n$ and $\Vec_{\bC}^m$ respectively, their Deligne tensor product exists and is equivalent to $\Vec_{\bC}^{nm}$.

Any functor making the diagram commute must send $A\boxtimes B$ to $A\boxtimes' B$. The definition of Deligne tensor produce tells us this is enough to conclude that the map is an equivalence of categories, since axiom 3 this map is always a functor, axiom 2 implies it is fully faithful, and axiom 1 implies it is essentially surjective, and hence we can apply proposition [ref].
\end{proof}

Now that we have defined the Deligne tensor product of $\bC$-linear categories equivalent to $\Vec_{\bC}^n$, we move on to defining the Deligne tensor product of fusion categories, spherical fusion categories, pre-modular categories, and modular categories.

\begin{prop} The following claims are all true.

\begin{enumerate}[(i)]
\item Let $\cC$, $\cD$ be fusion categories. On the level of objects, define a monoidal structure $\cC\boxtimes \cD$ by the formula

$$(A\boxtimes B)\otimes (A'\boxtimes B')= (A\otimes A')\boxtimes (B\otimes B').$$

Along with a natural choice of action of the tensor product on morphisms, unit $\bone_{\cC\boxtimes \cD}=\bone_{\cC}\boxtimes \bone_{\cD}$, and a natural choice of associator and unitors, this induces the structure of a monoidal category on $\cC$.

Define a right-rigid structure on $\cC\boxtimes \cD$ as follows. The dual of an object $A\boxtimes B$ is $A^*\boxtimes B^*$. Define $\ev_{A\boxtimes B}=\ev_{A}\boxtimes \ev_{B}$, $\coev_{A\boxtimes B}= \coev_{A}\boxtimes \coev_{B}$. This induces a well-defined right-rigid structure on $\cC\boxtimes \cD$.

The above definitions induce the structure of a fusion category on $\cC\boxtimes \cD$. 

\item Let $\cC,\cD$ be spherical fusion categories. The evaluation and coevaluation maps $\ev^{L}_{A\boxtimes B} = \ev^{L}_{A}\boxtimes \ev^{L}_B$ and $\coev^{L}_{A\boxtimes B}=\coev^{L}_{A}\boxtimes \coev^{L}_{B}$ induce a left-rigid structure on $\cC\boxtimes \cD$. Along with the canonial structure of a fusion category on $\cC\boxtimes \cD$, this induces the structure of a spherical fusion category on $\cC\boxtimes \cD$.

\item Let $\cC,\cD$ be pre-modular categories. The braiding map $\beta_{\cC\boxtimes \cD}=\beta_{\cC}\boxtimes \beta_{\cD}$ induces the structure of a pre-modular category on $\cC\boxtimes \cD$. The product $\cC\boxtimes \cD$ is modular if and only if $\cC,\cD$ are both modular.
\end{enumerate}
\end{prop}
\begin{proof} Given any of the above structures, all of the axioms on $\cC\boxtimes \cD$ immediately follow from their respective axioms on $\cC$ and $\cD$. Hence, the proof is an exercise is recalling definitons which we omit.
\end{proof}


\subsection{The category of $G$-graded $G$-representations}

\subsubsection{Overview}

We've talked about a lot of general theory of modular categories. It's time for us to focus on our main family of \textit{examples}. Namely, the categories $\fD(G)$ of $G$-graded $G$-representations. These categories describe discrete gauge theory based on the finite group $G$.

Before we can prove that $\fD(G)$ is a modular category, we need to endow $\fD(G)$ with the neccecary structures. In particular, we will endow $\fD(G)$ with $\bC$-linear, monoidal, braided, right-rigid, and left-rigid structures. We will need to show that all of these structures are comptabile with each other in the neccecary ways, and that $\fD(G)$ satisfies the non-degeneracy axiom. We will use this as an oppurtunity to introduce tools of general use for proving that categories satisfy the axioms of a modular category.

Additionally, we will also study two categories similar to $\fD(G)$ which will serve as extra examples to get our grip on definitions. These categories will also appear later as relevant in and of themselves. The first is $\Vec_G$, the category of $G$-graded vector spaces. It is defined as follows:

\Note{define $\Vec_G$ in terms of objects and composition.}

Our second structure of interest is $\Rep(G)$, the category of $G$-representations. It is defined as follows:

\Note{define $\Rep(G)$ in terms of objects and composition.}

We will show that both $\Vec_G$ and $\Rep(G)$ can be naturally equipped with the structures of spherical fusion categories. We then show that $\Rep(G)$ admits a braiding which turns it into a pre-modular category. This braiding is symmetric in the sense that $\beta_{B,A}\circ \beta_{A,B}=\id_{A}\otimes \id_{B}$ for all $A,B\in \cC$, and hence $\Rep(G)$ is not a modular category. The category $\Vec_G$ is shown to not admit a braiding whenever $G$ is non-abelian.


\subsubsection{Higher linear algebra}

\Note{ In this section we define the $\bC$-linear structures on $\Vec_G$, $\Rep(G)$, and $\fD(G)$. Our goal is to show that they are all equivalent to $\Vec_\bC^n$ for some $n\geq 1$.}

\Note{It seems like the best approach is through higher linear algebra. Namely, we show that if $\cC$ is abelian, $\bC$-linear, semisimple, and has finitely many isomorphism classes of simple obejcts then it must be isomorphism to $\Vec_{\bC}^n$. Its a good time to wax philosophical about higher linear algebra and 2-vector spaces. However, its not clear that this approach actually helps at all. It might be easier to immediately note that everybody is the direct sum of irriducibles, prove a Schur's lemma, and call it a day. Of course these approaches are all equivalent but its not clear what's best.}


\Note{Here's a lemma I would like to use. Since $\times$ and $\boxtimes$ look too similar, I should use $\boxplus$ to denote the Cartesian product and call it the direct sum. I should then prove this:
\begin{lem} Consider the category $\mathrm{\textbf{2Vec}}$ whose objects are $\bC$-linear categories equivalent to $\Vec_{\bC}^n$ for sme $n\geq 0$, and whose morphisms are $\bC$-linear functors. The direct sum $\boxplus$ is a biproduct in $\mathrm{\textbf{2Vec}}$
\end{lem}
\begin{proof} \Note{ do proof}
\end{proof}
}

\subsubsection{Spherical fusion structures}

\Note{ show that the categories have duals and monoidal structure. This should be pretty easy and painless. Pentagon identity should follow from the pentangon identity on $\Vec_{\bC}$.}

\subsubsection{Braiding and modularity}

\Note{Introduce braidings. Show that $\Rep(G)$ is symmetric. Show that $\Vec_G$ does not admit a braiding if $G$ is not abelian and does admit a symmetric braiding if $G$ is abelian. Show that $\fD(G)$ admits a non-degenerate braiding.}


\subsection{The modular representation}

\subsubsection{defn}

In this chapter we are going to talk about the \textit{modular representations} of modular categories. Here's the point. Let $\cC$ be a modular category. Let $\cL$ be the set of isomorphism classes of simple objects of $\cC$. We will define a group homomorphism

$$\rho_{\cC}:\SL_2(\bZ)\xrightarrow{}\Aut(\bC[\cL])$$

assocociated to $\cC$, where $\SL_2(\bZ)$ is the group of $2$-by-$2$ matrices with integer coefficients and unit determinant. The group $\SL_2(\bZ)$ is sometimes known as the \textit{modular group}, due to its connection with moduli spaces of elliptic curves. Hence, $\rho_{\cC}$ is known as the \textit{modular representation} of $\cC$.

The goal of this chapter is to introduce $\rho_{\cC}$, show it is well defined, and then prove a series of theorems related to $\rho_{\cC}$.

Before defining $\rho_{\cC}$, we recall the basic group theory of $\SL_2(\bZ)$. It is generated by the matrices

$$
s=
\begin{pmatrix}
0 & -1 \\
1 & 0 \\
\end{pmatrix},
\,\,\,\,
t=
\begin{pmatrix}
1 & 1 \\
0 & 1 \\
\end{pmatrix}.
$$

These two matrices satisfy the relations $s^2=-1$ and $(st)^3=-1$, where $1$ is used the represent the identity matrix. These relations generate $\SL_2(\bZ)$, in the sense that we have the following presentation:

\begin{prop} The following presentation is valid:

$$\SL_2(\bZ)=\Braket{s,t | s^4=1,\,\, (st)^3=s^2}.$$
\end{prop}
\begin{proof} This is a standard fact about $\SL_2(\bZ)$. Se for instance [ref]
\end{proof}

Hence, to define a homomorphism $\rho_{\cC}:\SL_2(\bZ)\to \Aut(\bC[\cL])$ it suffices to choose automorphisms $\rho_{\cC}(s)$, $\rho_{\cC}(t)$ of $\bC[\cL]$, and show that they satisfy the relations $\rho_{\cC}(s)^4=1$ and $(\rho_{\cC}(s)\rho_{\cC}(t))^3=\rho_{\cC}(s)^2$. Since $\bC[\cL]$ has a canonical basis, we can think of its automorphisms as being matrices with rows and collumns labeled by $\cL$. We define an operator $S:\bC[\cL]\to \bC[\cL]$ via the matrix coeffients

\begin{equation*}
\tikzfig{S-matrix}
\end{equation*}

We next define the matrix $T:\bC[\cL]\to\bC[\cL]$ to be the diagonal matrix with $([A],[A])$-entry $\theta_{A}$, for all $[A]\in \cL$.

As currently stated, the $S$ and $T$ matrices defined do not satify $S^4=1$ and $(ST)^3=S^2$. They only satisfy these formula up to phases in $\bC$. They still need to be normalized before we can defined $\rho_{\bC}$. The normalization factors come in terms of the \textit{Gauss sums},

$$p^{\pm}_{\cC}=\sum_{[A]\in \cL}\theta_A^{\pm 1}d_A^2.$$

We can now state the main theorem of this chapter:

\begin{thrm} Let $\cC$ be a modular category. The values $p_{\cC}^{+}$ and $p_{\cC}^{-}$ are nonzero, and the map

\begin{align*}
\rho_{\cC}:\SL_2(\bZ)&\xrightarrow{}\Aut(\bC[\cL])\\
s & \mapsto \frac{1}{\cD}\cdot S\\ 
t & \mapsto \left(p_{\cC}^{-}/p_{\cC}^+\right)^{1/6}\cdot T
\end{align*}

is a group homomorphism.
\end{thrm}

We will prove this theorem and motivate why it should be true over the course of this chapter. We will also prove key facts about the image and kernel of this representation, as well as other formulas of interest relating to twists, S-matrix entries, and Gauss sums.

\subsubsection{Torus perspective}

It's good to reflect on why MCs have $\SL_2(\bZ)$ representations assocated with them in the first place. Not only does the representation exist, but it is so fundamental to the modular category that it is chosen as the namesake. This begs the question. What's going on?

The answer has to do with topological phases on the torus.

\Note{add torus}

Every modular category $\cC$ is supposed to describe a topological order. Up to now we have only considered what happens when this topological order is applied to an infnitely large flat sheet. We have not examined what happens when this topological order is put on a space with nontrivial topology. For instance, the torus. Suppose we analyse the system of $\cC$ applied to the torus. This amounts to breaking up the torus into some microscopic lattice and applying some Hamiltonian. This Hamiltonian will have group states $V_{\text{g.s.}}^{T^2}$, which are independent of the choice of microscopic realization of $\cC$.

Suppose we start with a torus, cut it across, twist one of its legs, then glue it back together, as shown below:

\Note{ add Dehn twist picture.}

If the initial torus has some state $\ket{\psi}\in V_{\text{g.s.}}^{T^2}$ on it, then applying this procedure would give back another group state, though possibly a different one. The key phenominon is that continues transformations on physical space correspond to linear transformations on state space:

\Note{ add schematic.}

We can make this more formal as follows. We define the \textit{mapping class group} of a topological space $X$ as follows:

$$\MCG(X)=(\text{homeomorphisms $X\to X$})/(\text{continuous deformations}).$$

If two homeomorphisms can be continuously deformed from one another then they will act the same on the ground states $V_{\text{g.s.}}^{T^2}$. This is because ground states are topologically protected and hence slowly changing the diffeomorphism cannot affect them. Hence, we get a well-defined group homomorphism

$$\rho_{\cC}^{T^2}:\MCG(T^2)\xrightarrow{}\Aut\left(V_{\text{g.s.}}^{T^2}\right).$$

This homomorphism connects back to our modular representation as follows:

\begin{itemize}
\item \textbf{Claim 1:} $\MCG(T^2)\cong \SL_2(\bZ)$;

\item \textbf{Claim 2:} $V_{\text{g.s.}}^{T^2}\cong \bC[\cL]$;

\item \textbf{Claim 3:} $\rho_{\cC}^{T^2}\cong \rho_{\cC}$, passing through the identifications in claims 1 and 2.
\end{itemize}

In general, we see that associated to every modular category $\cC$ there should not only be a modular representation, but also a representation of $\MCG(\Sigma)$ for many other choices of topological space $\Sigma$. For instance, if $\Sigma=\Sigma_g$ is the $g$-holed torus then putting $\cC$ on $\Sigma_g$ we get a map

$$\rho_{\cC}^{\Sigma_g}:\MCG(\Sigma_g)\xrightarrow{}\Aut(V_{\text{g.s.}}^{\Sigma_g}).$$

\Note{: maybe say a few words about these representations. I'm sure they must have an explicit description in terms of generators and relations. A good reference (though a bit early) is this one: \cite{lyubashenko1995invariants}}

We now examine and motivate claims 1-3.

\textbf{Claim 1:} $\MCG(T^2)\cong \SL_2(\bZ)$. This claim is best seen by thinking of the torus as a a gluing diagram,

\Note{ add gluing diagram}

\Note{ add diagram with $s$ acting by rotating by 90 degrees. Clearly, $s^4=1$.}

\Note{add diagram with $t$ as a shift. $(st)^3=s^2$ can be left as an exercise.}

\Note{ writing presentation for $MCG(T^2)$, note that it is the same as $\SL_2(\bZ).$}

\textbf{Claim 2:} $V_{\text{g.s.}}^{T^2}\cong \bC[\cL]$.

\Note{explain this. Cut into cylinder, label by charge on boundary}

\textbf{Claim 3:} $\rho_{\cC}^{T^2}\cong \rho_{\cC}$.

\Note{Showing that the Dehn twist acts diagonally by $\theta_A$ is obvious. $\theta_A$ and Dehn twist are both defined as a $2\pi$ twist. For $S$ we need another argument, more subtle but not too hard. I think Simon has it.}


\Note{ Finish by saying this is something like TQFTs. TQFT = bundled collection of mapping class group representations. Link this to TQFT appendix.}

\Note{ I wrote a little extra about this for the Kapustin final, might be useful}

To demonstrate the TQFT perspective in action, we focus on the case of the genus one surface - the torus. As part of the data of every TQFT $(V_g, \rho_g, Z_{g,g'})$ there is a map

$$\rho_1:\MCG(\Sigma_1)\to \Aut(V_1).$$

We seek to understand this map in full detail, and as such gain some level of enlightenment about the behavior of topological order on the torus. On physical groups, the space $V_1$ is easy to describe. We argued before it should be $\bC[\cL]$. The mapping class group $\MCG(\Sigma_1)=\MCG(T^2)$ is also straightforward to describe. We state a classical theorem about $\MCG(T^2)$ below:

\begin{thrm}[\cite{farb2011primer}, Theorem 2.5] There is an isomorphism $\MCG(T^2)\cong \SL_2(\bZ)$, which sends the element of $\MCG(T^2)$ represented by $\pi/2$-rotation to $s$,

\Note{ add picture}

and which sends the element of $\MCG(T^2)$ represented by a Dehn twist around a handle of the torus to $t$. 
\end{thrm}
\begin{rem} For simplicity of notation, we will from now on identify $\MCG(T^2)$ with $\SL_2(\bZ)$, using the letters $s,t$ to represent both the elements of $\SL_2(\bZ)$ and their corresponding preimage in $\MCG(T^2)$.

A more elementary way of stating this theorem is that every self-homeomorphism of the torus can be deformed to a composition of Dehn twists and rotations, and that the rotation and Dehn twist satisfy the $\SL_2(\bZ)$ relations $s^4=1$ and $(st)^3=s^2$.

The Dehn twist is described as follows. \Note{describe Dehn twist in terms of theta-symbols $\theta_{A}$}.

The rotation is described as follows.\Note{describe the rotation in terms of the $S$-matrix.}



\end{rem}

Now that we have described $\MCG(T^2)=\SL_2(\bZ)$ and $V_1=\bC[\cL]$, we are tasked with understanding $\rho_1:\SL_2(\bZ)\to \bC[\cL]$. That is, we must describe the maps $\rho_1(s),\rho_1(t)\in \Aut(\bC[\cL])$.


]

\subsubsection{Bruguieres's modularity theorem and the Verlinde formula}

In this section we prove Bruguieres's modularity theorem. This theorem asserts that, given a pre-modular category $\cC$, the $S$-matrix $S$ is invertible if and only if $\cC$ is modular. Historically, this theorem is backwards. The original definition of modular category included that the $S$-matrix should be invertible. This was the only definition of modular category, until Bruguieres proved in [ref] that the invertability of the $S$-matrix is equivalent to $\cC$ having the non-degenerate braiding property that if

\begin{equation*}
\tikzfig{non-degeneracy}
\end{equation*}

for all $B\in \cC$ then $A\cong \bone$. We are thus stating a historically incorrect definition of modular category, and Bruguieres's modularity theorem tells us that this is equivalent to the original definition. The proof of the modularity theorem relies on the \textit{Verlinde algebra} of $\cC$. This algebra will be of use for us in proving other theorems in the future, in particular the Verlinde formula in section [ref].

We define an \textit{algebra} over $\bC$ to a vector space $V$ paired with a bilinear map $\cdot: V\times V\to V$ called multiplication, such that multiplication is associative and has a unit. An algebra is called \textit{commutative} if its multiplication is commutative.

We define the Verlinde algebra $K_{\bC}(\cC)$ of $\cC$ as follows:

$$K_{\bC}(\cC)=
\left\{
\bC[\cL] \text{ with algebra structure }
\ket{[A]}\cdot \ket{[B]}=\sum_{[C]\in \cL}N^{A,B}_{C}\ket{[C]}
\right\}.$$

We additionally define the function algebra

$$
\cfunc=
\left\{
\bC[\cL] \text{ with algebra structure }
\left(\sum_{[A]\in \cL}c_A \ket{[A]}\right)\cdot \left(\sum_{[A]\in \cL}c_A' \ket{[A]}\right)= \sum_{[A]\in \cL}c_A c_A' \ket{[A]}
\right\}.
$$

\begin{lem} Both $K_{\bC}(\cC)$ and $\cfunc$ are commutative algebras.
\end{lem}
\begin{proof} The fact that $K_{\bC}(\cC)$ is associative follows from the associativity of the tensor product. Its unit is $\ket{[\bone]}$. It is commutative because $\cC$ is braided. The fact that $\cfunc$ is a commutative algebra is a standard exercise in algebra. Its unit is $\sum_{[A]\in \cL}\ket{[A]}$.
\end{proof}


We now state and prove the core theorem which underlies the core properties of the $S$ matrix:

\begin{prop} The map

\begin{align*}
\cS:K_{\bC}(\cC)&\xrightarrow{} \cfunc\\
\ket{[A]}&\mapsto \sum_{[B]\in \cL}\frac{1}{d_B}S_{B,A}\ket{[B]}
\end{align*}

is a morphism of algebras.
\end{prop}
\begin{proof} Since it was defined on a basis, $\cS$ is clearly a linear map. We now verify that $\cS$ preserves multiplication. In the below computation, we identity endomorphisms of simple objects with the unique scalar they are times the identity. We let $A,B,D$ be simple objects.

\begin{equation*}
\tikzfig{verlinde-algebra-proof}
\end{equation*}

Note our key use of the fact that

\begin{equation*}
\tikzfig{circles-split}
\end{equation*}

which follows from the facts that $\id_{B\oplus C}$ can be decomposed as projection onto $B$ plus projection onto $C$, and composition is bilinear. We now conclude that

\begin{align*}
\cS\left(\ket{[A]}\right)\cdot \cS\left(\ket{[B]}\right) &= \sum_{[D]\in\cL} \left(\frac{1}{d_D}S_{D,A}\right)\left(\frac{1}{d_D}S_{D,B}\right) \ket{[D]}\\
&= \sum_{[D]\in \cL} \left(\sum_{[C]\in \cL}N^{A,B}_{C} \left(\frac{1}{d_D}S_{D,C}\right)\right)\ket{[D]}\\
&= \cS\left(\ket{[A]}\cdot \ket{[B]}\right)
\end{align*}

as desired.
\end{proof}


By Proposition [ref], we have constructed a map of algebras $\cS:K_{\bC}(\cC)\to \cfunc$. As a map of vector spacces, $\cS$ is equal to the $S$-matrix up to a rescaling of rows by nonzero factors. Hence, it is clear that the $S$-matrix is invertible if and only if the algebra map $\cS$ is invertible. We now use special properties of the map $\cS$ to prove the main theorem of the section:

\begin{thrm}[Bruguieres's modularity theorem] Let $\cC$ be a pre-modular category. The braiding on $\cC$ satisfies the non-degenerate braiding axiom if and only if the $S$-matrix is invertible.
\end{thrm}
\begin{proof} We observe that $\cC$ has a degenerate braiding if and only if there exists some $A\not\cong \bone$ such that

\begin{equation*}
\tikzfig{degenerate-braiding}
\end{equation*}

for all $D\in \cD$. If such an element $A$ exists, then clearly $\cS(\ket{[A]})=d_{A}\cS(\ket{[1]})$. Hence, two linearly independent vectors map to linearly dependent vectors and thus $\cS$ is not invertible. Thus, the invertibilty of the $\cS$-matrix implies that the braiding is non-degenerate.

We now prove the converse, and hence we suppose that $\cC$ has nondegenerate braiding. The proof is in two main steps.  First, we prove that $\ket{[\bone]}$ is in the image of $\cS$. Then, we use the fact that $\ket{[\bone]}$ is in the image of $\cS$ to construct the rest of the image, which proves that $\cS$ is surjective hence invertible.

\textbf{Part 1: $\ket{[\bone]}$ is in the image of $\cS$.} Since $\cC$ has nondegenerate braiding, for all simple objects $A\not\cong \bone$ there exists some simple object $\tilde{A}$ such that

\begin{equation*}
\tikzfig{braiding-non-degeneracy}
\end{equation*}

Thus, the vector $\cS(\ket{[\tilde{A}]})-\frac{S_{A,\tilde{A}}}{d_{A}}\cS(\ket{[\bone]})$ has a cooficient zero of for $\ket{[A]}$ but a non-zero cooficient for $\ket{[\bone]}$. Thus, using the product structure on $\cfunc$, we find that the vector

$$\prod_{\substack{[A]\in \cL \\ A\not\cong \bone}}\left(\cS(\ket{[\tilde{A}]})-\frac{S_{A,\tilde{A}}}{d_{A}}\cS(\ket{[\bone]})\right)$$

has a cooficient of zero for all $\ket{[A]}$, $A\not\cong \bone$, but a non-zero cooficient of $\ket{[\bone]}$. Hence, it is a scalar multiple of $\ket{[\bone]}$. Since $\cS$ is a morphism of algebras, it is in the image of $\cS$. Hence, $\ket{[\bone]}$ is in the image of $\cS$.

This completes the first part of the proof. We now use the fact that $\ket{[\bone]}$ is in the image of $\cS$ to construct the rest of the vectors.

\textbf{Part 2: $\cS$ is surjective.} Let $\omega = \sum_{[A]\in \cL}\omega_{A}\ket{[A]}\in K_{\bC}(\cC)$ be a vector such that $\cS (\omega)= \ket{[\bone]}$, which exists by part 1 of the proof. We now compute the quantity

\begin{equation*}
\tikzfig{null-vector-proof}
\end{equation*}

two ways, for all simple objects $X,Y\in \cC$. The first way folllows by expanding $X\otimes Y$ as a direct sum and using the fact that $\cS(\omega)=\ket{[\bone]}$:

\begin{equation*}
\tikzfig{null-vector-one}
\end{equation*}

In our second way of computing $h_{X,Y}$, we relate the string diagram trace to $S$-matrix values:

\begin{equation*}
\tikzfig{null-vector-two}
\end{equation*}

and thus, combining our two computations, we find

$$h_{X,Y}=\sum_{[A]\in \cL}\frac{\omega_{A}}{d_A}S_{X,A}S_{Y,A}=
\begin{cases}
1 & X\cong Y^*\\
0 & \text{otherwise}.
\end{cases}$$

We now define the vector

$$\omega^{(X)}=\sum_{[A]\in \cL}\frac{\omega_{A}}{d_A}S_{A,X^*}\ket{[A]}$$

for all simple objects $X\in \cC$. We compute

\begin{align*}
\cS(\omega^{(X)}) &= \sum_{[A]\in\cL}\frac{\omega_{A}}{d_A}S_{A,X^*}\left(\sum_{[Y]\in \cL}\frac{1}{d_Y}S_{A,Y}\ket{[Y]}\right)\\
&=\sum_{[Y]\in \cL}\frac{1}{d_Y}\left(\sum_{[A]\in\cL}\frac{\omega_{A}}{d_A}S_{A,X^*}S_{A,Y}\right)\ket{[Y]}\\
&= \sum_{[Y]\in \cL}\frac{h_{X^*,Y}}{d_Y}\ket{[Y]}=\frac{1}{d_X}\ket{[X]}.
\end{align*}

Hence $\ket{[X]}$ is in the image of $\cS$ for $[X]\in \cL$, as desired.

\end{proof}


\subsubsection{Verlinde formula}

In this section, we prove the \textit{Verlinde formula}. This formula was first conjectured by Verlinde \cite{verlinde1988fusion}, and proven the following year by Moore-Seiberg \cite{moore1989classical}. There are now many Verlinde-type formulas. Most imporantly, there is one for vertex operator algebras \cite{huang2008vertex} and one in algebraic geometry \cite{faltings1994proof}. With proposition [ref] in hand, the proof is very quick:

\begin{thrm}[Verlinde formula] Let $\cC$ be a modular category.

\begin{enumerate}[(i)]
\item For all simple objects $A,B,C\in \cC$,

$$N^{A,B}_C=\sum_{[E]\in \cL}\frac{S_{A,E}S_{B,E}\left(S^{-1}\right)_{C,E}}{d_E}$$

where $(S^{-1})_{C,E}$ denotes the $(C,E)$-coefficeint of the inverse of the $S$ matrix.

\item For all simple objects $A\in \cC$, the matrix

$$D^A=S N^{A}S^{-1}$$

is diagonal with $([B],[B])$-entry $S_{A,B}/d_B$, where $N^{A}=(N^{A,B}_C)_{([B],[C])\in \cL^2}$ is the fusion matrix of $A$.
\end{enumerate}
\end{thrm}
\begin{proof} We begin by proving part $(ii)$. The main observation of the proof is that that the operator $N^{A}:\bC[\cL]\to\bC[\cL]$ is exactly left multiplication by $A$ in $K_{\bC}(\cC)$. Proposition [ref] says that $\cS$ is a morphism of algebras, and hence we can commute $N^{A}$ past $\cS$, and turn it into multiplication by $A$ in $\cfunc$. Hence, using the appropriate multiplication in the appropriate algebra, we find

\begin{align*}
(\cS N^{A}\cS^{-1}) \ket{[B]} & = \cS \left(\ket{[A]} \cdot_{K_{\bC}(\cC)} \cS^{-1}(\ket{[B]})\right)\\
& = \cS(\ket{A}) \cdot_{\cfunc} \ket{[B]}\\
& = \frac{S_{A,B}}{d_B}\ket{[B]}.
\end{align*}

Thus, $\cS N^{A}\cS^{-1}$ is diagonal with $([B],[B])$ entry $S_{A,B}/d_B$. Scaling rows of $\cS$ does not change the effect of diagonalizition. Hence, we conclude that $S N^{A}S^{-1}$ is diagonal as well, with the same entries, and thus our proof of $(ii)$ is complete.

We now move on to proving part $(i)$. Expanding the formula $N^{A}=S^{-1}D^{A}S$, we find

\begin{align*}
N^{A}\ket{[B]} &= S^{-1}D^{A}S \ket{[B]}\\
&=S^{-1}\left(\sum_{[E]\in\cL}\frac{S_{A,E} S_{B,E}}{d_E}\ket{[E]}\right)\\
&=\sum_{[E]\in\cL}\frac{S_{A,E} S_{B,E}}{d_E}\left(\sum_{[C]\in \cL}(S^{-1})_{C,E}\ket{[C]}\right).\\
\end{align*}

Comparing coeffients with the definition of $N^{A}$, we conclude the result.

\end{proof}


\subsubsection{Proof of modularity}

In this section we prove that the $S$-matrix and $T$-matrix indeed give a representation of the modular group. That is, we will prove Theorem [ref]. At its heart, the fact that the modular representation of modular category is a homomorphisms comes down to proving a series of relations between the coeffients of the $S$-matrix and the coeffients of the $T$-matrix. That is, we are proving a series of relations between braiding and twisting. The general method is to take traces of certain diagrams, and then compute those traces in two ways. One way will involve more twists and the other will involve more braiding. This will give some algebraic relation, and choosing the right diagrams we will get enough algebraic relations to deduce Theorem [ref].

We begin with the most fundamental relationship between $S$-matrix and $T$-matrix entries:


\begin{lem} Let $\cC$ be a pre-modular category. We have that

$$S_{A,B}=\theta_A^{-1}\theta_{B}^{-1}\sum_{[C]\in \cL}N^{A,B}_{C}\theta_{C}d_C$$

and

$$S_{A^*,B}=\theta_A\theta_{B}\sum_{[C]\in \cL}N^{A,B}_{C}\theta^{-1}_{C}d_C.$$
\end{lem}
\begin{proof} By Proposition [ref], we have $\beta_{B,A}\circ \beta_{A,B}=(\theta_{A}^{-1}\otimes \theta_{B}^{-1}) \circ \theta_{A\otimes B }$. Taking the trace of this formula we get

$$S_{A,B}=\theta_{A}^{-1}\theta_{B}^{-1}\tr(\theta_{A\otimes B}).$$

Now, since $\theta$ is a natural transformation it splits over direct sums. Traces split over direct sums as well by proposition [ref] and hence $\tr(\theta_{A\otimes B})=\sum_{[C]\in \cL}N^{A,B}_{C}\theta_{C}d_C$. This concludes the proof of the first formula.

For the second formula, we observe that replacing overcrossings with undercrossings in the definition of $S_{A,B}$ has the effect of taking the dual of one of elements, replacing it with $S_{A^*,B}$. Hence $\tr(\beta_{A,B}^{-1}\circ \beta_{B,A}^{-1})=S_{A^*,B}$. Thus, taking the trace of the formula $\beta_{A,B}^{-1}\circ \beta_{B,A}^{-1}=\theta^{-1}_{A\otimes B }\circ (\theta_{A}\otimes \theta_{B})$ yields the desired result.
\end{proof}

Before continuing to our proof of theorem [ref] we observe a key lemma:

\begin{lem} Let $\cC$ be a pre-modular category. Let $A\in \cC$ be a (possibly non-simple) object. We have that

\begin{equation*}
\tikzfig{kirby-color-lemma}
\end{equation*}

and

\begin{equation*}
\tikzfig{kirby-color-second}
\end{equation*}

\end{lem}
\begin{proof} We only prove the first formula - the second follows by a formally dual argument. We restrict to the case that $A$ is simple. Seeing as both sides are linear with respect to direct sums, the case that $A$ is simple will immedaitely imply the general case. Since $A$ is simple, it suffices to prove that the traces of both sides are equal. The trace on the left hand side has the effect of replacing the diagram with $S_{A,B}$. Hence, we compute as follows using Lemma [ref] and the fact that $\sum_{[C]\in \cL}N^{A,B}_C d_B=d_{A}d_C$ from proposition [ref]:

\begin{align*}
\sum_{[B]\in \cL}d_{B}\theta_{B}S_{A,B}&= \sum_{[B]\in \cL}d_{B}\theta_{B}\left(\theta_A^{-1}\theta_{B}^{-1}\sum_{[C]\in \cL}N^{A,B}_{C}\theta_{C}d_C\right)\\
&= \theta_A^{-1}\sum_{[C]\in \cL}\theta_{C}d_C\left(\sum_{[B]\in \cL}N^{A,B}_{C}d_{B}\right)\\
&= \theta_A^{-1}d_A\sum_{[C]\in \cL}\theta_{C}d_C^2=p_{\cC}^{+}\theta_{A}^{-1}d_A.
\end{align*}

This result is exact the trace of the right hand sides of the lemma. Hence, the proof is complete.
\end{proof}


We now give the heart of the proof of theorem [ref]:

\begin{thrm} Let $\cC$ be a pre-modular category. Define the charge conjugation operator $\charge: \bC[\cL]\to \bC[\cL]$ to the matrix with $([A],[A^*])$ coefficient $1$ for all $[A]\in \cL$, and all other coefficients zero.

\begin{enumerate}[(i)]
\item $\charge S = S \charge$, $\charge T = T \charge$, and $\charge ^2 =1$;
\item $(ST)^3=p^{+}_{\cC}S^2$;
\item $(S T^{-1})^3 = p^{-}_{\cC}S^2 \charge$.
\end{enumerate}

If $\cC$ is modular, then

\begin{enumerate}[(iv)]
\item $S^2= p^{+}_{\cC}p^{-}_{\cC}\charge$.
\end{enumerate}

\end{thrm}
\begin{proof} Part $(i)$ follows from the fact that $S_{A^*,B^*}=S_{A,B}$, $\theta_{A^*}=\theta_A$, and $A^{**}\cong A$. Parts $(ii)$ and $(iii)$ have formally dual proofs, which arise from replacing $\theta$ with $\theta^{-1}$ at every opportunity. Part $(iv)$ follows algebraically from combining formulas $(ii)$ and $(iii)$ whenever $S$ is invertible, which is always the case when $\cC$ is modular by theorem [ref].

Hence, it suffices to prove part $(ii)$. The proof comes from comuting the quantity

\begin{equation*}
\tikzfig{modularity-proof}
\end{equation*}

two ways.

In the first way of computing $h_{X,Y}$, we use lemma [ref]. We find the following:

\begin{equation*}
\tikzfig{modularity-computation-1}
\end{equation*}

In our second way of computing $h_{X,Y}$, we use computation of the trace of two lines through a loop in the proof of Theorem [ref]. We find this way that

$$h_{X,Y}=\sum_{[A]\in \cL}\theta_{A}S_{X,A}S_{Y,A}.$$

Thus, we find that $\sum_{[A]\in \cL}\theta_{A}S_{X,A}S_{Y,A}=p_{\cC}^{+}\theta_{X}\theta_{Y}S_{X^*,Y}$. Thinking of these quantites as the $([X],[Y])$ entries in operators $\bC[\cL]\to\bC[\cL]$, we get the equation

$$STS=p_{\cC}^+ T S T \charge.$$

\Note{ This formula is WRONG. It should be 
$$STS=p_{\cC}^+ T^{-1} S T^{-1}.$$
From this we get
$$(ST)^3 = p_{\cC}^{+}S^2$$
as desired. I'm not sure where I went wrong, but something is off in here.
}
\end{proof}


We now know that the $S$ and $T$ matrices give a modular representation \textit{up to phase}! We still need to work out the details of the phases. A key part is the following computation:

\begin{cor} Let $\cC$ be a modular category. The quantities $p_{\cC}^{+}$ and $p_{\cC}^{-}$ are nonzero, and

$$p_{\cC}^+p_{\cC}^{-}=\cD^2.$$

\end{cor}
\begin{proof}The values $p_{\cC}^{+}$ and $p_{\cC}^{-}$ must be nonzero because $S$ is invertible and $S^2=p_{\cC}^{+}p_{\cC}^{-}\charge$. The formula $S^2=p_{\cC}^{+}p_{\cC}^{-}\charge$, when expanded, says that

$$
\sum_{[C]\in \cL}S_{C,A}S_{C,B}=
\begin{cases}
p_{\cC}^{+}p_{\cC}^- & A\cong B^*\\
0 & \text{otherwise}.
\end{cases}
$$

Applying this formula to $A=B=\bone$, we find

$$\sum_{[C]\in \cL}S_{C,\bone}S_{C,\bone}=\sum_{[C]\in\cL}d_{C}^2=p_{\cC}^{+}p_{\cC}^-$$

as desired.
\end{proof}

Now, it is clear that we can normalize the representation appropriately and conclude theorem [ref], as desired.

\subsubsection{Vafa's theorem, unitarity of $S$-matrix, and the Chiral central charge}
\label{vafa-theorem-unitarity-chiral-central-charge}

In this section we discuss some finer points of the structure of the modular representation. In particular, we will prove that modular representation of every modular category is \textit{unitary}. That is, the $S$ and $T$ matrices are both unitary operators on $\bC[\cL]$ when it is endowed with its canonical inner product coming from its basis

We begin with the matrix $T$. For a diagonal matrix to be unitary, it is neccecary and sufficient for its diagonal entries to have absolute value $1$. We will prove something even stronger: that all of the entries are roots of unity! We recall that a number $z\in \cC$ is called a root of unity if $z^n=1$ for some integer $n\geq 1$. We begin with a key topological lemma which will underscore our proof:

\begin{lem}[Lantern identity] Let $\cC$ be a pre-modular category. Let $A,B,C\in \cC$ be objects. As maps $A\otimes B\otimes C \to A\otimes B \otimes C$, we have the identity

$$\theta_{A\otimes B}\circ \theta_{A\otimes C}\circ \theta_{B\otimes C} = \theta_{A\otimes B\otimes C}\circ (\theta_{A}\otimes \theta_{B}\otimes \theta_{C}).$$
\end{lem}
\begin{proof} In the language of string diagrams, this formula becomes

\Note{ add diagram.}

It is a matter of elementary manipulations to convince one's self that these two diagrams are equal. An alternate algebraic approach is to expand the relation both sides using the formula $\theta_{X\otimes Y}=(\beta_{Y,X}\circ \beta_{X,Y})\circ (\theta_{X}\otimes \theta_{Y})$, cancel $\theta_{A}^2\otimes \theta_{B}^2\otimes \theta_{C}^2$, and compare the resulting braids using the hexagon and naturality.
\end{proof}

We state one more linear-algebraic lemma neccecary for the proof:

\begin{lem} Let $n\geq 1$ be an integer and let $M:\bR^n\to \bR^n$ be a linear operator whose $(a,b)$ entry is $M_{a,b}$. Suppose that all off diagonal entries of $M$ are positive, and that $\sum_{b=1}^n M_{a,b}<0$ for all $1\leq a\leq n$. Then, $M$ is invertible.
\end{lem}
\begin{proof}\Note{ This follows from the Gershgorin circle theorem, but that's way to general for our purposes. Find a nice proof which is appropriate for the situation.}
\end{proof}

\begin{thrm}[Vafa] Let $\cC$ be a pre-modular category. The values $\theta_{A}$ are roots of unity for all simple object $A\in \cC$.
\end{thrm}
\begin{rem} This theorem is interesting in part due to its history. \Note{ add history.}
\end{rem}
\begin{proof} To begin, we choose a simple obejct $A$. We observe that every endomorphism of $A\otimes A^*\otimes A$ induces a linear map

$$\Hom(A,A\otimes A^*\otimes A) \xrightarrow{} \Hom(A,A\otimes A^*\otimes A)$$

by postcomposition. We now consider the lantern identity (proposition [ref]) on the stands $A$, $A^*$, $A$,

$$\theta_{A\otimes A^*}\circ \theta_{A\otimes A}\circ \theta_{A^*\otimes A} = \theta_{A\otimes A^*\otimes A}\circ (\theta_{A}\otimes \theta_{A^*}\otimes \theta_{A}).$$

viewed as an equality of linear operators on $\Hom(A,A\otimes A^*\otimes A)$. We compute the determinant of both sides. We first compute the determinant of $\theta_{A\otimes A^*}$. The eigenvalues of the operator $\theta_{A\otimes A^*}$ are the twists $\theta_{B}$. The dimension of the $\theta_{B}$ eigenspace is exactly the dimension of the subspace of $\Hom(A,A\otimes A^*\otimes A)$ in which $A\otimes A^*$ fuse to $B$. The dimension of this space is $N^{A,A^*}_{B}N^{B,A}_{A}=(N^{A,A^*}_{B})^2$. The determinant is equal to the product of the eigenvalues counted with multiplicity, and hence

$$\det \theta_{A\otimes A^*}= \prod_{[B]\in \cL}\theta_{B}^{\left(N^{A,A^*}_{B}\right)^2}.$$

Continuing this way for all the other twists and plugging them into the lantern identity we get

$$\prod_{B}\theta_{B}^{2\left(N^{A,A^*}_{B}\right)^2+\left(N^{A,A}_{B}\right)^2}=\theta_{A}^{4\cdot \dim \Hom(A,A\otimes A^*\otimes A)}.$$

We now define the coefficients

$$
M_{A,B}=
\begin{cases}
2\left(N^{A,A^*}_{B}\right)^2+\left(N^{A,A}_{B}\right)^2 & A\not\cong B\\
2\left(N^{A,A^*}_{B}\right)^2+\left(N^{A,A}_{B}\right)^2 - 4\cdot \dim \Hom(A,A\otimes A^*\otimes A) & A\cong B.
\end{cases}
$$

We observe that all of the off-diagonal entries of $M_{A,B}$ are positive and

\begin{align*}
\sum_{[B]\in \cL}M_{A,B} &=\sum_{[B]\in \cL}2\left(N^{A,A^*}_{B}\right)^2+\left(N^{A,A}_{B}\right)^2 - 4\cdot \dim \Hom(A,A\otimes A^*\otimes A)\\
&=  -\dim \Hom(A,A\otimes A^*\otimes A)<0.
\end{align*}

Thus, by lemma [ref], we conclude that the matrix $M=(M_{A,B})_{([A],[B])\in \cL^2}: \bC[\cL]\to\bC[\cL]$ is invertible. Let $\tilde{M}$ be the adjugate of $M$. That is, an integer valued matrix such that $M\cdot \tilde{M}=\tilde{M}\cdot M = n$ where $n=\det M$. We find for all simple objects $A$ that

$$\theta_{A}^n=\prod_{[C]\in \cL}\left(\prod_{[B]\in \cL}\theta_{B}^{M_{C,B}}\right)^{\tilde{M}_{A,C}}=1,$$

and hence $\theta_{A}$ is a root of unity as desired.

\end{proof}


We immediately get several corollaries from this theorem. The first is that braiding enough times gets you back where you started:

\begin{cor} Let $\cC$ be a pre-modular category. There exists an integer $n\geq 1$ such that

$$(\beta_{B,A}\circ \beta_{A,B})^{n}=\id_{A\otimes B}$$

for all $A,B\in \cC$

\end{cor}
\begin{proof} Choose $n\geq 1$ so that $\theta_{C}^n=1$ for all simple objects $C\in \cC$, which exists by Vafa's theorem (theorem [ref]). Seeing that $\beta_{B,A}\circ \beta_{A,B}=\theta_{A\otimes B}\circ (\theta_{A}^{-1}\otimes \theta_{B}^{-1})$, in the direc sum decomposition $A\otimes B \cong \bigoplus_{[C]\in \cL}N^{A,B}_{C}\ket{[C]}$ the transformation $\beta_{B,A}\circ \beta_{A,B}$ acts by the scalar $\theta_{C}/(\theta_{A}\theta_{B})$ on every $\ket{[C]}$ summand. Thus, $(\beta_{B,A}\circ \beta_{A,B})^n$ acts by $(\theta_{C}/(\theta_{A}\theta_{B}))^n=1$ and thus $(\beta_{B,A}\circ \beta_{A,B})^n$ is the identity as desired.
\end{proof}

\begin{cor} Let $\cC$ be a modular category. The quantity $p_{\cC}^-/p_{\cC}^+$ is a root of unity.
\end{cor}
\begin{proof} We take determinants. From $S^2= p^{+}_{\cC}p^{-}_{\cC}\charge$ we find that $\det(S)^2=\pm p_{\cC}^+p_{\cC}^-$, since $\det \charge = \pm 1$. From the formula $(ST)^3=p^{+}_{\cC}S^2$ we find that

$$\left(p^{+}_C/\det(S)\right)^2=\det(T)^6$$

so $p^{+}_C/p^{-}_{C}=\pm \det(T)^6$. By Vafa's theorem $\det(T)$ is a root of unity. Hence, we conclude that $p^{+}_C/p^{-}_{C}$ is a root of unity as desired.
\end{proof}

We take a moment to observe that the quantity $p_{\cC}^-/p_{\cC}^+$ is of great interest to physicists. It is related to the \textit{chiral central charge} $c_{\text{top}}$ of the theory by the formula

$$p_{\cC}^-/p_{\cC}^{+}=e^{-\frac{2\pi i c_{\text{top}}}{4}}.$$

\Note{ probably say a bit more? Maybe there will be a better spot somewhere else.}

We now move on to proving that the $S$ matrix is unitary. The main technical ingredient is as follows:

\begin{prop} Let $\cC$ be a modular category. For all simple objects $A,B\in \cC$ we have that $S_{A^*,B}=\overline{S_{A,B}}$.
\end{prop}
\begin{proof} By the Verlinde formula [ref], we know that for all simple objects $A$ there exists a vector $\bold{v}_B\in \bC[\cL]$ such that

$$N^{A} \bold{v}_B= \frac{S_{A,B}}{d_B}\bold{v}_B$$

for all simple objects $A$. Namely, $\bold{v}_B$ is the $[B]$-collumn of the $S$ matrix. Let $\bold{v}_B^{*}$ be the row vector which is the Hermitian adjoint to $\bold{v}_B$. We have that

$$\bold{v}_{B}N^{A}\bold{v}_B=\frac{S_{A,B}}{d_B}|\bold{v}_B|^2.$$

Now, we observe that by Frobenius reciprocity (proposition [ref]) $(N^{A})^{\dagger}=N^{A^*}$. Hence,

\begin{align*}
\bold{v}_{B}N^{A}\bold{v}_B&=\left(N^{A^*}\bold{v}_B\right)^*\bold{v_B}\\
&=\left(\frac{S_{A^*,B}}{d_B}\bold{v}_B\right)^*\bold{v_B}\\
&=\frac{\overline{S_{A^*,B}}}{d_B}|\bold{v}_B|^2.
\end{align*}

Comparing, we get the desired result.

\end{proof}

We now get the following theorem:

\begin{thrm}[Etingof-Nikshych-Ostrik] Every matrix in the image of the modular represention $\SL_2(\bZ)\to \Aut(\bC[\cL])$ is a unitary operator on $\bC[\cL]$.
\end{thrm}
\begin{proof} We already know by theorem [ref] and corollary [ref] that the normalized $T$-matrix is unitary. It thus suffices to show that the modular $S$ matrix is unitary. From theorem [ref] we have that $\left(\frac{1}{\cD}S\right)^{-1} = (\frac{1}{\cD}S) \cdot \charge$. By proposition [ref] $(\frac{1}{\cD}S) \cdot \charge = (\frac{1}{\cD}S)^{\dagger}$ and thus $\frac{1}{\cD}S$ is unitary as desired.
\end{proof}

\subsection{Skeletonization}
\label{skeletonization}

\subsubsection{Principle}

\Note{ lots of big choices need to be made here. Do I call this the skeletonization, or do I call it something else? Do I work with multiplicity-free categories, or do I allow multiplicity? I don't know what the correct statements are or what the proofs look like so this section might be a tough one.
A good mathematical reference is \cite{davidovich2013arithmetic}.
}

\subsubsection{$F$-symbols}

\subsubsection{$R$-symbols}

\subsubsection{$\theta$-symbols}

\subsubsection{Reconstruction theorem}



\subsection{Quantum double modular categories}

\subsubsection{The Drinfeld center}

A quantum double modular category is a special type of modular category. They are particularly important because many of the constructions of topological order only deal with quantum double modular categories. For instance, there are constructions of modular categories/topological order coming from the theory of tensor networks [ref], subfactors [ref], vertex operator algebras [ref], \Note{add more sources}. All of these constructions only give quantum double modular categories. Hence, understanding quantum doubles is key to understanding how topological order work in practice.

At the heart of quantum doubles is a construction known as the \textit{Drinfeld center}. In its most basic form the Drinfeld center induces an assignment

$$\cZ: (\text{monoidal categories})\xrightarrow{}(\text{braided monoidal categories}).$$

In our context, we care about a more structured version of the Drinfeld center. It is a theorem of Muger that the Drinfeld center induces an assignment as follows:

$$\cZ: (\text{spherical fusion categories})\xrightarrow{}(\text{modular categories}).$$

This theorem is fantastic because it allows one to construct modular categories using much less data than would otherwise be neccecary. Without needing a braiding, non-degenerate or otherwise, the Drinfeld center allows one to construct a modular category. This makaes the Drinfeld center an abundant source of modular categories. We call an modular category $\cC$ a quantum double if it is of the form $\cZ(\cC_0)$ for some spherical fusion category $\cC_0$. A a major goal of this chapter is to set up and prove Muger's theorem

We now define the Drinfeld center. The Drinfeld center is a somewhat direct categorification of the usual notion of center for finite groups. If $G$ is a finite group, its center is defined as follows:

$$Z(G)=\{g\in G | gh=hg \,\, \forall h\in G\}.$$

The first guess at $\cZ(\cC)$ is thus

$$\cZ(\cC)=\{A\in \cC | A\otimes B \cong B\otimes A \,\, \forall B\in \cC\}.$$

This is almost correct, but not quite. The issue is that $\cZ(\cC)$ is not quite a braided monoidal category yet. Even though $A\otimes B\cong B\otimes A$ for all $A,B\in \cZ(\cC)$, we don't have a distinguished choice of isomorphism. A braided monoidal category requires a distinugished isomorphism $\beta_{A,B}:A\otimes B\xrightarrow{\sim} B\otimes A$. Moreover, these distinguished isomorphisms are required to satisfy the hexagon equations. Hence, we make a new definition of center which keeps track of the choice of isomorphism and enforces the hexagon equation along the way:


\begin{prop} The Drinfeld center $\cZ(\cC)$of a monoidal category $\cC$ is a braided monoidal category defined as follows:

\begin{itemize}
\item (Objects) Pairs $(A,\upbeta_{A,\--})$, where $A\in \cC$, and $\upbeta_{A,\--}$ is a natural isomorphism of monoidal natural isomorphism between the two functors $A\otimes \--$ and $\--\otimes A$ from $\cC$ to $\cC$, satisfying the additional condition that

$$\upbeta_{A,B\otimes C}=\left(\id_{B}\otimes \upbeta_{A,C}\right)\circ \left(\upbeta_{A,B}\otimes \id_C\right).$$

\item (Morphisms) Given $(A,\upbeta_{A,\--}), (B,\upbeta_{B,\--})\in \cZ(\cC)$, $\Hom_{\cZ(\cC)}((A,\upbeta_{A,\--}),(B,\upbeta_{B,\--}))$ is the subspace of morphisms $f\in \Hom_{\cC}(A,B)$ such that for all $C\in \cC$

$$\left(\id_{C}\otimes f\right)\circ \upbeta_{A,C}=\upbeta_{B,C}\circ (f\otimes \id_C).$$

\item (Tensor product) Given $(A,\upbeta_{A,\--}), (B,\upbeta_{B,\--})\in \cZ(\cC)$, we define

$$(A,\upbeta_{A,\--})\otimes (B,\upbeta_{B,\--})=\left(A\otimes B, (\upbeta_{A,\--}\otimes \id_{\cC})\circ (\id_{\cC}\otimes \upbeta_{B,\--}) \right).$$

\item (Unit) The element $(1,\rho\circ \lambda^{-1})$

\item (Braiding) We define the braiding between two elements $(A,\beta_{A,\--}), (B,\beta_{B,\--})\in \cZ(\cC)$ to be $\beta_{A,B}=\upbeta_{A,B}$.
\end{itemize}

Inheriting associativity, unitors, and composition from $\cC$, this gives $\cZ(\cC)$ the structure of a braided monoidal category.
\end{prop}
\begin{proof} Since morphisms in $\cZ(\cC)$ are a subspace of morphisms in $\cC$, commutative diagrams don't change when going from $\cC$ to $\cZ(\cC)$. Hence, the triangle and pentagon axioms for $\cZ(\cC)$ follow immediately from the triangle and pentagon axioms on $\cC$. One thing to be checked is that evaluation/co-evaluaiton satisfy the compatibility condition required to a morphism in $\cZ(\cC)$, but this is straightforward. We remark on the hexagon identities. The condition imposed on $\upbeta_{A,B\otimes C}$ given is technically incorrect. To make the parentheses work in the braiding one has to add associators, and impose the longer condition

$$\upbeta_{A,B\otimes C}=\alpha_{C,A,B}^{-1}\circ \left(\id_{B}\otimes \upbeta_{A,C}\right)\circ \alpha_{A,C,B}\circ\left(\upbeta_{A,B}\otimes \id_C\right)\circ \alpha^{-1}_{A,B,C}.$$

This condition makes the second hexagon identity tautological. Similarly, the definition of tensor product given is not strictly correct - one must add the correct associator terms, making the first hexagon identity immediate. Lastly one must verify the half-braidings defined on the tensor unit/tensor product are actually half braidings, i.e., that they satisfy the hexagon condition. These follow from straightforward computations, which we leave as exercises. This completes the proof.

\end{proof}

\subsubsection{Muger's theorem}

\Note{ through Mueger's theorem. The exposition will greatly differ based on what the proof looks like, which I haven't done before.
The standard proof people use is completeley steeped in the language of module categories. The original proof uses the tube algebra, and is ``elementary" in the sense that you don't need tube algebras but it is heineous, and I would very much like to avoid it. I'll figure out what to write hence once I've done the module category section and I've digested the proof. Maybe there's a way to de-module-ify it, but maybe not. Maybe I don't want to do that if it'll lose its essence.
}

\subsubsection{Discrete gauge theory as a quantum double and Morita equivalence}

We saw in chapter [ref] that $\Vec_G$ anad $\Rep(G)$ are both naturally spherical fusion categories. Thus, Muger's theorem tells us that $\cZ(\Vec_G)$ and $\cZ(\Rep(G))$ are both modular categories. Hence, given a finite group $G$ we have three different modular categories we can associate to it: $\fD(G)$, $\cZ(\Vec_G)$, $\cZ(\Rep(G))$. The amazing fact is that these are all the same category:

\begin{prop} Let $G$ be a finite group. There are equivalences of modular categories $\fD(G)\cong \cZ(\Vec_G)\cong \cZ(\Rep(G))$.
\end{prop}
\begin{proof}\Note{ do proof. Shouldn't be too hard.}
\end{proof}

We now make a few comments about this theorem. The first is that it proves that $\fD(G)$ is a quantum double modular category. Secondly, it gives a second proof that $\fD(G)$ has a non-degenearte braiding, using Muger's theorem. Thirdly, it demonstrates the concept of \textit{Morita equivalence}.

\Note{ introduce Morita equivalence. I know that there's some important basic facts to tell - I should include those.}

\subsubsection{Factorizability and time reversal symmetry}

Given a modular category $\cC$, we can forget the braiding on $\cC$ and only remember its structure as a spherical fusion category. Hence, Muger's theorem tells us that $\cZ(\cC)$ is canonically a modular category. It is a fantastic fact that in this case $\cZ(\cC)$ can be explicitely computed in terms of $\cC$. We describe this computation now.

\Note{ Define the time-reversal conjugate $\overline{\cC}$. Setup the map $\cC\boxtimes \overline{\cC}\xrightarrow{}\cZ(\cC)$}

\begin{prop} Let $\cC$ be a pre-modular category. The canonical map $\cC\boxtimes \overline{\cC}\xrightarrow{}\cZ(\cC)$ is an equivalence of categories if and only if $\cC$ is modular.
\end{prop}
\begin{proof}\Note{proof}
\end{proof}

This theorem is fantastic because it not only computes $\cZ(\cC)$ for every modular category $\cC$, but also it gives an equivalent definition of modularity. This gives us our third definition of modularity. Namely a pre-modular category $\cC$ is modular if and only if its braidings are all non-degenerate, or equivalently if its $S$-matrix is non-degenerate, or equivalently if it is factorizable in the above sense.

\subsubsection{Levin-Wen model}

\Note{: work though the Levin-Wen model.
I think that this model is fantastic because it shows how all of the ideas of tensor category theory can manifest themselves extremely concretelely on the level of gapped Hamiltonians. Namely, the coherence relations on the category theory side correspond exactly the the formulas needed to make terms in a Hamiltonian commute with one another. It would be nice if I could give a motivation for which the category which describes the Levin-Wen model is the Drinfeld center, though I've never seen that before.}




\subsection{Unitarity}

\subsubsection{Characterization of unitarizability}

\Note{ An early reference about unitary MTCs is \cite{kirillov1996inner}. I should read this.}

As we've mentioned beofre, the algebraic theory of topological order is not modular category theory but \textit{unitary} modular category theory because hom-spaces need Hilbert space structure to define valid quantum systems. We define unitary fusion categories and unitary modular categories in section [ref].

We've mostly ignored unitary fusion categories for the following reasons: their additional structure is a large semantic burden, and is in a way inessential. We mean this in the following sense:

\begin{enumerate}
\item Given a spherical fusion category, $\cC$ every unitary structure on $\cC$ is equivalent. That is, all the unitary fusion categories obtained by enriching $\cC$ with unitary structure are equivalent to one another as unitary fusion categories;

\item All braidings on unitary fusion categories are automattically unitary so there is no subtelty in passing to unitary modular categories;

\item A spherical fusion category admits a unitary structure if and only if its quantum dimensions are all positive.
\end{enumerate}

Still, it is worthwhile to study unitary fusion categories. They are the correct algebraic structure to describe topological order, so understanding \textit{why} points (1-3) above are true gives insight into unitarity in practice. There is also a characterization of unitary fusion categories in the skeletized perspective which is of much utility in practice. Namely, a fusion category admits a unitary structure if and only if its $F$-matrices and $R$-matrices can all be made unitary. We discuss this more and offer a proof in section [ref].

We now prove the first main result of the section:

\begin{prop} Let $\cC$ be a spherical fusion category. There is a compatible unitary structure on $\cC$ if and only if  the quanutum dimensions $d_A$ are positive for all simple objects $A\in \cC$. 
\end{prop}
\begin{proof} Suppose first that $\cC$ admits a unitary structure. Then, for all simple objects $A\in \cC$,

$$d_A=\tr(\id_{A})=\Braket{\id_{A} | \id_{A}}>0$$

because the inner product is positive definite. Conversely, suppose that all of the quantum dimensions of $\cC$ are positive. We will suppose for simplicity that $\cC$ is skeletal, which is possible by proposition [ref]. Suppose that $f:A\to B$ is a morphism between objects $A,B\in \cC$. We know that there are direct sum decompositions $A=\bigoplus_{i}A_i$ and $B=\bigoplus_{j}B_j$. Writing $f$ as an element of $\bigoplus_{i,j}\Hom(A_i,B_j)$, we find that $f$ acts by some scalar $f_{i,j}$ on each homspace $\Hom(A_i,B_j)$, where $f_{i,j}=0$ if $A\not\cong B$. We define $f^{\dagger}: B\to A$ to be the map whose $\Hom(B_j,A_i)$ is $f_{j,i}^{\dagger}=\overline{f_{i,j}}$.

We now prove that this defines a unitary structure on $\cC$. Given any two morphism $f:A\to B$, we compute

$$\Braket{f|f}=\tr(f^{\dagger}\circ f)=\sum_{i,j}|f_{i,j}|^2d_{A_i}.$$

Clearly, $\Braket{f|f}=0$ if and only if $f=0$ because all of the quantum dimensions $d_{A_i}$ are positive real numbers. Hence, $\Braket{\cdot|\cdot}$ is positive definition.

It is clear that $(f^{\dagger})^{\dagger}=f$, $(f\circ g)^{\dagger}=g^{\dagger}\circ f^{\dagger}$,  and $(f\otimes g)^{\dagger}=f^{\dagger}\otimes g^{\dagger}$ for appropriate choices of $f,g$. It remains to show that $\left(\ev^R_{A}\right)^{\dagger}=\coev^{L}_A$ and $\left(\coev^{R}_A\right)^{\dagger}=\ev^L_A$.

\Note{ I don't know how to do this part of the proof. It's something to come back to.}
\end{proof}

\subsubsection{Unitary braidings}

\Note{ Show that every braiding is automatically unitary. This is the content of \cite{galindo2014braided}.}

\subsubsection{Uniqueness of unitary structure}

\Note{ Show that unitary structures are unique. This is the content of \cite{reutter2023uniqueness}.}

\subsubsection{Skeletonization of unitarity}

\Note{ Show that a fusion category admits a unitary structure if and only if its $F$ and $R$ symbols can be made unitary in some basis. There are good references for this in the papers cited above.}



\subsection{Number theory in modular categories}

\subsubsection{Techniques and first results}

\Note{
A good general reference about this stuff is \cite{gannon2019algebraic}. To-read, for sure.
There is also interesting work in \cite{morrison2012non} which gives some counterexamples.
Another reference is \cite{davidovich2013arithmetic}.}

\subsubsection{Galois conjugation}

\Note{An early reference is \cite{coste1993remarks}.
Some other papers not to ignore in this area are \cite{plavnik2023modular, buican2022galois}.}

\subsubsection{Ocneanu rigidity}

\subsubsection{Rank-finiteness theorem}

\Note{Of course there is the original paper on the topic. However, there is also the generaliation for $G$-crossed MTCs and fermionic MTCs - \cite{jones2021rank}. Should I bring this up now or later?}


\subsubsection{Schauenberg-Ng theorem}

\Note{ Go through Schauenberg-Ng's original paper and understand the proof. It seems on the face of it like it is a hard theorem. Certainly, it uses strongly the theory of the Drinfeld center as well as the modular representation. It is good to push this proof as far down as possible since it will use a lot of machinery. I think it can be boiled down to something elegant, though, if the machinery has been set up.}


\Note{ I would quite like to show that fusion categories (or at least modular categories) have finitely automorphisms of the identity functor. The proof I know follows from generalities about the universal grading group - \cite{gelaki2008nilpotent}. I wonder if this proof can be done completely without the use of grading. Then, the fact that the universal grading group can be interpreted in terms of grading can be put as an exercise in the ``further structure" section!}


\Note{ this section is going to host a lot more theorems}



$\newline$
\fbox{\parbox{\dimexpr\linewidth-2\fboxsep-2\fboxrule\relax}{

\begin{center}
\textbf{History and further reading:}\\
\end{center}

Modular categories were born from conformal field theory in the late 1980s. In a series of papers, Moore and Seiberg analysed deeply the underlying content within conformal field theory to find what essential algebraic data lied within it \cite{moore1988polynomial, moore1989classical}. They wrote out the axioms of this essential algebraic data in their subsequent notes on conformal field theory \cite{moore1990lectures}. They used the name modular category to describe their data, as suggested by Igor Frenkel. This definition was then refined and re-introduced by Turaev \cite{turaev1992modular}. The first major application of modular categories was the Reshetikhin-Turaev construction \cite{reshetikhin1991invariants, turaev2010quantum}. Prior to this result nobody had succeed in constructing topological quantum field theories. In this way, modular categories and the Reshetikhin-Turaev construction completed Witten's programme of quantizing Chern-Simons theory.

$\newline$
By the early 2000s, the proposal of topological quantum computing was attracting a lot of interest in anyons and their algebraic properties. Seeing as topological order can be described by topological quantum field theory and topological quantum field theory is essentially equivalent to modular categories, it was understood that modular categories could be used to understand topological order. This latent description of anyons in terms of modular categories was made explicit in an appendix in the seminal 2006 paper of Kitaev \cite{kitaev2006anyons}. This approach to anyons in terms of modular categories was popularized by Wang's early monograph \cite{wang2010topological}. This has since become the standard approach towards the algebraic theory of topological quantum information.
}}


$\newline\newline$

\large \textbf{Exercises}:\normalsize

\begin{enumerate}[\thesection .1.]

\item \Note{ apply Verlinde formula to group-theoretical modular categories to recover classical theorem by Burnside}

\item \Note{show that irriducible $G$-graded $G$-reps are equivalent to irriducible reps of centralizers of conjugacy classes}

\item \Note{One of the pivotal axioms automatically holds in fusion categories. Namely, the one with the twists going the two different ways. The proof is very standard - reduce to the case $A,B$ are both simple, reduce to case $A=B$, $f=\lambda\cdot \id_{A}$ and hence commutes with everything. This would be a nice exercise.}

\item \Note{ there's a notion of prime factorization of MTCs. If $\cC$ is an MTC, then it can be decomposed as a tensor product
$$\cC=\cC_1\boxtimes \cC_2...\boxtimes \cC_n,$$
where $\cC_i\neq 0$ all have no proper non-degenerate braided fusion subcategories other than $\Vec_{\bC}$. We call suchMTCs \textit{prime MTCs}. Suppose that $\cC$ has NO abelian anyons. Then, the decomposition is unique. However, if $\cC$ has abelian anyons then the decomposition can fail to be unique. In particular, the color code is equivalent to both bilayer toric code and bilayer 3-fermion (c.f. \cite{kesselring2018boundaries}). The factorization results I asserted come from the (largely ignored?) paper by Muger: \cite{muger2002structure}.
This could make for a very nice exericse. First, prove Muger's double centralizer theorem. Then, prove the factorization into prime MTCs. Then, if $\cC$ has no abelian anyon anyons, prove that the decomposition is unique. Then, prove that bilayer toric code is equivalent to bilayer 3-fermion (or maybe just leave that as a comment)! }

\end{enumerate}
