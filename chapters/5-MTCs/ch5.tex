
\section{Modular tensor categories}

\subsection{Overview}

\subsubsection{Introduction}

In this chapter we will be giving a detailed analysis of modular tensor categories, the abstract algebraic structures used to describe anyons in topological order. We recall below how this fits into the general framework of this book:

\begin{equation*}
\tikzfig{mathematical-outline-MTC}
\end{equation*}

Describing exactly what an anyon and how it can transform in terms of states and unitary operators on a Hilbert space can be difficult. However, describing abstractly how these transformations compose with one another can be done realtively simply. Hence we take a composition-first category-theoretic approach to anyons. We will make heavy use of the diagramatic language of braided monoidal categories established in Chapter [ref]. More concretely, we will think of a modular tensor category as being the category with the following data:

\begin{equation*}
\left(\substack{
\mathbf{objects:}\text{ finite collections of anyons}\\
\mathbf{morphisms:}\text{ motions/behaviors of anyons}
}\right)
\end{equation*}

Up to topological equivalence, there are not that many things that a collection of anyons can do. The most basic thing is to move anyons around each other - this is known as braiding. If the anyons touch each other then they can congeal into a single composite anyon - this is known as fusion. Even if there are no anyons in a system, however, there is always something possible. Anyons can be spontaneously created, so long as every anyon which is created comes along with its corresponding antiparticle. This is known as \textit{pair-creation}. These three operations are the fundamental structures which we will building into modular tensor categories:

\begin{enumerate}
\item braiding;
\item fusion;
\item pair-creation.
\end{enumerate}

One potentially useful way of thinking about modular tensor categories comes from analogy with classical physics. We saw in Chapter [ref] that topological classical systems have an algebraic description in terms of finite groups. Namely, quasiparticles in the system of ordered media with order space $M$ is algebraically characterized by the fundamental group $\pi_1(M,m)$ of $M$ relative to some basepoint $m\in M$. Seeing as topological order is a vast quantum generalization of classical ordered media, we can think of modular tensor categories as being a vast quantum generalization of finite groups. Every finite group induces a modular tensor category, by first constructing the Kitaev quantum double model based on that finite group and then describing its anyons. Most modular tensor categories, however, lie beyond this description.

Due to the frequency of our use of the term, we will abbreviate modular tensor category to \textit{MTC}.

\subsubsection{Using the final product}

Before developping the theory of modular tensor categories (MTCs), it is good to get a feel for what using the final product is like. An MTC itself will be a big infinite thing, with infinitely many objects and infinitely many morphisms between those objects. However, all MTCs are in a real sense \textit{finitely generated}. What we mean by this is that plugging in a finite number of objects and morphisms, the rest of the obejcts and morphisms can be recovered by the abstract rules encoded in the formalism. For example, consider the 3-strand braid group $B_3$. This group has ininitely many elements and the group operation $\cdot: B_3\times B_3\to B_3$ takes a-priori an infinite amount of data to describe. However, the presentation

$$B_3=\Braket{\sigma_1, \sigma_2 | \sigma_1 \sigma_2 \sigma_1 = \sigma_2 \sigma_1 \sigma_2}$$

gives  completely finite description of $B_3$. It is important to note, however, that this presentation would \textit{not} have been enough to recover $B_3$ if we had just been told that $B_3$ is a monoid. The fact that $B_3$ is a group implied the existence of elements $\sigma_1^{-1}$, $\sigma_{2}^{-1}$, and defined how they interacted with $\sigma_1,\sigma_2$. We see in this way that the axioms of a group not only serve as a restriction on what mathematical objects are allowed to be groups, but they also serve as a compression technique. They give the rules by which a minimal collection of data can be used to generate the rest.

In a similar way, the axioms of MTCs are not only neccecary by the fact that they restrict which categories can be MTCs, but they are also vital in the fact that they allow us to generate a full description of anyons from a minimal collection of data. For practically-minded readers, this can be viewed as the main motivation for defining MTCs at all, instead of just working with important examples.

The final challenge in going from MTCs to their description in terms of a finite set of data is in comming up with an efficient standard way of descrbing morphisms in an MTC. This is done using the Yoneda perspective, as discussed in section [ref].

In the end, the data of an MTC will look like what we have below for the toric code:

[WORK: add toric code MTC data]

Or, for a more complicated example, we can consider the data for $G=S_3$:

[WORK: add $G=S_3$ MTC data]

A large table of these descriptions are found in Chapter [ref]. We now give a worked example of how this data is used to compute observable quantities.

[WORK: add good example, computing some probability of annhilation]

\subsection{First properties}

\subsubsection{Definition}

In this section we finally define modular tensor categories (MTCs), which are the main mathematical content of this book. Seeing as lots of data is involved, we spread out the definition over a series of steps as to not overload the senses. These intermediate definitions are also important in their own right, because they will be used in other places in the algebraic theory of topological phases.


\begin{definition}[Fusion category] A fusion category is the following data:

\begin{enumerate}
\item A category $\C$;
\item The structure of a right-rigid monoidal category on $\C$;
\item The structure of a $\CC$-linear category on $\C$.
\end{enumerate}

Such that:

\begin{enumerate}
\item The tensor product functor $\otimes: \C\times \C\to \C$ induces bilinear maps hom-spaces;
\item There is an equivalence $\C \cong \Vec_\CC^n$ as $\CC$-linear categories;
\item $\End_\C(\one)\cong \CC$ as $\CC$-vector spaces
\end{enumerate}

\raggedleft\qedsymbol{}
\end{definition}

A fusion category is part of the way towards having all of the requisite structures of a modular tensor category: it has a method for fusion inherited from the tensor product, and it has half of a method for pair-creation coming from right-rigidity. The $\CC$-linearity allows us to think of hom-spaces as vector spaces, which allows us to treat hom-spaces as quantum systems. The condition (1) is a comptability between the $\CC$-linear structure and the monoidal structure. The conditions (2)-(3) are strong niceness and finiteness conditions - we will explain them in detail later. We now move one step closer to our definition of modular tensor category:

\begin{definition}[Spherical fusion category] A spherical fusion category is the following data:

\begin{enumerate}
\item A fusion category $\C$;
\item A left-rigid structure on $\C$.
\end{enumerate}

Such that:

\begin{enumerate}
\item The left-rigid and right-rigid structures on $\C$ satisfy the axioms of a pivotal structure on $\C$;
\item For every object $A\in \C$ and for every morphism $f: A \to A$, we have

[WORK: add spherical diagram.]
\end{enumerate}

\raggedleft\qedsymbol{}
\end{definition}

A spherical fusion category now has a structure for fusion, and a full structure for pair-creation. The 2nd condition is known as the \textit{spherical axiom}. We will explain this axiom in more detail later.

Adding on a braiding, we can get all of the structures of a modular tensor category. However, adding this structure still misses one key structure of being a modular tensor category. Hence, we call it \textit{pre-modular}:

\begin{definition}[Pre-modular tensor category] A pre-modular tensor category is the following data:

\begin{enumerate}
\item A spherical fusion category $\C$;
\item A braided structure on $\C$;
\end{enumerate}

Such that:

\begin{enumerate}
\item .[WORK: compatibility between braiding + $\CC$-linearity, or compatibility between braiding and rigidity?]
\end{enumerate}

\raggedleft\qedsymbol{}
\end{definition}

This definition now has all of the structure we wanted it to have: fusion, pair-creation, and braiding. The final axiom is a non-degeneracy condition. It is subtle in its interpretation, and we will explain it several different ways throughout this chapter:


\begin{definition}[Modular tensor category] A modular tensor category is a pre-modular tensor category satisfying the following condition. Let $A\in \C$ be an object. If

\begin{equation*}
\tikzfig{non-degeneracy}
\end{equation*}

for all $B\in \C$, then $A\cong \one$.

\raggedleft\qedsymbol{}
\end{definition}


\subsubsection{Anyons in MTCs}

Modular tensor categories (MTCs) are supposed to be theories of anyons in topological order. So, now that we have the definition of MTC, it is natural to ask: what do anyons mathematically correspond to, in MTCs? The answer lies within the condition in a fusion category $\C$ that there is an equivalence $\C\cong \Vec_\CC^n$ as $\CC$-linear categories. We explore the importance of this condition.

Suppose we are given an object $V=(V_1,V_2...V_n)\in \Vec_\CC^n$. For all $1\leq i\leq n$, let  $\CC_i\in \Vec_\CC^n$ denote the object which has dimension zero in every index $j\neq i$ and is equal to $\CC$ in index $i$. We observe the isomorphism

\begin{align*}
V& \cong \bigoplus_{i=1}^n (0... V_i ... 0)\\
& \cong \bigoplus_{i=1}^n (0... \CC^{\dim (V_i)}.. 0)\\
& \cong \bigoplus_{i=1}^n \dim(V_i)\cdot \CC_i
\end{align*}

where $\dim(V_i)\cdot (\CC_i)=\CC_i\oplus \CC_i...\oplus \CC_i$, $\dim(V_i)$ many times. This computation shows that any object in $\Vec_\CC^n$ can be decomposed into irriducible components $\CC_i$. These objects $\CC_i$ are in a real sense the building blocks of $\Vec_\CC^n$. They will correspond physically to anyons. More concretely, we make the following definition:

\begin{definition} A \textit{simple object} $A$ in a fusion category $\C$ is an object which has no direct sum decomposition into smaller objects. That is, $A\ncong B\oplus C$ for any non-zero objects $B,C\in \C$ where $\oplus$ denotes the biproduct in $\C$.
\end{definition}

Our physics-math dictionary is that anyon types correspond to isomorphism classes of simple objects.

We now state the basic proposition which ensures that the neccecary properties from $\Vec_\CC^n$ follow through the equivalence of categories.

\begin{proposition} Let $\C$ be a fusion category. The biproduct of any two elements in $\C$ exists. Let $\LL$ denote the set of isomorphism classes of simple objects in $\C$. The set $\LL$ is finite. Choose an object $X\in \C$. There exist unique nonnegative integers $c_{[A]}$, $[A]\in \LL$ such that 

$$X\cong \bigoplus_{[A]\in \LL}N_{[A]}\cdot A.$$
\end{proposition}
\begin{proof}.[WORK: do proof.]
\end{proof}

The set of simple objects has an alternative description, known as Schur's lemma:

\begin{proposition}[Schur's Lemma] Let $\C$ be a fusion category. An object $A\in \C$ is simple if and only if its endomorphism ring $\End(A)$ is one-dimensional. Additionally, if $A,B\in \C$ are nonisomorphic simple objects then $\Hom(A,B)=0$.
\end{proposition}
\begin{proof}.[WORK: do proof.]
\end{proof}

As an immediate application of Schur's lemma, we observe that the monoidal unit $\one$ is a simple object in every fusion category. By our physics-math dictionary, this means that $\one$ corresponds to an anyon type. This type is the \textit{vaccuum} type - empty space. The anyon $\one$ is the trivial no-anyon type.

Another application of Schur's lemma is to make a first verification that simple objects are a good choice of mathematical characterization of anyons. If $A,B$ are distinct anyon types, then there should not be any physical process which goes from one to another. There is no physical mechanism for locally turning one anyon type into another. This is captured by the formula $\Hom(A,B)=0$. Similarly, given an anyon $A$, there is no nontrivial action that can be locally performed on $A$. This comes from the fact that information is topologically protected, and thus cannot be changed by acting on a single particle - topological information processing requires global braiding between multiple particles. This is encoded in the fact that $\Hom(A,A)\cong \CC$ is one dimensional and hence consists only of trivial phase gates.

Expanding our physics-math dictionary, we say that for every anyon $A$ its \textit{antiparticle} is the dual $A^*$ which comes from right-rigidity. This gives a valid anyon type by the following computation:

\begin{proposition} Let $\C$ be a fusion category. If $A\in \C$ is a simple object, then so is $A^*$.
\end{proposition}
\begin{proof}. [WORK: do proof]
\end{proof}

An important part of of understanding simple objects in MTCs is making sense of the direct sum decompositions coming from Proposition [ref]. Let $\C$ be a fusion category with simple objects $A,B\in \C$. Consider the decomposition

$$A\otimes B \cong \bigoplus_{[C]\in \LL}N^{A,B}_{C}\cdot C$$

where $N^{A,B}_{C}\geq 0$ are nonnegative integers, and $\LL$ is the set of isomorphism classes of simple objects. The integers $\{N^{A,B}_C\}_{[C]\in \LL}$ are known as fusion coefficents, because they specify the behavior of $A$ and $B$ when they fuse.

The tensor product $\otimes$ physically corresponds to joining anyons, forming a composite anyon configuration. The object $A\otimes B$ corresponds to the configuration with one $A$-type anyon and one $B$-type anyon. The direct sum decomposition is physically interpreted as saying that when $A\otimes B$ are fused, the possible results of that fusion are all of the anyon types $[C]\in \LL$ for which $N^{A,B}_{C}\neq 0$.

[WORK: add nontrivial example from Kitaev quantum double model]

A more detailed understanding of the physical meaning of the direct sum will have to wait for later.

This concludes our basic picture of anyons in fusion categories.

\subsubsection{States in MTCs and unitarity}

It is now worth reflecting on what exactly states correspond to in MTCs. In particular, objects in MTCs are \textit{not} quantum systems. They don't have vector space structure. The spaces with vector space structure are the hom-spaces, by $\CC$-linearity. Objects will correspond to anyon configurations. States will correspond to normalized vectors in certain hom-spaces. In particular:

\begin{equation*}
\left(\substack{\text{space of states of topological order $\C$} \\ \text{on the infinite plane $\RR^2$} \\ \text{with anyon configuraiton $A_1,A_2...A _n$}}\right)
=
\left(
\substack{
\text{normalized vectors in the Hilbert space}\\
\Hom_\C(\one, A_1\otimes A_2... \otimes A_n)
}
\right)
\end{equation*}

where by ``anyon configuration $A_1,A_2...A_n$" we mean that the state has anyons present in $n$ sites, arranged left to right on a one dimensional subspace of $\RR^2$, with corresponding anyon type $A_1,A_2...A_n$. For the sake of concreteness, one can imagine that at the point $(i,0)\in \RR^2$ thee state has an anyon of type $A_i$.

The remainder of this subsection is a series of loosely-related observations about this choice of state space:

$\newline$
\textbf{Observation 1:} \textit{In the definition of an MTC hom-spaces are vector spaces and not Hilbert spaces, so this choice of physics-math correspondance is incorrect as literally written}.

To make this definition work, all of the hom-spaces of the MTC $\C$ should be equipped with Hilbert space structures. Furthermore, the natural operators we wish to perform like braiding should all be unitary with respect to these inner products. This amounts to adding a large number of compatibility conditions on the Hilbert space structures. An MTC with this choice of structure is known as a \textit{unitary} MTC. We give the formal definition below:


\begin{definition}[Unitary fusion category] A unitary fusion category is the following data:

\begin{enumerate}
\item An fusion category $\C$.
\item (Conjugation) A linear map $\dagger: \Hom(A,B)\to \Hom(B,A)$ for all $A,B\in \C$.
\end{enumerate}

Additionally, a unitary fusion category is required to satisfy the following properties:

\begin{enumerate}
\item (Unitarity) Given $f:A\to A$ an endomorphism of $A\in \C$, define


$$\tr(f)=\ev_A \circ (\id_{A^*}\otimes f)\circ \left(\ev_A\right)^{\dagger}.$$

The map $\left<\cdot|\cdot\right>:\Hom(A,B)\times \Hom(A,B)\to \CC$ defined by $\left<f|g\right>=\tr(f\circ g^{\dagger})$ is an inner product, endowing $\Hom(A,B)$ with the structure of a Hilbert space.
\item $\left(f^{\dagger}\right)^{\dagger}=f$ for all $f\in \Hom(A,B)$, $A,B\in \C$.
\item $(f\circ g)^{\dagger}=g^{\dagger}\circ f^{\dagger}$ for all $f\in \Hom(B,C)$,$g\in \Hom(A,B)$, $A,B,C\in \C$.
\item $(f\otimes g)^{\dagger}=f^{\dagger}\otimes g^{\dagger}$ for all $f\in \Hom(A,B)$,$g\in \Hom(C,D)$, $A,B,C,D\in \C$.
\item $\left(\coev_A\right)^{\dagger}\circ (f \otimes \id_{A^*})\circ \coev_A=\tr(f)$ for all $A\in \C$
\end{enumerate}

\raggedleft\qedsymbol{}
\end{definition}

Unitary fusion categories make for a pleasant object of study because the distinguished maps $(\ev_A)^{\dagger}:1\to A^{*}\otimes A$ and $(\coev_A)^{\dagger}:A\otimes A^{*}\to 1$ induce a pivotal structure:

\begin{proposition} All unitary fusion categories are spherical fusion categories. [WORK: reword this to make it an actual mathematical statemenet]
\end{proposition}
\begin{proof}.[WORK: Should be easy.]
\end{proof}

We can now define unitary MTCs. The compatibility conditions for the twist are chosen so that the definition of trace as an MTC and the definition of trace as a unitary fusion category coincide. [WORK: The definition below is outdated. It uses the twist-theoretic definition of MTCs. Should be re-done.]

\begin{definition}[Unitary Modular Tensor Category] A unitary Modular Tensor Category is the following data:

\begin{enumerate}
\item An MTC $\C$.
\item (Conjugation) A linear map $\dagger: \Hom(A,B)\to \Hom(B,A)$ for all $A,B\in \C$.
\end{enumerate}

Additionally, a unitary Modular Tensor Category is required to satisfy the following properties:

\begin{enumerate}
\item Forgetting the twist and braiding, $\left(\C,\dagger\right)$ forms a unitary fusion category.
\item $\left(\beta_{A,B}\right)^{\dagger}=\beta_{A,B}^{-1}$ for all $A,B\in \C$.
\item $\left(\theta_A\right)^{\dagger}=\left(\theta_A\right)^{-1}$ for all $A\in \C$.
\item $\left(\ev_{A}\right)^{\dagger}= \left(\id_{A^*}\otimes \theta_A^{-1}\right)\circ \beta_{A^*,A}^{-1}\circ \coev_A$ for all $A\in \C$
\item $\left(\coev_{A}\right)^{\dagger}= \ev_{A}\circ\beta_{A,A^*}\circ\left(\theta_A \otimes \id_{A^*}\right)$ for all $A\in \C$
\end{enumerate}

\raggedleft\qedsymbol{}
\end{definition}

For this reason, the correct algebraic structure to underlie the theory of topological order is not MTC, but unitary MTC. We have chosen to not emphasize this before because the difference between unitary MTCs and non-unitary MTCs is very small. [WORK: talk about uniqueness + positive q.d. criterion this will make more sense once we write the actual section about unitarity. A good thing to emphaize is that unitary MTCs don't let you use less data in your definition, and you can still do essentially everything you want to do. It's just way more cumbersome. They're all equivalent but you still have to choose, c.f. the fact that the category of vector spaces and Hilbert spaces with linear maps as morphisms are equivallent].

$\newline$
\textbf{Observation 2:} \textit{The physical space is assumed to be an infinitely large flat plane.}

The reason that the physical space is the infinitely large flat plane is that it has a unique ground state, and does not have boundaries. Finite contractible regions would have boundaries to worry about, and spaces with topology like the torus would have non-equivalent ground states. The formula $\Hom(\one,\one)\cong \CC$ implies that there is a unique ground state, whence the conclusion.

Describing anyons on spaces with non-trivial topology is a more difficult question, and requires more machinery [WORK: reference something later? Will I talk about this? Seems important.] 

$\newline$
\textbf{Observation 3:} \textit{The anyons are always assumed to be arranged in a line.}

The anyon configurations are always assumed to be linear. The main reason to do this is because it makes the mathematics much simpler. If we kept track of the positions of each of the anyons in two dimensional space it would add more pieces of data and structures to keep track of. Seeing as every anyon configuration can be pushed onto a one-dimensional space, only working with a one-dimensional configuration does not affect the generality of the answers and hence it is very much prefered.

$\newline$
\textbf{Observation 4:} \textit{The formula $\Hom_{\C}(\one, A_1\otimes A_2...\otimes A_n)$ encodes the fact that states can be specified by their history.}

A good first question to ask when seeing the Hilbert space $\Hom_\C(\one, A_1\otimes A_2... A_n)$ is \textit{why} this should describe a state with anyon configuration $A_1$... $A_n$. The answer is that states can be described their history. [WORK: give good example of making a state by specifying its history; argue why it has to be this way in general].

\subsubsection{Topological charge measurement}

When two anyons are fused together, they will form a superposition of other anyon types. Measuring the result of the fusion will collapse the answer into a specific anyon type. The outcome of this measurment is an observable quantity, which allows for the measurement of topological quantum information. In many cases this is the \textit{only} local observable quanitity. We give the formalism behind computing these probabilities now.

[WORK: do this right - I don't know it well but it shouln't be hard to learn. Don't introduce anything too general, like trace or whatnot. Just quantum dimension, which should already have been introduced in previous chapter.]

[WORK: The correct reference for this subsection is \cite{bonderson2021measuring}. The paper \cite{cong2017universal} claims to introduce the term topological charge measurement and gives a nice formal treatmenet. Clarifying the situation seems important.]
 


\subsection{The MTC toolkit}

In this section, we will introduce and prove the basic facts about the most important structures in the theory of modular tensor categories (MTCs). These facts and structures are the tools used for solving problems about the algebraic theory of anyons. 

\subsubsection{Trace}

The first structure to define in the theory of MTCs is the \textit{trace}. Let $\C$ be a spherical fusion category. Given any object $A\in \C$ and any endomorphism $f:A\to A$, we define the \textit{trace of $f$} by the following formula:

\begin{equation*}
\tikzfig{trace}
\end{equation*}

Initially, the trace is a morphism,  $\tr(f):\one\to\one$. However, we will choose to think of the trace of a morphism as a \textit{complex number}, $\tr(f)\in \CC$. This can be done because the definition of a fusion category $\End(\one)\cong \CC$. This isomorphism can be made canonical by identifying an endomorphism $g\in \End(\one)$ with the unique $\lambda\in \CC$ such that $g = \lambda \cdot \id_{\one}$.

The trace is used mainly as a tool for linearization. Morphisms and objects are hard to describe, but the trace is a complex number.

\begin{proposition}\label{trace} Let $\C$ be a spherical fusion category. For all $A,B\in \C$, $f\in \End(A)$ the following claims are all true:

\begin{enumerate}
\item $\tr: \End(A)\to \CC$ is a linear map of vector spaces,
\item $\tr(f^{*})=\tr(f)$,
\item $\tr(f\circ g)=\tr(g\circ f)$ for all $g\in \End(A)$,
\item $\tr(f\oplus g)=\tr(f)+\tr(g)$ for all $g\in \End(B)$,
\item $\tr(f\otimes g)=\tr(f)\cdot \tr(g)$ for all $g\in \End(B)$,
\item $\tr(g\circ f\circ g^{-1})$ for any isomorphism $g:A\xrightarrow{\sim}B$,
\item Trace is preserved by functors. That is, let $\C,\Dcat$ be spherical categories with traces $\tr_{\C},\tr_{\Dcat}$ respectively. Let $F:\C\to \Dcat$ be a pivotal functor. We have that $\tr_{\C}(f)=\tr_{\Dcat}(F(f))$;
\end{enumerate}

\end{proposition}
\begin{proof} We prove the claims one by one.

\begin{enumerate}
\item This follows immediately from the bilinearity of composition.

\item This is a straightforward computation.

\item Using Proposition \ref{pivotal-alternative} part (3), we find that

\begin{equation*}
\tikzfig{composition-commutes}
\end{equation*}

\item From the explicit definition of evaluation and coevaluation for $A\oplus B$ given in the proof of Proposition \label{rigidity}, we find that $\tr(f\oplus g)$ as a morphism is

$$1\to 1\oplus 1 \xrightarrow{\tr(f)\oplus \tr(g)}1\oplus 1\to 1$$

where $1\to 1\oplus 1$ is the sum of the right and left inclusion maps and $1\oplus 1\to 1$ is the sum of the right and left projection maps. Seeing as composition is linear and inclusion composed with projection is the identity, we get that this map is $\tr(f)+\tr(g)$ times the identity as desired.

\item Using Proposition \ref{pivotal-alternative}, we compute

\begin{equation*}
\tikzfig{trace-tensor}
\end{equation*}

\item To begin, we observe that $F(A^*)$ can be canonically given the structure of a dual for $F(A)$, by the maps $\coev_{F(A)}=F(\coev_A)$, $\ev_{F(A)}=F(\ev_A)$. Since $F$ is a pivotal functor, we find that

\begin{equation*}
\tikzfig{pivotal-functor}
\end{equation*}

Now, the fact that $F$ is $\CC$-linear on hom spaces and $F(\id_{1_{\C}})=\id_{\Dcat}$ implies that $\tr_{\C}(f)=F(\tr_{\C}(f))$ as complex numbers. Passing trough $F$, the desired conclusion is immediate.

\end{enumerate}

This completes the proof.
\end{proof}

With these properties in hand, we can explicitely compute the trace using a straightforward procedure:

\begin{corollary}\label{explicit-trace} Let $f:A\to A$ be an endomorphism in a fusion category $\C$. Fix a decomposition $A\cong \bigoplus_{i\in I}A_i$ of $A$ into simple objects $A_i$. Moreover, we take the decomposition such that if $A_i\cong A_j$ then $A_i=A_j$. We can decompose

$$\Hom(A,A)\cong \Hom(\bigoplus_{i\in I} A_i,\bigoplus_{i\in I}A_i)=\bigoplus_{i\in I, j\in I}\Hom(A_i,A_j).$$

Let $M$ be the matrix whose collums and rows are labeled by $I$, and whose $(i,j)$ entry is $0$ if $A_i\not\cong A_j$ and $\lambda \cdot d_{A_i}$ if $A_i=A_j$, where $\lambda\in \CC$ is the unique value such that the $\Hom(A_i,A_j)$ component of $f$ is $\lambda\cdot \id_{A_i}$. We have that

$$\tr_{\C}(f)=\tr_{\Vec}(M).$$

\end{corollary}
\begin{proof} Since $\tr_{\Vec}$ adds over direct sums, verifying this construction immediately reduces to the case that $A$ is a simple object by Proposition \ref{trace} (4). Now, linearity of $\tr_{\C}$ and the definition of quantum dimension completes the proof.
\end{proof}

\subsubsection{Duality}

Duality is baken into our definition of MTCs as a fundamental part of the structure.


The dual of an object $A$ can be identified from its fusion coeffients.

\begin{proposition} Let $\C$ be a fusion category. Let $A,B\in \C$ be simple objects. The following are equivalent:

\begin{enumerate}[(i)]
\item $B\cong A^*$;
\item $N^{B,A^*}_{\one}>0$;
\item $N^{A^*,B}_{\one}>0$;
\item $N^{A,A^*}_{\one}=N^{A^*,A}_{\one}=1$.
\end{enumerate}
\end{proposition}
\begin{proof}. [WORK: do proof.]
\end{proof}

In particular, we conclude the following:

\begin{corollary} If $\C$ is a fusion category, then $A\cong A^{**}$ for all $A\in \C$.
\end{corollary}
\begin{proof} .[WORK: do proof.]
\end{proof}

However, despite this corollary, we \textit{cannot} conclude that every fusion category admits a pivotal structure. The isomorphism $A\cong A^{**}$ may fail to form a monoidal natural transformation. It is an open problem whether or not every fusion category admits a pivotal structure, and it is furthermore an open problem whether every pivotal fusion category admits a spherical structure \cite{etingof2005fusion}.

Duality acts a controlled way on fusion coefficients:

\begin{proposition}Let $\C$ be a fusion category and let $A,B,C\in \C$ be simple objects. We have the following:

\begin{enumerate}[(i)]
\item (Anti-involution) $N^{A,B}_C=N^{B^*,A^*}_{C^*}$;
\item (Frobenius reciprocity) $N^{A,B}_C = N^{A^*,C}_B = N^{C, B^*}_{A}$.
\end{enumerate}

\end{proposition}
\begin{proof} . [WORK: do proof.]
\end{proof}


\subsubsection{Quantum dimension and Frobenius-Perron dimension}

.[WORK: include something about global quantum dimension $D$.]

Our next tool to discuss is the \textit{quantum dimension}. Given any spherical fusion category $\C$ and any object $A\in \C$, we define its quantum dimension using the following formula:

[WORK: define $d_A$.]

As usual, we identitfy $d_A$ with a complex number via the canonical isomorphism $\End(\one)\cong \CC$. The quantum dimension is clearly equal to the trace of the identity map on $A$, $d_A=\tr(\id_A)$. The first properties of quantum dimension follow from our general analysis of trace:

\begin{proposition} For every spherical fusion category $\C$ and any objects $A,B\in \C$, we have the following formulas:

\begin{enumerate}[(i)]
\item If $A\cong B$, then $d_A=d_B$;
\item $d_{A\oplus B}=d_{A}+d_B$;
\item $d_{A\otimes B}=d_{A}\cdot d_{B}$;
\item $d_{A^*}=d_A$.
\item $d_A\neq 0$.
\end{enumerate}
\end{proposition}
\begin{proof}.[WORK: do proof]
\end{proof}

The above propositions tell us that the values $d_A$ as $A$ ranges over isomorphism classes of simple objects determs all the other values of $d_A$. Moreover, proposition [ref] tells us that the quantum dimensions of simple objects determines the trace of \textit{every} endomorphism! Hence, computing $d_{A}$ for each isomorphism class $[A]\in \LL$ is an important step in analysing an MTC. The following formula and its linear-algebraic reformulaion are the primary insight in performing the computation:

\begin{proposition} Let $\C$ be a spherical fusion category.

\begin{enumerate}[(i)]
\item Let $A,B\in \C$ be simple objects. We have that

$$d_Ad_B=\sum_{[C]\in \LL}N^{A,B}_C d_C.$$

\item Let $A\in \C$ be a simple object. Define an operator

 \begin{align*}
N^{A}:\CC[\LL]&\xrightarrow{} \CC[\LL].\\
\ket{[B]}&\mapsto \sum_{[C]\in \LL} N^{A,B}_C \ket{[C]}
\end{align*}

Define $\bold{d} = \sum_{[B]\in \LL} d_{B}\ket{[B]}\in \CC[\LL]$. We have that

$$N^{A}\bold{d}=d_{A}\bold{d}.$$

\end{enumerate}

\end{proposition}
\begin{proof}.[WORK: do proof]
\end{proof}

We now make commentary about the above proposition. It tells us that $d_A$ is an eigenvalue of $N^A$. Since $N^A$ is an operator with integer coefficients, this immediately tells us that $d_A$ is the root of polynomial with integer coeffients. Namely, the characteristic polynomial of $N^A$. We can even be more precise about the nature of $d_A$:

\begin{corollary} Let $\C$ be a spherical fusion category. The quantum dimensions of all simple objects in $\C$ are real numbers.
\end{corollary}
\begin{proof}.[WORK: do proof.]
\end{proof}

The question is whether or not the quantum dimensions are \textit{positive} real numbers. We recall that we defined a unitarizable spherical fusion category to be one in which the quantum dimensions are all positive. It is at this point that this becomes relevant. In particular, if $\C$ is unitarizable then its quantum dimensions are eigenvalues of $N^A$, and their corresponding eigenvector $\bold{d}$ has positive real entries. There is a theorem about eigenalues of non-negative matrices with positive eigenvectors:

\begin{theorem}[Frobenius-Perron theorem] .[WORK: state version which suffices for our case]
\end{theorem}

We call the largest positive real eigenvalue of a matrix with non-negative entries its \textit{Frobenius-Perron eigenvalue}. The above theorem tells us the following:

\begin{corollary} Let $\C$ be a unitarizable spherical fusion category. Let $A\in \C$ be a simple object. The quantum dimension $d_A$ is equal to the Frobenius-Perron eigenvalue of $N^A$.
\end{corollary}
\begin{proof}.[WORK: do proof]
\end{proof}

In this chapter we will mostly work with generic spherical fusion categories with no conditions on unitarizability. Hence, it is useful to make the following definition. Let $A\in\C$ be a simple object in a spherical fusion category. We define

$$\FPdim(A)=(\text{Frobenius-Perron eigenvalue of $N^A$}).$$

When $\C$ is unitarizable, $\FPdim(A)=d_A$. Many formulas about quantum dimension in the unitary world apply to the Frobenius-Perron dimension in the non-unitary world. An interesting observation is that the definition of quantum dimension strongly uses the spherical structure on $\C$. However, the Frobenius-Perron dimension only uses the fusion coeffients, and those are well-defined in any fusion category. Hence, the Frobenius-Perron dimension also derives utility from being applicable in a broader set of situations than the quantum dimension.

We now give an alternate interpretation of the Frobenius-Perron dimension in terms of growth in tensor powers. This sort of alternate perspective of dimension applies to several types of objects outside the scope of tensor category theory \cite{coulembier2024growth}.

\begin{proposition} Let $\C$ be a fusion category, and let $A\in \C$ be a simple object.

\begin{enumerate}[(i)]
\item $\FPdim(A)=\lim_{n\to\infty}\dim(\Hom(A^{\otimes n},A^{\otimes n}))^{1/(2n)}$
\item $\FPdim(A)=\lim_{n\to\infty}\dim(\Hom(\one,A^{\otimes n}))^{1/n}$
\item $\,$

$$\FPdim(A)=\lim_{n\to\infty}(\text{\# of simple objects in the direct sum decomposition of $A^{\otimes n}$})^{1/n}.$$
\end{enumerate}
\end{proposition}
\begin{proof}.[WORK: do proof]
\end{proof}

This proposition can be interpreted as saying that the simple object $A$ has $\FPdim(A)$ internal degrees of freedom ``on average". Elements of the vector space $\Hom(\one,A^{\otimes n})$ correspond to states in the system with $n$ anyons of type $A$ arranged in a line. If the internal configuration space of each anyon was $\FPdim(A)$-dimensional, then the overall dimension would be $\FPdim(A)^n$. By Proposition [ref], $\FPdim(A)^n$ is approximately $\Hom(\one,A^{\otimes n})$ for large $n$. Hence, each anyon has approximately $\FPdim(A)$ internal degrees of freedom. Of course, $\FPdim(A)$ has no reason to be an integer! In the Fibonacci theory $\FPdim(\tau)=\phi=1.61...$. Frobenius-Perron dimension just gives an average amount for large values.

[WORK: re-do this explanation way better + add diagram for it.]

\subsubsection{Twist}

In this section we will discuss \textit{twists}. The twist is a subtle concept, which we have not explicitely mentioned up to now. The idea is that anyons can \textit{rotate in place}. Since the space of endomorphisms of an anyon is one dimensional, this rotation must act by a phase. This phase is physically relevant, and can be measured in experiment.

For example, consider the $Y$-type on the toric code. It consists of the fusion of an $X$-type anyon and a $Z$-type anyon, as shown below:

[WORK: add figure of Y as a thick X and Z together.]

Twisting $Y$ in place will correspond to twisting $X$ and $Z$ around each other. This twsiting thus results in a phase of $-1$. In general, we can imagine anyons as having some thickness to them. Anyons are not localized at points - they are localized at small regions. Twisting this region all the way around can be viewed visually as

[WORK: twisted anyon.]

This is the twist. One way of working with the twist is to work with thickened diagrams, where strings are replaced with ribbons. While popular in some parts of the literature, we will continue to work with string diagrams for simplicity. The key observation is that the twist can be constructed using string diagramatic structures we already have as follows:

[WORK: twist as a swirl diagram.]

Hence, letting $\C$ be a pre-modular fusion category, we \textit{define} the twist $\theta_{A}$ of an object $A\in \C$ to be

[WORK: define $\theta_A$ in terms of a swirl.]

For every simple object $A\in \C$, the map $\theta_A\in \End(A)$ can be identified with the unique complex number $\lambda$ such that $\theta_A=\lambda\cdot \id_A$. Equivilantly, we can identify $\theta_A$ with the complex number $\lambda=\tr(\theta_A)/d_A$ which gives the graphical formula

[WORK: $\theta_A=\frac{1}{d_A}$ times a nice diagram.]

We now characterize the key properties of the twist:

\begin{proposition} Let $\C$ be a pre-modular fusion category. The twists $\theta$ induce a monoidal natural isomorphism $\id_{\C}\xrightarrow{\sim}\id_{\C}$. Additionally, $\theta$ satisfies the identity

$$\theta_{A\otimes B}=\beta_{B,A}\circ \beta_{A,B}\circ (\theta_{A}\otimes \theta_{B})$$

for all $A,B\in \C$, and $\theta_{A^*}=(\theta_A)^*$.
\end{proposition}
\begin{proof}. [WORK: do proof]
\end{proof}

The naive reason to care about twists is that they descrbe a physically relevant quantity and hence should be studied. The more subtle reason to care about twists is that they are the most efficient way of encoding the spherical structure on $\C$. A spherical structure is first and foremost a pivotal structure, meaning that it has a right and left rigid structure which are compatible. Given a spherical structure one can always obtain twists. Conversely, given a right-rigid structure and twists one can recover the left-rigid structure via the formulas

\begin{equation*}
\tikzfig{co-dual}
\end{equation*}

In this way, giving a spherical structure on a right-rigid monoidal category is the \textit{same} as giving a twist structure. This is codified in the following lemma:

\begin{proposition}[Deligne's twisting lemma, \cite{yetter1992framed}] Let $\C$ be a right-rigid braided monoidal category. Every pivotal structure on $\C$ naturally gives a twist natural transformation $\theta:\id_\C\to\id_\C$. This assignment induces a canonical bijection between the set of pivotal structures on $\C$ and the set of natural isomorphism $\theta:\id_\C\to\id_\C$ satisfying $\theta_{A\otimes B}=\beta_{B,A}\circ \beta_{A,B}\circ (\theta_A\otimes \theta_B)$ for all $A,B\in \C$.

Moreover, restricting the assignment to the space of spherical structures on $\C$ induces a canonical bijection between the set of spherical structures on $\C$ and the set of isomorphisms $\theta:\id_\C\to\id_\C$ satisfying $\theta_{A\otimes B}=\beta_{B,A}\circ \beta_{A,B}\circ (\theta_A\otimes \theta_B)$ for all $A,B\in \C$ and $\theta_{A^*}=(\theta_A)^*$.
\end{proposition}
\begin{proof} We show that the distinuished morphisms defined satisfy the axioms of Proposition \ref{pivotal-alternative}.

\begin{enumerate}
\item Rigidity is immediate.

\item We compute as follows:

\begin{equation*}
\tikzfig{something-property-proof}
\end{equation*}

yielding the desired result.

\item By the naturality of $\theta_A$, we have that $\theta_A\circ f \circ \theta^{-1}_B=f$ for all $f:B\to A$. Hence,

\begin{equation*}
\tikzfig{morphism-duals-agree-proof}
\end{equation*}

as desired.
\end{enumerate}

Conversely, that we are given a pivotal structure on $\C$. Define $\theta_A$ by

\begin{equation*}
\tikzfig{graphical-twist-reverse}
\end{equation*}

The arguments above for parts (1)-(3) apply equally well in reverse, and allow us to deduce that $\theta_A$ is natural and satisfies the twist condition. Additionally, we find by

\begin{equation*}
\tikzfig{deligne-lemma-proof}
\end{equation*}

that this serves as an inverse to our stated bijection, so we are done. [WORK: do spherical part]
\end{proof}

\subsubsection{Functors, natural transformations, and equivalence}

.[WORK: this subsection isn't very sexy but I really think it should be included. Questions to answer:

\begin{enumerate}
\item What is the appropriate notion of functor between fusion cats/ spherical fusion cats / MTCs?
\item What is the appropriate notion of natural transformation?
\item What is the appropriate notion of equivalence of categories?
\end{enumerate}


Then, I should talk about how the things we want to be preserved under equivalence really are preserved. For instance, number of isomorphism classes of simple objects, fusion coefficients. A more in-depth discussion of this is going to be in the Yoneda-perspective section so no need to harp on it too much.

Maybe this is a good time for strict vs. skeletal, but that could also wait for the Yoneda perspective section.

This answers to the questions in this section might inform my definition-writing in the previous sections.
]

\subsubsection{Deligne tensor product}

In the theory of any class of mathematical object, an important consideration is the ways in which examples can be put together to give new examples. In the case of fusion categories, this basic operation is known as the \textit{Deligne tensor product}. Given any fusion categories $\C$, $\Dcat$, their Deligne tensor product $\C\boxtimes \Dcat$ is a new fusion category. The Deligne tensor product of spherical fusion categories will be equipped with the structure of a spherical fusion category, and the Deligne tensor product of (pre-)modular tensor categories will be equipped with the structure of a (pre-)modular tensor category.

Physically, the Deligne tensor product corresponds to \textit{stacking}. Consider two sheets of material. We choose two MTCs $\C$, $\Dcat$. We enodow the top sheet with the structure of a topologically ordered quantum system described by $\C$ and we endow the bottome with the structure of a topologically ordered quantum system described by $\Dcat$. The algebraic description of this bilayer system is $\C\boxtimes \Dcat$. This can be viewed as the physical definition of $\C\boxtimes \Dcat$.

[WORK: add bilayer system diagram]

We now mathematically define the Deligne tensor product.

[WORK: this is subtle. There's a definition in general, but we're in a special case where things could be a whole lot simpler. Is there a completely explicit definition in terms of object and morphisms? I do not really want to define right-exact functor if I don't have to.]


[WORK: after the basic definition, I have to define it for fusion/spherical/modular categories. This could take up some room.]

[WORK: show that morphisms between box products of simple objects corresponds to the tensor product of the hom-spaces. Conclude that simple objects in the box product are box products of simple objects.]

\subsection{The category of $G$-graded $G$-representations}

\subsubsection{Overview}

We've talked about a lot of general theory of MTCs. It's time for us to focus on our main family of \textit{examples}. Namely, the categories $\D(G)$ of $G$-graded $G$-representations. These categories describe discrete gauge theory based on the finite group $G$.

Before we can prove that $\D(G)$ is a modular tensor category, we need to endow $\D(G)$ with the neccecary structures. In particular, we will endow $\D(G)$ with $\CC$-linear, monoidal, braided, right-rigid, and left-rigid structures. We will need to show that all of these structures are comptabile with each other in the neccecary ways, and that $\D(G)$ satisfies the non-degeneracy axiom. We will use this as an oppurtunity to introduce tools of general use for proving that categories satisfy the axioms of a modular tensor category.

Additionally, we will also study two categories similar to $\D(G)$ which will serve as extra examples to get our grip on definitions. These categories will also appear later as relevant in and of themselves. The first is $\Vec_G$, the category of $G$-graded vector spaces. It is defined as follows:

[WORK: define $\Vec_G$ in terms of objects and composition.]

Our second structure of interest is $\Rep(G)$, the category of $G$-representations. It is defined as follows:

[WORK: define $\Rep(G)$ in terms of objects and composition.]

We will show that both $\Vec_G$ and $\Rep(G)$ can be naturally equipped with the structures of spherical fusion categories. We then show that $\Rep(G)$ admits a braiding which turns it into a pre-modular tensor category. This braiding is symmetric in the sense that $\beta_{B,A}\circ \beta_{A,B}=\id_{A}\otimes \id_{B}$ for all $A,B\in \C$, and hence $\Rep(G)$ is not an MTC. The category $\Vec_G$ is shown to not admit a braiding whenever $G$ is non-abelian.


\subsubsection{Higher linear algebra}

.[WORK: In this section we define the $\CC$-linear structures on $\Vec_G$, $\Rep(G)$, and $\D(G)$. Our goal is to show that they are all equivalent to $\Vec_\CC^n$ for some $n\geq 1$.

It seems like the best approach is through higher linear algebra. Namely, we show that if $\C$ is abelian, $\CC$-linear, semisimple, and has finitely many isomorphism classes of simple obejcts then it must be isomorphism to $\Vec_{\CC}^n$. Its a good time to wax philosophical about higher linear algebra and 2-vector spaces. However, its not clear that this approach actually helps at all. It might be easier to immediately note that everybody is the direct sum of irriducibles, prove a Schur's lemma, and call it a day. Of course these approaches are all equivalent but its not clear what's best.
]

\subsubsection{Spherical fusion structures}

.[WORK: show that the categories have duals and monoidal structure. This should be pretty easy and painless. Pentagon identity should follow from the pentangon identity on $\Vec_{\CC}$.]

\subsubsection{Braiding and modularity}

.[WORK: Introduce braidings. Show that $\Rep(G)$ is symmetric. Show that $\Vec_G$ does not admit a braiding if $G$ is not abelian and does admit a symmetric braiding if $G$ is abelian. Show that $\D(G)$ admits a non-degenerate braiding.]


\subsection{The modular representation}

\subsubsection{Definition}

\subsubsection{Torus perspective}

\subsubsection{Bruguieres's modularity theorem}

\subsubsection{Verlinde formula}

\subsubsection{Proof of modularity}

\subsubsection{Proof of unitarity}

\subsubsection{Schauenberg-Ng theorem}



\subsection{The Yoneda perspective}

\subsubsection{Principle}

\subsubsection{$F$-symbols}

\subsubsection{$R$-symbols}

\subsubsection{$\theta$-symbols}

\subsubsection{Reconstruction theorem}



\subsection{Quantum double MTCs}

\subsubsection{The Drinfeld center}

\subsubsection{Muger's theorem}

\subsubsection{Levin-Wen model}

\subsubsection{Discrete gauge theory as a quantum double and Morita equivalence}

\subsubsection{Factorizability and time reversal symmetry}



\subsection{Unitarity}

\subsubsection{Characterization of unitarizable MTCs}

[WORK: the main theorem of this section is that an MTC is unitarizable if and only if it has positive quantum dimensions]

[WORK: there also needs to be mention of the fact that braiding is automatically unitary in a unitary category]

\subsubsection{Uniqueness of unitary structure}

\subsubsection{Yoneda perspective on unitarity}




\subsection{Number theory in MTCs}

\subsubsection{.[prerequisites and introduction]}

\subsubsection{Galois conjugation}

\subsubsection{Ocneanu rigidity}

\subsubsection{Rank-finiteness theorem}

\subsubsection{Vafa's theorem}

.[WORK: this section is going to host a lot more theorems]




\subsection{Adjectives in MTCs}

\subsubsection{Solvability}

\subsubsection{Nilpotence}

.[WORK: surely more adjectives will be hosted here. Not sure if this section is neccecary, we'll see.]


$\newline$
\fbox{\parbox{\dimexpr\linewidth-2\fboxsep-2\fboxrule\relax}{

\begin{center}
\textbf{History and further reading:}\\
\end{center}

Modular tensor categories were born from conformal field theory in the late 1980s. In a series of papers, Moore and Seiberg analysed deeply the underlying content within conformal field theory to find what essential algebraic data lied within it \cite{moore1988polynomial, moore1989classical}. They wrote out the axioms of this essential algebraic data in their subsequent notes on conformal field theory \cite{moore1990lectures}. They used the name modular tensor category to describe their data, as suggested by Igor Frenkel. This definition was then refined and re-introduced by Turaev \cite{turaev1992modular}. The first major application of modular tensor categories was the Reshetikhin-Turaev construction \cite{reshetikhin1991invariants, turaev2010quantum}. Prior to this result nobody had succeed in constructing topological quantum field theories. In this way, modular tensor categories and the Reshetikhin-Turaev construction completed Witten's programme of quantizing Chern-Simons theory.

$\newline$
By the early 2000s, the proposal of topological quantum computing was attracting a lot of interest in anyons and their algebraic properties. Seeing as topological order can be described by topological quantum field theory and topological quantum field theory is essentially equivalent to modular tensor categories, it was understood that modular tensor categories could be used to understand topological order. This latent description of anyons in terms of modular tensor categories was made explicit in an appendix in the seminal 2006 paper of Kitaev \cite{kitaev2006anyons}. This approach to anyons in terms of modular tensor categories was popularized by Wang's early monograph \cite{wang2010topological}. This has since become the standard approach towards the algebraic theory of topological quantum information.
}}


$\newline\newline$

\large \textbf{Exercises}:\normalsize

\begin{enumerate}[\thesection .1.]

\item .[WORK: apply Verlinde formula to group-theoretical MTCs to recover classical theorem by Burnside]

\end{enumerate}