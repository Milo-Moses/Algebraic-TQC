\section{Category theory}
\label{category-theory}

\subsection{Overview}

\subsubsection{Introduction}

There is a lot of math in the world. The development of the subject has spanned thousands of years, and has enjoyed a large uptick in progress the last two hundred or so. This has given ample time for the most important ideas to rise to the top. Among these important concepts there is one which is the focus of chapter: \textbf{composition}.

Let $A,B,C$ be sets. Let $f:A\to B$ and $g:B\to C$ be functions. The {\em composition} of $f$ and $g$ is the function $g\circ f: A\to C$ defined by the formula $(g\circ f)(x)=g(f(x))$ for all $x\in A$. More generally, composition is the act of performing one process followed by performing a second process. Composition is distinguished in its importance for two reasons:

\begin{enumerate}
\item Composition is ubiquitous;
\item Many complicated structures can be described in terms of composition.
\end{enumerate}

These two primary sources of importance lead to several emergent applications of composition:

\begin{enumerate}
\item It's a good organization principle - thinking in terms of composition gives a unifed approach to disperate subjects, which highlights the universality latent within mathematics;
\item It's a good compression technique - in a composition-first approach there's no need to remember details about objects or functions between them, only the way that those functions compose is used;
\item Sometimes composition rules are the only data we have about an object of study, making a composition-first technique the only approach possible.
\end{enumerate}

This third point is the situation we find ourselves in with the algebraic theory of topological quantum information. We're trying to give a usable mathematical description of topologically ordered systems. The way we do this is by focusing on anyons (local quasiparticle excitations in topological order). In doing so we run into three important ponts:

\begin{enumerate}
\item Describing anyons exactly is hard. They are emergent phenomina, found within highly-entangled energy eigenstates of arbitrarily complicated gapped Hamiltonians;
\item Describing the ways anyons can transform is hard. This involves specifying intricate unitary operators on high-dimensional Hilbert spaces.
\item Describing how these transformations compose with one another is relatively simple. It can be done using explicit-to-describe rules, which are independent of the system size or choice of gapped Hamiltonian.
\end{enumerate}

What to do in this situation is clear: we will take a composition-first approach to anyons. The mathematical structure which allows for an intelligent discussion of composition is known as a {\em category}. The composition-first approach to mathematics is known as {\em category theory}. Of course, to describe anyons we will need more than just the structure of composition. We will also need a way to encode what happens we we put anyons together, braid them, and fuse them. There structures are all completely compatible with the compostion-first approach, and correspond to adding extra structures onto the category. The type of category which fully describes anyons is known as a {\em modular tensor category}, and these categories will be the subject of much of this book. This chapter deals with introducing category theory, as well as some of the structures which will be important for discussing anyons and modular tensor categories.

\subsubsection{Definition and important obervations}

As discussed before, a category is the structure which allows for a composition-first approach to mathematics. Before going forward lets define what a category is:

\begin{defn}[Category]\label{category-def} A category is the following data:

\begin{enumerate}
\item (Objects) A set $\cC$;
\item (Morphisms) A set $\Hom(A,B)$ for all $A,B\in \cC$;
\item (Composition) Functions

$$\circ: \Hom(B,C)\times \Hom(A,B)\to \Hom(A,C)$$

for all $A,B,C\in \cC$;
\end{enumerate}

Such that:

\begin{enumerate}

\item  For all morphisms $f\in \Hom(A,B)$, $g\in \Hom(B,C)$, $h\in\Hom(C,D)$,  and objects $A,B,C,D\in \cC$,

$$(h\circ g)\circ f = h\circ (g\circ f).$$

\item (Identity) For all objects $A\in \cC$ there exists a morphism $\id_{A}: A\to A$ such that for all $B\in \cC$, $f\in \Hom(A,B)$, and $g\in \Hom(B,A)$,

\begin{align*}
f\circ \id_{A}=f, && \id_{A}\circ g = g.
\end{align*}

\end{enumerate}
\end{defn}

\begin{rem} The structure of definition \ref{category-def} is very typical of algebra. Roughly, algebra is defined to be the study of algebraic structures. An algebraic structure is some collection operations on some space, with rules outlining how these operations interact with each other. The general way of definding an algebraic structure is to first list its operations, and then list the axioms of how these operations inteact with each other. We will see many definitions of this sort throughout the rest of the book, so it is good to get used to it now.
\end{rem}

\begin{ex}\label{category-examples} In this text we have already seen many examples of categories. We list some of them here:

\begin{itemize}
\item $\Set$, the category of sets. The objects are sets and the morphisms are functions.

\item $\Top$, the category of topological spaces. The objects are topological spaces and the morphisms are continuous functions.

\item $\Vec_k$, the category of finite dimensional vector spaces over a field $k$. The objects are finite dimensional vector spaces over $k$ and the morphisms are linear operators.

\item $\Grp$, the category of groups. The objects are groups and the morphisms are group homomorphisms.

\item $\Hilb$, the category of quantum systems. The objects are finite dimensional Hilbert spaces and the morphisms are unitary operators.

\item ${\bf Prob}$, the category of probability spaces. The objects are finite dimensional real vector spaces with distinguished bases and the morphisms are operators which send normalized vectors to normalized vectors.

\item ${\bf Ord}_M$, the category associated with ordered media with order space $M$. The objects are continuous maps $\phi: \bR^2\to M$ and the morphisms are continuous deformations.

\item $\fD(G)$, the category associated with discrete gauge theory based on the finite group $G$. The objects are $G$-graded $G$-representations and the morphisms are linear maps which respect both the $G$-grading and the $G$-action.

\end{itemize}
\end{ex}

\begin{warn}
One subtelty of the defintions in example \ref{category-examples} is that the collection of objects in some of these definitions are not sets. For instance, the collection of all sets does not itself form a set, due to logical paradoxes such as Bertrand's paradox. A more proper treatment of category theory would restrict to smaller categories (such as the category {\em finSet} of finite sets) where the collection of objects does really form a set. Alternatively, one can introduce the notation of a {\em class}, with is a collection of sets defined by some unambiguous property that all objects share. The collection of all objects in $\Set$ is not a set, but it is a class. In this framework, we call categories whose collection objects and whose hom spaces are all sets a {\em finite category}. The concerned reader can make all of our discussion correct by restricting to the case of small categories, and defining $\Set$ to be the space of all sets whose cardinality is at most the size of the real numbers, and in general restricting any definition of a category (such as $\Grp$) to spaces whose underlying set whose cardinality has at most the size of the real numbers.
\end{warn}

\begin{rem}The objects and morphisms of a category do not have much complexity implicit to them.  All of the interesting structure is encoded within the composition structure. This is despite the fact that when we listed our examples in example \ref{category-examples} we only described the objects and morphisms, and not the composition structure. The reason for this is that the composition structure between morphisms in all of our examples is clear. In all our examples the objects are sets with extra strcture, and the morphisms are maps of sets. The composition structure is inhereted from the composition structure on functions between sets. Going further, we remark that objects in abstract categories are {\em not} required to be sets and the morphisms are {\em not} required to be functions of sets. It is important to be aware of the fact that there are some categories for which there is no interpreation of morphisms as functions between sets \cite{freyd1970homotopy}.
\end{rem}

\begin{rem} We will make several notational shorthands when dealing with categories.  We will conflate a category and its set of objects whenever convenient. That is, we will label a category by its set of objects. Additionally, instead of writing ``$\Hom_{\cC}(A,B)$" we will write ``$\Hom(A,B)$" when $\cC$ is clear, and we will write ``$f:A\to B$" to mean ``$f\in \Hom_{\cC}(A,B)$". Additionally, we will use a single quantifier to instanciate multiple objects simultaneously. For example, we will use ``$\forall f:A\to B \text{ in }\cC$" to mean ``$\forall A,B\in \cC,\text{ and }\forall f:A\to B$".
\end{rem}

\begin{rem} A category isn't just a space with a good notion of composition - it also has identity maps. These identity maps are important, and we include them in the definition purposefully. There are two primary reasons: firstly that all of the relevant examples of categories will have identity maps, and secondly that most interesting properties of categories only make sense because of the identity maps. Hence if we didn't require identity maps then we would find ourselves constantly requiring them as a condition, which is a waste of space.

It is important to take a closer look at what the identity map means, though. The identity map is trying to capture a very general phenominon about transformations: there is always the trivial transformation which results from doing nothing. This do-nothing map is the identity. In the category of sets, the identity maps on the set $A$ is given by the formula $\id_A(x)=x$ for all $x\in A$ by lemma \ref{identity-lemma}. The fact that these maps are the identities in the category of sets is the reason that the identity axiom for categories is defined like it is.
\end{rem}

\begin{lem}\label{identity-lemma} Let $A$ be a set. For all sets $B$ and for all $f:A\to B$, $g:B\to A$ we have

\begin{align*}
f\circ \id_{A}=f, && \id_{A}\circ g = g.
\end{align*}

In particular, $\id_A$ satisfies the axiom of an identity in the category of sets, and hence $\Set$ forms a category.
\end{lem}
\begin{proof} The associativity axiom is satisfied because composition of set functions is associative, and for all $f:A\to B$, $g:B\to A$,

$$(f\circ \id_{A})(x)=f(\id_{A}(x))=f(x),$$

$$(\id_{A}\circ g)(x)=\id_{A}(g(x))=g(x),$$

so the identity axiom is satisfied.
\end{proof}

\begin{defn}[Isomorphism] Let $\cC$ be a category, let $A,B\in C$ be objects, and let $f:A\to B$ be a morphism. We say that $f$ is an {\em isomorphism} if there exists a morphism $f^{-1}:B\to A$ such that $f^{-1}\circ f= \id_A$ and $f\circ f^{-1}=\id_B$. We call $f^{-1}$ the {\em inverse} of $f$. In this case, we say that $A$ and $B$ are {\em isomorphic objects}.
\end{defn}

\begin{lem} Let $A,B$ be sets, and let $f:A\to B$ be a function. The map $f$ is a bijection if and only if there exists a function $f^{-1}: B\to A$ such that $f^{-1}\circ f= \id_A$ and $f\circ f^{-1}=\id_B$. In particular, a function $f$ in the category $\Set$ is an isomorphism if and only it is a bijection.
\end{lem}
\begin{proof} Suppose that $f$ is a bijection. Then, we can define a map $f^{-1}:B\to A$ which sends $b\in B$ to the unique element $f^{-1}(b)$ such that $f(f^{-1}(b))=b$, which exists since $f$ is surjective and is unique because $f$ is injective. By definition of $f^{-1}$, $f\circ f^{-1}=\id_{B}$. To show that the composition the other direction is the identity, we observe that for all $a\in A$

$$f(f^{-1}(f(a))=f(a),$$

so $f^{-1}(f(a))=a$ by the injectivity of $f$. Thus, $f$ has an inverse. Conversely, suppose that $f$ has an inverse $f^{-1}$. Then, $f(a)=f(a')$ implies $a=f^{-1}(f(a))=f^{-1}(f(a'))=a'$ so $f$ is injective. Additionally, for all $b\in B$ we have $b=f(f^{-1}(b))$ so $f$ is surjective. Thus, $f$ is a bijection. We have proved both directions, so our proof is complete.
\end{proof}

\begin{rem} Just like how the category-theoretic definitions of identity maps and isomorphisms are modeled after the abstract properties of identity maps and isomorphisms in the category of sets, many other definitions will be implicitely modeled after the abstract properties of the category of sets or vector spaces. Accompanied with most definitions, there is often an implicit lemma that the usual examples satisfy the axioms of the definition. Going forward, we will rarely remark on these implicit lemmas.
\end{rem}

\begin{prop}\label{identity-unique} Let $\cC$ be a category. Identities in $\cC$ are unique. Explicitely, let $A\in \cC$ be an object and let $\id_A,\tilde{\id}_A:A\to A$ be morphisms satisfying the identity axiom. We have that $\id_A=\tilde{\id}_A$.
\end{prop}
\begin{proof}. Using the fact that $\id_A \circ f = f$ and $f\circ \tilde{\id}_A=f$ for any $f:A\to A$, we compute that

$$\id_A= \id_A \circ \tilde{\id}_A = \tilde{\id}_A$$

as desired.
\end{proof}

\begin{prop}
\label{inverse-unique}
Let $\cC$ be a category. Let $A,B$ be objects and let $f:A\to B$ be an isomorphism. The inverse of $f$ is unique. That is, let $f^{-1},\tilde{f}^{-1}$ be morphisms satisfying the definition of the inverse of $f$. We have that $f^{-1}=\tilde{f}^{-1}$.
\end{prop}
\begin{proof} Using the associativity axiom, we compute

$$f^{-1}=f^{-1}\circ \id _{B} = f^{-1}\circ (f \circ \tilde{f}^{-1})=(f^{-1}\circ f)\circ \tilde{f}^{-1}=\id_A \circ \tilde{f}^{-1}=\tilde{f}^{-1}$$

as desired.
\end{proof}

\begin{rem} Statements in category theory can be very broadly applied. This is in some sense obvious by the fact that there are so many different examples of categories, but it's good to state the observation explicitely. For instance, look at proposition \ref{inverse-unique}. It applied equally well for showing that inverse elements in groups are unique and for showing that inverses of matrices are unique. Abstractly, proposition \ref{inverse-unique} demonstrates why the inverse of any reversible process is unique.
\end{rem}

\subsection{Structures in category theory}

\subsubsection{Products and universal properties}

In this section we will work on defining important structures in category, with a focus on the broadly applicable principles behind the definitions. In this first subsection we focus on products, our first example of a definition via universal property.

\begin{defn}[Product]\label{product-definition} Let $A,B\in \cC$ be objects in a category. A {\em product} of $A$ and $B$ is the following data:

\begin{enumerate}
\item An object $A\times B\in \cC$;
\item A morphism $\pi_A:A\times B\to A$;
\item A morphism $\pi_B:A\times B\to B$;
\end{enumerate}

such that for all other objects $C\in \cC$ with morphisms $f_A:C\to A$, $f_B:C\to B$, there exists a unique morphism $f:C\to A\times B$ such that the diagram

% https://q.uiver.app/#q=WzAsNCxbMSwxLCJBXFx0aW1lcyBCIl0sWzAsMiwiQSJdLFsxLDAsIkMiXSxbMiwyLCJCIl0sWzAsMSwiXFxwaV9BIl0sWzAsMywiXFxwaV9CIiwyXSxbMiwxLCJmX0EiLDJdLFsyLDMsImZfQiJdLFsyLDAsImYiLDFdXQ==
\[\begin{tikzcd}
	& C \\
	& {A\times B} \\
	A && B
	\arrow["f"{description}, from=1-2, to=2-2]
	\arrow["{f_A}"', from=1-2, to=3-1]
	\arrow["{f_B}", from=1-2, to=3-3]
	\arrow["{\pi_A}", from=2-2, to=3-1]
	\arrow["{\pi_B}"', from=2-2, to=3-3]
\end{tikzcd}\]

commutes.


\end{defn}

\begin{rem}\label{product-examples-remark} At first glance, the categorical definition of a product may look strange. For a first level of comfort, one should observe that the categorical notion of product agrees with the usual notion of Cartesian product in the category $\Set$, by proposition \ref{product-in-set}. More generally, the same argument as in proposition \ref{product-in-set} can be used to show that the Cartesian product endowed with the product topology is a product in the category $\Top$, the direct sum of vector spaces is a product in the category $\Vec_k$ for all fields $k$,  the Cartesian product endowed with component-wise multiplication is a product in the category $\Grp$, and so on.
\end{rem}

\begin{prop}\label{product-in-set} For all pairs of sets $A,B$, the triple $(A\times B,\pi_A,\pi_B)$ is a product of $A,B$ in the category $\Set$, where $A\times B$ is the Cartesian product, $\pi_A$ is the projection of $A\times B$ onto the $A$ component, and $\pi_B$ is the projection of $A\times B$ onto the $B$ component.
\end{prop}
\begin{proof} Consider a set $C$ and functions $f_A:C\to A$, $f_B:C\to B$. We can define a function $f:C\to A\times B$ by $f(c)=(f_A(c),f_B(c))$. Clearly, this morphism $f$ satisfies $f_A=\pi_A\circ f$ and $f_B=\pi_B\circ f$. Moreover, suppose $f:C\to A\times B$ is any function with $f_A=\pi_A\circ f$ and $f_B=\pi_B\circ f$. Then, the $A$ component of $f(c)$ is $f_A(c)$ and the $B$ component of $f(c)$ is $f_B(c)$. Thus, $f(c)=(f_A(c),f_B(c))$. Thus, we conclude that there is a unique map $f:C\to A\times B$ making the relevant diagram commute, and since $C$, $f_A$, $f_B$ were chosen arbitrarily we conclude the result.
\end{proof}

\begin{rem} Even though the Cartesian product is a product in the category of sets, it is {\em not} true that every categorical product of two sets $A,B$ in $\Set$ is equal to the Carteisan product. In particular, suppose that $D$ is a set and $i:D\xrightarrow{\sim}A\times B$ is a bijection from $D$ to $A\times B$. Define $g_A=\pi_A\circ i$ and $g_B=\pi_B\circ i$. Then, $(D,g_A,g_B)$ is also a product of $A$ and $B$. This fact can be seen as follows. Suppose $C$ is set, and $f_A:C\to A$, $f_B:C\to B$ are functions. We can define $f:C\to D$ by $f(c)=i^{-1}((f_A(c),f_B(c)))$. This map satisfies $f_A=g_A\circ f$ and $f_B=g_B\circ f$ since

$$g_A\circ f =(\pi_A\circ i)\circ (i^{-1}(f_A(c),f_B(c)))=f_A(c).$$

This is, however, the only freedom we have for choosing products. Every product of $A,B$ in $\Set$ will be obtained by starting with $(A\times B,\pi_A,\pi_B)$ and composing with a bijection. To summarize this situation, we say that categorical products are not unique but they are {\em unique up to isomorphism}. Moreover, given another product $(D,g_A,g_B)$, there is a {\em unique} isomorphism $i:D\to A\times B$ such that $f_A=\pi_A\circ f$ and$g_B=\pi_B\circ i$. For this reason we say that products are {\em unique up to unique isomorphism}. The proof for the category of sets is no easier than the general case, which is given in proposition \ref{product-unique}
\end{rem}

\begin{prop}\label{product-unique} Let $A,B\in\cC$ be objects in a category. Let $(C,f_A,f_B)$, $(D,g_A,g_B)$ be products of $A$ and $B$. There exists a unique isomorphism $i:C\to D$ such that $f_A= g_A\circ i$ and $f_B=g_B\circ i$.
\end{prop}
\begin{proof} By the universal property of $(D,g_A,g_B)$, there exists a unique morphism $i:C\to D$ making the relevant diagram commute ($f_A= g_A\circ i$ and $f_B=g_B\circ i$). Similarly, by the universal property of $C$, there exists a unique morphism $j:D\to C$ making the relevant diagram commute ($g_A= f_A\circ j$ and $g_B=f_B\circ j$). Composing, we find that $j\circ i:C\to C$ makes the relevant diagram commute ($f_A=f_A\circ (j\circ i)$ and $f_B=f_B\circ (j\circ i)$). The map $\id_C:C\to C$, however, also makes the relevant diagram commute ($f_A=f_A\circ \id_C$ and $f_B=f_B\circ \id_C$). The universal property of the product says that there is a {\em unique} map making the diagram commute. Thus, we must have $j\circ i =\id_C$. By an analagous argument using the universal property of $D$, we find that $i\circ j=\id_D$. Thus, $j=i^{-1}$ and $i$ is an isomorphism as desired.
\end{proof}

\begin{rem} Definition \ref{product-definition} is our first example of a definition by a {\em universal property}. The property that the triple $(A\times B,\pi_A,\pi_B)$ is asked to satisfy in the definition is the universal property. In words, we will sometimes say that the product of $A,B$ is universal with respect to the property of having morphisms into $A$ and $B$. In light of proposition \ref{product-unique}, the categorical definition of product is unique up to isomorphism, and in light of proposition \ref{product-in-set} this unique product is isomorphic to the usual Cartesian product. Thus, at least for the category of sets, definition \ref{product-definition} is a more-complicated and less-precise way of defining the Cartesian product. There are several general reasons why one might prefer definitions by universal property:

\begin{enumerate}
\item Universal properties are common throughout mathematics (and specifically in the study of topological quantum information). It is good to understand their general structure;
\item \Note{make this list better. There's a very well-written list on Wikipedia.}
\end{enumerate}
\end{rem}

\begin{defn}[Coproduct]\label{coproduct-definition} Let $A,B\in \cC$ be objects in a category. A {\em coproduct} of $A$ and $B$ is the following data:

\begin{enumerate}
\item An object $A\sqcup B\in \cC$;
\item A morphism $i_A:A\to A\sqcup B$;
\item A morphism $i_B:B\times A\sqcup B$;
\end{enumerate}

such that for all other objects $C\in \cC$ with morphisms $f_A:A\to A\sqcup B$, $f_B:B\to A\sqcup B$, there exists a unique morphism $f:A\sqcup B\to C$ such that the diagram

% https://q.uiver.app/#q=WzAsNCxbMSwxLCJBXFxzcWN1cCBCIl0sWzAsMiwiQSJdLFsxLDAsIkMiXSxbMiwyLCJCIl0sWzAsMywiaV9CIiwyXSxbMCwyLCJmIiwxXSxbMSwyLCJmX0EiXSxbMywyLCJmX0IiLDJdLFsxLDAsImlfQSIsMl1d
\[\begin{tikzcd}
	& C \\
	& {A\sqcup B} \\
	A && B
	\arrow["f"{description}, from=2-2, to=1-2]
	\arrow["{i_B}"', from=2-2, to=3-3]
	\arrow["{f_A}", from=3-1, to=1-2]
	\arrow["{i_A}"', from=3-1, to=2-2]
	\arrow["{f_B}"', from=3-3, to=1-2]
\end{tikzcd}\]

commutes.


\end{defn}

\begin{defn}[Opposite category] Let $\cC$ be a category. The {\em opposite category} of $\cC$ is the category defined as follows $\cC^{\op}$. The set of objects of $\cC^{\op}$ is the same as the set of objects as $\cC$, with the object $A\in \cC$ corresponding to the object $A^{\op}\in \cC^{\op}$. The hom-sets of $\cC^{\op}$ are defined as

$$\Hom_{\cC^{\op}}(A^{\op},B^{\op})=\Hom_{\cC}(B,A).$$

The morphism in $\Hom_{\cC^{\op}}(A^{\op},B^{\op})$ corresponding to $f\in \Hom_{\cC}(B,A)$ is denoted $f^{\op}$. Given any $A,B,C\in \cC$, and functions $f:A\to B$, $g:B\to C$ we define

$$f^{\op}\circ g^{\op}=(g\circ f)^{\op}.$$
\end{defn}

\begin{rem} The category $\cC^{\op}$ is intuitively described as ``reversing all the arrows in $\cC$". It is often a useful category to consider, especially in the context of mathematical physics were we see that the opposite category can be seen as the {\em time reversal} of $\cC$. That is, the physical system analogous to $\cC$ where the arrow of time has been reversed. This is also a mathematically fruitful perspective to take, because flipping source and target for arrows is the same as changing the direction of causation.
\end{rem}

\begin{rem} Definition \ref{coproduct-definition} of the coproduct is another definition by universal property. The universal property, of course, is very similar to the universal property of the product. In a formal sense it is the same universal property but with all of the arrows reversed. In particular, proposition \ref{product-coproduct-duality} shows that products and coproducts are formally dual in the sense that products (resp. coproducts) in a category $\cC$ correspond to coproducts (resp. products) in the opposite category $\cC^{\op}$. This is a common theme in category theory. Many notions have corresponding dual notions, and the general terminology for the dual notion is to add the prefix ``co-".
\end{rem}

\begin{prop}\label{product-coproduct-duality} Let $A,B\in \cC$ be objects in a category. Let $(A\times B,\pi_A,\pi_B)$ be a product of $A,B$ in $\cC$. The triple $((A\times B)^{\op},\pi_A^{\op},\pi_B^{\op})$ is a coproduct of $A^{\op},B^{\op}$ in $\cC^{\op}$. Similarly, if $(A\sqcup B,i_A,i_B)$ is a coproduct of $A,B$ in $\cC$ then $((A\sqcup B)^{\op},i_A^{\op},i_B^{\op})$ is a product in $A^{\op},B^{\op}$ in $\cC^{\op}$.
\end{prop}
\begin{proof} We show that $((A\times B)^{\op},\pi_A^{\op},\pi_B^{\op})$ is a coproduct. Choose an arbitrary triple $C^{\op}\in \cC^{\op}$,$f^{\op}_A:A^{\op}\to C^{\op}$, and $f^{\op}_B:B^{\op}\to C^{\op}$. Then considering the triple $(C,f_A,f_B)$ and applying the universal property of the product, we conclude that there is a unique map $f:C\to A\times B$ making the relevant diagram for the product commute. The dual map $f^{\op}:(A\times B)^{\op}\to C^{\op}$ is thus the unique map making the relevant diagram for the coproduct commute, so $((A\times B)^{\op},\pi_A^{\op},\pi_B^{\op})$ is indeed a coproduct. The argument for why $((A\sqcup B)^{\op},i_A^{\op},i_B^{\op})$ is a product is analagous.
\end{proof}

\begin{cor} Let $A,B \in \cC$ be objects in a category. The coproduct of $A,B$, if it exists, is unique up to unique isomorphism. That is, if $(C,f_A,f_B)$, $(D,g_A,g_B)$ are coproducts of $A$ and $B$, there exists a unique isomorphism $i:C\to D$ such that $g_A=i\circ f_A$ and $g_B=i\circ f_B$.
\end{cor}
\begin{proof} By proposition \ref{product-coproduct-duality} we find that $(C^{\op},f_A^{\op},f_B^{\op})$, $(D^{\op},g_A^{\op},g_B^{\op})$ are products in $\cC^{\op}$. By proposition \ref{product-unique}, we conclude that there is a unique isomorphism between $C^{\op}$ and $D^{\op}$ making the relevant diagram commute. Taking the opposite of this isomorphism, we conclude that there is a unique isomorphism between $C$ and $D$ making the relevant diagram commute, as desired.
\end{proof}

\begin{rem}\label{coproduct-examples-remark} Unlike how the product looks roughly similar in all our basic examples of categories, the coproduct can change quite dramatically. For instance, in the category $\Set$ the coproduct is the disjoint union by proposition \ref{coproduct-example}. However, the coproduct of two vector spaces $V,W$ in $\Vec_k$ is the direct sum $V\oplus W$ with $i_V,i_W$ given by $i_V(v)=(v,0)$, $i_W(v)=(0,v)$, for all fields $k$. The coproduct in $\Grp$ is especially exotic. Given two groups $G,H$, their coproduct is the {\em free product}, defined to be the group whose elements are formal words where each letter is either an element of $G$ or an element of $H$, and two words are considered equivalent if it is possible to go from one to another by adding/removing copies of the identity elements of $G,H$ or by replacing a product $g_1g_2$ of adjacent elements by their product (in $G$ or $H$). For example, the free product of $\bZ$ and $\bZ$ is the free group on two generators. The definition of free product can sometimes feel cumbersome - it is an example of a case where the universal property can be helpful. Seeing as the free product of finite groups can be infinite, we conclude that the category ${\bf finGrp}$ does not have coproducts. Even though coproducts are unique up to unique isomorphism if they exist, they are not gauranteed to exist.

\end{rem}

\begin{prop}\label{coproduct-example} For all pairs of sets $A,B$, the triple $(A\sqcup B,i_A,i_B)$ is a coproduct of $A,B$ in the category of $\Set$, where $A\sqcup B$ is the disjoint union of $A$ and $B$ and $i_A$ (resp. $B$) is the inclusion of $A$ (resp. $B$) into $A\sqcup B$.
\end{prop}
\begin{proof} Suppose $C$ is a set and $f_A:A\to C$, $f_B:B\to C$ are maps. Then, we can define a map $f:A\sqcup B\to C$ by the formula

$$
f(x)=
\begin{cases}
f_A(x) & \text{if } x\in A\\
f_B(x) & \text{if } x\in B.
\end{cases}
$$

Moreover, suppose we had any other map $f:A\sqcup B\to C$ such that $f_A=f\circ i_A$ and $f_B=f\circ i_B$. The first condition says that $f(x)=f_A(x)$ for $x\in A$ and the second condition says that $f(x)=f_B(x)$ for $x\in B$, so $f$ must be equal to the map defined above. Thus, we conclude that $(A\sqcup B,i_A,i_B)$ as a coproduct of $A,B$.
\end{proof}

\subsubsection{Functors and natural transformations}

Category theory philosophy tells us to care about the relationships between things (or more precisely, how those relationships compose). Moreover, category theory philosophy tells us to care about categories (any self-respecting theory should care about its central object of study). Naively combining these two principles, category theory thus suggests we should care about relationships between different categories. These relationships are called {\em functors}. Going further, seeing as we care about functors, a naive application of the principles of category theory suggests we should care about the relationships between different functors. These relationships are known as {\em natural transformations}. Thankfully, this line of reasoning does not go on indefinitely. One could try to define a notation of a relationship between natural transformations, but the resulting notion is trivial. Thus, one is left with exactly two important and fundamental notions to define: functors and natural transformations.

\begin{defn}[Functor] A {\em functor} from a category $\cC$ to a category $\cD$ is the following data:

\begin{enumerate}
\item A function of objects $ F:\cC\to\cD$;
\item A function of morphisms $F:\Hom_{\cC}(A,B)\to \Hom_{\cD}(F(A),F(B))$ for all $A,B\in \cC$;
\end{enumerate}

such that for all $A,B,C\in \cC$, $f:A\to B$, $g:B\to C$,

$$F(g\circ f)=F(g)\circ F(f).$$
\end{defn}

\begin{defn}[Natural transformation] A {\em natural transformation} from a functor $F$ to a functor $G$ between categories $\cC,\cD$ is a family of morphisms $\eta_A:F(A)\to G(A)$ for all $A\in \cC$ such that the diagram

% https://q.uiver.app/#q=WzAsNCxbMCwwLCJGKEEpIl0sWzEsMCwiRyhBKSJdLFswLDEsIkYoQikiXSxbMSwxLCJHKEIpIl0sWzAsMiwiRihmKSIsMl0sWzEsMywiRyhmKSJdLFswLDEsIlxcZXRhX0EiXSxbMiwzLCJcXGV0YV9CIiwyXV0=
\[\begin{tikzcd}
	{F(A)} & {G(A)} \\
	{F(B)} & {G(B)}
	\arrow["{\eta_A}", from=1-1, to=1-2]
	\arrow["{F(f)}"', from=1-1, to=2-1]
	\arrow["{G(f)}", from=1-2, to=2-2]
	\arrow["{\eta_B}"', from=2-1, to=2-2]
\end{tikzcd}\]

commutes for all $f:A\to B$ in $\cC$.
\end{defn}

\begin{ex} A important family of examples of functors are {\em forgetful functors}. A forgetful functor is a functor which acts on the level of objects by taking an algebraic structure and forgetting that it satisfies certain axioms or is equipped with certain operations so that it becomes a different algebraic structure, and it acts as the identity on the level of morphisms. For instance, there is a forgetful functor $F:\Top\to\Set$ which takes a topological space and assigns to it its underlying set. There is a forgetful functor $F:\Vec\to\Grp$ which takes a vector space and assigns to it its underlying additive group. There is also a forgetful functors $F:\Field\to \Grp$ from the category of fields to the category of groups which assigns to every fields its underlying additive group. Some functors which are almost like forgetful functors also get called forgetful functors, such as the functor $F:\Field\to\Grp$ which assigns to every field its multiplicative group of non-zero elements. The inclusion functors $F:{\bf finSet}\to \Set$ and $F:{\bf finGrp}\to\Grp$ from the categories of finite sets (resp. finite groups) to to the category of all sets (resp. groups) can also be viewed as forgetful functors, where we are forgetting the fact that the input objects are finite.
\end{ex}

\begin{ex} For every $n\geq 1$, there is a functor $\GL_n:\Field\to\Grp$ which assigns to a field $k$ to the general linear group of $n$ by $n$ matrices, $\GL_n(k)$. This map is {em functorial} (that is, it can be extended to a functor) since every field homomorphism $k\to k'$ induces a group homomorphism $\GL_n(k)\to \GL_n(k')$ by acting via the field homomorphism element-wise on a matrix. It is clear that this assignment is compatible with composition, and is thus a functor. There are many natural transformations between the functors $\GL_n$. For instance, the inverse-transpose $((\dash)^{-1})^T$ is a natural transformation between the functor $\GL_n$ and itself. For every field $k$, there is a map $((\dash)^{-1})^T:\GL_n(k)\to \GL_n(k)$ which acts by first taking the inverse of a matrix and then taking its transpose. These maps form a natural transformation, because for any field homomorphism $k\to k'$ there is a commutative diagram

% https://q.uiver.app/#q=WzAsNCxbMCwwLCJcXEdMX24oaykiXSxbMSwwLCJcXEdMX24oaykiXSxbMCwxLCJcXEdMX24oaycpIl0sWzEsMSwiXFxHTF9uKGsnKS4iXSxbMCwyXSxbMCwxLCIoKFxcZGFzaCleey0xfSleVCJdLFsyLDMsIigoXFxkYXNoKV57LTF9KV5UIiwyXSxbMSwzXV0=
\[\begin{tikzcd}
	{\GL_n(k)} & {\GL_n(k)} \\
	{\GL_n(k')} & {\GL_n(k').}
	\arrow["{((\dash)^{-1})^T}", from=1-1, to=1-2]
	\arrow[from=1-1, to=2-1]
	\arrow[from=1-2, to=2-2]
	\arrow["{((\dash)^{-1})^T}"', from=2-1, to=2-2]
\end{tikzcd}\]

\Note{the determinant is also a good example, and so is the map which pads a matrix with an extra row/collumn. Lots of nice linear algebra natural transformations. Not sure how relevant they are to this book - maybe I can do something better? Perhpas the additive group of matrices, with trace as an example. Want something that I could pull on later.}
\end{ex}

\begin{defn} The way that functors relate categories and natural transformations relate functors is completely analagous to the way that morphisms relate objects in a category. In particular, we observe that there is a category $\Cat$ whose objects are categories, and whose morphisms are functors between categories. The identity morphisms in this category are the idenity functors $\id_\cC$, which act as the identity both on objects and on hom spaces. For any two categories $\cC\to\cD$, we can define a catgegory ${\bf Hom}(\cC,\cD)$ whose objects are functors $\cC\to\cD$ and whose morphisms are natural transformations. The identity morphisms in this category are the identity natural transformations $\id_F$ from a functor to itself which acts by the identity map on $F(A)$ for objects $A$.
\end{defn}

\begin{defn} Two categories are called {\em isomorphic} if they are isomorphic in the category of categories (denoted $\cC\cong \cD$), and two functors are called {\em naturally isomorphic} if they are isomorphic in the appropriate category of functors (denoted $F\cong G$). Two categories $\cC$, $\cD$ are called {\em equivalent} if there exists functors $F:\cC\to \cD$ and $G:\cD\to \cC$ such that $G\circ F\cong\id_\cC$ and $F\circ G\cong \id_\cD$ (denoted $\cC\simeq \cD$).
\end{defn}

\begin{rem} If two categories are isomorphic, then they are also equivalent. There are, however, equivalent categories which are not isomorphic. Let $\cC$ be a non-empty category for which between any two objects are is always a unique morphism. Then, by proposition \ref{unique-morphism-contractible}, $\cC$ is equivalent to the category ${\bf 1}$ which is defined to have a unique object labeled $X$ and whose only morphism is $\id_X$. We need not have $\cC\cong {\bf 1}$, however. In particular, if $\cC$ has at least two objects than it cannot be isomorphic to $\bf 1$ since $\bf 1$ has a single objects and isomorphisms of categories must be bijections of sets of objects.

As an example, given objects $A,B\in \cD$ in a category we define a category ${\bf Prod}(A,B)$ as follows. The objects of ${\bf Prod}(A,B)$ are products $(C,f_A,f_B)$ of $A,B$. A morphism between $(C,f_A,f_B)$ and $(D,g_A,g_B)$ is a map $h:C\to D$ such that $f_A=g_A\circ h$ and $f_B=g_B\circ h$. That is, morphisms in ${\bf Prod}(A,B)$ are the morphisms in $\cC$ which respect the additional structure of the triples. By the definition of a product, there is a unique morphism between any two objects in ${\bf Prod}(A,B)$ so by proposition \ref{unique-morphism-contractible} we have ${\bf Prod}(A,B)\simeq {\bf 1}$ whenever the product of $A,B$ exists. Philosophicaly, this language of equivalence says that even though products are not unique, the space of possible products between any two objects is equivalent to a point.
\end{rem}


\begin{prop}\label{unique-morphism-contractible} Let $\cC$ be a non-empty category for there is a unique morphism between any two objects. Then, $\cC\simeq {\bf 1}$.
\end{prop}
\begin{proof}
Define a functor $F:{\bf 1}\to \cC$ by $F(X)=C$, $F(\id_X)=\id_C$ for some $C\in \cC$. Define a functor $G:\cC\to{\bf 1}$ by $G(A)=X$, $G(f)=\id_X$ for all $A,B\in \cC$, $f:A\to B$. It is clear that $G\circ F=\id_{\bf 1}$. Conversely, we can show that $F\circ G\cong \id_{\cC}$. To do this, we can define a natural transformation $\eta:(F\circ G)\to \id_{\cC}$ which acts on each component by the unique morphism with the correct source and target. Namely, for all $A\in \cC$, $\eta$ is the unique morphism from $(F\circ G)(A)=F(X)=C$ to $\id_{\cC}(A)=A$. The relevant diagram for showing that $\eta$ is a natural transformation must commute, because there is a unique morphism between any two objects in $\cC$ and thus every diagram in $\cC$ commutes! Thus, $\eta$ is indeed a natural transformation. We can see that $\eta$ is invertible, it acts by invertible maps on all components (since all maps in $\cC$ are invertible). Thus, $\eta$ is a natural isomorphism as desired, so $\cC\simeq {\bf 1}$.
\end{proof}

\begin{ex} For all categories $\cC$, the functor $i:(\cC^{\op})^{\op}\to \cC$ defined by $(A^{\op})^{\op}\mapsto A$ and $(f^{\op})^{\op}\mapsto f$ is an isomorphism of categories.
\end{ex}

\begin{rem} We can give an alternate definition of the opposite category via a universal property. Given two categories $\cC,\cD$, we define a {\em contravariant functor} $F:\cC\to \cD$ to be an assignment $F:\cC\to \cD$ of objects and $F:\Hom_{\cC}(A,B)\to \Hom_{\cD}(F(B),F(A))$ of morphisms for all $A,B\in \cC$, such that for all $A,B,C\in \cC$, $f:A\to B$, $g:B\to C$,

$$F(g\circ f)=F(f)\circ F(g).$$

Note that a contravariant functor is not a functor. We will sometimes call a standard functor a {\em covariant functor} to highlight that it is not contravariant. We observe that by the definition of the opposite category, every contravariant functor $F:\cC\to \cD$ naturally induces a covariant functor $G:\cC^{\op}\to \cD$ by defining $G(A^{\op})=F(A)$, $G(f^{\op})=F(f)$. Moreover, this basic fact about the opposite category can be stated in terms of a universal property. This goes as follows. There is a canonical contravariant functor $i:\cC\to\cC^{\op}$, which takes $A\to A^{\op}$. This functor has the property that for any category $\cD$ and for any contravariant functor $F:\cC\to \cD$, there exists a unique covaraint functor $G:\cC^{\op}\to\cD$ as defined before, making the diagram

% https://q.uiver.app/#q=WzAsMyxbMCwwLCJcXGNDIl0sWzEsMCwiXFxjQ157XFxvcH0iXSxbMCwxLCJcXGNEIl0sWzAsMiwiRiIsMl0sWzAsMSwiaSJdLFsxLDIsIkciXV0=
\[\begin{tikzcd}
	\cC & {\cC^{\op}} \\
	\cD
	\arrow["i", from=1-1, to=1-2]
	\arrow["F"', from=1-1, to=2-1]
	\arrow["G", from=1-2, to=2-1]
\end{tikzcd}\]

commute. Moreover, for any other pair $(\mathcal{E},i)$ with $\mathcal{E}$ a category $j:\cC\to\mathcal{E}$ a contravariant with the same universal property then there would be a unique isomorphism of categories $I:\mathcal{E}\to \cC^{\op}$ such that $I\circ j = i$. That is, the universal property that $\cC^{\op}$ turns contravariant functors into covariant functors defines $\cC^{\op}$ up to unique isomorphism.
\end{rem}

\begin{ex} Let $\cC$ be a category. For all $A\in \cC$, there is a functor $\Hom_{\cC}(A,\dash):\cC\to\Set$ defined by $\Hom_{\cC}(A,\dash)(B)=\Hom_{\cC}(A,B)$. This map is functorial because any morphism $f:B\to C$ induces a map $\Hom_{\cC}(A,B)\to \Hom_{\cC}(A,C)$ by postcomposition. Similarly, there is a contravariant functor $\Hom_{\cC}(\dash,A)$. If a functor $F:\cC\to\Set$ is of the form $\Hom_{\cC}(A,\dash)$ or $\Hom_{\cC}(\dash,A)$ for some $A\in\cC$, we call it {\em representable}.
\end{ex}

\begin{defn} Let $\cC$, $\cD$ be categories. We define the {\em product category} $\cC\times \cD$ of $\cC$ to $\cD$ to be the category whose objects are pairs $(A,B)$, $A\in \cC$, $B\in \cD$, whose morphisms between $(A,A')$ and $(B,B')$ are pairs $(f,g)$, $f\in \Hom_{\cC}(A,B)$, $g\in \Hom_{\cD}(A',B')$, and whose composition is defined component-wise. It is simple to check that this definition of the product of categories is an abstract product in the category $\Cat$.
\end{defn}

\begin{ex} There is a nice rephrasing of the definition of product and coproduct in terms of representable functors. In particular, we claim that products of $A,B\in \cC$ are in bijective correspondance with objects $A\times B\in \cC$ paired with natural isomorphisms

$$\eta:\Hom(\dash,A\times B)\xrightarrow{\sim}\Hom(\dash,A)\times\Hom(\dash,B).$$

This can be seen as follows. Suppose $\eta$ is a natural isomorphism as above. Then, we can define $(\pi_A,\pi_B)=\eta_{A\times B}(\id_{A\times B})$. We claim that $(A\times B,\pi_A,\pi_B)$ is a product in $\cC$. Suppose that $C\in \cC$, $f_A:C\to A$, $f_B:C\to B$ is another triple. Then, we can define $f=\eta_{C}^{-1}((f_A,f_B))$. By naturality the following diagram commutes:

% https://q.uiver.app/#q=WzAsNCxbMCwwLCJcXEhvbShBXFx0aW1lcyBCLEFcXHRpbWVzIEIpIl0sWzEsMCwiXFxIb20oQVxcdGltZXMgQixBKVxcdGltZXMgXFxIb20oQVxcdGltZXMgQixCKSJdLFswLDEsIlxcSG9tKEMsQVxcdGltZXMgQikiXSxbMSwxLCJcXEhvbShDLEEpXFx0aW1lc1xcSG9tKEMsQikiXSxbMiwzLCJcXGV0YV9DIl0sWzAsMSwiXFxldGFfe0FcXHRpbWVzIEJ9Il0sWzAsMiwiXFxIb20oZixBXFx0aW1lcyBCKSIsMl0sWzEsMywiXFxIb20oZixBKVxcdGltZXNcXEhvbShmLEIpIl1d
\[\begin{tikzcd}
	{\Hom(A\times B,A\times B)} & {\Hom(A\times B,A)\times \Hom(A\times B,B)} \\
	{\Hom(C,A\times B)} & {\Hom(C,A)\times\Hom(C,B)}
	\arrow["{\eta_{A\times B}}", from=1-1, to=1-2]
	\arrow["{\Hom(f,A\times B)}"', from=1-1, to=2-1]
	\arrow["{\Hom(f,A)\times\Hom(f,B)}", from=1-2, to=2-2]
	\arrow["{\eta_C}", from=2-1, to=2-2]
\end{tikzcd}\]

following $\id_{A\times B}$ through this square, we get that $f_A=\pi_A\circ f$ and $f_B=\pi_B\circ f$ as desired. Applying this same procedure in reverse allows one to define a natural transformation $\eta$ given a product $(A\times B,\pi_A,\pi_B)$. Dually, we find that coproducts of $A,B\in \cC$ are in bijective correspondance with objects $A\sqcup B\in \cC$ paired with natural isomorphisms

$$\eta:\Hom(A\sqcup B,\dash)\xrightarrow{\sim}\Hom(A,\dash)\times\Hom(B,\dash).$$
\end{ex}

\subsubsection{Linear categories}

\Note{I need to add the definition of {\em zero object}, so that proposition \ref{fusion-category-simples} makes sense.}

To understand topological quantum information, we will need our categories to be in some real sense quantum mechanical. Quantum systems are always vector spaces over $\bC$. Eventually, the morphisms in our categories will directly correspond to quantum states in certain quantum systems. Thus, we need the hom-spaces in our categories to be vector spaces over $\cC$. This leads us to study the abstract properties of {\em linear categories}.

\begin{defn}[$\bC$-linear category] A $\bC$-linear category is the following data:

\begin{enumerate}
\item A category $\cC$;
\item The structure of a $\bC$-vector space on $\Hom(A,B)$ for all $A,B\in \cC$;

\end{enumerate}

such that the composition maps $\circ:\Hom(B,C)\times \Hom(A,B)\to \Hom(A,C)$ are bilinear maps of vector spaces for all $A,B,C\in \cC$.
\end{defn}

\begin{defn}[$\bC$-linear functor] A $\bC$-linear functor between $\bC$-linear categories $\cC,\cD$ is a functor $F:\cC\to\cD$ such that $F:\Hom_\cC(A,B)\to\Hom_\cD(F(A),F(B))$ is a linear map of vector spaces for all $A,B\in\cC$.
\end{defn}

\begin{rem} When we refer to a functor between two $\bC$-linear categories, we will always assume that that functor is $\bC$-linear unless otherwise stated. In general, whenever two categories are equipped with extra structure we will assume that functors between those categories respect that structure. There is no notion of a $\bC$-linear natural transformation (or, if there were, it would have to be the same as the notion of usual natural transformation).
\end{rem}

\begin{ex} The category $\Vec_{\bC}$ of vector spaces over the complex numbers is  $\bC$-linear category, since the space of linear maps $\Hom_{\Vec_{\bC}}(V,W)$ between any two vector spaces $V,W$ is itself a vector space. Additionally, the product of two $\bC$-linear categories is again a $\bC$-linear category. So, the category $\Vec_{\bC}^n$ of ordered $n$-tuples of vector spaces (equivalent to the product of $n$ copies of $\Vec_{\bC}^n$) is a $\bC$-linear category.
\end{ex}

\begin{rem} A noteworthy feature of $\bC$-linear categories is the existence of {\em zero morphisms}. Suppose that $A,B\in \cC$ are objects in a $\bC$-linear category. There is a distingished morphism $0:A\to B$, called the {\em zero morphism}, corresponding to the zero element of $\Hom(A,B)$ as a vector space. The zero element is uniquely identified by the fact that for all $f:B\to C$ and for all $g:C\to A$,

$$ f\circ 0=0, \qquad\qquad 0\circ g=0.$$

Note that the notation being used is ambiguous, since in the formulas above $0$ refers to both the zero morphism $A\to B$ but also to the zero morphisms $A\to C$ and $C\to B$. The reason that the zero morphism satisfies these properties is that $\circ$ is a bilinear map, and bilinear maps evaluate to zero on any pair which has zero as one of its components.
\end{rem}

\begin{defn}[Biproduct]\label{biproduct-definition} Let $A,B\in \cC$ be objects in a $\bC$-linear category category. A {\em biproduct} (or {\em direct sum}) of $A$ and $B$ is the following data:

\begin{enumerate}
\item An object $A\oplus B\in \cC$;
\item The structure of a product $(A\oplus B,\pi_A,\pi_B)$ on $A\oplus B$;
\item The structure of a coproduct $(A\oplus B,i_A,i_B)$ on $A\oplus B$;
\end{enumerate}

such that:

\begin{enumerate}
\item $\pi_A\circ i_A=\id_A$ and $\pi_B\circ i_B=\id_B$;
\item $\pi_A\circ i_B=0$ and $\pi_B\circ i_A=0$.
\end{enumerate}
\end{defn}

\begin{prop} For all vector spaces $V,W$, the direct sum $A\oplus B$ paired with its projections $(\pi_A,\pi_B)$ onto its components and its inclusions $i_V(v)=(v,0)$, $i_W(v)=(0,v)$ is a biproduct in $\Vec_{\bC}$.
\end{prop}
\begin{proof} We have observed in remarks \ref{product-examples-remark} and \ref{coproduct-examples-remark} that $A\oplus B$ is individually a product and a coproduct. It is immediate that the structures are compatible in the sense of definition \ref{biproduct-definition}, and thus $A\oplus B$ is a biproduct.
\end{proof}

\subsection{Monoidal categories}

\subsubsection{Motivation, definition, and string diagrams}

The goal of this section is to introduce the languages of {\em monoidal categories} and {\em string diagrams}, which are neccecary for a proper algebraic discussion of anyons and modular tensor categories.

The structures in monoidal category theory and their interpretation in terms of string diagrams will allow us to discuss situations like the one below, where we create and braid quasiparticles:

\begin{ex} Suppose that we are braiding quasiparticles in a topological system. In spacetime, the trajectories of the anyons will look something like diagram \ref{anyon-example-picture}. 

\begin{equation}\label{anyon-example-picture}
\tikzfig{category-theory-motivating-example}
\end{equation}

Using the formalism of monoidal categories and the language of string diagrams, we will be able to interpret the above diagram \ref{anyon-example-picture} as a certain morphism in a category. The exact category in which the morphism should live depends on the underlying topological system. The quasiparticle labels $A,B,A^*,B^*$ represent objects in $\cC$.  The objects $(A,A^*)$ form a particle/antiparticle pair, and the objects $(B,B^*)$ form a particle/antiparticle pair. The diagram is broadly interpreted as follows. To begin, there are no particles. Then, we have creation maps $\text{create}_{A,A^*}$ and $\text{create}_{B^*,B}$ which pairs of particles and their antiparticles. Then we have three different braiding operatations ($\text{braid}_{A^*,B^*}$, $\text{braid}_{A,B^*}$, and $\text{braid}_{A^*,B}$) which swap the positions of adjacent quasiparticles. The overall process is the composition of these sub-processes:

$$(\text{braid}_{A^*,B})\circ (\text{braid}_{A,B^*})\circ (\text{braid}_{A^*,B^*}) \circ (\text{create}_{B^*,B})\circ (\text{create}_{A,A^*}).$$

It is thus clear why categories-with-structure are the right framwork for understanding diagrams like diagram \ref{anyon-example-picture}. The additional structures on $\cC$ will give the braiding and creation maps meaning, and the composition structure on $\cC$ will tell us how these components fit together to make larger processes.
\end{ex}

We now begin to introduce the language of string diagrams. The most basic principle os string diagrams is that morphisms are represented as follows:

\begin{equation*}
\tikzfig{basic-morphism-string-diagram}
\end{equation*}

The direction of time going from bottom to up and the space being two-dimensional slices is the same in every diagram, and hence is left implicit from here on out. Composition is expressed cleanly in this langauge as stacking. That is, for all $f:A\to B$, $g:B\to C$, we define

\begin{equation*}
\tikzfig{composition-string-diagram}
\end{equation*}

Accordingly, the identity map has a simple implementation:

\begin{equation*}
\tikzfig{identity-string-diagram}
\end{equation*}

We now give our first major example of adding structure, and how that structure can be interpreted in terms of string diagrams. This structure is that of a {\em monoidal category}. For technical reasons we only define {\em strict} monoidal categories for now - we will come back to the general definition later. Monoidal categories give a way to put objects together. For instance, in diagram \ref{anyon-example-picture} we had four particles all together. We need to way to discuss composite-particle systems. In quantum mechanics, forming a composite system is done by taking the tensor product. Hence, we will use the notation $\otimes$ for joining particles in our current setting. We will even use the term ``tensor product" to discuss it. In general, joining two systems is one way of going from pairs of systems to individual systems:

\begin{align*}
(\text{systems})\times (\text{systems})&\xrightarrow{}(\text{systems}).\\
(\text{system 1}, \text{system 2})&\mapsto (\text{system 1})\otimes (\text{system 2})
\end{align*}

In the world of category theory, we only require some basic properties of this joining. Namely, it should be functorial and satisfy some simple conditions: 

\begin{defn}[Strict monoidal category] A strict monoidal category is the following data:

\begin{enumerate}
\item A category $\cC$;
\item (Tensor product) A functor $\otimes: \cC \times \cC \to \cC$;
\item (Unit) A distinguished element $\bone\in \cC$;
\end{enumerate}

Such that:

\begin{enumerate}
\item (Unit axiom) Let $A,A'\in \cC$ be objects and let $f:A\to A'$ be a morphism. We have

\begin{align*}
A\otimes \bone = \bone \otimes A =A, && f\otimes \id_{\bone} = \id_{\bone}\otimes f = f.
\end{align*}

\item (Associativity) Let $A,B,C,A',B',C'\in\cC$ be objects, and let $f:A\to A'$, $g:B\to B'$, $h:C\to C'$ be morphisms. We have

\begin{align*}
(A\otimes B)\otimes C = A\otimes (B\otimes C), && (f\otimes g)\otimes h = f\otimes (g\otimes h).
\end{align*}
\end{enumerate}


\end{defn}

\begin{rem}
The object $\bone\in \cC$ is important. Just like how groups of symmetries always include the ``do-nothing" symmetry, strict monoidal categories should always include the unit. In this case, $\bone\in \cC$ represents the empty particle - no particle at all. In every particle theory there should be the possibility of not having any particles. Joining the empty particle with any other particle should obviously do nothing, hence the axiom $\bone\otimes A = A\otimes \bone = A$.
\end{rem}

We can now work strict monoidal categories into our graphical language. The tensor product of two objects is represented by putting two lines adjacent to one another. For instance, let $\cC$ be a strict monoidal category, let $A,B,C,D\in \cC$ be four objects, and let $f:A\to C$, $g:B\to D$ be morphisms. We have

\begin{equation*}
\tikzfig{monoidal-string-diagram}
\end{equation*}

The monoidal unit $\bone$ is distinguished in monoidal categories, and hence is represented with a special line. We will either use a dotted line, or no line at all:

\begin{equation*}
\tikzfig{monoidal-unit-string-diagram}
\end{equation*}

\begin{rem} We {\em do not} require that the lines drawn in string diagrams be straight. They can curve any amount so long as it is clear that they are directly connecting an output to an input. The lines cannot cross each other or double back. Additionally, when it is clear from context, we {\em do not} require ourselves to include every label.
\end{rem}

\begin{ex} 
Diagram \ref{diagramatics-example-picture} is valid in all strict monoidal categories $\cC$, where $A$,$B$,$C$,$D$,$E$,$F$,$G$,$H\in \cC$ are objects, and $f:A\otimes B\otimes C \to E\otimes F$, $g: E \to I\otimes G$, $h: F\otimes D\to H$, and $k:G\otimes H \to \bone$ are morphisms:

\begin{equation}\label{diagramatics-example-picture}
\tikzfig{big-example-string-diagram}
\end{equation}

\end{ex}

\subsubsection{Braided monoidal categories}

We continue our definitions of structures on monoidal categories, and their expression in the language of string diagrams. Our next definition is that of a strict braided monoidal category:


\begin{defn}[Strict braided monoidal category] A strict braided monoidal category is the following data:

\begin{enumerate}
\item A strict monoidal category $\cC$;
\item (Braiding) Isomorphisms $\beta_{A,B}: A\otimes B \xrightarrow{} B\otimes A$ for all $A,B\in \cC$ which form a natural isomorphism between the functors $\cC\times \cC\to \cC$ given by $(A,B)\mapsto A\otimes B$ and $(A,B)\mapsto B\otimes A$.
\end{enumerate}

Such that for all $A,B,C\in \cC$, the diagrams

% https://q.uiver.app/#q=WzAsMyxbMCwyLCJCXFxvdGltZXMgQ1xcb3RpbWVzIEEiXSxbMSwxLCJCXFxvdGltZXMgQVxcb3RpbWVzIEMiXSxbMCwwLCJBXFxvdGltZXMgQlxcb3RpbWVzIEMiXSxbMSwwLCJcXGlkX3tCfVxcb3RpbWVzIFxcYmV0YV97QSxDfSJdLFsyLDEsIlxcYmV0YV97QSxCfVxcb3RpbWVzIFxcaWRfe0N9Il0sWzIsMCwiXFxiZXRhX3tBLEJcXG90aW1lcyBDfSIsMl1d
\[\begin{tikzcd}
	{A\otimes B\otimes C} \\
	& {B\otimes A\otimes C} \\
	{B\otimes C\otimes A}
	\arrow["{\beta_{A,B}\otimes \id_{C}}", from=1-1, to=2-2]
	\arrow["{\beta_{A,B\otimes C}}"', from=1-1, to=3-1]
	\arrow["{\id_{B}\otimes \beta_{A,C}}", from=2-2, to=3-1]
\end{tikzcd}
,\qquad
\begin{tikzcd}
	{A\otimes B\otimes C} \\
	& {B\otimes A\otimes C} \\
	{B\otimes C\otimes A}
	\arrow["{\beta_{B,A}^{-1}\otimes \id_{C}}", from=1-1, to=2-2]
	\arrow["{\beta^{-1}_{B\otimes C,A}}"', from=1-1, to=3-1]
	\arrow["{\id_{B}\otimes \beta^{-1}_{C,A}}", from=2-2, to=3-1]
\end{tikzcd}
\]

commute.


\end{defn}

The idea for how to implement braided monoidal categories in the language of string diagrams is to introduce a special symbol for the braiding map $\beta_{A,B}$. Namely, we graphically define overcrossing and undercrossing as follows:

\begin{equation*}
\tikzfig{braiding-definition-string-diagram}
\end{equation*}

\begin{rem}
The fact that overcrossing and undercrossing are related by an inverse encodes the following topological fact:

\begin{equation*}
\tikzfig{over-under-crossing-string-diagram}
\end{equation*}
\end{rem}

We can now describe the conditions on a strict braided monoidal category in a graphical way. The fact that $\beta$ is a natural transformation can be reinterpreted as follows:

\begin{lem} Let $\cC$ be a strict braided monoidal category. For all $A,B,C,D\in \cC$ and $f:A\to C$, $g:B\to D$, we have the following equality of string diagrams:

\begin{equation*}
\tikzfig{braiding-naturality-string-diagram}
\end{equation*}

The same formula holds replacing overcrossing with undercrossing on both sides.
\end{lem}
\begin{proof} Consider the morphism $(f,g):(A,B)\xrightarrow{}(C,D)$ in $\cC\times \cC$. The naturality of $\beta$ implies the following commutative square:


% https://q.uiver.app/#q=WzAsNCxbMCwwLCJBXFxvdGltZXMgQiJdLFsxLDAsIkNcXG90aW1lcyBEIl0sWzEsMSwiRFxcb3RpbWVzIEMiXSxbMCwxLCJCXFxvdGltZXMgQSJdLFswLDMsIlxcYmV0YV97QSxCfSJdLFsxLDIsIlxcYmV0YV97QyxEfSJdLFswLDEsImZcXG90aW1lcyBnIl0sWzMsMiwiZ1xcb3RpbWVzIGYiXV0=
\[\begin{tikzcd}
	{A\otimes B} & {C\otimes D} \\
	{B\otimes A} & {D\otimes C}
	\arrow["{f\otimes g}", from=1-1, to=1-2]
	\arrow["{\beta_{A,B}}", from=1-1, to=2-1]
	\arrow["{\beta_{C,D}}", from=1-2, to=2-2]
	\arrow["{g\otimes f}", from=2-1, to=2-2]
\end{tikzcd}\]

exanding this square in diagramatic language gives the first part of the proposition. Reversing the direction of the arrows by taking inverses gives the second part.
\end{proof}

\begin{rem}
The first coherence axiom can be stated diagrammatically as follows,

\begin{equation*}
\tikzfig{braiding-coherence-string-diagram}
\end{equation*}

and the second coherence axiom can be stated similarly replacing overcrossing with undercrossing. The importance of this axiom is that it means that our graphical langauge can express braid diagrams without other ambiguity. We can safely deform strings behind braids and not need to worry about whether we are applying $\beta_{A,B\otimes C}$ or $(\id_{B}\otimes \beta_{A,C})\circ (\beta_{A,B}\otimes \id_{C})$.
\end{rem}

\begin{rem} When discussing the theory of braiding in subsection \ref{topological-classical-computation}, we discusses braiding operations. It was asserted that two braiding operations are topologically equivalent if and only if they can be manipulated from one to another via manipulations like the one in equation \ref{yang-baxter-condition}. Thus, proposition \ref{yang-baxter proposition} proves that any two topologically equivalent braids will correspond to the same morphism in a braided monoidal category.
\end{rem}

\begin{prop}[Yang-Baxter equation]\label{yang-baxter proposition} Let $\cC$ be a strict braided monoidal category. Let $A,B,C\in \cC$ be objects. We have

\begin{equation}\label{yang-baxter-condition}
\tikzfig{yang-baxter}
\end{equation}

\end{prop} 
\begin{proof} We offer a graphical proof, using first the coherence condition and then naturality:


\begin{equation*}
\tikzfig{yang-baxter-proof}
\end{equation*}
\end{proof}

\begin{cor}\label{braid-group-rep-prop} Let $\cC$ be a strict braided monoidal category. Let $A\in \cC$ be an object. The map

\begin{align*}
B_n &\xrightarrow{} \Aut(A^{\otimes n})\\
\sigma_{i} & \mapsto \id_{A^{\otimes i-1}} \otimes \beta_{A,A}\otimes \id_{A^{n-i-1}}
\end{align*}

is a homomorphism of groups.
\end{cor}
\begin{proof} By definition $B_n$, the only relations we need to verify are $\sigma_{i+1}\sigma_{i}\sigma_{i+1}=\sigma_{i}\sigma_{i+1}\sigma_{i}$ for $1\leq i\leq n-1$. These conditions are satisfied by the map by proposition \ref{yang-baxter proposition}.
\end{proof}

\subsubsection{Examples, equivalences, and MacLane's coherence theorem}

\begin{warn} This section is not neccecary for a conceptual understanding of the subject matter. It is material of technical importance, and thus of interest to those who want a correct formal understanding of the mathematics at play.
\end{warn}

In this section we will give concrete examples of monoidal categories and braided monoidal categories. What we will find, however, is that these examples will all demonstrate the same subtle problem. For example, here is the first category which we would want to give as an example of a monoidal category:

$$\cC=\Set,\,\, \otimes = \text{Cartesian product}.$$

The Cartesian product is certainly functorial. Namely, given morphisms $f:A\to C$ and $g:B\to D$ we get a morphism

\begin{align*}
(f\times g): A\times B &\xrightarrow{} C\times D.\\
(a,b)&\mapsto (f(a), g(b))
\end{align*}

However we get a key issue $(A\times B)\times C \neq A\times (B\times C)$. We have an isomorphism

\begin{align*}
\alpha : (A\times B )\times C &\xrightarrow{} A \times (B\times C),\\
((a,b),c)&\mapsto (a,(b,c))
\end{align*}

but this isomorphism is {\em not} an equality. This means that $\Set$ does not satisfy the definiton of a strict monoidal category! In general, all the examples we would want to give of monoidal categories fail to be strict monoidal categories because the tensor product is not litereally associative. In this section we discuss a method for loosening the definition of monoidal category so that $\Set$ and other examples can be included in the definition.

\begin{rem}
The most naive way of loosening the definition of monoidal category is to only enforce the condition $(A\otimes B)\otimes C\cong A\otimes (B\otimes C)$ instead of equality. However, this leads to a problem. The associativity axiom on morphisms $(f\otimes g)\otimes h = f\otimes (g\otimes h)$  no longer makes sense because there is no way of comparing morphisms on $(A\otimes B)\otimes C$ and $A\otimes (B\otimes )C$. It is for this reason that we need to require specific isomorphisms $\alpha_{A,B,C}:(A\otimes B)\otimes C\xrightarrow{\sim} A\otimes (B\otimes C)$ and require that  those isomorphisms satisfy certain coherence conditions. In general, when looseing equalities to isomorphisms in category theory it is good to posit the existence of specific isomorphisms instead of only forcing that an isomorphism exists.
\end{rem}

\begin{defn}[Monoidal category] A monoidal category is the following data:

\begin{enumerate}
\item A category $\cC$.
\item (Tensor product) A functor $\otimes: \cC \times \cC \to \cC$.
\item (Unit) A distinguished element $\bone\in \cC.$
\item (Associator) A natural isomorphism

$$\alpha: \dash\otimes (\dash \otimes \dash) \xrightarrow{\sim} (\dash\otimes \dash)\otimes \dash , $$

where $\dash \otimes (\dash\otimes \dash)$ denotes the functor $\cC\times \cC\times \cC\to\cC$ sending $(A,B,C)$ to $A\otimes (B\otimes C)$, and similarly for $(\dash \otimes \dash )\otimes\dash$.
\item (Left unitor) A natural isomorphism $\lambda: \bone\otimes \dash \xrightarrow{\sim} \dash$, where $\bone\otimes \dash$ denotes the functor $\cC\to \cC$ sending $A$ to $\bone\otimes A$, and $\dash$ denotes the identity.
\item (Right unitor) A natural isomorphism $\rho: \dash\otimes \bone \xrightarrow{\sim} \dash$, where $\dash\otimes \bone$ is the functor $\cC\to \cC$ sending $A$ to $A\otimes \bone$.
\end{enumerate}

Additionally, a monoidal category is required to satisfy the following properties:

\begin{enumerate}
\item (Triangle identity) The diagram

\[\begin{tikzcd}
	{} & {} & {\left(A\otimes \bone\right)\otimes B} & {} & {A\otimes (\bone\otimes B)} \\
	&& {} & {A\otimes B} \\
	&&&& {}
	\arrow["{\alpha_{A,1,B}}", from=1-3, to=1-5]
	\arrow["{\rho_A\otimes \id_B}"', from=1-3, to=2-4]
	\arrow["{\id_A\otimes \lambda_B}", from=1-5, to=2-4]
\end{tikzcd}\]

commutes for all $A,B\in \cC$.

\item (Pentagon identity) The diagram

\[\begin{tikzcd}
	& {(A\otimes B)\otimes(C\otimes D)} \\
	{((A\otimes B)\otimes C)\otimes D} && {A\otimes (B\otimes (C\otimes D))} \\
	{(A\otimes (B\otimes C))\otimes D} && {A\otimes((B\otimes C)\otimes D)}
	\arrow["{\alpha_{A\otimes B, C,D}}", from=2-1, to=1-2]
	\arrow["{\alpha_{A,B,{C\otimes D}}}"', from=2-3, to=1-2]
	\arrow["{\alpha_{A,B,C}\otimes \id_D}"', from=2-1, to=3-1]
	\arrow["{\id_A\otimes_{B,C,D}}"', from=3-3, to=2-3]
	\arrow["{\alpha_{A,B\otimes C,D}}"', from=3-1, to=3-3]
\end{tikzcd}\]

commutes for all $A,B,C,D\in \cC$.
\end{enumerate}
\end{defn}

\begin{ex}
The following collections of data form monoidal categories

\begin{itemize}
\item The category $\cC=\Set$, with tensor product $\otimes = \text{Cartesian product}$, monoidal unit $\bone =\{*\}$, associator

\begin{align*}
\alpha_{A,B,C}: A\times (B\times C) &\xrightarrow{\sim}(A\times B)\times C,\\
(a,(b,c))&\mapsto ((a,b),c)
\end{align*}

and unitors 

\begin{align*}
\lambda: \bone \otimes A &\to A && \rho:  A\otimes \bone \to A.\\
(*, a)&\mapsto a && (a,*)\mapsto a
\end{align*}

\item The plain category $\cC=\Vec_{\bC}$, with its standard tensor product, monoidal unit $\bone =\bC$, associator

\begin{align*}
\alpha_{A,B,C}: A\times (B\times C) &\xrightarrow{\sim}(A\times B)\times C,\\
a\otimes (b\otimes c ) & \mapsto (a\otimes b)\otimes c
\end{align*}

and unitors 

\begin{align*}
\lambda: \bone \otimes A &\to A && \rho:  A\otimes \bone \to A.\\
1\otimes a &\mapsto  a && a\otimes 1 \mapsto a
\end{align*}

\item The category $\cC=\Set$ with tensor product $\otimes=\text{Disjoint union}$ and $\bone=\{\}$, with a standard choice of associators and unitors;

\item The category $\cC=\Vec_{\bC}$ with tensor product $\otimes = \text{Direct sum}$, and $\bone = 0$, with a standard choice of assoicators and unitors.
\end{itemize}
\end{ex}

\begin{rem}\label{no-string-diagrams}
In expanding our definition from strict monoidal category to monoidal category we have introduced a subtle problem. The diagram

\begin{equation*}
\tikzfig{three-strand-identity}
\end{equation*}

no longer makes sense! The map $\id_{A\otimes B\otimes C}$ no longer exists, because $A\otimes B \otimes C$ no longer exists. One must make a choice of $(A\otimes B)\otimes C$ or $A\otimes (B\otimes C)$. These maps may be isomorphic, but they have no need to be equal! The correct diagram would be

\begin{equation*}
\tikzfig{associativity-example}
\end{equation*}

In general, string diagrams for non-strict monoidal categories need $\alpha$ maps thrown in at key points to make a well-defined language. This is quite complicated, and has issues that need to be adressed. Hence, we maintain that our graphical langauge only applies to strict monoidal categories.
\end{rem}

\begin{rem}\label{monoidal-category-workflow}
In light of remark \ref{no-string-diagrams}, we seem to have made very little progress. We defined the notion of a non-strict monoidal category so that we could include our favorite examples, but then we observed that string diagrams fail to describe those examples!  This seemingly bad situation is rectified by theorem \ref{maclane-coherence-theorem}, which we first state informally.

\begin{itemize}
\item MacLane's cohrence theorem: {\em every monoidal category is equvialent to a strict monoidal category}.
\end{itemize}

This gives us a workflow for the book. We will frame our discussion so that it applies to arbitrary monoidal categories. That way, all our usual examples are included. Then, when we want to use string diagrams, we use MacLane's coherence theorem to pass to an equivalent strict category, in which our diagrams make sense. Then, when we are done using the diagram, we pass the conclusion of the argument through the equivalence! We will be using this subtle technique repeatedly throughout the book. To save time and energy, we won't explicitely mention it. We will implicitely pass to an equivalent strict category without making any special note. Sometimes we will want to pass to a strict monoidal category even before string diagrams come into play. Alternatively one can adopt the following simpler policy, with the price of making the usual examples only heuristic:

\begin{center}
\fbox{We assume monoidal categories are strict whenever it is convenient.}
\end{center}
\end{rem}

\begin{ex} To illustrate the workflow proposed in \ref{monoidal-category-workflow} we take a closer look at corollary \ref{braid-group-rep-prop}, where we proved that every strict braided monoidal category $\cC $ comes paired with a group homomorphism

$$B_n \xrightarrow{} \Aut\left(A^{\otimes n}\right)$$

for all $A\in \cC$, $n\geq 1$. Once we generalize strict braided monoidal categories to possibly non-strict braided monoidal categories, this proposition will become false. The object $A^{\otimes n}$ does not exist - a choice of parenthesization needs to be made. Every time that an element of the braid group acts on $A^{\otimes n}$, the parentheses need to be re-arranged using associators, then the braiding map $\beta$ can be applied, and then the parentheses need to be re-arranged back into their original position using associators again. It is possible to formalize corollary \ref{braid-group-rep-prop} for non-strict categories, but it is space-consuming and makes the key insights less clear.
\end{ex}

\begin{rem}
To state MacLane's coherence theorem, we need a notion of {\em equivalence} of monoidal categories. Our notion of equivalence is modeled after the more general notation of equivalence of categories - a pair of functors whose compositions are both naturally isomorphic to the identity. To translate to the present setting, we need a good notion of monoidal functor and monoidal natural transformation so that equivalence can preserve information about monoidal structure.
\end{rem}

\begin{defn}[Monoidal functor] A monoidal functor between monoidal categories $(\cC,\otimes_{\cC}, \alpha_{\cC},\lambda_{\cC},\rho_{\cC},\bone_{\cC})$ and $(\cD,\otimes_{\cD},\alpha_{\cD},\lambda_{\cD},\rho_{\cD},1_{\cD})$ is the following data:

\begin{enumerate}
\item A functor $F: \cC\to \cD$.
\item A morphism $\epsilon:1_{\cD}\to F(1_{\cC})$.
\item A natural isomorphism $\mu: F(\dash)\otimes_{\cD}F(\dash)\xrightarrow{\sim}F(\dash\otimes_{\cC}\dash)$.
\end{enumerate}

Additionally, a monoidal functor is required to satisfy the following properties:

\begin{enumerate}
\item (Associativity) The diagram

\[\begin{tikzcd}
	{(F(A)\otimes_{\cD}F(B))\otimes_{\cD}F(C)} &&& {F(A)\otimes_{\cD}(F(B)\otimes_{\cD}F(C))} \\
	{F(A\otimes_{\cC}B)\otimes_{\cD}F(C)} &&& {F(A)\otimes_{\cD}F(B\otimes_{\cC}C)} \\
	{F((A\otimes_{\cC} B)\otimes_{\cC}C)} && {} & {F(A\otimes_{\cC}(B\otimes_{\cC}C))}
	\arrow["{\mu_{A,B}\otimes \id_{F(C)}}", from=1-1, to=2-1]
	\arrow["{\mu_{A\otimes_{\cC}B,C}}", from=2-1, to=3-1]
	\arrow["{\mu_{A,B\otimes_{\cC}C}}", from=2-4, to=3-4]
	\arrow["{\id_{F(A)}\otimes\mu_{B,C}}", from=1-4, to=2-4]
	\arrow["{F(\alpha_{\cC;A,B,C})}"{description}, from=3-1, to=3-4]
	\arrow["{\alpha_{\cD;F(A),F(B),F(C)}}"{description}, from=1-1, to=1-4]
\end{tikzcd}\]

commutes for all $A,B,C\in \cC$.

\item (Unitality) The diagrams

\[\begin{tikzcd}
	{1_{\cD}\otimes_{\cD}F(A)} && {F(1_{\cC})\otimes F(A)} \\
	{F(A)} && {F(1_{\cC}\otimes A)}
	\arrow["{\lambda_{\cC;F(A)}}", from=1-1, to=2-1]
	\arrow["{F(\lambda_{\cC;A})}"', from=2-3, to=2-1]
	\arrow["{\mu_{1_{\cC},A}}"', from=1-3, to=2-3]
	\arrow["{\epsilon\otimes \id_{F(A)}}"', from=1-1, to=1-3]
\end{tikzcd}\]

and

\[\begin{tikzcd}
	{F(A)\otimes_{\cD}1_{\cD}} && {F(A)\otimes_{\cD}F(1_{\cC})} \\
	{F(A)} && {F(1_{\cC}\otimes A)}
	\arrow["{\rho_{\cC;F(A)}}", from=1-1, to=2-1]
	\arrow["{F(\rho_{\cC;A})}"', from=2-3, to=2-1]
	\arrow["{\mu_{A,1_{\cC}}}"', from=1-3, to=2-3]
	\arrow["{\id_{F(A)}\otimes\epsilon}"', from=1-1, to=1-3]
\end{tikzcd}\]

commute for all $A\in \cC$.
\end{enumerate}
\end{defn}


\begin{defn}[Monoidal natural transformation] A monoidal natural transformation between two monoidal functors $(F_0,\mu_0,\epsilon_0)$ and $(F_1,\mu_1,\epsilon_1)$ between monoidal categories $(\cC,\otimes_{\cC},\bone_{\cC})$ and $(\cD,\otimes_{\cD},1_{\cD})$ is a natural transformation $\eta$ between the underlying functors $F_0,F_1$. Additionally, a monoidal natural transformation is required to satisfy the following properties:

\begin{enumerate}
\item (Compatibility with tensor product) For all objects $A,B\in \cC$, the diagram

\[\begin{tikzcd}
	{F_0(A)\otimes_{\cD}F_1(B)} & {F_1(A)\otimes_{\cD}F_1(B)} \\
	{F_0(A\otimes_{\cC} B)} & {F_1(A\otimes_{\cC} B)}
	\arrow["{\mu_{0;A,B}}", from=1-1, to=2-1]
	\arrow["{\mu_{1;A,B}}", from=1-2, to=2-2]
	\arrow["{\eta_A\otimes \eta_B}", from=1-1, to=1-2]
	\arrow["{\eta_{A\otimes B}}", from=2-1, to=2-2]
\end{tikzcd}\]

commutes.

\item (Compatibility with unit) The diagram

\[\begin{tikzcd}
	& {1_{\cD}} \\
	{F_0(1_{\cC})} && {F_1(1_{\cC})}
	\arrow["{\eta_{1_{\cC}}}", from=2-1, to=2-3]
	\arrow["{\epsilon_0}"', from=1-2, to=2-1]
	\arrow["{\epsilon_1}", from=1-2, to=2-3]
\end{tikzcd}\]

commutes.
\end{enumerate}
\end{defn}

\begin{defn}
A {\em monoidal equivalence} between two monoidal categories $\cC,\cD$ is a pair of monoidal functors $F:\cC\to \cD$, $G:\cD\to \cC$ such that $G\circ F$ is monoidally naturally isomorphic to $\id_{\cC}$ and $F\circ G$ is monoidally naturally isomorphic to $\id_{\cD}$. We say two categories are {\em monoidally equivalent} if there is a monoidal equivalence between them.
\end{defn}

\begin{thrm}[MacLane's coherence theorem]\label{maclane-coherence-theorem} Every monoidal category is monoidally equivalent to a strict monoidal category.
\end{thrm}

\begin{rem}
MacLane's coherence theorem allows us to enjoy a managable string diagram workflow for monoidal categories. However, as we add more structure onto our categories, it will be a non-trivial task to verify that we can still apply MacLane's coherence theorem. In particular, we will need to strengthen our notion of equivalence to make sure it is strong enough to pass through information about our additional structures. We can see this in the case of braidings already - if we have a braided monoidal category whose associativity is non-strict, will it be equivalent to a braided monoidal category whose associativity is strict? The answer is yes, by proposition \ref{braided-coherence-theorem}.
\end{rem}


\begin{defn}[Braided monoidal category] A braided monoidal category is the following data:

\begin{enumerate}
\item A monoidal category $(\cC,\otimes,\alpha, \bone)$.
\item (Braiding) A natural isomorphism $\beta$ between the functor $\cC\times \cC\to \cC$ sending $(A,B)$ to $A\otimes B$, and the functor sending $(A,B)$ to $B\otimes A$.
\end{enumerate}

Additionally, a braided monoidal category is required to satisfy the following properties:

\begin{enumerate}
\item (Hexagon identities) The diagrams

\[\begin{tikzcd}
	{A\otimes (B\otimes C)} && {(A\otimes B)\otimes C} && {C\otimes (A\otimes B)} \\
	{A\otimes (C\otimes B)} && {(A\otimes C)\otimes B} && {(C\otimes A)\otimes B}
	\arrow["{\alpha_{A,B,C}}", from=1-1, to=1-3]
	\arrow["{\beta_{A\otimes B,C}}", from=1-3, to=1-5]
	\arrow["{\alpha^{}_{B,C,A}}", from=1-5, to=2-5]
	\arrow["{\id_A\otimes \beta_{B,C}}"', from=1-1, to=2-1]
	\arrow["{\alpha_{A,C,B}}"', from=2-1, to=2-3]
	\arrow["{\beta_{A,C}\otimes \id_B}"', from=2-3, to=2-5]
\end{tikzcd}\]

and

\[\begin{tikzcd}
	{(A\otimes B)\otimes C} && {A\otimes (B\otimes C)} && {(B\otimes C)\otimes A} \\
	{(B\otimes A)\otimes C} && {B\otimes (A\otimes C)} && {B\otimes (C\otimes A)}
	\arrow["{\alpha^{-1}_{A,B,C}}", from=1-1, to=1-3]
	\arrow["{\beta_{A,B\otimes C}}", from=1-3, to=1-5]
	\arrow["{\alpha^{-1}_{B,C,A}}", from=1-5, to=2-5]
	\arrow["{\beta_{A,B}\otimes \id_C}"', from=1-1, to=2-1]
	\arrow["{\alpha^{-1}_{B,A,C}}"', from=2-1, to=2-3]
	\arrow["{\id_B\otimes \beta_{A,C}}"', from=2-3, to=2-5]
\end{tikzcd}\]

commute for all $A,B,C\in \cC$.
\end{enumerate}


\end{defn}

\begin{defn}[Braided monoidal functor] A braided monoidal functor between braided monoidal categories $(\cC,\otimes_{\cC},\beta_{\cC})$, $(\cD,\otimes_{\cD},\beta_{\cD})$ is a monoidal functor $(F,\mu):\cC\to \cD$ such that the diagram

\[\begin{tikzcd}
	{F(A)\otimes_{\cD}F(B)} && {F(B)\otimes_{\cD}F(A)} \\
	\\
	{F(A\otimes_{\cC}B)} && {F(B\otimes_{\cC}A)}
	\arrow["{\mu_{A,B}}", from=1-1, to=3-1]
	\arrow["{\beta_{\cD;F(A),F(B)}}"', from=1-1, to=1-3]
	\arrow["{\mu_{B,A}}"', from=1-3, to=3-3]
	\arrow["{F(\beta_{\cC;F(A),F(B)})}", from=3-1, to=3-3]
\end{tikzcd}\]

commutes for all $A,B\in \cC$.
\end{defn}

\begin{rem}
There is no notion of braided monoidal natural transformation - any monoidal natural transformation will automatically repsect the braiding. Hence, we can define two braided monoidal categories to be equivalent if there are braided monoidal functors between them which have compositions which are naturally isomorphic to the identity.
\end{rem}


\begin{prop}[Braided MacLane coherence theorem]\label{braided-coherence-theorem} Every braided monoidal equivalent is equivalent as a braided monoidal category to a strict braided monoidal category.
\end{prop}
\begin{proof} Let $\cC$ be a braided monoidal category. Let $\cC'$ be a strict monoidal category equivalent to $\cC$ (which exists by theorem \ref{maclane-coherence-theorem}), and let $F:\cC\to\cC'$, $G:\cC'\to \cC$ be monoidal functors inducing the equivalence. Let $\eta:F\circ G\cong \id_{\cC}$ be a monoidal natural isomorphism. We define a natural transformation $\beta'$ on the $A,B$ component by the followiing composition:

% https://q.uiver.app/#q=WzAsNixbMCwwLCJBXFxvdGltZXMgQiJdLFsxLDAsIkYoRyhBXFxvdGltZXMgQikpIl0sWzIsMCwiRihHKEEpXFxvdGltZXMgRyhCKSkiXSxbMiwxLCJGKEcoQilcXG90aW1lcyBHKEEpKSJdLFsxLDEsIkYoRyhCXFxvdGltZXMgQSkpIl0sWzAsMSwiQlxcb3RpbWVzIEEiXSxbMCwxLCJcXGV0YV57LTF9X3tBXFxvdGltZXMgQn0iXSxbMSwyLCJGKFxcbXVeey0xfV97RztBLEJ9KSJdLFsyLDMsIkYoXFxiZXRhX3tHKEEpLEcoQil9KSJdLFszLDQsIkYoXFxtdV97RztCLEF9KSJdLFs0LDUsIlxcZXRhX3tCXFxvdGltZXMgQX0iXV0=
\[\begin{tikzcd}
	{A\otimes B} & {F(G(A\otimes B))} & {F(G(A)\otimes G(B))} \\
	{B\otimes A} & {F(G(B\otimes A))} & {F(G(B)\otimes G(A))}
	\arrow["{\eta^{-1}_{A\otimes B}}", from=1-1, to=1-2]
	\arrow["{F(\mu^{-1}_{G;A,B})}", from=1-2, to=1-3]
	\arrow["{F(\beta_{G(A),G(B)})}", from=1-3, to=2-3]
	\arrow["{\eta_{B\otimes A}}", from=2-2, to=2-1]
	\arrow["{F(\mu_{G;B,A})}", from=2-3, to=2-2]
\end{tikzcd}\]

It is a straightforward to show that the axioms of a braided monoidal category on $\cC$, the axioms of a monoidal functor on $F,G$ and the axioms of a natural transformation on $\eta$ imply that $\beta'$ is the structure of a braiding on $\cC'$. Moreover, the monoidal equivalence of categories between $\cC$, $\cC'$ is a braided monoidal equivalence.
\end{proof}

\begin{rem}
As we go through this text, we will define increasingly more structure on monoidal categories. We will be implicitely assuming theorems which assert that every structured monoidal categories is equivalent to a structure monoidal category whose underlying monoidal category is strict. Importantly, we will assume that this equivalence respects the relevant structure. We will not state these theorems as we go along the way, but they are true and neccecary for our discussion.
\end{rem}


\subsubsection{Pivotal monoidal categories}

So far we have defined a language for putting particles together and braiding them. The next frontier is to introduce a langauge for creating and fusing particles/antiparticles. Categories with a mechanism for creating and fusing particles/antiparticles is known as a {\em pivotal monoidal category}.

\begin{rem}
In the relevant physical systems, every particle has a dual {\em antiparticle}. Particle/antiparticle pairs can always spontaneously be created from the vaccuum. Often, particles/antiparticles can annhilate each other to go back to the vaccum. This process of annhilation is subtle however, because a particle/antiparticle pair could also fuse to make a particle which is not the vacuum. We delay the subtleties of fusion to our chapter on modular tensor categories, and focus instead on the abstract underpinnings of antiparticles (which we call {\em duals}), pair-creation (which we call {\em coevaluation}) and fusion (which we all {\em evaluation}).
\end{rem}

\begin{defn}[Right-rigid monoidal category] A right-rigid monoidal category is the following data:

\begin{enumerate}
\item A monoidal category $\cC$.
\item Objects $A^*$ for all $A\in \cC$.
\item Morphisms $\ev_{A}: A\otimes A^*\to 1$, and $\coev_{A}: 1\to A^*\otimes A$ for all $A\in \cC$.
\end{enumerate}

Such that $(\ev_A \otimes \id_A)\circ (\id_A\otimes \coev_A)=\id_A$ and $(\id_{A^*}\otimes \ev_A)\circ (\coev_{A}\otimes \id_{A^*})=\id_{A^*}$ for all $A\in \cC$. 


\end{defn}

We implement right-rigid monoidal categories in string diagrams as follows. We denote evaluation and coevaluation as follows:

\begin{equation*}
\tikzfig{eval-coeval-definition}
\end{equation*}

The compatibility conditions are stated graphically as follows:

\begin{equation*}
\tikzfig{eval-coeval-coherence}
\end{equation*}

\begin{defn}[Left-rigid monoidal category] A left-rigid monoidal category is the following data:

\begin{enumerate}
\item A monoidal category $\cC$.
\item Objects $A^*$ for all $A\in \cC$.
\item Morphisms $\ev_{A}: A^*\otimes A\to 1$, and $\coev_{A}: 1\to A\otimes A^*$ for all $A\in \cC$.
\end{enumerate}

Additionally, a rigid category is required to satisfy the property that $(\id_A\otimes\ev_A)\circ (\coev_A\otimes \id_A)=\id_A$ and $(\ev_A\otimes \id_{A^*})\circ (\id_{A^*}\otimes \coev_{A})=\id_{A^*}$ for all $A\in \cC$. 


\end{defn}

\begin{rem} We want to discuss categories which have a full theory of particles/antiparticles. This means that they should be able to create particle/antiparticle pairs on both sides, leading to a left-rigid and right-rigid structure on $\cC$. As per usual, there should be some compatibility conditions between these two rigid structures.
\end{rem}

\begin{defn}[Pivotal monoidal category] A pivotal monoidal category is the following data:

\begin{enumerate}
\item A monoidal category $\cC$;
\item A right-rigid structure $(\ev^R, \coev^R)$ on $\cC$;
\item A left-rigid structure $(\ev^L, \coev^L)$ on $\cC$.
\end{enumerate}

Such that:

\begin{enumerate}
\item The right-duals and left-duals of all objects are equal;
\item For all $A,B\in \cC$, we have an equality of morphisms $B^*\otimes A^*\xrightarrow{} (A\otimes B)^*$,

\begin{equation*}
\tikzfig{pivotal-coherence-1}
\end{equation*}

\item For all $A,B\in \cC$ and $f:A\to B$,

\begin{equation*}
\tikzfig{morphism-duals-agree}
\end{equation*}
\end{enumerate}


\end{defn}

\begin{rem}
The first thing to observe is that even though there is a lot of structure involved in the definition of a rigid monoidal category, most of it is in a real sense innessential. Proposition \ref{rigidity} tells us we could have chosen different duals and the result would have been essentially the same, or in other words, duals are unique up to unique isomorphism.
\end{rem}

\begin{prop}\label{rigidity} Let $\cC$ be right (resp. left) rigid monoidal category. Let $A\in \cC$ be an object, and let $(\tilde{A}^*, \tilde{\ev}_{A}, \tilde{\coev}_{A})$ be another triple satisfying the axioms of rigidity. There is a unique morphism $i: A^*\xrightarrow{} \tilde{A}^*$ making the diagram

\[\begin{tikzcd}
	& {A^{*}\otimes A}\\
	1 \\
	& {A\otimes \tilde{A}^*}
	\arrow["{\coev_A}", from=2-1, to=1-2]
	\arrow["{\tilde{\coev}_A}"', from=2-1, to=3-2]
	\arrow["\sim", from=1-2, to=3-2]
\end{tikzcd}\]

commute (resp. reverse order of tensor factors). This unique morphism is an isomorphism, and it is given by

\begin{equation*}
\tikzfig{unique-rigidity-morphism}
\end{equation*}
\end{prop}
\begin{proof} By the computation

\begin{equation*}
\tikzfig{rigidity-iso-uniqueness-proof}
\end{equation*}

we find that $i$ is unique, and it has the desired formula. To prove that $i$ is an isomorphism we observe that the map

\begin{equation*}
\tikzfig{unique-rigidity-morphism-dual}
\end{equation*}

serves as an inverse. This gives a proof of the result.
\end{proof}
 
\begin{rem} Seeing as we will be working with rigid and pivotal categories, it behooves us to make sure that we have the correct notion of functors and natural transformations between these categories. Let $F:\cC\to \cD$ be a functor between pivotal categories. Given an object $A\in \cC$, the evaluation and coevaluation maps naturally extend through the functor to endow $F(A^*)$ with the structure of a dual for $A$. Thus, by proposition \ref{rigidity}, we have a canonical isomorphism $F(A^*)\cong F(A)^*$. This isomorphism exists without needing to add any extra conditions on $F$. In this way, the correct notion of functor between right/left rigid categories is just functor! There is, however, extra an compatibility condition needed for pivotal category. Both the left-rigid {\em and} right-rigid structures induce isomorphisms $F(A^*)\cong F(A)^*$. These induced isomorphisms should be the same. A functor between pivotal categories with this property is known as a {\em pivotal functor}.
\end{rem}

\begin{defn} Define a monoidal category $\overline{\cC}$ as follows. The underlying category on $\overline{\cC}$ is the opposite category for $\cC$. The tensor product is given by $A\overline{\otimes} B = B\otimes A$, and the monoidal unit is $\bone \in \cC$. \Note{I'm not sure if I like this definition, notation, or location.}
\end{defn}

\begin{rem} An important feature of rigid categories is that duality is automatically {\em functorial}, by proposition \ref{rigidity-functorial}. This perspective can be used to motivate the axioms of a pivotal category. By proposition \ref{rigidity-functorial} both the right and left rigid structures in a pivotal category induce functors $\cC\to \overline{\cC}$. The coherence condition is that these two functors should be equal.
\end{rem}

\begin{prop}\label{rigidity-functorial} Let $\cC$ be right (resp. left) rigid monoidal category. 

\begin{enumerate}[(i)]
\item The right (resp.left) rigid structure on $\cC$ induces a left (resp. right) rigid structure on $\overline{\cC}$;

\item Given any morphism $f: A\to B$ in $\cC$, define

\begin{equation*}
\tikzfig{rigidity-functor}
\end{equation*}

to be the dual for $f$ (resp. same diagram using left rigidity). The assignment $A\mapsto A^*$, $f\mapsto f^*$ induces a functor from $\cC$ to $\overline{\cC}$ which we denote $(\dash)^{*}$.

\item Given any objects $A,B\in \cC$, define the map

\begin{equation*}
\tikzfig{rigidity-functor-monoidal}
\end{equation*}

from $B^*\otimes A^*$ to $(A\otimes B)^*$ (resp. same diagram using left rigidity). These maps endows $(\dash)^{*}$ with the structure of a monoidal functor.

\item The functor $(\dash)^*$ is fully faithful. If $\cC$ is a pivotal category, then then functor above is an equivalence of monoidal categories between $\cC$ and $\overline{\cC}$.

\end{enumerate}
\end{prop}
\begin{proof} We do only  the proofs for right-rigid categories. The left-rigid proof is identitical.

\begin{enumerate}[(i)]
\item This follows immediately from the definitions;

\item Functoriality is the condition that $(f\circ g)^{*}=g^{*}\circ f^{*}$. The follows from the following argument:

\begin{equation*}
\tikzfig{rigidity-functor-proof}
\end{equation*}

\item This is an unenlightening and straightforward computation;

\item  We first prove that $(\dash)^*$ is fully faithful. Given any objects $A,B\in \cC$ and any morphism $g:B^*\to A^*$, the morphism

\begin{equation*}
\tikzfig{rigidity-functor-reverse}
\end{equation*}

has the property that $f^*=g$. Hence, $(\dash)^*$ is bijective on hom-spaces as desired.

We now show that $(\dash)^*$ is an equivalence of categories with $\cC$ is pivotal.  By part $(i)$, $\overline{\cC}$ is a pivotal monoidal category. Hence duality once again induces a monoidal functor, this time $\overline{\cC}\to\overline{\overline{\cC}}$. Clearly, by our definition of $\overline{\cC}$, $\overline{\overline{\cC}}=\cC$. Hence we have a pair of functors $\cC\to \overline{\cC}$ and $\overline{\cC}\to\cC$, each given by duality. Proving this proposition hence amounts to showing that the double dual map is monoidally naturally isomorphic to the identity.

To do this, we define a natural isomorphism explicitely by the isomorphisms $i:A\xrightarrow{\sim}A^{**}$

\begin{equation*}
\tikzfig{double-dual-natural-isomorphism}
\end{equation*}

for all $A\in \cC$. To show that these morphisms induce a natural transformation, we observe that for all $f:A\to B$

\begin{equation*}
\tikzfig{pivotal-naturality}
\end{equation*}

The fact that $\cC$ is compatible with the tensor product is a straightforward computation, using the fact that computing the tensor product using right-rigidity and left-rigidity gives the same answer, and compatibility of $\cC$ with the unit is immediate.

\end{enumerate}
\end{proof}

\begin{rem}
As a key part of proposition \ref{rigidity-functorial}, we showed that every pivotal structure on a right-rigid monoidal category induces a natural isomorphism between the identity functor and the double dual functor. This gives an alternate description of pivotal categories as rigid categories equipped with isomorphisms between the identity functor and the double dual functor. This is stated precisely in corollary \ref{pivotal-alternate}.
\end{rem}

\begin{cor}\label{pivotal-alternate} Let $\cC$ be a right-rigid monoidal category. Let $i:\id_{\cC}\xrightarrow{\sim}(\dash)^{**}$ be a monoidal natural isomorphisms between the identity functor and the double dual functor. The maps

\begin{equation*}
\tikzfig{pivotality-condition-maps}
\end{equation*}

induce a pivotal structure on $\cC$. Moreover, this assignement induces a bijection between pivotal structures on $\cC$ and monoidal natural isomorphisms $\id_{\cC}\xrightarrow{\sim}(\id_{\cC})^{**}$.
\end{cor}
\begin{proof} Proving that the maps provided satisfy the axioms of a left-rigid structure is immediate. Proving that they satisfy the axioms of a pivotal structure comes from running the arguments in the proof of proposition \ref{rigidity-functorial} in reverse. The operations of inducing a monoidal natural isomorphism from a pivotal structure and inducing a pivotal structure from a monoidal natural isomorphism are inverses of one another. Hence, they induce a bijection between the two types of structures as desired.
\end{proof}

$\newline$
\fbox{\parbox{\dimexpr\linewidth-2\fboxsep-2\fboxrule\relax}{

\begin{center}
\textbf{History and further reading:}\\
\end{center}

\Note{ an early reference for string diagrams is \cite{moussouris1984quantum}. Maybe I should cite it?}

Category theory was first introduced and formalized by Saunders Mac Lane and Samuel Eilenberg in 1945 \cite{eilenberg1945general}. Of course, the ideas underlying category theory were present earlier and can be traced back arbitrarily far. In the subsequent decades the formalism of category theory spread far and wide, bringing with it the discovery of many deep theorems. The first major explicit appearance of category theory in physics was Vladimir Drinfeld's work on so-called {\em quantum groups} in the early 1980s \cite{drinfeld1986quantum}. Quantum groups are certain kinds of mathematical objects righly related to content in this book. They were introduced as tools to help generate exactly-solvable models in condensed matter physics. Very quickly quantum groups were absorbed into the theory of the ideas of string theory of topological quantum field theory, which were both new at the time \cite{belavin1984infinite, witten1988topological}. The physics in this area has since become and remained extremely categorical in nature \cite{lurie2008classification, bartlett2015modular}.

$\newline$
There are many excellent introductory texts to category theory. Some authors find it fruitful to reformulate all of quantum mechanics, and especially quantum information, in terms of category theory. A good source outlining this approach and introducing category theory through it is Coecke-Kissinger's textbook \cite{coecke2018picturing}. The Kong-Zhang textbook \cite{kong2022invitation} gives an introduction to category theory in the context of topological order. A good general-purpose textbook on category theory is Fong-Spivak \cite{fong2019invitation}, and a classical but slightly dated reference is \cite{mac2013categories}.
}}


$\newline\newline$

\large \textbf{Exercises}:\normalsize

\begin{enumerate}[\thesection .1.]

\item \Note{ If $\cC$ is a category with products, then the product forms a monoidal structure (with a good $\bone$ given of course), and same for coproducts.}

\item \Note{ Show that endofunctors form a {\em strict} monoidal category.}

\item \Note{ Add an exercise giving some compatibility conditions between monoidal/rigid structures and direct sums. Namely, they distribute nicely.}

\Note{I'm not consistent about up/down orientation for my string diagrams. I need to go through and fix this.}

\end{enumerate}