\section{Preface}
\label{Preface}

This book is a mathematical treatment of topological quantum information, with a focus on formal algebraic aspects and a special eye towards topological quantum computation. This manuscript began as an extended set of notes from a course on topological quantum field theory given by Zhenghan Wang in the winter of 2022 at UC Santa Barbara. Through his courses, his private tutoring, and his reccomendations, Zhenghan took me from a state of almost complete ignorance of mathematical physics to being a young researcher in the field. I am greatly emdebted to him for this, and it is certain that this book would not have existed without his guidance - he richly deserves of my apple.

Great pains have been taken to make this book as pedagogical and accessable as possible. The hope is that it should be readable by both mathematicians unfamilar with quantum mechanics as well as theoretical physicists unfamiliar with category theory. A primary focus of this text is balancing powerful algebraic generalities with concrete examples, principles, and applications. The prerequisites for this book are a undergraduate-level understanding of topology, linear algebra, and group theory, as well as a popular-science level of familarity with quantum mechanics.

There are already many great references to learn aspects of the material covered in this book. An excellently written and relatively complete book on topological quantum information from the perspective of a physicist is Steven Simon's text \cite{simon2023topological}. Simon's book is algebraic, but does \textit{not} include any category theory. The main references for the relevant category theory are Bakalov-Kirillov \cite{bakalov2001lectures} and Etingof-Gelaki-Nikshych-Ostrik \cite{etingof2016tensor}. While both excellent texts, they suffer notable shortcomings for learning topological quantum information. Bakalov-Kirillov was written in 2001, making it outdated. Etingof-Gelaki-Nikshych-Ostrik is modern, but makes no connections to physics and does not use the language of string diagrams. The manuscript most similar to this one is Kong-Zhang's preprint \cite{kong2022invitation}. We distinguish ourself from Kong-Zhang by our different choice of topics, our different choice of treatment, and our extended scope. Other relevant books and review articles include Wang's monograph \cite{wang2010topological} and Kauffman-Lomonaco's quantum topology themed review \cite{kauffman2009topological}.

[WORK: I will add a section detailing the structure of this book, and how it should be read. I have not written enough for this to be useful yet.]