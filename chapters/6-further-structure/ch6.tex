\section{Further structure}
\label{Further structure}

\subsection{Overview}

\subsubsection{Introduction}

We've seen this general picture throughout the book:

[WORK: add triangle.]

In this chapter we will talk about aspects of the algebraic theory of topological quantum information \textit{beyond} this basic model. In particular, we will talk about physical phenomina beyond plain topological order. These phenomina are not just a phisical or mathematical curiosity. They are central to almost all proposals for topological quantum computation, including the majority of the proposals we will present in chapter [ref].

We will show by example how easy it is to start running into phenomina beyond plain topological order. Suppose we want to make a topological quantum computer on some chip in the lab. Realistically, this chip is going to be \textit{flat}. It's difficult and unpractical to make chips bend into spheres or tori for all sorts of reasons. The chip is also going to be finite. Hence, the chip is going to have \textit{boundary}. This boundary and the way it is manipulated have serious and subtle impacts on the computation being performed. It cannot be ignored. Up to now, we have given no indication about how boundaries are to be dealt with.

[WORK: add picture of chip with boundary.]

These boundaries are not described by modular categories - we need additional algebraic structures to define them. In general, whenever we have any phenominon beyond plain topological order we will need to introduce new algebraic structures.

In this chapter, we will discuss three different classes of important physical phenomina:

\begin{enumerate}
\item \textbf{Boundaries and domain walls}. In this section, we will talk about how topological phases can interact with one another. If you have one topological phase next to another, they can be physically seperated by a \textit{domain wall}:

[WORK: add picture of domain wall.]

If tuned correctly, this domain wall can be chosen so that the composite system keeps its topological gap. That is, we can make a composite topological system with two different sub-phases! The theory of domain walls includes the theory of boundaries as a special case, because a boundary can be viewed as a domain wall between a topological phase and empty space (which is itself a trivial topological phase).

\item \textbf{Symmetry enriched topological order}. [WORK: describe SETs]

\item \textbf{Fermionic topological order}. [WORK: describe fermionic TO.]
\end{enumerate}


One of the fantastic features of the above structures is that they are all mathematically based on the same ideas: \textit{module categories} and \textit{graded categories}. The basic theory of module categories and graded categories will take us very far. Hence even though physically we will be stopping-and-starting every section with new phenomina, mathematically we will see a very smooth narrative.

\subsection{Boundaries and domain walls}

\subsubsection{Physical picture}

In this section we will talk about boundaries and domain walls in topological systems. To begin, we'll talk about boundaries in topological phases. We have the following general picture:

[WORK: topologically ordered system with boundary. DON'T use the categorical ``$\C$". Use the words ``topologically ordered system". Keep with this convention for the whole section. I'm not assuming that the reader knows any category theory.]

This topologically ordered system with boundary will be described by some Hamiltonian. Within the bulk of the system the Hamiltonian would be the same as if the topological system had no boundary. On the boundary, however, it will neccecarily be different. A-priori, we could chose our boundary terms in the Hamiltonian to behave arbitrarily. This will cause a problem, however. It can close the topological gap!

The main point of a topologically ordered system is that it has an energy gap in its spectrum between its lowest energy eigenstate and its next lowest energy eigenstate:

[WORK: add picture of gap.]

Adding the wrong terms on the boundary could close the gap. Namely, the size of the gap could tend to zero as the system size tends to infinity. To preserve the gap, the boundary terms in the Hamiltonian need to have a very special form. This special form implies a quite rigid structure on the space of gapped boundaries.

[WORK: Talk about how gapped boundaries can be equivalent to one another, they can be stable or unstable. When we say boundary theory, we mean \textit{stable gapped boundary} theory. State that there are finitely many equivalence classes of boundary theories for a given topological order. 

I don't really know how this equivalence works and how stability works. This is something to learn.]

Within a boundary there can be \textit{boundary defects}. These are points in which the boundary theory changes from one theory to another, as shown below:

[WORK: add basic picture of a boundary defect.]

Just like with boundaries, it is important that the terms of the Hamiltonian around the boundary defect are chosen so that the topological gap is maintained. There can be stable and unstable boundary defects. There is a natural equivalence relation on the space of stable boundary defects, under which there are finitely many equivalence classes.

One important feature of boundary defects is that they can be \textit{fused}. This happens when two defects are brought very close to one another, as shown below:

[WORK: add a picture for the fusion of boundary defects.]

This fusion can be nondeterministic. That is, repeating the fusion of two boundary defects can yield different results. In this sense boundaries can host non-abelian defects. Importantly, we will find that abelian bulk theories can host non-abelian boundary defects! This is the heart of our proposal for quantum computing with $\D(\ZZ_2)$ topological order. Instead of working with the abelian anyons, we work with the nonabelian boundary defects!

More generally, in this section we will also discuss \textit{domain walls}. Domain walls are boundaries between two different phases. That is, a domain wall is a one-dimensional line seperating two-dimensional topologically ordered systems, as shown below:

[WORK: add picture.]

Domain walls include boundaries as a special case. Empty space can be viewed as a topologically ordered system. Namely, it is the topologically ordered space with a unique ground state - emptyness! This is the trival topological theory. A domain wall can be chosen between any topological theory and the empty theory. This is a boundary!

Of course, domain walls should be chosen to maintain the topological gap. There is a notion of stable and unstable domain walls. There is a notion of equivalnece of stable domain walls. There are finitely many equivalence classes of stable domain walls between any two phases.

Within a domain wall there can be a boundary defect, just like before, which can fuse:

[WORK: add picture of boundary defect in domain wall, alongside a picture for fusion of defects.]

Domain walls also have the added structure of a notion of fusion. Given three topological phases and successive domain walls between them, bringing the domain walls will result in a new domain wall:

[WORK: fusion of domain walls.]

It is important to observe that just like how every boundary can be interpreted as a domain wall, we can interpret every domain wall in terms of a boundary. This goes as follows. Given a domain wall between two phases, we can \textit{fold} the domain wall to get a boundary of the resulting bilayer theory:

[WORK: add folding picture.]

This establishes a bijection,

$$
\left(\substack{
\text{domain walls between}\\
\text{phase 1 and phase 2}}\right)
\xrightarrow{\sim}
\left(\substack{
\text{boundaries of the bilayer phase} \\
\text{obtained by stacking phase 1} \\
\text{and a flipped copy of phase 2} \\
}\right).
$$

This bijection is known as the \textit{folding trick}.[WORK: Should I attribute it to Wen? People at the conference at the Perimeter institute were doing that. Kitaev and Kong are an early reference for the trick.]

In this section, we will be discussing the theory of domain walls, boundary defects, and their fusion.

\subsubsection{Boundaries and domain walls in the toric code}

.[WORK: Do boundaries and domain walls in the toric code. The original reference is \cite{bravyi1998quantum}. There's also a good discussion in ``Topological quantum". I suppose I'll stay away from the surface code literature because they all do it for Bravyi's $\pi/4$ twisted version.]



\subsubsection{The algebraic theory of gapped boundaries}

In this subsection, we will take all of the principles discussed in section [ref] and interpret them in terms of category theory. This will give us an algebraic theory of gapped boundaries.

[WORK: Motivate why gapped boundaries correspond to Lagrangian algebras. As you approach a gapped boundary, every anyon is \textit{confied} (become a boundary excitation) or \textit{condensed} (dissapear at the boundary). A nice talk about this perspective is \cite{burnell2018anyon}.

I still don't quite get how you get from this to Lagrangian algebras. What does the multiplication map mean, physically? Why does it need to be commutative? Why does it need to be seperable? Why does it need to be connected? What's with the condition on dimensions?

Also, there's this quote by Burnell which is very mysterious to me:

``In a conventional ordering transition, the order parameter results from condensing a bosonic excitation–
for example, the ordering of a magnet can be viewed as
the Bose condensation of spinons."

What does it mean?

What do boundary defects correspond to? If the boundary defects go from one boundary type to the same boundary type then the answer is that it is the idempotent completion of the quotient of the original category by the algebra object. This quotient procedure is supposed to describe condensation of anyons at the domain wall. What is the general definition of boundary defect? I can't find it in Cong-Cheng-Wang.

I'll talk to Zhenghan and he will explain everything to me.
]

[WORK:

\begin{itemize}
\item A discussion of anyon condensation at boundaries as a physical mechanism for gapped boundaries - smoothly get to Lagrangian algebras without needing to go through module categories or the Levin-Wen model!
\item Prove that every module category is equivalent to the category of modules over some algebra (this is is Ostrik's original paper);
\item Classification of irriducible module categories in terms of Lagrangian algebras - this is proposition 4.8 of \cite{davydov2013witt}.
\item Characterization of Lagrangian algebras in terms of algebras with nice properties - this is due to Cong-Cheng-Wang;
\item Proof that an MTC is doubled iff it contains a Lagrangian algebra
\item Characterization of Lagrangian algebras in $\Z(\Vec_{G})$;
\end{itemize}
]

[WORK:

There's the very important issue of what the ground state degeneracy of systems with boundaries and boundary defects is. What is a basis for these states? How does braiding act on them? All that stuff. This is done well in Cong-Cheng-Wang.

Another part of this is giving a treatment of topological charge measurement. What are observables? What are the associated probabilities?
]

\subsubsection{Lagrangian algebras and algebra objects}

.[WORK: Give a nice mathematical treatment of Lagrangian algebras and algebra objects.]


\subsubsection{Module categories and boundaries in the Levin-Wen model}

[WORK: This subsection needs to be totally re-done. This is now an \textit{alternate} perspective on gapped boundaries. I'm including it because the theory of module categories is neccecary for the next section. I'm also including it because its nice to have multiple perspectives on this sort of stuff. I'm also including it because the theory of modules is very powerful. This might be the only reasonable way to get at Muger's theorem and Lagrange's theorem for fusion categories.]

In this subsection, we will take all of the principles discussed in section [ref] and interpret them in terms of category theory.

To begin, we re-iteratre that not every topological phase admits a gapped boundary. In fact, we have the following physical principle \cite{kaidi2022higher}:

\begin{equation*}
\left(
\substack{
\text{the topological order described}\\
\text{by the modular category $\BB$}\\
\text{admits a gapped boundary}
}
\right)
\iff
\left(
\substack{
\text{there is an equivalence of categories}\\
\text{$\BB\cong \Z(\C)$ for some}\\
\text{spherical fusion category $\C$}
}
\right).
\end{equation*}

For this reason, we will restrict our attention to \textit{doubled} topological order for the rest of the section. That is, modular categories $\BB$ which are equivalent to $\Z(\C)$ for some spherical fusion category $\C$. We will simplify notion for doubled topological order. We will write

[WORK: bulk theory with letter $\C$ in it $=$ topological ordered described by $\Z(\C)$.]

This now leads us to the big question: what, algebraically, are boundaries theories in the $\Z(\C)$ topological order?

[WORK: The answer is $\C$-module categories. I am suspecting that to motivate this answer one needs to take some physical principles from the string-nets developped in the Levin-Wen model. I haven't written the Levin-Wen section yet, so this can wait until that.]

More generally, a domain wall between the phases described by $\Z(\C)$ and $\Z(\Dcat)$ will be a $(\C,\Dcat)$-bimodle category. [WORK: motivate this as well].

We rigorously define the above intuitive notions as follows.

\begin{definition}[Left module category] A \textit{left module category} over a fusion category $\C$ is the following data:

\begin{enumerate}
\item A $\CC$-linear category $\M$;
\item (Action) A functor $\otimes: \C \times \M \xrightarrow{} \M$;
\item (Left associator) A natural isomorphism $m: \left((\--)\otimes (\--)\right)\otimes (\--) \xrightarrow{\sim} (\--)\otimes \left((\--)\otimes (\--)\right)$ of functors $\C\times \C\times \M \xrightarrow{} \M$;
\item (Unit) A natural isomorphism $\lambda_m:\one \otimes (\--)\xrightarrow{} (\--)$ of functors $\M\xrightarrow{}\M$.
\end{enumerate}

Such that:

\begin{enumerate}

\item There is an equivalence $\C \cong \Vec_{\CC}^n$ as $\CC$-linear categories for some integer $n\geq 1$.

\item The functor $\otimes$ is $\CC$-linear.

\item For all $A,B,C\in \C$ and $M\in \M$ the diagram

% https://q.uiver.app/#q=WzAsNSxbMSwwLCIoKEFcXG90aW1lcyBCKVxcb3RpbWVzIEMpXFxvdGltZXMgTSJdLFswLDEsIihBXFxvdGltZXMgKEJcXG90aW1lcyBDKSlcXG90aW1lcyBNIl0sWzAsMiwiQVxcb3RpbWVzICgoQlxcb3RpbWVzIEMpXFxvdGltZXMgTSkiXSxbMiwxLCIoQVxcb3RpbWVzIEIpXFxvdGltZXMgKENcXG90aW1lcyBNKSJdLFsyLDIsIkFcXG90aW1lcyAoQlxcb3RpbWVzIChDXFxvdGltZXMgTSkpIl0sWzAsMSwiXFxhbHBoYV97QSxCLEN9XFxvdGltZXMgXFxpZF97TX0iLDFdLFsxLDIsIm1fe0EsQlxcb3RpbWVzIEMsIE19Il0sWzAsMywibV97QVxcb3RpbWVzIEIsQyxNfSIsMV0sWzMsNCwibV97QSxCLENcXG90aW1lcyBNfSIsMl0sWzIsNCwiXFxpZF97QX1cXG90aW1lcyBtX3tCLEMsTX0iXV0=
\[\begin{tikzcd}
	& {((A\otimes B)\otimes C)\otimes M} \\
	{(A\otimes (B\otimes C))\otimes M} && {(A\otimes B)\otimes (C\otimes M)} \\
	{A\otimes ((B\otimes C)\otimes M)} && {A\otimes (B\otimes (C\otimes M))}
	\arrow["{\alpha_{A,B,C}\otimes \id_{M}}"{description}, from=1-2, to=2-1]
	\arrow["{m_{A\otimes B,C,M}}"{description}, from=1-2, to=2-3]
	\arrow["{m_{A,B\otimes C, M}}", from=2-1, to=3-1]
	\arrow["{m_{A,B,C\otimes M}}"', from=2-3, to=3-3]
	\arrow["{\id_{A}\otimes m_{B,C,M}}", from=3-1, to=3-3]
\end{tikzcd}\]

commutes.

\item For all $A\in \C$ and $M\in \M$, the diagram

% https://q.uiver.app/#q=WzAsMyxbMCwwLCIoQVxcb3RpbWVzIFxcb25lKVxcb3RpbWVzIE0iXSxbMSwwLCJBXFxvdGltZXMgKFxcb25lIFxcb3RpbWVzIE0pIl0sWzAsMSwiQVxcb3RpbWVzIE0iXSxbMCwyLCJcXHJob197QX0iXSxbMSwyLCJpZF97QX1cXG90aW1lcyBcXGxhbWJkYV9tIl0sWzAsMSwibV97QSxcXG9uZSxNfSJdXQ==
\[\begin{tikzcd}
	{(A\otimes \one)\otimes M} & {A\otimes (\one \otimes M)} \\
	{A\otimes M}
	\arrow["{m_{A,\one,M}}", from=1-1, to=1-2]
	\arrow["{\rho_{A}}", from=1-1, to=2-1]
	\arrow["{id_{A}\otimes \lambda_m}", from=1-2, to=2-1]
\end{tikzcd}\]

commutes.
\end{enumerate}

\raggedleft\qedsymbol{}
\end{definition}

The definition of a \textit{right} module category is completely symmetric, with no surprises. The definition of a bimodule category comes from putting these definitions together:

\begin{definition}[Bimodule category] A \textit{$(\C,\Dcat)$-bimodule category} $\M$ over fusion categories $\C,\Dcat$ is the following data:

\begin{enumerate}
\item The structure of a right $\C$-module on $\M$;
\item The structure of a left $\Dcat$-module on $\M$;
\item (Middle associator) A natural isomorphism $b: ((\--)\otimes (\--))\otimes (\--)\xrightarrow{\sim} (\--)\otimes ((\--)\otimes (\--))$ of functors $\C\times\M \times \Dcat \xrightarrow{} \M$;
\end{enumerate}

Such that:

\begin{enumerate}
\item For all $A,B\in \C$, $M\in \prescript{}{\C}\M_{\Dcat}$, $A',B\in \Dcat$, the diagrams

% https://q.uiver.app/#q=WzAsNSxbMSwwLCIoKEFcXG90aW1lcyBCKVxcb3RpbWVzIE0pXFxvdGltZXMgQyJdLFswLDEsIihBXFxvdGltZXMgKEJcXG90aW1lcyBNKSlcXG90aW1lcyBDIl0sWzAsMiwiQVxcb3RpbWVzICgoQlxcb3RpbWVzIE0pXFxvdGltZXMgQykiXSxbMiwxLCIoQVxcb3RpbWVzIEIpXFxvdGltZXMgKE1cXG90aW1lcyBDKSJdLFsyLDIsIkFcXG90aW1lcyAoQlxcb3RpbWVzIChNXFxvdGltZXMgQykpIl0sWzAsMSwibV5MX3tBLEIsTX0iLDJdLFswLDMsImJfe0FcXG90aW1lcyBCLCBNLCBDfSJdLFsxLDIsImJfe0EsQlxcb3RpbWVzIE0sQ30iXSxbMiw0LCJcXGlkX3tBfVxcb3RpbWVzIGJfe0IsTSxDfSJdLFszLDQsIm1ee0x9X3tBLEIsTVxcb3RpbWVzIEN9IiwyXV0=
\[\begin{tikzcd}
	& {((A\otimes B)\otimes M)\otimes C} \\
	{(A\otimes (B\otimes M))\otimes C} && {(A\otimes B)\otimes (M\otimes C)} \\
	{A\otimes ((B\otimes M)\otimes C)} && {A\otimes (B\otimes (M\otimes C))}
	\arrow["{m^L_{A,B,M}}"', from=1-2, to=2-1]
	\arrow["{b_{A\otimes B, M, C}}", from=1-2, to=2-3]
	\arrow["{b_{A,B\otimes M,C}}", from=2-1, to=3-1]
	\arrow["{m^{L}_{A,B,M\otimes C}}"', from=2-3, to=3-3]
	\arrow["{\id_{A}\otimes b_{B,M,C}}", from=3-1, to=3-3]
\end{tikzcd}\]

and

% https://q.uiver.app/#q=WzAsNSxbMSwwLCIoKEFcXG90aW1lcyBNKVxcb3RpbWVzIEIpXFxvdGltZXMgQyJdLFswLDEsIihBXFxvdGltZXMgKE1cXG90aW1lcyBCKSlcXG90aW1lcyBDIl0sWzIsMSwiKEFcXG90aW1lcyBNKVxcb3RpbWVzIChCXFxvdGltZXMgQykiXSxbMCwyLCJBXFxvdGltZXMgKChNXFxvdGltZXMgQilcXG90aW1lcyBDKSJdLFsyLDIsIkFcXG90aW1lcyAoTVxcb3RpbWVzIChCXFxvdGltZXMgQykpIl0sWzMsNCwiXFxpZF97QX1cXG90aW1lcyBtXntSfV97TSxCLEN9Il0sWzAsMiwibV57Un1fe0FcXG90aW1lcyBNLCBCLEN9Il0sWzIsNCwiYl97QSxNLEJcXG90aW1lcyBDfSIsMV0sWzAsMSwiYl97QSxNLEJ9XFxvdGltZXMgXFxpZF97Q30iLDJdLFsxLDMsImJfe0EsTVxcb3RpbWVzIEIsQ30iLDFdXQ==
\[\begin{tikzcd}
	& {((A\otimes M)\otimes B)\otimes C} \\
	{(A\otimes (M\otimes B))\otimes C} && {(A\otimes M)\otimes (B\otimes C)} \\
	{A\otimes ((M\otimes B)\otimes C)} && {A\otimes (M\otimes (B\otimes C))}
	\arrow["{b_{A,M,B}\otimes \id_{C}}"', from=1-2, to=2-1]
	\arrow["{m^{R}_{A\otimes M, B,C}}", from=1-2, to=2-3]
	\arrow["{b_{A,M\otimes B,C}}"{description}, from=2-1, to=3-1]
	\arrow["{b_{A,M,B\otimes C}}"{description}, from=2-3, to=3-3]
	\arrow["{\id_{A}\otimes m^{R}_{M,B,C}}", from=3-1, to=3-3]
\end{tikzcd}\]

commute.
\end{enumerate}

\raggedleft\qedsymbol{}
\end{definition}

Given two fusion categories $\C,\Dcat$ and a $(\C,\Dcat)$-bimodule categories $\M$, we define $\M$ to be a \textit{simple} module if for all simple objects $N,M\in \M$ there exists some $A\in \C$ such that $\Hom(N,A\otimes M)\neq 0$. This condition of irriducibility corresponds exactly the condition of stability of boundaries. Hence, we arrive at the following correspondance:

[WORK: boundary theories correspond to simple module categories;]

[WORK: domain walls correspond to simple bimodule categories.]

We now turn our attention towards boundary defects.

[WORK: motivate why boundary defects are bimodule functors. I think this one can be easily seen by passing boundary excitations through the defect and thus noticing that the defect must act like a functor.]

Formally, we get the following definition:

\begin{definition}[Left module functor] A \textit{left module functor} between left $\C$-module categories $\M$, $\N$ is the following data:

\begin{enumerate}
\item A functor $F: \M \xrightarrow{}  \N$;
\item A natural isomorphism $s:F((\--)\otimes (\--))\xrightarrow{\sim}(\--)\otimes (\--)$ of functors $\C\times \M\xrightarrow{} \N$.
\end{enumerate}

Such that:

\begin{enumerate}

\item For all $A,B\in \C$, $M\in \M$, the diagram

% https://q.uiver.app/#q=WzAsNSxbMSwwLCJGKChBXFxvdGltZXMgQilcXG90aW1lcyBNKSJdLFswLDEsIkYoQVxcb3RpbWVzIChCXFxvdGltZXMgTSkpIl0sWzAsMiwiQVxcb3RpbWVzIEYoQlxcb3RpbWVzIE0pIl0sWzIsMSwiKEFcXG90aW1lcyBCKVxcb3RpbWVzIEYoTSkiXSxbMiwyLCJBXFxvdGltZXMgKEJcXG90aW1lcyBGKE0pKSJdLFsyLDQsIlxcaWRfe0F9XFxvdGltZXMgc197QixNfSJdLFswLDMsInNfe0FcXG90aW1lcyBCLE19Il0sWzMsNCwibV57XFxOfV97QSxCLEYoTSl9IiwxXSxbMCwxLCJGKG1ee1xcTX1fe0EsQixNfSkiLDJdLFsxLDIsInNfe0EsQlxcb3RpbWVzIE19IiwxXV0=
\[\begin{tikzcd}
	& {F((A\otimes B)\otimes M)} \\
	{F(A\otimes (B\otimes M))} && {(A\otimes B)\otimes F(M)} \\
	{A\otimes F(B\otimes M)} && {A\otimes (B\otimes F(M))}
	\arrow["{F(m^{\M}_{A,B,M})}"', from=1-2, to=2-1]
	\arrow["{s_{A\otimes B,M}}", from=1-2, to=2-3]
	\arrow["{s_{A,B\otimes M}}"{description}, from=2-1, to=3-1]
	\arrow["{m^{\N}_{A,B,F(M)}}"{description}, from=2-3, to=3-3]
	\arrow["{\id_{A}\otimes s_{B,M}}", from=3-1, to=3-3]
\end{tikzcd}\]

commutes.

\item For all $M\in \M$ the diagram

% https://q.uiver.app/#q=WzAsMyxbMCwwLCJGKFxcb25lXFxvdGltZXMgTSkiXSxbMSwwLCJcXG9uZVxcb3RpbWVzIEYoTSkiXSxbMSwxLCJGKE0pIl0sWzAsMiwiRihcXGxhbWJkYV5cXE1fe20sTX0pIiwyXSxbMSwyLCJcXGxhbWJkYV57XFxOfV97bSxGKE0pfSJdLFswLDEsInNfe1xcb25lLE19Il1d
\[\begin{tikzcd}
	{F(\one\otimes M)} & {\one\otimes F(M)} \\
	& {F(M)}
	\arrow["{s_{\one,M}}", from=1-1, to=1-2]
	\arrow["{F(\lambda^\M_{m,M})}"', from=1-1, to=2-2]
	\arrow["{\lambda^{\N}_{m,F(M)}}", from=1-2, to=2-2]
\end{tikzcd}\]

commutes.
\end{enumerate}

\raggedleft\qedsymbol{}
\end{definition}

A right module functor is defined in complete analogy to left module functor. A $(\C,\Dcat)$-bimodule functor is defined to be a functor which is simulaneously equipped with the structure of a left and right module functor. We denote by $\Fun_{\C}(\M, \N)$, the space of $\C$-modue functors from $\M$ to $\N$, and we denote by $\Fun_{\C | \Dcat}( \M, \N)$ the space bimodule functors.

To turn these spaces of module functors into categories, we will need to define an appropriate notion of natural transformation

\begin{definition}[Natural transformation of left module functors] A \textit{natural transformation of left module functions} between left $\C$-module functors $F,G:\M\to \N$ is a natural transformation $\eta:F\to G$ such that for all $A\in \C$, $M\in \M$

% https://q.uiver.app/#q=WzAsNCxbMCwwLCJGKEFcXG90aW1lcyBNKSJdLFsxLDAsIkcoQVxcb3RpbWVzIE0pIl0sWzAsMSwiQVxcb3RpbWVzIEYoTSkiXSxbMSwxLCJBXFxvdGltZXMgRyhNKSJdLFswLDEsIlxcZXRhX3tBXFxvdGltZXMgTX0iXSxbMiwzLCJcXGlkX3tBfVxcb3RpbWVzIFxcZXRhX3tNfSJdLFswLDIsInNee0Z9X3tBLE19IiwxXSxbMSwzLCJzXkdfe0EsTX0iLDFdXQ==
\[\begin{tikzcd}
	{F(A\otimes M)} & {G(A\otimes M)} \\
	{A\otimes F(M)} & {A\otimes G(M)}
	\arrow["{\eta_{A\otimes M}}", from=1-1, to=1-2]
	\arrow["{s^{F}_{A,M}}"{description}, from=1-1, to=2-1]
	\arrow["{s^G_{A,M}}"{description}, from=1-2, to=2-2]
	\arrow["{\id_{A}\otimes \eta_{M}}", from=2-1, to=2-2]
\end{tikzcd}\]

\raggedleft\qedsymbol{}
\end{definition}

Just like in our general theory, we define an object in a functor category to be irriducible if it cannot be expressed as the biproduct of two nonzero elements. We arrive at the following correspondance:

[WORK: boundary defects correspond to simple bimodule functors.]

The fusion of boundary defects is clear. Given two boundary defects $F: \M\to \M'$ and $G: \M'\to \M''$ the composition $(G\circ F): \M\to \M''$ gives another boundary defect. This is the correct physical choice becase [WORK: depends on the physical motivation I gave for defects corresponding to functors].

The fusion of domain walls is a bit more sublte. It requires the notion of relative tensor product:

\begin{definition}[Relative Deligne tensor product] [WORK: give definition]
\end{definition}

Physically, we can define fusion of domain walls $\M$ and $\N$ to be the relative Deligne tensor product $\M\boxtimes_{\K}\N$, where $\M$ is a $(\C\,\K)$-bimodule and $\N$ is a $(\K,\Dcat)$-bimodule. This is the correct definition because [WORK: give physical motivation].

All together, this gives us the following physics-math dictionary:

[WORK: add dictionary.]

As we have just seen, the algebraic theory of of topological boundaries and domain walls is modules and bimodules. So, mathematically, we now take the time to set up the theory of modules and bimodules over fusion categories.

The first thing we are going to do is reinterpret bimodules in terms of modules. This will allow us to state all of our results in this section in terms of modules without loss of generality, because dealing with bimodules adds unnececary notational difficulty. This is done via the folding trick. As mentioned before, we have a correspondance as follows:

[WORK: add folding trick. The flat plane should have the label $(\C,\Dcat)$-bimodules and the folded plane shuld have the label $(\C\boxtimes \overline{\Dcat})$-modules.]

This is codifed mathematically in the following proposition:

\begin{proposition}[Folding trick] Let $\C,\Dcat$ be fusion categories. Given a $(\C,\Dcat)$-bimodule $\M$, define a left ($\C\boxtimes \overline{\Dcat}$)-module structure on $\M$ via the formula

$$(A\boxtimes \overline{B})\otimes M = (A\otimes M)\otimes B$$

for any $A\in \C$, $B\in \Dcat$, $M\in \M$. This is a well-defined left $(\C\boxtimes \overline{\Dcat})$-module structure. This assignment induces a bijection between the set of  $(\C,\Dcat)$-bimodules and the set of left $(\C\boxtimes \overline{\Dcat})$-modules.
\end{proposition}
\begin{proof} This is a straightforward and unenlightening exercise.
\end{proof}


Next, we take a closer look at simple modules. That is, module categories with no proper nontrivial submodule categories. Given two categories $\M$ and $\N$, we can define their Cartesian product $\M\times \N$, whose objects are pairs $(A,B)$, $A\in \M$, $B\in \N$. To avoid confusion with the Deligne tensor product, we will use the notation $\boxplus$ to denote the Cartesian product in this context, and refer to it as the direct sum. This is consistent with our genereal principles of higher linear algebra (see section [ref]). If $\M$ and $\N$ are left $\C$-module categories, then $\M\boxplus \N$ can be equipped with the structure of a left $\C$-module via the action

$$A\otimes (M\boxplus N)= (A\otimes M)\boxplus (A\otimes N).$$

We seek to prove that every module is the direct sum of simple modules. This follows after a lemma.

\begin{lemma} Let $\C$ be a fusion category. Let $\M$ be a left $\C$-module. For all $A\in \C$ and $N,M\in \M$, we have that

$$\Hom_{\M}(N,A\otimes M)\cong \Hom_{\M}(A^*\otimes N, M)$$

as $\CC$-vector spaces.
\end{lemma}
\begin{proof} The proof is exactly the same as of proposition [ref]. We opt for a proof using the graphical language of morphisms instead of the algebraic approach, because it is more lucid. We have not formally introduced the graphical language for module categories, but it is completely analogous to the graphical language for fusion categories. We define the map

\begin{equation*}
\tikzfig{frobenius-reciprocity}
\end{equation*}

Twisting the $A^*$ the other way serves as an inverse, completing the proof.
\end{proof}

\begin{proposition} Let $\C$ be a fusion category. Every left $\C$-module category is equivalent to a direct sum of simple $\C$-module categories.
\end{proposition}
\begin{remark} The usual way of summarizing this proposition is by saying that the space of left $\C$-modules is \textit{semsimple}.
\end{remark}
\begin{proof} Let $\M$ be a left $\C$-module. Denote by $\LL_{\M}$ the set of isomorphism classes of simple objects in $\M$. We define a equivalence relation $\sim$ on $\LL_{\M}$ by saying that $[N]\sim [M]$ if there exists some $A\in \C$ such that $\Hom(N,A\otimes M)\neq 0$. This relation is reflexive since $\Hom(M,\one\otimes M)\neq 0$, it is transitive since if $\Hom(M,A\otimes M')\neq 0$ and $\Hom(M',B\otimes M'')\neq 0$, then $\Hom(M,A\otimes B\otimes M'')\neq 0$ by composing morphisms. Seeing as $\M\cong \Vec_{\CC}^n$ for some $n\geq 1$ as a category, we have that $\Hom(N,M)\cong \Hom(M,N)$ as vector spaces for all $N,M\in \M$. Hence, the equivalence relation is symmetric by lemma [ref].

For all $i\in \LL_{M}/\sim$, we define a full subcategory of $\M$ via

$$\M_{i}=\left\{\left.M\in \M\right| M\cong \bigoplus_{[N]\in i}n_{N}\cdot N,\text{ for some integers } n_{A}\geq 0\right\}.$$

Now, our equivalence relation gaurantees that $\M_{i}$ is closed under the action of $\C$ for all $i\in \LL_{\M}/\sim$. That is, it forms a left $\C$-submodule of $\M$. We claim that the map

\begin{align*}
\bplus_{i\in \LL_{\M}/\sim}\M_{i}&\xrightarrow{}\M\\
(M_{i})_{i\in \LL_{\M}/\sim}&\mapsto \bigoplus_{i\in \LL_{\M}/\sim}M_{i}
\end{align*}

is an equivalence of categories. It is faithful by the general properties of the direct sum. It is full because $\Hom(M_i,M_j)=0$ for all $i\neq j$ since $M_i$, $M_j$ are the direct sums of disjoint sets of simple objects, and hence every morphisms can be broken into its $i$-components for each $i\in \LL_{\M}/\sim$. It is essentially surjective because every simple objects appears in the category $\M_{i}$ for some $i\in \LL_{\M}/\sim$. Hence, it is an equivalence of categories by proposition [ref] so our proof is complete.
\end{proof}

With propositions [ref] and [ref] in hand, we can set up more directly the theory of fusion of domain walls. Let $\C$, $\K$, $\Dcat$ be fusion categories. Let $\M$ be a $(\C,\K)$-bimodule and let $\N$ be a $(\K,\Dcat)$ bimodule. We have an equivalence

$$\N\boxtimes\M \cong \bplus_{[\M']} n^{\N,\M}_{\M'}\cdot \M',$$

where $[\M']$ runs over the set of equivalence classes of simple $(\C,\Dcat)$-bimodules. This is completely analogus to the algebraic theory of fusion of anyons introduced in section [ref]. One of the major differenes is that we are now up a categorical level. Instead of fusing objects in a category, we are fusing the categories themselves.

We now turn our attention towards functors between module categories. When the source and target of our functors are the same, we have the following main result:

\begin{proposition} Let $\C$ be a fusion category. Let $\M$ be a left $\C$-module. Define

$$\C^{\vee}_{\M}=\Fun_{\C}(\M,\M).$$

We call $\C^{\vee}_{\M}$ the \textit{dual of $\C$ with respect to $\M$}. We define a monoidal structure on $\C^{\vee}_{\M}$ via the formula

$$F\otimes G = G\circ F.$$

Along with a natural choice of rigid structure, $\C^{\vee}_{\M}$ has the structure of a fusion category.

If $\C$ is a spherical fusion category, then $\C^{\vee}_{\M}$ is naturally equipped with the structure of a spherical fusion category.
\end{proposition}
\begin{proof}.[WORK: do proof. This is going to use some stuff from ``on fusion categories". Might need the same results for the next proposition as well.

I'm in trouble the proof uses Yetter cohomology and other scary-looking tools. Is there a simpler proof? The hard part is showing that if $\N,\M$ are module categories the space of module functors is equivalent to $\Vec_{\CC}^n$ for some $n\geq 0$. In particular, we need to show that there are finitely many isomorphism classes of simple module functors. How do I do it?! It's a sort of generalized Ocneanu rigidity.
]
\end{proof}

When the source and target are different, we have the follwing result:

\begin{proposition} Let $\C$ be a fusion category. Let $\N$, $\M$ be left $\C$-modules. The category $\Fun_{\C}(\N,\M)$ is naturally equipped with the structure of a left $\C_{\N}^{\vee}$-module and a right $\C_{\M}^{\vee}$-module.
\end{proposition}
\begin{proof}.[WORK: do proof. Again, the finiteness part is hard.]
\end{proof}

Now, we verify the finiteness result that we claimed in section [ref]:

\begin{proposition} Let $\C$ be a fusion category. There are finitely many equivalence classes of simple left $\C$-modules.
\end{proposition}
\begin{proof}.[WORK: do proof. In ``On fusion categories" they say that this follows from the finiteness condition on functors, along with a decategorified version of the same result which follows from counting.]
\end{proof}

The following result serves as the centerpiece of our study of module categories:

\begin{theorem} Let $\C$ be a fusion category. Let $\M$ be a left $\C$ module. The map

\begin{align*}
\can: \C &\xrightarrow{\sim } \left(\C^{\vee}_{\M}\right)^{\vee}_{\M}\\
A &\mapsto \left(M\mapsto A\otimes M\right)
\end{align*}

is an equivalence of categories. The $\C^{\vee}_{\M}$-module functor structure on $\can_{A}$ is defined for all $A\in \C$, $M\in \M$, $F\in \C^{\vee}_{\M}$ by

$$s^{\can_{A}}_{F,M}:\can_{A}(F\otimes M )= A\otimes F(M)\xrightarrow{s^{-1}_{A,M}}F(A\otimes M)=F\otimes \can_{A}(M).$$

If $\C$ is a spherical fusion category, $\can$ is an equivalence of spherical fusion categories.
\end{theorem}
\begin{remark} This result is an analogue of the \textit{double centralizer theorem} from classical representation theory.
\end{remark}
\begin{proof}.[WORK: do proof. The proof in EGNO uses algebras an algebra structures. It would be nice to get around this.]
\end{proof}

As a corollary, we get the following:

\begin{corollary} Let $\C$ be a fusion category and let $\M$ be a left $\C$-module. The module $\M$ is simple as a right $\C_{\M}^{\vee}$-module.
\end{corollary}
\begin{proof}.[WORK: do proof]
\end{proof}

Of course, the above discussion gaurantees a proper theory of fusion of boundary defects. Suppose that $\C,\D$ are fusion categories, and $\M$, $\M'$, $\M''$ are simple $(\C,\D)$-bimodules. Suppose that $F:\M\to \M'$ and $G:\M'\to\M''$ are simple $(\C,\D)$ bimodule functors. Since $\Fun_{\C|\D}(\M,\M'')$ is equivalent to $\Vec_{\CC}^n$ for some $\geq 0$ as a $\CC$-linear category, we have a decomposition

$$G\circ F \cong \bigoplus_{[H]}n^{F,G}_{H}\cdot H$$

in $\Fun_{\C,\D}(\M,\M'')$ where $[H]$ runs over isomorphism classes of simple functors in $\Fun_{\C|\D}(\M,\M'')$. Again, this is completely analagous to the fusion of anyons.

\subsubsection{The Drinfeld center via bimodules}

[WORK: The paper \cite{kong2008morita} has some interesting additional insights. Maybe if I took the time to understand it it would be relevant as a comment?]


So far we have been setting up a powerful theory of domain walls to study doubled topological order.

We already know one thing about doubled topological order - its anyons are described by the Drinfeld center $\Z(\C)$ of the input spherical fusion category $\C$. These anyonic bulk excitations can interact with the boundaries, and produce non-trivial effects. This means that the category $\Z(\C)$ neccecarily has deep connections to the theory of boundaries, and thus mathematically will have a module-theoretic interpretation. This will also help answer the question of why the Drinfeld center $\Z(\C)$ is the category which describes anyons in the Levin-Wen model.

To start, consider an anyon in the bulk. This anyon is a stable localized excitation. Above it and below it the theory is in its ground state. Hence, we can think of this anyon as an interface between the trivial boundary and itself! This is seen visually below:

[WORK: add picture of anyon as interface between trivial boundary and itself.]

Thus, from our general description of boundary defects, we should conclude the following:

\begin{equation*}
\left(\text{anyon types}\right)
= \left(\substack{
\text{boundary defects}\\
\text{between trivial theory and itself}
}\right)
= \left(\substack{
\text{simple objects in }\\
\Fun_{\C|\C}(\C,\C)
}\right).
\end{equation*}

However, we know from our first discussion of the Levin-Wen model that anyon types should correspond to simple objects in $\Z(\C)$. Hence, we conclude that simple objects in $\Z(\C)$ should correspond to simple objects in $\Fun_{\C|\C}(\C,\C)$. Of course, this equivalence should repsect fusion, duality, braiding, and all the other algebraic structures of anyons. Thus, if our boundary theory is correct, it predicts an equivalence of categories $\Z(\C)\cong\Fun_{\C|\C}(\C,\C)$. This mathematical prediction is accurate, by the following proposition:

\begin{proposition} Let $\C$ be a fusion category. The map

\begin{align*}
\Fun_{\C|\C}(\C,\C)&\xrightarrow{\sim}\Z(\C)\\
F &\mapsto \left(F(1), \upbeta_{F(1),\--}\right)
\end{align*}

is an equivalence of categories, where $\upbeta_{F(1),\--}$ is defined by the composition

% https://q.uiver.app/#q=WzAsMyxbMCwwLCJcXHVwYmV0YV97RigxKSxcXC0tfTpGKFxcb25lKVxcb3RpbWVzIEIiXSxbMSwwLCJGKFxcb25lXFxvdGltZXMgQik9RihCKT1GKEJcXG90aW1lcyBcXG9uZSkiXSxbMiwwLCJCXFxvdGltZXMgRihcXG9uZSkiXSxbMCwxLCJzXlJfe1xcb25lLEJ9Il0sWzEsMiwiKHNeTF97QixcXG9uZX0pXnstMX0iXV0=
\[\begin{tikzcd}
	{\upbeta_{F(1),\--}:F(\one)\otimes B} & {F(\one\otimes B)=F(B)=F(B\otimes \one)} & {B\otimes F(\one)}
	\arrow["{s^R_{\one,B}}", from=1-1, to=1-2]
	\arrow["{(s^L_{B,\one})^{-1}}", from=1-2, to=1-3]
\end{tikzcd}\]

and $s^L$, $s^R$ are the compatibility maps implicit in the left/right module functor structure of $F$.

If $\C$ is a spherical category, then the map $\Fun_{\C|\C}(\C,\C)\xrightarrow{\sim}\Z(\C)$ induces an equivalence as pre-modular categories.
\end{proposition}
\begin{proof}. [WORK: do proof]
\end{proof}

This equivalence of categories gives the proper interpretation for $\Z(\C)$ in the theory of boundaries. If we wanted we could have taken it as a \textit{definition} of $\Z(\C)$. We now examine how anyons interact with boundaries to get more module-theoretic properties of the Drinfeld center.

Suppose $\C$, $\Dcat$ are spherical fusion categories, and $\prescript{}{\C}\M_{\Dcat}$ is a bimodule. If a $A\in \C$ is a simple object describing an anyon, we can imagine trying to push $A$ from $\C$ to $\Dcat$ through the boundary theory $\prescript{}{\C}\M_{\Dcat}$ :

[WORK: add picture: first its on one side, then a bubble starts to form, then the bubble detaches. It leaves behind a dotted line with label $\prescript{}{\Dcat}\M^{op}_{\C}\boxtimes_{\C}\prescript{}{\C}\M_{\Dcat}$ ]

Suppose that the domain wall is \textit{invertible}. This means that $\prescript{}{\Dcat}\M^{op}_{\C}\boxtimes_{\C}\prescript{}{\C}\M_{\Dcat}\cong \Dcat$. Thus, the anyon can pass directly through the boundary, and the boundary can re-close without leaving a trace of the fact that the anyon went through.

This gives an assignment of anyons in the $\Z(\C)$ to anyons in the $\Z(\Dcat)$ phase. This assignement is reversible, because anyons in the $\Z(\Dcat)$ phase can pass through the domain wall in the other direction. Thus, our boundary theory predicts that every invertible bimodule $\prescript{}{\C}\M_{\Dcat}$ will induce an equivalence of categories $F_{\M}:\Z(\C)\xrightarrow{\sim}\Z(\Dcat)$. We verify this prediction with a theorem. Before we can prove it, we need a lemma.

\begin{lemma} Let $\C$ be a fusion category. The forgetful functor

\begin{align*}
\Z(\C)&\xrightarrow{} \C\\
(A,\upbeta_{A,\--})&\mapsto A
\end{align*}

is surjective onto the objects of $\C$.
\end{lemma}
\begin{proof}.[WORK: this proof is steeped in the theory of module categories.]
\end{proof}

\begin{theorem} Let $\C$, $\Dcat$ be spherical fusion categories. There is a bijection

$$
\left(
\substack{
\text{invertible $(\C,\Dcat)$-bimodule}\\
\text{categories }\M}
\right)
\xleftrightarrow{\sim}
\left(
\substack{
\text{equivalences of pre-modular}\\
\text{categories }\Z(\C)\xrightarrow{\sim}\Z(\Dcat)}
\right),$$

defined by sending a $(\C,\Dcat)$-bimodule category $\M$ to the functor $F_{\M}:\Z(\C)\to\Z(\D)$ defined by

\begin{align*}
F_{\M}: \Z(\C)\cong &\Fun_{\C|\C}(\C,\C) \xrightarrow{} \Fun_{\Dcat|\Dcat}(\Dcat,\Dcat)\cong \Z(\Dcat).\\
G&\mapsto \left(\Dcat\xrightarrow{}\M^{op}\boxtimes_{\C}\C\boxtimes_{\C}\M \xrightarrow{\id\boxtimes G \boxtimes \id}\M^{op}\boxtimes_{\C}\C\boxtimes_{\C}\M \xrightarrow{}\Dcat\right)
\end{align*}

For all triples $\C$,$\K$, $\Dcat$ of fusion categories, $(\C,\K)$-bimodules $\M$, and $(\K,\Dcat)$-bimodules $\N$, we have that

$$F_{\M\boxtimes_{\K}\N}=F_{\N}\circ F_{\M}.$$

If $\C$ and $\Dcat$ are spherical fusion categories, then the assignement $\M\mapsto F_{\M}$ induces a bijection.

[WORK: what is the correct statement? Do I have to reduce the class of bimodules I am considering?]

\end{theorem}
\begin{remark} This theorem is originally due to Etingof-Nikshych-Ostrik \cite{etingof2010fusion}. It was observed in the theory of gapped boundaries by Kitaev and Kong \cite{kitaev2012models}. The proof presented here is original.
\end{remark}
\begin{proof}
.[WORK: do proof. The first direct is making sure this is a well-defined map and filling in details. That's not very hard.

The big question is showing that this is a \textit{bijection}. That means that we need an inverse to this map. The inverse is suprisingly cute and nice. The forgetful functor $\Z(\C)\to \C$ induces a $\Z(\C)$-bimodule structure on $\C$. Given any equivalence $F:\Z(\C)\xrightarrow{\sim}\Z(\Dcat)$, we can thus construct the tensor product

$$\C \boxtimes_{\Z(\C)\cong \Z(\Dcat)}\Dcat.$$

This tensor product is a $(\C,\Dcat)$ bimodule. The fact that it is an invertible bimodule uses in a key way the fact that the forgetful functors $\Z(\C)\to \C$ and $\Z(\Dcat)\to \Dcat$ are surjective. It's not hard to show that these two constructions are inverses to one another. This gives the proof.
]
\end{proof}


As a corollary we find the following theorem, which was first proved by Kitaev and Muger and communicated by Etingof-Nikshych-Ostrik \cite{etingof2011weakly}. We recall that two fusion categories $\C$, $\Dcat$ are called \textit{Morita equivalent} if their Drinfeld centers are equivalent as braided fusion categories

[WORK: What is the correct statement of this theorem for spherical fusion categories?]

\begin{corollary} Two fusion categories $\C$, $\Dcat$ are Morita equivalent if and only if there exists an invertible bimodule category between them.
\end{corollary}
\begin{proof} This is clear from Theorem [ref].
\end{proof}



\subsubsection{Muger's theorem and Lagrange's theorem for fusion categories}

[WORK: There's some results about Frobenius-Perron dimensions which will play a key role in the proof. Can I use quantum dimension instead? Probably not because I'm not assuming spherical.]

[WORK: Out of all this we get Lagrange's theorem for fusion categories, that given a fusion $\C$ with full subcategory $\C$, the ratio $\FPdim(\Dcat)/\FPdim(\C)$ is an algebraic integer. Do I care? Should I include this? Maybe I should have a section with a proof of Muger's theorem plus a proof of Lagrange's theorem for fusion categories.]


\subsubsection{Skeletonization of gapped boundaries}

[WORK:To be explicitely workable it is nice to reinterpret gapped boundaries in the skeletal language just like how one does with modular categories. In this case the symbols are M3j and M6j symbols, as discussed in Cong-Cheng-Wang. Not sure if I want to do this, though.]

\subsection{Symmetry enriched topological order}

\subsubsection{Physical picture}

[WORK: Two additional good references to read for this stuff are \cite{aasen2016topological, aasen2020topological}]

[WORK:

This section should give an elementary physical picture of what symmetry enriched topological order. The basic idea is that a topological gap might be closable, but maybe closing it has to break some symmetry! This is easy to explain and motivate.

To talk about the next part, people need some idea about how phase transitions correspond to symmetry breaking. Moreover, it would be nice if it was clear how symmetry breaking corresponds to condensing quasiparticles. Burnell has a very nice quote about this which is mysterious to me:

``In a conventional ordering transition, the order parameter results from condensing a bosonic excitation–
for example, the ordering of a magnet can be viewed as
the Bose condensation of spinons."

I think that getting the example of the magnet phase transition across would be great. If people can understand the magnet, they can understand everything. I don't understand the magnet yet.

]

[WORK: I would like to include the Curie principle - the symmetry of the causes are to be found in the effects! The symmetries $\Aut^{\otimes}(\C)$ are nessecarily interesting objects. Why? Curie principle! The curie principle in this case manifests in the possible defects which can appear. While $\C$ and $\C\otimes \overline{\C}$ might \textit{look} like very similar categories, $\C\otimes \overline{\C}$ may have way more symmetries than $\C$ and hence might look like a better candidate for TQC in this guise

The original reference for the Curie principle is \cite{curie1894symetrie}. The theory of magnetic space groups gives a quite nice example.
]

\subsubsection{Haldane spin chain}

[WORK:

This section should include the easiest example possible of an SPT phase, or at least one that maximizes simplicity and relevance. I'm sure Kitaev has a nice model. People online say that the Haldane phase of the Heisenberg spin chain is easy but I haven't seen it before so I can't be sure.

]

\subsubsection{Twist defects in the toric code}

[WORK:

Talk about the toric code. Introduce the twist defects, and show how they work. This gives the main example of an SET phase with its defects nice and visible.
]

\subsubsection{The algebraic theory of SET phases}

In this section, we will introduce the algberaic theory of symmetry enriched topological phases.

[WORK:

Why categorical $G$-actions? Why $G$-graded categories? It would be great if I could work up to the picture

$$\C <-- confinement, defectification --> \C^{\times}_{G} $$

If I could even get in the fact that there are obstructions that would be awesome. Of course, I can't really prove much. The hope is that I can state everything trivially, and not have to make any big claims...
]


[WORK: Another good thing to note is that there are generalizations of the notion of symmetry in this context! In particular we have the work of Cui-Zini-Wang \cite{cui2019generalized}. It's not clear that group actions are the only symmetries which could protect defects.]

[WORK:

A really important theorem in this space is EGNO's observation that $\Aut$ and $\Pic$ are isomorphic. Does this come up on its own already, or do I need to add this in? I'm hazy now on the physical significance of this isomorphism, and on what categorical level it holds.

Zhenghan made a point to me that noninvertible defects break the theorem (they aren't in the picard group), but the philosophy is still there. Namely, the category (appropriately defined) $\Pic\left(\mathcal{B}^{\times}_{G}\right)$ should be isomorphic to some endomorphism group. The philosophy holds (though I am hazy on it), but there is no known theorem. This is something to look into!
]

\begin{definition}[Categorical group action] A \textit{categorical group action} of a finite group $G$ on a fusion category $\C$ is the following data:

\begin{enumerate}
\item A function $\rho: G \xrightarrow{}\Aut^{\otimes}(\C)$;
\item A natural isomorphism of functions of fusion categories

$$\eta_{g,h}: \rho(g)\circ \rho(h)\xrightarrow{\sim}\rho(gh)$$

for all $g,h\in G$.
\end{enumerate}

Such that:

\begin{enumerate}

\item The square

% https://q.uiver.app/#q=WzAsNCxbMCwwLCJcXHJobyhnXzApXFxjaXJjIFxccmhvKGdfMSlcXGNpcmMgXFxyaG8oZ18yKSJdLFsxLDAsIlxccmhvKGdfMGdfMSlcXGNpcmMgXFxyaG8oZ18yKSJdLFswLDEsIlxccmhvKGdfMClcXGNpcmMgXFxyaG8oZ18xZ18yKSJdLFsxLDEsIlxccmhvKGdfMGdfMWdfMikiXSxbMCwxLCJcXGV0YV97Z18wLGdfMX0iXSxbMCwyLCJcXGV0YV97Z18xLGdfMn0iLDJdLFsyLDMsIlxcZXRhX3tnXzAsZ18xZ18yfSJdLFsxLDMsIlxcZXRhX3tnXzBnXzEsZ18yfSJdXQ==
\[\begin{tikzcd}
	{\rho(g_0)\circ \rho(g_1)\circ \rho(g_2)} & {\rho(g_0g_1)\circ \rho(g_2)} \\
	{\rho(g_0)\circ \rho(g_1g_2)} & {\rho(g_0g_1g_2)}
	\arrow["{\eta_{g_0,g_1}}", from=1-1, to=1-2]
	\arrow["{\eta_{g_1,g_2}}"', from=1-1, to=2-1]
	\arrow["{\eta_{g_0g_1,g_2}}", from=1-2, to=2-2]
	\arrow["{\eta_{g_0,g_1g_2}}", from=2-1, to=2-2]
\end{tikzcd}\]

commutes for all $g_0,g_1,g_2\in G$.

\item There is an equality of functors $ \rho(e)=\id_{\C}$, where $e\in G$ is the identity element\footnote{The condition is added for simplicity of notation down the line - the axioms already imply that there is an equivalence $\rho(e)\cong \id_{\C}$.}.

\end{enumerate}

\raggedleft\qedsymbol{}
\end{definition}

We can similarly define a categorical group action on a spherical fusion category and modular category. Namely, the group of automorphisms is replaced by the group of automorphisms which respect the additional structure, and the natural isomorphisms between those functors are required to respect that extra structure as well. In this case we allow ourselves extra adjectives to emphasize the additonal structure. Namely, we refer to a \textit{braided} categorical group action of a group on a braided fusion category. We can thus give the following physical principle:

\begin{equation*}
\left(\text{Symmetry enriched topological phases} \right) \iff
\left(
\substack{
\text{triples $(\C,G,\rho)$,}\\
\text{where $\C$ is a modular category, $G$ is a finite group}\\
\text{and $\rho$ is a braided categorical group action}
}
\right).
\end{equation*}

Algebraically, the physical phenominon of \textit{symmetry breaking} is extremly clear. It is the act of taking a symmetry enriched phase $(\C,G,\rho)$ and passing some subgroup $H\leq G$,  yield $(\C,H,\left.\rho\right|_{H})$. Passing all the way to the trivial group yields the ordinary topological phase $\C$.

We now move on to discussing symmetry fractionalization.

[WORK: motivate the answer physically.

It's quite important that NOT every symmetry can be fractionalized. This is where the gauging anomaly shows up. Talk about this.
]

\begin{definition}[$G$-graded fusion category] A \textit{$G$-graded fusion category} over a finite group $G$ is the following data:

\begin{enumerate}
\item A fusion category $\C$;
\item A full subcategory $\C_{g}$ of $\C$ for each $g\in G$.
\end{enumerate}

Such that:

\begin{enumerate}
\item The functor

\begin{align*}
\bplus_{g\in G}\C_g &\xrightarrow{\sim} \C\\
(A_g)_{g\in G}&\mapsto \bigoplus_{g\in G}A_g
\end{align*}

is an equivalence of categories;

\item The tensor product on $\C$ restricts to a map $\otimes: \C_g\times \C_h \xrightarrow{}\C_{gh}$;

\item (Faithfullness\footnote{This condition is not included in the definition by many authors.}) The subcategories $\C_g$ are nonzero for all $g\in G$.
\end{enumerate}

\raggedleft\qedsymbol{}
\end{definition}

\begin{definition}[$G$-crossed braided fusion category] A \textit{$G$-crossed braided fusion category} over a finite group $G$ is the following data:

\begin{enumerate}
\item A $G$-graded fusion category $\C^{\times}_{G}$ with $g$-component $\C_{g}$;
\item A categorical group action $\rho$ of $G$ on $\C^{\times}_{G}$;
\item A natural isomorphism

$$\beta_{g,A,B}:A\otimes B \xrightarrow{}\rho(g)(B)\otimes A$$

for all $A\in \C_{g}$, $B\in \C^{\times}_{G}$, $g\in G$.
\end{enumerate}

Such that:

.[WORK: add compatibility. Namely, heptagon.

Also there's another big one:

$$\rho(g)(\C_h)\subseteq \C_{g h g^{-1}}.$$
]

\raggedleft\qedsymbol{}
\end{definition}

Finally, we arrive at the definition of a $G$-crossed modular category:

\begin{definition}[$G$-crossed modular category] A \textit{$G$-crossed modular category} over a finite group $G$ is the following data:

\begin{enumerate}
\item A $G$-crossed braided fusion category $\C_{G}^{\times}$;
\item A spherical structure on $\C_{G}^{\times}$;
\end{enumerate}

Such that:

\begin{enumerate}
\item .[WORK: the spherical structure should be compatible with the $G$-crossed braided structure].
\item $\C_{e}$ is a modular category, where $e\in G$ is the identity element. [WORK: say this nicer. Should I introduce better notation for group elements?]
\end{enumerate}

\raggedleft\qedsymbol{}
\end{definition}

We observe from our definitoins that the identity graded component $\C_{e}$ of any $G$-crossed modular category is itself a modular category. The assignment $\C^{\times}_{G}\mapsto \C=\C_{e}$ corresponds to the physical transition from a phase where defects exist and can be freely manipulated, and a phase in which no free defects can be made. This is known physically as \textit{confinement}. This gives us the following picture:

% https://q.uiver.app/#q=WzAsNCxbMSwwLCJcXENee1xcdGltZXN9X3tHfSJdLFsxLDEsIlxcc3Vic3RhY2t7XFx0ZXh0eyRHJC1jcm9zc2VkfVxcXFwgXFx0ZXh0e21vZHVsYXIgY2F0ZWdvcnl9fSJdLFswLDAsIlxcQyJdLFswLDEsIlxcc3Vic3RhY2t7XFx0ZXh0e21vZHVsYXIgY2F0ZWdvcnl9XFxcXCBcXHRleHR7d2l0aCBicmFpZGVkfSBcXFxcIFxcdGV4dHtjYXRlZ29yaWNhbCAkRyQtYWN0aW9ufX0iXSxbMSwwLCIiLDAseyJzdHlsZSI6eyJib2R5Ijp7Im5hbWUiOiJzcXVpZ2dseSJ9fX1dLFsyLDAsIlxcdGV4dHtmcmFjdGlvbmFsaXphdGlvbn0iLDAseyJvZmZzZXQiOi0xfV0sWzAsMiwiXFx0ZXh0e2NvbmZpbmVtZW50fSIsMCx7Im9mZnNldCI6LTF9XSxbMywyLCIiLDAseyJzdHlsZSI6eyJib2R5Ijp7Im5hbWUiOiJzcXVpZ2dseSJ9fX1dXQ==
\[\begin{tikzcd}
	\C & {\C^{\times}_{G}} \\
	\begin{array}{c} \substack{\text{modular category}\\ \text{with braided} \\ \text{categorical $G$-action}} \end{array} & \begin{array}{c} \substack{\text{$G$-crossed}\\ \text{modular category}} \end{array}
	\arrow["{\text{fractionalization}}", shift left, from=1-1, to=1-2]
	\arrow["{\text{confinement}}", shift left, from=1-2, to=1-1]
	\arrow[squiggly, from=2-1, to=1-1]
	\arrow[squiggly, from=2-2, to=1-2]
\end{tikzcd}\]

as mentioned, it is not always possible to fractionalize symmetries. We will discuss this in more detail in section [ref].


\subsubsection{Gauging symmetries}

[WORK: This is the section where I talk about gauging. Not sure how much I would have mentioned it before. First off, it would be nice to talk about gauging from a physical perspective. The end game is to state the maps

$$ \C^{\times}_{G}  <-- condensation, gauging --> \left(\C^{\times}_{G}\right)^{G}$$

properly. General discussion gauge symmetry breaking phase transition would be nice to talk about. If I could state everything clearly mathematically that would be great.
]

\begin{proposition} Let $\C$ be a fusion category with a categorical $G$ action. Define a category $\C^{G}$ as follows. It's objects are pairs $(A,\gamma)$, where $A\in \C$ is an object and

$$\gamma_{g,A}:\rho(g)(A)\xrightarrow{\sim}A$$

is an isomorphism for all $g\in G$, $A\in \C$ such that

[WORK: add diagram]

commutes. The morphisms in $\C^{G}$ are morphisms in $f$ such that

[WORK: add diagram]

commutes. [WORK: define the rest of the structure]. This endows $\C^{G}$ with the structure of a fusion category. We call $\C^{G}$ the \textit{equivariantization} of $G$.

[WORK: the space of possible maps $\gamma$ making $(n\cdot \one, \gamma)$ an element of $\C^{G}$ is equivalent to the space of $n$-dimensional representations of $G$. To make this canonical, however, one needs to choose a choice of trivialization of $\rho(g)(\one)$ for all $g\in G$. That is, an isomorphism $\rho(g)(\one)\cong \one$.

I figured it out. I added the axiom that $\rho(e)$ is the identity. We thus have the distinguished map

\begin{align*}
&\rho(g)(\one)\\
&\to\rho(g)(\one)\otimes \rho(e)(\one)\\
&\to\rho(g)(\one)\otimes \rho(g)\rho(g^{1})(\one)\\
&\to \rho(g)(\one \otimes \rho(g^{-1})(\one))\\
&\to \rho(g)(\rho(g^{-1})(\one))\\
&\to \rho(e)(\one)\\
&\to \one
\end{align*}


which fixes the issue of canonical trivialization. Should I add this as a footnote maybe? Feels overly technical.
]

[WORK: figure out how to pass this lemma through once there's extra structure added onto $\C$.]
\end{proposition}

\begin{proof}.[WORK: do proof]
\end{proof}

In light of the above proposition, we give the following definition:

\begin{definition}[Internal $G$-symmetry] An \textit{internal $G$-symmetry} of a fusion category $\C$ is a fully faithful functor $\Rep(G)\xrightarrow{} \C$.
\end{definition}

[WORK: I'm realizing that I haven't done much with representation categories up to here. Should I talk about them more? Say some generalities about representation theory?]

Proposition [ref] tells us that every fusion category with a categorical $G$-action induces a fusion category with an internal $G$-symmetry. [WORK: give motivation for why this is gauging].

We now discuss \textit{condensation}. As discussed above, condensation should serve as an inverse to gauging. Thus, it should take as an input a category with an internal $G$-symmetry and output a category with a categorical $G$-action. We define this procedure as follows:

\begin{proposition}.[WORK: define equivariantization and assert that it behaves well]
\end{proposition}
\begin{proof}.[WORK: do proof]
\end{proof}

We now verify that these two constructions are indeed inverses to one another:

\begin{proposition}.
[WORK: define maps

$$\C\to (\C^{G})_{G}$$

and

$$\C\to (\C_G)^G.$$

State that these are equivalences of categories. Furthermore, assert that these equivalences of categories go through when we add all of our extra structure back on.

The principle here is that the data of a $G$-grading is the same as the data of an internal $G$-symmetry. This is the ``main principle" of Bruguieres and Muger \cite{bruguieres2000categories, muger2000galois}. A good exposition of it is in \cite{drinfeld2010braided}.
]
\end{proposition}

Thus, combining the results of this section with the ones from the previous section we get the following picture:

[WORK: add picture; gauging, condensing, defectifying, confining]



\subsubsection{Obstruction theory, anomalies, and zesting}

[WORK: this section is devoted to mathematical results which are neccecary from the first two sections. This might have to be broken up for readability]

[WORK: A lot of the stuff in this section is going to be about graded categories. There needs to be a good discussion of equivariantization and deequivariantization. I suppose there also needs to be some discussion of Tannakian subcategories. Lots of stuff to do - this section could be quite mathematically intricate. I will NOT prove the main obstruction theory theorem from ENO. It would be nice to have a proper statement of it though... Maybe the modern theory of zesting is relevant for a good statement?

The theory of zesting seems to be pretty good. There are several papers which have made major progress on it: it was formally introduced in \cite{delaney2021braided}. The paper  \cite{delaney2021zesting} explains how zesting gives MTCs with the same modular data - this could be a good exercise for the end of the section. The paper I care about is \cite{delaney2024g}, which allows me to go from solution-to-solution in obstruction theory, realizing the torsor action.
]
\subsubsection{$G$-crossed modularity}

[WORK:

There's nice stuff to be said about the modular representation of a $G$-crossed MTC. After the original Barkeshli paper, the follow up paper \cite{babakhani2023g} included a lot of the nice results.

There is a $G$-crossed verlinde formula - \cite{deshpande2019crossed}. Seeing as it is new it is not very well known, and could easily be overlooked. I'm sure the proof isn't that bad, and it would be really good to have included. It feels important to me.
]

[WORK: Here are some tricks I found useful a few months ago when I was doing this

\begin{enumerate}
\item Given any $A\in \C_{g}$, all other elments of $\C_g$ an be obtained by fusion with an element of $\C_{e}$;

\item Rank($\C_g$)=$\#$ fixed points of $\C_{e}$ under action of $g$
\end{enumerate}

is this a good section to include them in?
]

\subsubsection{Skeletonization of SET phases}

.[WORK:
I want to list data for SET phases soon, so skeletonization is important. Talk about it. Set it up right. Do it.
]




\subsection{Fermionic topological order}

\subsubsection{Physical picture}

\subsubsection{Kitaev spin chain}

[WORK: talk about Kitaev spin chain. Emphasize that bulk defects are fermions! These fermions can interact with the boundary Majorana modes.]

\subsubsection{The algebraic theory of fermionic topological order}


.[WORK: give the algebraic theory. This should be very easy to set up based on the work I've done in the previous sections.

I should talk about 16 fold way too. Prove that if a gauging exists then there are 16 solutions. State the 16 fold way theorem, but don't prove it. The proof is \cite{johnson2024minimal} and it is very deep.  Talking to Theo, he emphasizes that the proof is highly nontrivial and is not very physically motivated.]

\subsubsection{The 16-fold way on the trivial fermionic theory}

.[WORK: Talk about Kitaev's 16-fold way on the trivial fermionic theory.

This means showing that the obstruction vanishes, and then constructing all 16 theories, and saying the relevant things about them.
]

\subsubsection{Modular representations of fermionic topological order}

[WORK:

One easy thing to do is the modular representation of fermionic topological order. It just has a small subtlety (you have to pass to a congruence subgroup for spin structure reasons) but then everything works out! This is shown here: \cite{bonderson2018congruence}.

Maybe there's more to say in this section? Talk about Verlinde formula and other hallmarks of modular categories, but just insofar as they have special applications to fermionic topological order.
]

\subsubsection{Super-groups and Deligne's theorem}


[WORK: 

A very important theorem in this area is Deligne's theorem for fusion categories: Every symmetric fusion category is equivalent to the category of representations of a finite group or super-group. The original proof is in \cite{deligne2002categories}, but that's way too general and I'm sure there are better/more modern proofs for my case.

Depending on how important this theorem is, I might have to move this up. In Theo's paper \cite{johnson2024minimal} he talks about Deligne's theorem + his obstruction results gives a complete picture for modularization and gauging away centers. This is an interesting perspective which I would like to include.
]



$\newline$
\fbox{\parbox{\dimexpr\linewidth-2\fboxsep-2\fboxrule\relax}{

\begin{center}
\textbf{History and further reading:}\\
\textit{(Boundaries and domain walls)}\\
\end{center}

The first major work on boundaries in conformal field theory was Cardy's semimal 1989 paper \cite{cardy1989boundary}. This work was then reinterpreted in terms of category theory by Fuchs and Schweigert in 2001 \cite{fuchs2001category}, who introduced the idea of interpreting boundaries in terms of module-categories and Frobenius algebras. This theory was then generalized by Frohlich, Fuchs, Runkel, and Schweigert to include a categorical description of domain walls as well \cite{frohlich2005picard}.

$\newline$
In parallel, Bravyi and Kitaev (along with similar work by Freedman and Meyer \cite{freedman2001projective}) studied the toric code with boundary and discovred that it had two different boundary types \cite{bravyi1998quantum}. In 2008 Bombin and Martin-Delgado defined a version of the Kitaev quantum double model with boundary based on any finite group \cite{bombin2008family, bombin2011nested}. In Kitaev and Kong's 2012 paper \cite{kitaev2012models}, they generalized this to an arbitary doubled topological order using the Levin-Wen model. They brought the well-established theory of boundaries in conformal field theory to topological order. The full theory of boundaries and domain walls in topological order was then established in 2016 by Cong, Cheng, and Wang \cite{cong2016topological}.

$\newline$
The canonical reference for the theory of module categories is Etingof-Gelaki-Nikshych-Ostrik \cite{etingof2016tensor}. This reference collected a lot of the earlier literature references, such as \cite{ostrik2003module, etingof2005fusion, etingof2011weakly}.
}}

$\newline$
\fbox{\parbox{\dimexpr\linewidth-2\fboxsep-2\fboxrule\relax}{

\begin{center}
\textbf{History and further reading:}\\
\textit{(Symmetry enriched topological order)}\\
\end{center}

Many of the most famous examples of topological quantum systems are not topologically ordered systems, but symmetry protected topological systems. The first example of symmetry protected topological order was the Haldane phase of the odd-integer spin chain \cite{haldane1983nonlinear}, which is protected by $\SO(3)$ spin rotation symmetry. Another famous example is that of the topological insulator, which is portected by $U(1)$ phase symmetry and time-reversal symmetry \cite{kane2005quantum, hasan2010colloquium}. The existence of these important examples begs for a description in the category-theoretic language.

$\newline$
This mathematical language, however, had to wait for a while. There are lots of subtlties involved, and people were still working out the theory for ordinary topological order. The notion of $G$-crossed modular category was introduced in Turaev's giant work on homotopy quantum field theory \cite{turaev2010homotopy}. Homotopy quantum field theory was not intended as a physically relevant work - it was mostly a mathematical curiosity, which was worked on because it seemed natural. The theory of $G$-crossed fusion categories was then applied to the process of orbifolding in confomral field theory \cite{kirillov2004g}, though most of the semial work such as Etingof-Nikshych-Ostrik's application of obstruction theory to $G$-crossed categories was still being done as a purely mathematical exercise.

$\newline$
It was not until the work of the work of Barkeshli-Bonderson-Cheng-Wang that $G$-crossed modular categories were introduced as the correct formalism for symmetry enriched topological order \cite{barkeshli2019symmetry}. This bridging of the mathematial literature and the physical literature led to a huge amount of progress in the area.
}}

$\newline$
\fbox{\parbox{\dimexpr\linewidth-2\fboxsep-2\fboxrule\relax}{

\begin{center}
\textbf{History and further reading:}\\
\textit{(Fermionic topological order)}\\
\end{center}

There is a very strong link between fermions and $\ZZ_2$-gradings. This link was made popular by the discovery of \textit{supersymmetry} in string theory, which mean passing to a $\ZZ_2$-graded geometry \cite{varadarajan2004supersymmetry}. The fact that topological order whose fundamental degrees of freedom are fermionic need special attention was brough to light by Gu, Wang, and Wen, in a series of foundational papers \cite{gu2014lattice, gu2014symmetry, gu2015classification}. The formalism of super-modular categories was introduced in 2017 \cite{bruillard2017fermionic}. This paper brought together a lot of the surrounding literature into a coherent physical narrative. The central conjecture of this narrative was the 16-fold way, which was established in 2024 by Johnson-Freyd and Reutter \cite{johnson2024minimal}
}}


$\newline\newline$

\large \textbf{Exercises}:\normalsize

\begin{enumerate}[\thesection .1.]

\item .[WORK: show that there is a canonical bijection between structures of a left $\C$-module on $\M$ and $\CC$-linear monoidal functors

$$\M\xrightarrow{}\End(\C).$$

This is a categorical version of the fact that modules and representations are in bijection.]

\item . [WORK: people love the Tambara-Yamagami theorem, and consider it to be very deep. One of the original motivation of the algebraic theory of SET phases is that it gives a quick proof! It would thus be nice to have the Tambara-Yamagmi result here as an exercise. \cite{tambara1998tensor}.]


\item .[WORK: categorical group actions can be reinterpreted in terms of funtors. Namely, a category group action is a monoidal functor from the categorified version of the group. The categorified version of the group is defined so that it has an object for each group element, its only morphisms are the identity, and its monoidal structure is given by the group law.]


\item . [WORK: There is the principle of \textit{modularization}. Here, we take the center (which will be some symmetric fusion category) and ``gauge it away". The fact that this is possible comes from our setup of symmetry enriched toplogical order and fermionic topological order. I don't quite see how it is physically relevant. It seems best left here as an exercise.]

\item .[WORK: I think that a SFC is abelian if and only if its Drinfeld center is. This would be a nice exercise to include.]
\end{enumerate}