\documentclass{article}

\usepackage[utf8]{inputenc}
\usepackage{enumerate}
\usepackage{amsfonts}
\usepackage{amsmath}
\usepackage{amsthm}
\usepackage{blindtext}
\usepackage{graphicx}
\usepackage[numbers]{natbib}
\usepackage{amssymb}
\usepackage{mathtools}
\usepackage{stmaryrd}
\usepackage{tikz-cd}
\usepackage{relsize}
\usepackage{mathrsfs}
\usepackage{ upgreek }
\usepackage[normalem]{ulem}
\usepackage{quiver} 




\newtheorem{theorem}{Theorem}
\newtheorem{lemma}{Lemma}
\newtheorem{corollary}{Corollary}
\newtheorem{conjecture}{Conjecture}
\newtheorem{proposition}{Proposition}
\theoremstyle{definition}
\newtheorem*{definition}{Definition}
\newtheorem{remark}{Remark}
\newtheorem{experiment}{Experiment}
\newtheorem{proposition-definition}{Proposition-Definition}


\graphicspath{ {./images/} }
\numberwithin{figure}{section}
\setcounter{section}{-1}


\title{The Quantum Double Model for Abelian Groups}
\author{Milo Moses}


\begin{document}


\maketitle

\newcommand{\RR}{\mathbb{R}}
\newcommand{\HH}{\mathbb{H}}
\newcommand{\NN}{\mathbb{N}}
\newcommand{\QQ}{\mathbb{Q}}
\newcommand{\DD}{\mathcal{D}}
\newcommand{\CC}{\mathbb{C}}
\newcommand{\FF}{\mathbb{F}}
\newcommand{\ZZ}{\mathbb{Z}}
\newcommand{\Zcal}{\mathcal{Z}}
\newcommand{\Ncal}{\mathcal{N}}
\newcommand{\LL}{\mathscr{L}}
\newcommand{\TT}{\mathcal{T}}
\newcommand{\Ccat}{\mathscr{C}}
\newcommand{\Dcat}{\mathscr{D}}
\newcommand{\Ecat}{\mathscr{E}}
\newcommand{\st}{\,\,\mathrm{s.t.}\,\,}
\newcommand{\mm}{\mathfrak{m}}
\newcommand{\pp}{\mathfrak{p}}
\newcommand{\Hom}{\mathrm{Hom}}
\newcommand{\Aut}{\mathrm{Aut}}
\newcommand{\Frac}{\mathrm{Frac}}
\newcommand{\tr}{\mathrm{tr}}
\newcommand{\op}{\mathrm{op}}
\newcommand{\res}{\mathrm{res}}
\newcommand{\im}{\mathrm{im}}
\newcommand{\ev}{\mathrm{ev}}
\newcommand{\coev}{\mathrm{coev}}
\newcommand{\id}{\mathrm{id}}
\newcommand{\coker}{\mathrm{coker}}
\newcommand{\SL}{\mathrm{SL}}
\newcommand{\End}{\mathrm{End}}
\newcommand{\Rep}{\bold{Rep}}
\newcommand{\Set}{\bold{Set}}
\newcommand{\Vecc}{\bold{Vec}}
\newcommand{\Top}{\bold{Top}}
\newcommand{\Grp}{\bold{Grp}}
\newcommand{\Hilb}{\bold{Hilb}}
\newcommand{\Bord}{\bold{Bord}}
\newcommand{\Cat}{\bold{Cat}}
\newcommand{\0}{\left|0\right>}
\newcommand{\1}{\left|1\right>}
\newcommand{\nullclass}{\left|\bold{0}\right>}
\newcommand{\alphaclass}{\left|\alpha\right>}
\newcommand{\betaclass}{\left|\beta\right>}
\newcommand{\alphabetaclass}{\left|\alpha\beta\right>}
\newcommand{\ppsi}{\left|\psi\right>}
\newcommand{\pphi}{\left|\phi\right>}
\newcommand{\func}{\mathrm{func}}
\newcommand{\bigleadsto}{\mathlarger{\mathlarger{\mathlarger{\leadsto}}}}
\newcommand{\vin}{\rotatebox[origin=c]{-90}{$\in$}}

Kitaev's quantum double model gives a systematic way of creating topological order based on any finite group. Given a group $G$, we create topological order $\DD(G)$\footnote{$\DD$ for double, and for Drinfeld.}. These topological orders then go on to induce theories of anyons/topological quantum computation, and error correcting codes.

The simplest case is when $G=\ZZ_2$, where one recovers the toric code. The typical exposition of Kitaev's model will first deal with the toric code, and then go to generic groups. Going to generic groups comes with lots of difficulties, however - you have to deal with intricate operator algebras, you have to move along sites/ribbons, and your non-abelian anyons have non-trivial internal states to moniter.

For abelian groups, however, none of these fancy techniques are needed. It is very straightforward to generalize the toric code to all finite abelian groups without introducing the technical machinery present in the non-abelian case. It is this abelian group specific analysis we present. Thus, for all of the discussion fix a lattice on the torus and a finite abelian group $G$. We will define the $\DD(G)$ topological order.

On every edge of the lattice, place of copy of the group algebra $\CC[G]$. In other words, we work in the algebra

$$\Ncal=\bigotimes_{\text{edges on lattice}}\CC[G].$$

We now define the operators

\begin{align*}
T_g:\CC[G]&\to\CC[G],\,\, g\in G\\
\left| h\right>&\mapsto \left| gh\right>
\end{align*}

and

\begin{align*}
R_\chi:\CC[G]&\to\CC[G],\,\, \chi\in \widehat{G}\\
\left| g\right>&\mapsto \chi(g)\left| g\right>
\end{align*}

where $\widehat{G}=\Hom(G,\CC^\times)$ is the character group of $G$. We use these to define our stabilizers.

Unlike in the case of $G=\ZZ_2$, orientations matter because elements are not equal to their inverses. Hence, add orientations to each edge (i.e. make them directed), and fix a choice of orientation on the torus. We now let

$$A_{g,v}=\bigotimes_{\substack{\text{edges} \\ \text{going into }v}}T_g \otimes \bigotimes_{\substack{\text{edges} \\ \text{going out of }v}} T_{g^{-1}}$$

for

$$B_{\chi,p}=\bigotimes_{\substack{\text{edges}\\ \text{clockwise to }p}}R_\chi \otimes \bigotimes_{\substack{\text{edges}\\ \text{counterclockwise to }p}}R_{\chi^{-1}}$$

where $g\in G$, $\chi\in \widehat{G}$, $v$ is a vertex, $p$ is a face, and clockwise/counterclockwise are defined using the orientation on the torus. These are the stabilizers in our code. Namely, the codespace $\LL$ is defined to be the common $+1$ eigenspace of all of the $A_{g,v}$ and $B_{\chi,p}$. The following properties summarize the basic properties of these operators:

\begin{lemma} For all $g\in G$, $\chi\in \widehat{G}$, verticies $v$, and faces $p$, we have that

$$R_{\chi}T_g=\chi(g) T_gR_\chi$$

and

$$B_{\chi,p}A_{g,v}=A_{g,v}B_{\chi,p}.$$

Both $T_g$ and $R_\chi$ are unitary.
\end{lemma}
\begin{proof}.[WORK: do proof well]
\end{proof}

With these operators, we can define the Hamiltonian

$$H=-\sum_{\substack{\text{verticies }v \\ g\in G}}A_{g,v} - \sum_{\substack{\text{faces }p \\ \chi \in \widehat{G}}}B_{\chi,p}.$$

It's ground states will be the ones in the $+1$ eigenspace of each $A_{g,v}$ and $B_{\chi,p}$, by the fact that $A_{g,v}$ and $B_{\chi,p}$ commute. The lowest energy is thus

$$-|G| \cdot |\text{verticies}|-|G|\cdot |\text{faces}|=-|G|\cdot |\text{edges}|,$$

where we here use the fact that the Euler characteristic of the torus is $0$. We now discus what anyonic excitations in this model:

\begin{proposition} Let $\ppsi\in\Ncal$ be a simultaneous eigenvector of all of the stabilizers. For all verticies $v$ there exists a unique $\chi_{v}\in \widehat{G}$ such that

$$A_{g,v}\ppsi = \chi_{v}(g)\ppsi$$

for all $g\in G$. For all faces $p$ there exists a unique $g_p\in G$ such that

$$B_{\chi,p}\ppsi=\chi(g_p)\ppsi$$

for all $\chi\in \widehat{G}$. We have that

$$\prod_{\text{verticies }v}\chi_v=1,\,\, \prod_{\text{faces }p}g_p=1.$$
\end{proposition}
\begin{proof}.[WORK: write out details]
\end{proof}

We summarize the above proposition as follows. There are anyonic excitations at verticies, corresponding to failiures of the $A_{g,v}$ stabilizers. There is one vertex quasiparticle type for each non-trivial character in $\widehat{G}$. There are anyonic excitations at faces, corresponding to failiures of the $B_{g,p}$ stabilizers. There is one face quasiparticle type for each non-trivial group element in $G$.

In the following proposition we will study these anyonic excitations more:

\begin{proposition} The Hamiltonian $H$ is a Hermitian matrix. Let $\ppsi\in\Ncal$ be a simultaneous eigenvector of all the stabilizers. Let $n$ be the number of anyonic excitations it contains. Then

$$H\ppsi=\left(-\left| G\right| \cdot |\text{edges}|+ n|G|\right)\ppsi.$$
\end{proposition}
\begin{proof}.[WORK: do proof]
\end{proof}

This proposition can be summarized as saying that this is a gapped system, in the sense that energy levels are discretely seperated. All of the anyon types have the same energy, namely, $|G|$.

The movement and fusing of these excitations is also very simple. Namely, we have the following:

\begin{proposition} Let $\ppsi\in \Ncal$ be a simultaneous eigenvector of all the stabilizers. Suppose that $v_0,v_1$ are verticies and $e:v_0\to v_1$ is an edge connecting them. Suppose that $\ppsi$ has a $\chi_0$-type quasiparticle at $v_0$ and a $\chi_1$-type quasiparticle at $v_1$. The state $R^{\otimes e}_{\chi_0}\ppsi$ has no quasiparticle at $v_0$, and a $\chi_0\chi_1$ type quasiparticle at $v_1$. Here, $R^{\otimes e}_{\chi_0}$ denotes the application of $R_{\chi_0}$ on the $e$-tensor factor of $\Ncal$.

Similarly, suppose that $p_0,p_1$ are faces and $e$ is an edge between $p_0$ and $p_1$, whose orientation is clockwise relative to $p_0$. Suppose that $\ppsi$ has a $g_0$-type quasiparticle at $p_0$ and a $g_1$-type quasiparticle at $p_1$. The state $T_{g_0}^{\otimes e}\ppsi$ has no quasiparticle at $p_0$, and a $g_0g_1$-type quasiparticle at $p_1$.
\end{proposition}
\begin{proof}.[WORK: write out the (trivial) proof]
\end{proof}

This above proposition can be summarized as saying that $\chi$-type quasiparticles move along the lattice by $R_\chi$, and $g$-type quasiparticles move the dual lattice by $T_g$. Additionally, this proposition says that when a $\chi_0$ type quasiparticle and a $\chi_1$ type quasiparticle are fused you get a $\chi_0\chi_1$ type quasiparticle. When a $g_0$ and $g_1$ type quasiparticle are fused you get a $g_0g_1$ type quasiparticle.

With this, we can finally compute the dimension of the codespace $\LL$:

\begin{proposition} The dimension of $\LL$ is $|G|^2$. 
\end{proposition}
\begin{proof}.[WORK: do proof]
\end{proof}

Even more strongly, we can exactly describe what this ground space looks like:

\begin{proposition} There is a canonical isomorphism

$$\LL\xrightarrow{\sim}\CC[H^1(T;G)],$$

where $H^1(T;G)$ is the $1$st cohomology of the torus $T$ with $G$ coefficients.
\end{proposition}
\begin{proof}.[WORK: do proof]
\end{proof}

We now discuss the action of braiding anyons. The first result is pretty clear:

\begin{proposition} Braiding a $g$-type and $\chi$-type quasiparticle results in a $\chi(g)$ global phase.
\end{proposition}
\begin{proof}.[WORK: do proof]
\end{proof}

Now, we examine the action of braiding quasiparticles around the non-trivial loops of the torus. Quasiparticles on verticies move along the lattice. Hence, homotopy classes of paths will correspond to elements of $H_1(T;\ZZ)$. Quasiparticles of faces move along the dual lattice. Hence, homotopy classes of paths will correspond to elements of $H^1(T;\ZZ)$.

\begin{proposition} Moving a $\chi$-type quasiparticle along a path of the lattice induces an action $\LL\to \LL$. This action is well defined along homology classes of paths. Let $\alpha\in H_1(T;\ZZ)$ be the homotopy class of path $\chi$ is moved along. We denote the action $\LL\to \LL$ by $U_{\chi,\alpha}$. Indentifying $\LL$ with $\CC[H^1(T;G)]$, we have the explicit description

\begin{align*}
U_{\chi,\alpha}:\CC[H^1(T;G)]&\xrightarrow{}\CC[H^1(T;G)],\\
\left| \omega \right>& \mapsto \chi(\omega(\alpha))\left| \omega \right>
\end{align*}

where $\omega\in H^1(T;G)$ is considered as a map $H_1(T;G)\to G$, and we identify elements of $H_1(T;\ZZ)$ with elements of $H_1(T;G)=H_1(T;\ZZ)\otimes_\ZZ G$ by tensoring with the identity group element.
\end{proposition}
\begin{proof}.[WORK: do proof]
\end{proof}


Similarly, we have the following for $g$-type quasiparticles:

\begin{proposition} Moving a $g$-type quasiparticle along a dual path of the lattice induces an action $\LL\to \LL$. This action is well defined along cohomology classes of paths. Let $\omega \in H_1(T;\ZZ)$ be the homotopy class of of the path $g$ is moved along. We denote the action $\LL\to \LL$ by $U_{g,\omega}$. Identifying $\LL$ with $\CC[H^1(T;G)]$, we have the explicit description

\begin{align*}
U_{g,\omega}: \CC[H^1(T;G)]&\xrightarrow{}\CC[H^1(T;G)]\\
\left|\omega'\right>&\mapsto \left|(\omega\otimes g)+\omega'\right>
\end{align*}

where we treat $\omega\otimes g$ as element of $H_1(T;G)=H_1(T;\ZZ)\otimes_\ZZ G$.
\end{proposition}
\begin{proof}.[WORK: do proof]
\end{proof}

Composing maps of the form $U_{\chi,\alpha}$ and $U_{g,\omega}$ gives a complete description of the quantum computation possible using the $\DD(G)$ topological order. All that is left now is to make some miscilaneous observations about the above theory.

We now describe briefly the quantum circuits arrising from these error correcting codes.

As seen, all of the stabilizers and movements are are implemented by performing 

\bibliographystyle{alpha}
\bibliography{ref}


\end{document}






